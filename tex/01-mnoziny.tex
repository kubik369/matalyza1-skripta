\chapter{Množiny}

V daľšom budeme používať tieto označenia:
\begin{itemize}[label=]
  \item $\mathbb{N}$ množina všetkých prirodzených čísel ($= \{ 1, 2, 3, \ldots \}$)
  \item $\mathbb{Z}$ množina všetkých celých čísel ($= \{ 0, -1, 1, 2, -2, \ldots \}$)
  \item $\mathbb{Q}$ množina všetkých racionálnych čísel
    ($= \{ \frac{p}{q}; p \in \mathbb{Z} \land q \in \mathbb{N} \}$)
  \item $\mathbb{R}$ množina všetkých reálny čísel
  \item $\mathbb{R}^+$ množina všetkých kladných reálny čísel ($= ( 0, \infty )$)
  \item $\mathbb{R}^+_0$ množina všetkých nezáporných reálny čísel
    ($= \langle 0, \infty )$)
\end{itemize}
Ak pre niektorý prvok $a$ neprázdnej množiny $A \subset \mathbb{R}$ platí
$$\forall x \in A : x \leq a (\forall x \in A: q \geq a)$$
nazývame tento prvok \emph{maximum (minimum) množiny $A$} a označujeme ho
$max(A)$ ($min(A)$)

\section{Reálne čísla}

\begin{enumerate}[resume]
    \foreachproblem[realne-cisla]{\item\thisproblem}
\end{enumerate}

\section{Ohraničené množiny reálnych čísel, supremum a infimum}

Neprázdna množina $A \subset \mathbb{R}$ sa nazýva \textit{zhora (zdola)
ohraničená}, ak platí

$$
\exists K \in  \mathbb{R} \quad \forall x \in A: x \leq K \:
(\exists K \in \mathbb{R} \quad \forall x \in A: x \geq K).
$$

Číslo $K$ s uvedenou vlastnosťou sa nazýva \textit{horné (dolné)
ohraničenie množiny $A$.} $\emptyset$ považujeme za ohraničenú zhora aj zdola.

Množina, ktorá je zhora aj zdola ohraničená, sa nazýva \textit{ohraničená}.
Množina, ktorá nie je ohraničená sa nazýva \textit{neohraničená}.

\begin{enumerate}[resume]
	\item \useproblem[supremum-infimum]{supremum-infimum-1}
	\item \useproblem[supremum-infimum]{supremum-infimum-2}
	\item \useproblem[supremum-infimum]{supremum-infimum-3}
\end{enumerate}


Číslo $\alpha \in \mathbb{R}$ sa nazýva \textit{supremum množiny $A$}
$\subset \mathbb{R}, A \neq \emptyset$, ak
\begin{align}
\forall x \in A: x \leq \alpha \label{eq:supremum-i} \\
(\forall \varepsilon > 0) (\exists \: x_\varepsilon \in A):
x_\varepsilon > \alpha - \varepsilon
\end{align}

Podľa \ding{34} je $\alpha$ horné ohraničenie množiny $A$; \ding{37} je negácia
výroku ``pre niektoré $\varepsilon > 0$ je číslo $\alpha - \varepsilon$ horným
ohraničením množiny $A$'', hovorí teda, že neexistuje horné ohraničenie množiny
$A$, ktoré by bolo menšie než $\alpha$. Teda $\alpha$ je najmenšie horné
ohraničenie množiny $A$. Supremum množiny $A$ označujeme $\sup \{ A \}$.

Číslo $\beta \in \mathbb{R}$ sa nazýva \textit{infimum množiny} $A \subset
\mathbb{R}, A \neq \emptyset$, ak:

\begin{align*}
  \forall x \in A&: x \geq \beta \tag{\ding{34}} \\
  (\forall \varepsilon > 0) (\exists \: x_\varepsilon \in A)&:
  x_\varepsilon < \beta + \varepsilon \tag{\ding{37}}
\end{align*}

To znamená, že $\beta$ je najväčšie z dolných ohraniční množiny $A$. Infimum
množiny $A$ označujeme $\inf\{A\}$

Ak usporiadané pole $\mathbb{R}$ reálnych čísel konštruujeme z poľa
$\mathbb{Q}$ racionálnych čísel pomocou Dedekindových rezov, môžeme dokázať
nasledujúce dve ekvivalentné tvrdenia:

\begin{veta}
Každá neprázdna zhora ohraničená množina reálnych čísel má supremum.
\end{veta}

\begin{veta}
Každá neprázdna zdola ohraničená množina reálnych čísel má infimum.
\end{veta}

Ak usporiadané pole $\mathbb{R}$ zavádzame axiomaticky, považujeme prvé z
uvedených tvrdení za axiómu, z nej možno odvodiť vetu o existencií infima.

\showanswers
\begin{enumerate}[resume]
  \item \useproblem[supremum-infimum]{supremum-infimum-4}
  \hideanswers
  \item \useproblem[supremum-infimum]{supremum-infimum-5}
  \item \useproblem[supremum-infimum]{supremum-infimum-6}
  \item \useproblem[supremum-infimum]{supremum-infimum-7}
  \item \useproblem[supremum-infimum]{supremum-infimum-8}
\end{enumerate}

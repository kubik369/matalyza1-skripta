\chapter{Množiny}%\label{chapter:gramatiky}

V daľšom budeme používať tieto označenia:

$\mathbb{N}$ množina všetkých prirodzených čísel ($= \{ 1, 2, 3, \ldots \}$)

$\mathbb{Z}$ množina všetkých celých čísel ($= \{ 0, -1, 1, 2, -2, \ldots \}$)

$\mathbb{Q}$ množina všetkých racionálnych čísel
  ($= \{ \frac{p}{q}; p \in \mathbb{Z} \land q \in \mathbb{N} \}$)

$\mathbb{R}$ množina všetkých reálny čísel

$\mathbb{R}^+$ množina všetkých kladných reálny čísel ($= ( 0, \infty )$)

$\mathbb{R}^+_0$ množina všetkých nezáporných reálny čísel
  ($= \langle 0, \infty )$)

Ak pre niektorý prvok $a$ neprázdnej množiny $A \subset \mathbb{R}$ platí

$$\forall x \in A : x \leq a (\forall x \in A: q \geq a)$$

nazývame tento prvok \textit{maximum (minimum) množiny $A$} a označujeme ho
$max(A)$ ($min(A)$)

\section{Reálne čísla}


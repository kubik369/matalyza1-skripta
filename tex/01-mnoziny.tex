\chapter{Množiny}%\label{chapter:gramatiky}

V daľšom budeme používať tieto označenia:

$\mathbb{N}$ množina všetkých prirodzených čísel ($= \{ 1, 2, 3, \ldots \}$)

$\mathbb{Z}$ množina všetkých celých čísel ($= \{ 0, -1, 1, 2, -2, \ldots \}$)

$\mathbb{Q}$ množina všetkých racionálnych čísel
  ($= \{ \frac{p}{q}; p \in \mathbb{Z} \land q \in \mathbb{N} \}$)

$\mathbb{R}$ množina všetkých reálny čísel

$\mathbb{R}^+$ množina všetkých kladných reálny čísel ($= ( 0, \infty )$)

$\mathbb{R}^+_0$ množina všetkých nezáporných reálny čísel
  ($= \langle 0, \infty )$)

Ak pre niektorý prvok $a$ neprázdnej množiny $A \subset \mathbb{R}$ platí

$$\forall x \in A : x \leq a (\forall x \in A: q \geq a)$$

nazývame tento prvok \textit{maximum (minimum) množiny $A$} a označujeme ho
$max(A)$ ($min(A)$)

\section{Reálne čísla}

\begin{enumerate}
  \item \useproblem[mnoziny]{real3}
\end{enumerate}
3. Nech $a$ je racionálne, $b$ je iracionálne číslo. Potom $a + b$ je
iracionálne číslo. Dokážte!

4. Nech $a, b \in \mathbb{Q}$ a $\sqrt{ab}$ je iracionálne číslo. Potom
$\sqrt{a} + \sqrt{b}$ je iracionálne číslo. Dokážte!

5. Dokážte iracionálnosť čísel
  1. $\sqrt{5}$
  2. $\sqrt{15}$
  3. $\sqrt{2} + \sqrt{3}$
  4. $\sqrt{3} + \sqrt{5}$
  5. $(4\sqrt{3} - 3) / 6$
  6. $\sqrt{3} - \sqrt{2}$

6. Dokážte nasledujúce nerovnosti:
  1. $\forall a, b \in \mathbb{R}^+: \frac{1}{2} ln(a) + \frac{1}{2} ln(b) \leq ln(\frac{a+b}{2})$;
  2. nech $a \in \mathbb{R}, b > 0, c > 0, a < b$; potom  $\frac{a}{b} < \frac{a + c}{b + c}$;
  3. $(a + b) \geq 0 \Rightarrow a^3 + b^3 \geq a^2 b + ab^2$;
  4. $a^2 + b^2 + c^2 \geq ab + bc + ac$;
  5. $\forall a, b \mathbb{R}^+ : \frac{a^3 + b^3}{2} \geq (\frac{a + b}{2})^3$;
  6. $\frac{x^2 + 2}{\sqrt{x^2 + 1}} \geq 2$.

7. Nech $b_1 > 0, \ldots, b_n > 0$. Dokážte, že zlomok $\frac{a_1 + a_2 +
\ldots + a_n}{b_1 + b_2 + \ldots + b_n}$ nie je menší ako najmenší a nie je
väčší ako najväčší zo zlomkov $\frac{a_1}{b_1}, \frac{a_2}{b_2},
\ldots, \frac{a_n}{b_n}$

8. Dokážte nasledujúce tvrdenia:
  1. $\forall n \in \mathbb{N}: 1^2 + 3^2 + \ldots + (2n - 1)^2 = \frac{1}{3} n(2n - 1)(2n + 1)$;
  2. $\forall n \in \mathbb{N}: 1^2 - 2^2 + 3^2 - 4^2 + \ldots + (-1)^{n-1}n^2 = (-1)^{n-1}\frac{1}{2}n(n + 1)$
  3. $\forall n \in \mathbb{N}, n \geq 2: 1 + \frac{1}{\sqrt{2}} + \frac{1}{\sqrt{3}} + \ldots + \frac{1}{\sqrt{n}} > \sqrt{n}$;
  4. $\forall n \in \mathbb{N}, n > 1: \frac{1}{n+1} + \frac{1}{n+2} + \ldots + \frac{1}{2n} > \frac{13}{24}$

9. Absolútna hodnota $|x|$ reálneho čísla $x$ je definovaná nasledovne:

\[   \left\{
\begin{array}{ll}
  x & ak x \geq 0 \\
  -x & ak x < 0 \\
\end{array} 
\right. \]
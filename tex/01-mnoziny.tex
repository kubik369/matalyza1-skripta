\chapter{Množiny}%\label{chapter:gramatiky}

V daľšom budeme používať tieto označenia:

$\mathbb{N}$ množina všetkých prirodzených čísel ($= \{ 1, 2, 3, \ldots \}$)

$\mathbb{Z}$ množina všetkých celých čísel ($= \{ 0, -1, 1, 2, -2, \ldots \}$)

$\mathbb{Q}$ množina všetkých racionálnych čísel
  ($= \{ \frac{p}{q}; p \in \mathbb{Z} \land q \in \mathbb{N} \}$)

$\mathbb{R}$ množina všetkých reálny čísel

$\mathbb{R}^+$ množina všetkých kladných reálny čísel ($= ( 0, \infty )$)

$\mathbb{R}^+_0$ množina všetkých nezáporných reálny čísel
  ($= \langle 0, \infty )$)

Ak pre niektorý prvok $a$ neprázdnej množiny $A \subset \mathbb{R}$ platí

$$\forall x \in A : x \leq a (\forall x \in A: q \geq a)$$

nazývame tento prvok \textit{maximum (minimum) množiny $A$} a označujeme ho
$max(A)$ ($min(A)$)

\section{Reálne čísla}

\begin{enumerate}[resume]
    \foreachproblem[realne-cisla]{\item\thisproblem}
\end{enumerate}

\section{Ohraničené množiny reálnych čísel, supremum a infimum}

Neprázdna množina $A \subset \mathbb{R}$ sa nazýva \textit{zhora (zdola)
ohraničená}, ak platí

$$
\exists K \in  \mathbb{R} \quad \forall x \in A: x \leq K \:
(\exists K \in \mathbb{R} \quad \forall x \in A: x \geq K).
$$

Číslo $K$ s uvedenou vlastnosťou sa nazýva \textit{horné (dolné)
ohraničenie množiny $A$.} $\emptyset$ považujeme za ohraničenú zhora aj zdola.

Množina, ktorá je zhora aj zdola ohraničená, sa nazýva \textit{ohraničená}.
Množina, ktorá nie je ohraničená sa nazýva \textit{neohraničená}.

\begin{enumerate}[resume]
	\item \useproblem[supremum-infimum]{supremum-infimum-1}
	\item \useproblem[supremum-infimum]{supremum-infimum-2}
	\item \useproblem[supremum-infimum]{supremum-infimum-3}
\end{enumerate}


Číslo $\alpha \in \mathbb{R}$ sa nazýva \textit{supremum množiny $A$}
$\subset \mathbb{R}, A \neq \emptyset$, ak
\begin{align}
\forall x \in A: x \leq \alpha \label{eq:supremum-i} \\
(\forall \varepsilon > 0) (\exists \: x_\varepsilon \in A):
x_\varepsilon > \alpha - \varepsilon
\end{align}

\chapter{Množiny}

V daľšom budeme používať tieto označenia:
\begin{itemize}[label=]
  \item $\mathbb{N}$ množina všetkých prirodzených čísel ($= \{ 1, 2, 3, \ldots \}$)
  \item $\mathbb{Z}$ množina všetkých celých čísel ($= \{ 0, -1, 1, 2, -2, \ldots \}$)
  \item $\mathbb{Q}$ množina všetkých racionálnych čísel
    ($= \{ \frac{p}{q}; p \in \mathbb{Z} \land q \in \mathbb{N} \}$)
  \item $\mathbb{R}$ množina všetkých reálny čísel
  \item $\mathbb{R}^+$ množina všetkých kladných reálny čísel ($= ( 0, \infty )$)
  \item $\mathbb{R}^+_0$ množina všetkých nezáporných reálny čísel
    ($= \langle 0, \infty )$)
\end{itemize}
Ak pre niektorý prvok $a$ neprázdnej množiny $A \subset \mathbb{R}$ platí
$$\forall x \in A : x \leq a (\forall x \in A: q \geq a)$$
nazývame tento prvok \emph{maximum (minimum) množiny $A$} a označujeme ho
$max(A)$ ($min(A)$)

\section{Reálne čísla}

\begin{enumerate}[resume]
    \foreachproblem[realne-cisla]{\item\thisproblem}
\end{enumerate}

\section{Ohraničené množiny reálnych čísel, supremum a infimum}

Neprázdna množina $A \subset \mathbb{R}$ sa nazýva \textit{zhora (zdola)
ohraničená}, ak platí

$$
\exists K \in  \mathbb{R} \quad \forall x \in A: x \leq K \:
(\exists K \in \mathbb{R} \quad \forall x \in A: x \geq K).
$$

Číslo $K$ s uvedenou vlastnosťou sa nazýva \textit{horné (dolné)
ohraničenie množiny $A$.} $\emptyset$ považujeme za ohraničenú zhora aj zdola.

Množina, ktorá je zhora aj zdola ohraničená, sa nazýva \textit{ohraničená}.
Množina, ktorá nie je ohraničená sa nazýva \textit{neohraničená}.

\begin{enumerate}[resume]
	\item \useproblem[supremum-infimum]{supremum-infimum-1}
	\item \useproblem[supremum-infimum]{supremum-infimum-2}
	\item \useproblem[supremum-infimum]{supremum-infimum-3}
\end{enumerate}


Číslo $\alpha \in \mathbb{R}$ sa nazýva \textit{supremum množiny $A$}
$\subset \mathbb{R}, A \neq \emptyset$, ak
\begin{align}
\forall x \in A: x \leq \alpha \label{eq:supremum-i} \\
(\forall \varepsilon > 0) (\exists \: x_\varepsilon \in A):
x_\varepsilon > \alpha - \varepsilon
\end{align}

Podľa \ding{34} je $\alpha$ horné ohraničenie množiny $A$; \ding{37} je negácia
výroku ``pre niektoré $\varepsilon > 0$ je číslo $\alpha - \varepsilon$ horným
ohraničením množiny $A$'', hovorí teda, že neexistuje horné ohraničenie množiny
$A$, ktoré by bolo menšie než $\alpha$. Teda $\alpha$ je najmenšie horné
ohraničenie množiny $A$. Supremum množiny $A$ označujeme $\sup \{ A \}$.

Číslo $\beta \in \mathbb{R}$ sa nazýva \textit{infimum množiny} $A \subset
\mathbb{R}, A \neq \emptyset$, ak:

\begin{align*}
  \forall x \in A&: x \geq \beta \tag{\ding{34}} \\
  (\forall \varepsilon > 0) (\exists \: x_\varepsilon \in A)&:
  x_\varepsilon < \beta + \varepsilon \tag{\ding{37}}
\end{align*}

To znamená, že $\beta$ je najväčšie z dolných ohraniční množiny $A$. Infimum
množiny $A$ označujeme $\inf\{A\}$

Ak usporiadané pole $\mathbb{R}$ reálnych čísel konštruujeme z poľa
$\mathbb{Q}$ racionálnych čísel pomocou Dedekindových rezov, môžeme dokázať
nasledujúce dve ekvivalentné tvrdenia:

\begin{veta}
Každá neprázdna zhora ohraničená množina reálnych čísel má supremum.
\end{veta}

\begin{veta}
Každá neprázdna zdola ohraničená množina reálnych čísel má infimum.
\end{veta}

Ak usporiadané pole $\mathbb{R}$ zavádzame axiomaticky, považujeme prvé z
uvedených tvrdení za axiómu, z nej možno odvodiť vetu o existencií infima.

\showanswers
\begin{enumerate}[resume]
  \item \useproblem[supremum-infimum]{supremum-infimum-4}
  \hideanswers
  \item \useproblem[supremum-infimum]{supremum-infimum-5}
  \item \useproblem[supremum-infimum]{supremum-infimum-6}
  \item \useproblem[supremum-infimum]{supremum-infimum-7}
  \item \useproblem[supremum-infimum]{supremum-infimum-8}
\end{enumerate}

\section{Funkcie}
\subsection{Definícia funkcie. Zložené funkcie. Elementárne funkcie}
Nech $A\subset \mathbb{R}$ je neprázdna množina. Ak je každému číslu $x\in A$ priradené práve jedno číslo $y\in\mathbb{R}$, ktoré označíme $f(x)$, hovoríme, že $f$ je funkcia (funkcia definovaná na množine $A$). Číslo $f(x)$ sa nazýva funkčná hodnota (v bode $x$), množina $A$ definičný obor funkcie $f$ (túto množinu budeme označovať $D(f)$).Na označenie funkcií budeme používať písmená latinskej a gréckej abecedy. Ak chceme zdôraznič, že definičným oborom funkcie $f$ je množina $A$, použijeme zápis $f:A \rightarrow \mathbb{R}$ (prápadne $f:A \rightarrow B$, ak pre každé $x\in A$ platí $f(x)\in B$) alebo $f(x)$, $x\in A$. Okrem označení "funkcia $f$", "funkcia $f:A \rightarrow \mathbb{R}^n$" sa možno stretnúť aj so spojeniami "funkcia $y=f(x)$" alebo "funkcia $f(x)$" (istá nepresnosť posledných dvoch spojení spočíva v tom, že symbol $f(x)$ sa zvykne označovať funkčná hodnota v danom bode $x$; mnohí autori preto rozlišujú označenie $f(x)$ pre funkčnú hodnotu a $f(.)$ pre funkciu).

Hovoríme, že funkcie $f$ a $g$ sa rovnajú, ak $D(f)=D(g)$ a pre každé $x\in D(f)$ platí $f(x)=g(x)$ (teda funkcia je jednoznačne určená predpisom priradenia a definičným oborom). 

Funkciu $a$, ktorej definičným oborom je množina $N$, nazývame postupnosť a označujeme ju spravidla $\{a_n\}_{n=1}^\infty$; funkčná hodnota v bode $n$ sa nazýva $n$-tý člen postupnosti $\{a_n\}_{n=1}^\infty$ a označuje sa $a_n$.

Ak $f:A \rightarrow \mathbb{R}$ je funkcia a $B \subset A$ neprázdna množina, tak množina $f(B):=\{f(x);x\in B \}$ sa nazýva obraz množiny $B$ (pri zobrazení $f$). Špeciálne množina $f(A)$ sa nazýva obor hodnôt funkcie $f$.

Nech sú dané funkcie $f$, $g$.

\begin{itemize}
\item Ak je množina $D_1:=D(f)\cap D(g)$ neprázdna, nazývajú sa funkcie $p$, $q$, $r$ definované na množine $D_1$ predpismi
\begin{center}
\item $p(x)=f(x)+g(x)$,
\item $q(x)=f(x)-g(x)$,
\item $r(x)=f(x)\cdot g(x)$
\end{center}
súčet, rozdiel a súčin funkcií $f$, $g$ a označujú sa $f+g$, $f-g$, $f\cdot g$.
\item Ak je množina $D_2:=D(f)\cap \{x\in D(g);g(x)\neq 0\}$ neprázdna, nazýva sa funkcia $s:D_2 \rightarrow \mathbb{R}$ definovaná predpisom 
\begin{itemize}
\item $a(x)=\frac{f(x)}{g(x)}$
\end{itemize}
podiel funkcií $f$, $g$ a označuje sa $\frac{f}{g}$.
\item Ak je množina $D_3:=\{x\in D(f);f(x)\in D(g)\}$ neprázdna, nazýva sa funkcia $t:D_3\rightarrow\mathbb{R}$ daná predpisom
\begin{itemize}
\item $t(x)=g(f(x))$
\end{itemize}
zložená funkcia z funkcií $f$, $g$ (superpozícia funkcií $f$ a $g$) a označuje sa $g \circ f$. Funkcia $f$ sa nazýva vnútorná zložka, funkcia $g$ vonkajšia zložka funkcie $g \circ f$.
\end{itemize}

Základnými elementárnymi funkciami nazývame nasledujúce funkcie:
\begin{tabular}{|l|l|p{10cm}|}
  \hline
  {\bf Názov} & {\bf Predpis} & {\bf Definičný obor} \\
  \hline \hline
  konštantné        & $f(x)\equiv a$\footnote{symbol $\equiv$ čítame "identicky rovné" }, $a\in\mathbb{R}$ & $\mathbb{R}$ \\
  mocninové        & $f(x)=x^a,a\in\mathbb{R}\ \{0\}$ & \begin{itemize}
  \item ak $a>0$:
  \begin{itemize}
  \item ak $a=\frac{p}{q},p,q\in\mathbb{N},$ $p$ je párne alebo $p$ aj $q$ sú nepárne: $\mathbb{R}$
  \item vo všetkých ostatných prípadoch: $\langle 0,\infty)$;
  \end{itemize}
  \item ak $a<0$:
  \begin{itemize}
  \item ak $a=-\frac{p}{q},p,q\in\mathbb{N}$, $p$ je párne lebo $p$ aj $q$ sú nepárne: $\mathbb{R}\ \{0\}$
  \item vo všetkých ostatných prípadoch: $(0,\infty)$
  \end{itemize}
\end{itemize}   \\
  exponenciálne        & $f(x)=a^x,a>0,a\neq 1$ & $\mathbb{R}$ \\
  logaritmické          & $f(x)=\log_a x,a>0,a\neq 1$\footnote{špeciálne v prípade $a=10$ budeme používať označenie $log_a x$, v prípade $a=e$ označenie $\ln x$} & $\mathbb{R}^{+}$\\
  goniometrické         & $f(x)=\sin x$ & $\mathbb{R}$ \\
   & $f(x)=\cos x$ & $\mathbb{R}$ \\
   & $f(x)=\tan x$ & $\mathbb{R}\ \{\frac{\pi}{2}+k\pi;k\in\mathbb{Z}\}$ \\
   & $f(x)=\cot x$ & $\mathbb{R}\ \{k\pi;k\in\mathbb{Z}\}$ \\
  cyklometrické\footnote{pozri odsek $1.3.4$}    & $f(x)=\arcsin x$ & $\langle -1,1 \rangle$ \\
   & $f(x)=\arccos x$ & $\langle -1,1 \rangle$ \\
   & $f(x)=\arctan x$ & $\mathbb{R}$ \\
   & $f(x)=arccot x$ & $\mathbb{R}$ \\
  \hline
\end{tabular}

Funkcie, ktoré vzniknú zo základných elementárnych funkcií len použitím operácií súčtu, rozdielu, súčinu, podielu a superpozície funkcií, sa nazývajú elementárne funkcie.

Všimnime si, že definičný obor funkcie, ktorá je súčtom, rozdielom, súčinom, podielom alebo superpozíciou daných funkcií $f$ a $g$, je jednoznačne určený množinami $D(f)$ a $D(g)$. Teda ak funkcia $h$ vznikne z funkcií $f_1,..,f_n$ len použitím operácií súčtu, rozdielu, súčinu, podielu a superpozície funkcií, je množina $D(h)$ jednoznačne určená množinami $D(f_1),...,D(f_n)$. Preto ak napíšeme predpis takejto funkcie $h$ bez toho, aby sme výslovne určili jej definičný obor, považujeme funkciu $h$ za definovanú práve na tej množine, ktorá je určená množinami $D(f_1),...,D(f_n)$ na základe definícií súčtu, rozdielu, súčinu, podielu a superpozície funkcií. (Teda trocha nepresne povedané, za definičný obor takejto funkcie $h$ považujeme množinu všetkých tých $x\in\mathbb{R}$, pre ktoré má predpis funkcie $h$ "zmysel".)

\begin{enumerate}[resume]
  \item \useproblem[supremum-infimum]{supremum-infimum-9}
  \item \useproblem[supremum-infimum]{supremum-infimum-10}
  \item \useproblem[supremum-infimum]{supremum-infimum-11}
  \item \useproblem[supremum-infimum]{supremum-infimum-12}
\end{enumerate}

Ak $f:A\rightarrow\mathbb{R},g:B\rightarrow\mathbb{R}$ sú funkcie, $A\subset B$ a pre všetky $x\in A$ platí $f(x)=g(x)$, hovoríme, že funkcia $f$ je zúženie funkcie $g$ na množinu $A$ ($f$ je funkcia $g$ zúžená na množinu $A$) a označujeme $f=\frac{g}{A}$.

\begin{enumerate}[resume]
  \item \useproblem[supremum-infimum]{supremum-infimum-13}
\end{enumerate}

\subsection{Graf funkcie}
Nech je v rovine daná pravouhlá súradnicová sústava, pričom jednotky dĺžky na súradnicových osiach $Ox$ a $Oy$ sú rovnaké. Množina $\{(x,f(x))\}$ bodov roviny, kde $f:A\rightarrow\mathbb{R}$ je daná funkcia, sa nazýva graf funkcie $f$ $((x,f(x))$ je zápis bodu roviny pomocou jeho súradníc v danej súradnicovej sústave).

V nasledujúcej tabuľke sú opísané elementárne transformácie grafov funkcií:

\begin{tabular}{|l|p{10cm}|}
  \hline
  {\bf funkcia $y=g(x)$} & {\bf transformácia grafu funkcie $y=f(x)$}  \\
  \hline \hline
  $y=f(x)+c$  & posunutie o $c$ v smere osi $Oy$ \\
  $y=f(x-c)$  & posunutie o $c$ v smere osi $Ox$ \\
  $y=f(-x)$  & symetria podľa osi $Oy$ \\
  $y=-f(x)$  & symetria podľa osi $Ox$ \\
  $y=a\cdot f(x)$  & vynásobenie každej $y$-ovej súradnice číslom $a$ \\
  $y=f(ax)$  & vydelenie každej $x$-ovej súradnice číslom $a,(a\neq 0)$ \\
  \hline
\end{tabular}

\begin{enumerate}[resume]
  \item \useproblem[supremum-infimum]{supremum-infimum-14}
\end{enumerate}

\textbf{Riešenie j):} Najprv zostrojíme graf funkcie $f(x)=\sin x$, vydelením každej $x$-ovej súradnice číslom $2$ (teda jeho "dvojnásobným zhustením") z neho dostaneme graf funkcie $f(2x)=g(x)=\sin 2x$. Naša funkcia $\sin 2(x+3)$ má tvar $g(x+3)$, jej graf teda získame, ak graf funkcie $g$ posunieme pozdĺž osi $Ox$ o $3$ jednotky dĺžky doľava.

(Pri zostrojovaní takýchto grafov sa často chybne zamieňa poradie transformácií. Zistime, graf ktorej funkcie by sme dostali pri ich opačnom poradí: posunutím grafu funkcie $f(x)=\sin x$ o $3$ jednotky dĺžky vľavo získame graf funkcie $g_1(x)=f(x+3)=\sin (x+3);$ ak teraz v tomto grafe vydelíme každú $x$-ovú súradnicu číslom $2$, dostaneme graf funkcie $g_2(x)=g_1)(2x)=\sin (2x+3)$.)

\begin{enumerate}[resume]
  \item \useproblem[supremum-infimum]{supremum-infimum-15}
  \item \useproblem[supremum-infimum]{supremum-infimum-16}
  \item \useproblem[supremum-infimum]{supremum-infimum-17}
\end{enumerate}

\textbf{Riešenie c):} Pre $x=k\pi,k\in\mathbb{Z}$, je $y=0$, pre $x=\frac{\pi}{2}+2k\pi,k\in\mathbb{Z}$, ležia príslučné body grafu funkcie $y=x\sin x$ na priamke $y=x$, pre $x=\frac{3\pi}{2}+2k\pi,k\in\mathbb{Z}$, na priamke $y=-x$. Pretože pre $x\geq 0$ je $-x\leq x\sin x\leq x$, pre $x<0$ je $x\leq x\sin x\leq -x$, leží graf funkcie $y=x\sin x$ "medzi" priamkami $y=x$ a $y=-x$. Na základe toho už vieme približne načrtnúť tento obrázok:


\begin{enumerate}[resume]
  \item \useproblem[supremum-infimum]{supremum-infimum-18}
\end{enumerate}

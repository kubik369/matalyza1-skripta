\chapter{Množiny}

V daľšom budeme používať tieto označenia:
\begin{itemize}[label=]
  \item $\mathbb{N}$ množina všetkých prirodzených čísel ($= \{ 1, 2, 3, \ldots \}$)
  \item $\mathbb{Z}$ množina všetkých celých čísel ($= \{ 0, -1, 1, 2, -2, \ldots \}$)
  \item $\mathbb{Q}$ množina všetkých racionálnych čísel
    ($= \{ \frac{p}{q}; p \in \mathbb{Z} \land q \in \mathbb{N} \}$)
  \item $\mathbb{R}$ množina všetkých reálny čísel
  \item $\mathbb{R}^+$ množina všetkých kladných reálny čísel ($= ( 0, \infty )$)
  \item $\mathbb{R}^+_0$ množina všetkých nezáporných reálny čísel
    ($= \langle 0, \infty )$)
\end{itemize}
Ak pre niektorý prvok $a$ neprázdnej množiny $A \subset \mathbb{R}$ platí
$$\forall x \in A : x \leq a (\forall x \in A: q \geq a)$$
nazývame tento prvok \emph{maximum (minimum) množiny $A$} a označujeme ho
$max(A)$ ($min(A)$)

\section{Reálne čísla}

\begin{enumerate}[resume]
    \foreachproblem[realne-cisla]{\item\thisproblem}
\end{enumerate}

\section{Ohraničené množiny, supremum a infimum}

Neprázdna množina $A \subset \mathbb{R}$ sa nazýva \textit{zhora (zdola)
ohraničená}, ak platí

$$
\exists K \in  \mathbb{R} \quad \forall x \in A: x \leq K \:
(\exists K \in \mathbb{R} \quad \forall x \in A: x \geq K).
$$

Číslo $K$ s uvedenou vlastnosťou sa nazýva \textit{horné (dolné)
ohraničenie množiny $A$.} $\emptyset$ považujeme za ohraničenú zhora aj zdola.

Množina, ktorá je zhora aj zdola ohraničená, sa nazýva \textit{ohraničená}.
Množina, ktorá nie je ohraničená sa nazýva \textit{neohraničená}.

\begin{enumerate}[resume]
	\item \useproblem[supremum-infimum]{supremum-infimum-14}
	\item \useproblem[supremum-infimum]{supremum-infimum-15}
	\item \useproblem[supremum-infimum]{supremum-infimum-16}
\end{enumerate}


Číslo $\alpha \in \mathbb{R}$ sa nazýva \textit{supremum množiny $A$}
$\subset \mathbb{R}, A \neq \emptyset$, ak
\begin{align}
\forall x \in A: x \leq \alpha \label{eq:supremum-i} \\
(\forall \varepsilon > 0) (\exists \: x_\varepsilon \in A):
x_\varepsilon > \alpha - \varepsilon
\end{align}

Podľa \ding{34} je $\alpha$ horné ohraničenie množiny $A$; \ding{37} je negácia
výroku ``pre niektoré $\varepsilon > 0$ je číslo $\alpha - \varepsilon$ horným
ohraničením množiny $A$'', hovorí teda, že neexistuje horné ohraničenie množiny
$A$, ktoré by bolo menšie než $\alpha$. Teda $\alpha$ je najmenšie horné
ohraničenie množiny $A$. Supremum množiny $A$ označujeme $\sup \{ A \}$.

Číslo $\beta \in \mathbb{R}$ sa nazýva \textit{infimum množiny} $A \subset
\mathbb{R}, A \neq \emptyset$, ak:

\begin{align*}
  \forall x \in A&: x \geq \beta \tag{\ding{34}} \\
  (\forall \varepsilon > 0) (\exists \: x_\varepsilon \in A)&:
  x_\varepsilon < \beta + \varepsilon \tag{\ding{37}}
\end{align*}

To znamená, že $\beta$ je najväčšie z dolných ohraniční množiny $A$. Infimum
množiny $A$ označujeme $\inf\{A\}$

Ak usporiadané pole $\mathbb{R}$ reálnych čísel konštruujeme z poľa
$\mathbb{Q}$ racionálnych čísel pomocou Dedekindových rezov, môžeme dokázať
nasledujúce dve ekvivalentné tvrdenia:

\begin{veta}
Každá neprázdna zhora ohraničená množina reálnych čísel má supremum.
\end{veta}

\begin{veta}
Každá neprázdna zdola ohraničená množina reálnych čísel má infimum.
\end{veta}

Ak usporiadané pole $\mathbb{R}$ zavádzame axiomaticky, považujeme prvé z
uvedených tvrdení za axiómu, z nej možno odvodiť vetu o existencií infima.

\showanswers
\begin{enumerate}[resume]
  \item \useproblem[supremum-infimum]{supremum-infimum-17}
  \hideanswers
  \item \useproblem[supremum-infimum]{supremum-infimum-18}
  \item \useproblem[supremum-infimum]{supremum-infimum-19}
  \item \useproblem[supremum-infimum]{supremum-infimum-20}
  \item \useproblem[supremum-infimum]{supremum-infimum-21}
\end{enumerate}

\section{Funkcie}
\subsection{Definícia funkcie. Zložené funkcie. Elementárne funkcie}

Nech $A\subset \mathbb{R}$ je neprázdna množina. Ak je každému číslu $x\in A$
priradené práve jedno číslo $y\in\mathbb{R}$, ktoré označíme $f(x)$, hovoríme,
že $f$ je funkcia (funkcia definovaná na množine $A$). Číslo $f(x)$ sa nazýva
funkčná hodnota (v bode $x$), množina $A$ definičný obor funkcie $f$ (túto
množinu budeme označovať $D(f)$).Na označenie funkcií budeme používať písmená
latinskej a gréckej abecedy. Ak chceme zdôraznič, že definičným oborom funkcie
$f$ je množina $A$, použijeme zápis $f:A \rightarrow \mathbb{R}$ (prápadne $f:A
\rightarrow B$, ak pre každé $x\in A$ platí $f(x)\in B$) alebo $f(x)$, $x\in
A$. Okrem označení "funkcia $f$", "funkcia $f:A \rightarrow \mathbb{R}^n$" sa
možno stretnúť aj so spojeniami "funkcia $y=f(x)$" alebo "funkcia $f(x)$" (istá
nepresnosť posledných dvoch spojení spočíva v tom, že symbol $f(x)$ sa zvykne
označovať funkčná hodnota v danom bode $x$; mnohí autori preto rozlišujú
označenie $f(x)$ pre funkčnú hodnotu a $f(.)$ pre funkciu).

Hovoríme, že funkcie $f$ a $g$ sa rovnajú, ak $D(f)=D(g)$ a pre každé $x\in
D(f)$ platí $f(x)=g(x)$ (teda funkcia je jednoznačne určená predpisom
priradenia a definičným oborom).

Funkciu $a$, ktorej definičným oborom je množina $N$, nazývame postupnosť a
označujeme ju spravidla $\{a_n\}_{n=1}^\infty$; funkčná hodnota v bode $n$ sa
nazýva $n$-tý člen postupnosti $\{a_n\}_{n=1}^\infty$ a označuje sa $a_n$.

Ak $f:A \rightarrow \mathbb{R}$ je funkcia a $B \subset A$ neprázdna množina,
tak množina $f(B):=\{f(x);x\in B \}$ sa nazýva obraz množiny $B$ (pri zobrazení
$f$). Špeciálne množina $f(A)$ sa nazýva obor hodnôt funkcie $f$.

Nech sú dané funkcie $f$, $g$.

\begin{itemize}
\item
  Ak je množina $D_1:=D(f)\cap D(g)$ neprázdna, nazývajú sa funkcie $p, q, r$
  definované na množine $D_1$ predpismi
  \begin{align*}
    p(x) &= f(x)+g(x) \\
    q(x) &= f(x)-g(x) \\
    r(x) &= f(x)\cdot g(x)
  \end{align*}
  súčet, rozdiel a súčin funkcií $f, g$ a označujú sa $f+g$, $f-g$, $f\cdot g$.
\item
  Ak je množina $D_2:=D(f)\cap \{x\in D(g);g(x)\neq 0\}$ neprázdna, nazýva sa
  funkcia $s:D_2 \rightarrow \mathbb{R}$ definovaná predpisom
  $$a(x)=\frac{f(x)}{g(x)}$$
  podiel funkcií $f$, $g$ a označuje sa $\frac{f}{g}$.
\item
  Ak je množina $D_3:=\{x\in D(f);f(x)\in D(g)\}$ neprázdna, nazýva sa funkcia
  $t:D_3\rightarrow\mathbb{R}$ daná predpisom
  $$t(x)=g(f(x))$$
  zložená funkcia z funkcií $f$, $g$ (superpozícia funkcií $f$ a $g$) a označuje
  sa $g \circ f$. Funkcia $f$ sa nazýva vnútorná zložka, funkcia $g$ vonkajšia
  zložka funkcie $g \circ f$.
\end{itemize}

Základnými elementárnymi funkciami nazývame nasledujúce funkcie:

% TODO Fix tabuľku
\begin{tabular}{|l|l|p{10cm}|}
  \hline
  {\bf Názov} & {\bf Predpis} & {\bf Definičný obor} \\
  \hline \hline
  konštantné &
  $\underset{a \in \mathbb{R}}{f(x)}\equiv a$
  \footnote{symbol $\equiv$ čítame "identicky rovné" } &
  $\mathbb{R}$ \\
  mocninové &
  $\underset{a \in \mathbb{R} \setminus {0}}{f(x)}\equiv a$ &
  \begin{enumerate}
    \item ak $a>0$:
    \begin{enumerate}
      \item ak $a=\frac{p}{q},p,q\in\mathbb{N},$ $p$ je párne alebo $p$ aj $q$
            sú nepárne: $\mathbb{R}$
      \item vo všetkých ostatných prípadoch: $\langle 0,\infty)$;
    \end{enumerate}
    \item ak $a<0$:
    \begin{enumerate}
      \item ak $a=-\frac{p}{q},p,q\in\mathbb{N}$, $p$ je párne lebo $p$ aj $q$
            sú nepárne: $\mathbb{R}\ \{0\}$
      \item vo všetkých ostatných prípadoch: $(0,\infty)$
    \end{enumerate}
  \end{enumerate} \\
  exponenciálne & $f(x)=a^x,a>0,a\neq 1$ & $\mathbb{R}$ \\
  logaritmické & $f(x)=\log_a x,a>0,a\neq 1$\footnote{špeciálne v prípade $a=10$ budeme používať označenie $log_a x$, v prípade $a=e$ označenie $\ln x$} & $\mathbb{R}^{+}$\\
  goniometrické & $f(x)=\sin x$ & $\mathbb{R}$ \\
   & $f(x)=\cos x$ & $\mathbb{R}$ \\
   & $f(x)=\tan x$ & $\mathbb{R}\ \{\frac{\pi}{2}+k\pi;k\in\mathbb{Z}\}$ \\
   & $f(x)=\cot x$ & $\mathbb{R}\ \{k\pi;k\in\mathbb{Z}\}$ \\
  cyklometrické\footnote{pozri odsek $1.3.4$}    & $f(x)=\arcsin x$ & $\langle -1,1 \rangle$ \\
   & $f(x)=\arccos x$ & $\langle -1,1 \rangle$ \\
   & $f(x)=\arctan x$ & $\mathbb{R}$ \\
   & $f(x)=arccot x$ & $\mathbb{R}$ \\
  \hline
\end{tabular}

Funkcie, ktoré vzniknú zo základných elementárnych funkcií len použitím
operácií súčtu, rozdielu, súčinu, podielu a superpozície funkcií, sa nazývajú
elementárne funkcie.

Všimnime si, že definičný obor funkcie, ktorá je súčtom, rozdielom, súčinom,
podielom alebo superpozíciou daných funkcií $f$ a $g$, je jednoznačne určený
množinami $D(f)$ a $D(g)$. Teda ak funkcia $h$ vznikne z funkcií
$f_1,\ldots,f_n$ len použitím operácií súčtu, rozdielu, súčinu, podielu a
superpozície funkcií, je množina $D(h)$ jednoznačne určená množinami
$D(f_1),\ldots,D(f_n)$. Preto ak napíšeme predpis takejto funkcie $h$ bez toho,
aby sme výslovne určili jej definičný obor, považujeme funkciu $h$ za
definovanú práve na tej množine, ktorá je určená množinami
$D(f_1),\ldots,D(f_n)$ na základe definícií súčtu, rozdielu, súčinu, podielu a
superpozície funkcií. (Teda trocha nepresne povedané, za definičný obor takejto
funkcie $h$ považujeme množinu všetkých tých $x\in\mathbb{R}$, pre ktoré má
predpis funkcie $h$ "zmysel".)

\begin{enumerate}[resume]
  \item \useproblem[supremum-infimum]{supremum-infimum-22}
  \item \useproblem[supremum-infimum]{supremum-infimum-23}
  \item \useproblem[supremum-infimum]{supremum-infimum-24}
  \item \useproblem[supremum-infimum]{supremum-infimum-25}
\end{enumerate}

Ak $f:A\rightarrow\mathbb{R},g:B\rightarrow\mathbb{R}$ sú funkcie, $A\subset B$ a pre všetky $x\in A$ platí $f(x)=g(x)$, hovoríme, že funkcia $f$ je zúženie funkcie $g$ na množinu $A$ ($f$ je funkcia $g$ zúžená na množinu $A$) a označujeme $f=\frac{g}{A}$.

% má tam byť číslovanie v príklade
\begin{enumerate}[resume]
  \item \useproblem[supremum-infimum]{supremum-infimum-26}
\end{enumerate}

\subsection{Graf funkcie}
Nech je v rovine daná pravouhlá súradnicová sústava, pričom jednotky dĺžky na súradnicových osiach $Ox$ a $Oy$ sú rovnaké. Množina $\{(x,f(x))\}$ bodov roviny, kde $f:A\rightarrow\mathbb{R}$ je daná funkcia, sa nazýva graf funkcie $f$ $((x,f(x))$ je zápis bodu roviny pomocou jeho súradníc v danej súradnicovej sústave).

V nasledujúcej tabuľke sú opísané elementárne transformácie grafov funkcií:

\begin{tabular}{|l|p{10cm}|}
  \hline
  {\bf funkcia $y=g(x)$} & {\bf transformácia grafu funkcie $y=f(x)$}  \\
  \hline \hline
  $y=f(x)+c$  & posunutie o $c$ v smere osi $Oy$ \\
  $y=f(x-c)$  & posunutie o $c$ v smere osi $Ox$ \\
  $y=f(-x)$  & symetria podľa osi $Oy$ \\
  $y=-f(x)$  & symetria podľa osi $Ox$ \\
  $y=a\cdot f(x)$  & vynásobenie každej $y$-ovej súradnice číslom $a$ \\
  $y=f(ax)$  & vydelenie každej $x$-ovej súradnice číslom $a,(a\neq 0)$ \\
  \hline
\end{tabular}

\begin{enumerate}[resume]
  \item \useproblem[supremum-infimum]{supremum-infimum-27}
\end{enumerate}

\textbf{Riešenie j):} Najprv zostrojíme graf funkcie $f(x)=\sin x$, vydelením každej $x$-ovej súradnice číslom $2$ (teda jeho "dvojnásobným zhustením") z neho dostaneme graf funkcie $f(2x)=g(x)=\sin 2x$. Naša funkcia $\sin 2(x+3)$ má tvar $g(x+3)$, jej graf teda získame, ak graf funkcie $g$ posunieme pozdĺž osi $Ox$ o $3$ jednotky dĺžky doľava.

(Pri zostrojovaní takýchto grafov sa často chybne zamieňa poradie transformácií. Zistime, graf ktorej funkcie by sme dostali pri ich opačnom poradí: posunutím grafu funkcie $f(x)=\sin x$ o $3$ jednotky dĺžky vľavo získame graf funkcie $g_1(x)=f(x+3)=\sin (x+3);$ ak teraz v tomto grafe vydelíme každú $x$-ovú súradnicu číslom $2$, dostaneme graf funkcie $g_2(x)=g_1)(2x)=\sin (2x+3)$.)

\begin{enumerate}[resume]
  \item \useproblem[supremum-infimum]{supremum-infimum-28}
  \item \useproblem[supremum-infimum]{supremum-infimum-29}
  \item \useproblem[supremum-infimum]{supremum-infimum-30}
\end{enumerate}

\textbf{Riešenie c):} Pre $x=k\pi,k\in\mathbb{Z}$, je $y=0$, pre $x=\frac{\pi}{2}+2k\pi,k\in\mathbb{Z}$, ležia príslučné body grafu funkcie $y=x\sin x$ na priamke $y=x$, pre $x=\frac{3\pi}{2}+2k\pi,k\in\mathbb{Z}$, na priamke $y=-x$. Pretože pre $x\geq 0$ je $-x\leq x\sin x\leq x$, pre $x<0$ je $x\leq x\sin x\leq -x$, leží graf funkcie $y=x\sin x$ "medzi" priamkami $y=x$ a $y=-x$. Na základe toho už vieme približne načrtnúť tento obrázok:


\begin{enumerate}[resume]
  \item \useproblem[supremum-infimum]{supremum-infimum-31}
\end{enumerate}

\textbf{Riešenie a):} $y=\sqrt{3}\cos x +1\cdot\sin x=2(\frac{\sqrt{3}}{2}\cos x +\frac{1}{2}\sin x)=2(\sin \frac{\pi}{3}\cos x +\cos \frac{\pi}{3}\sin x)=2\sin (x+\frac{\pi}{3})$ (táto úprava je analogická úprava algebraického tvaru komplexného čísla $a+bi$ na goniometrický tvar $\sqrt{a^2+b^2}(\cos\varphi +i\sin\varphi)$).

\begin{enumerate}[resume]
  \item \useproblem[supremum-infimum]{supremum-infimum-32}
  \item \useproblem[supremum-infimum]{supremum-infimum-33}
  \item \useproblem[supremum-infimum]{supremum-infimum-34}
\end{enumerate}

% ja by som dala do stĺpcov posledný príklad

\subsection{Niektoré vlastnosti funkcií (ohraničené, monotónne, periodické, prosté funkcie)}

Nech $A,B\subset\mathbb{R}$ sú neprázdne množiny, $B\subset A$. Funkcia $f:A\rightarrow\mathbb{R}$ sa nazýva ohraničená na množine $B$, ak je ohraničená množina $f(B)$. Funkcia ohraničená na svojom definičnom obore sa nazýva ohraničená. Analogicky sa zavádza pojem funkcie ohraničenej zhora, resp. zdola a funkcie ohraničenej zhora, resp. zdola na množine $B$.

Ak je množina $f(B)$ zhora, resp. zdola ohraničená, tak jej supremum, resp. infimum sa nazýva supremum, resp. infimum funkcie $f$ na množine $B$ a označuje sa $\sup_{x\in B} f(x)$, resp. $\inf_{x\in B} f(x)$. Ak existuje maximum, resp. minimum množiny $f(B)$? nazýva sa toto číslo maximum, resp. minimum funkcie $f$ na množine $B$ (často sa používa názov globálne maximum, resp. globálne minimum funkcie $f$ na množine $B$) a označuje sa $\max_{x\in B}f(x)$, resp. $\min_{x\in B}f(x)$.

% max a min  sa asi inak zapisuje nech tie x su pod tým

\begin{enumerate}[resume]
  \item \useproblem[supremum-infimum]{supremum-infimum-35}
  \item \useproblem[supremum-infimum]{supremum-infimum-36}
  \item \useproblem[supremum-infimum]{supremum-infimum-37}
  \item \useproblem[supremum-infimum]{supremum-infimum-38}
  \item \useproblem[supremum-infimum]{supremum-infimum-39}
\end{enumerate}

Na $A,B\subset\mathbb{R}$ sú neprázdne množiny, $B\subset A$. Funkcia $f:A\rightarrow\mathbb{R}$ sa nazýva rastúca na množine $B$, resp. neklesajúca na množine $B$, ak platí $$\forall x,y\in B:x<y\Rightarrow f(x)<f(y),$$ resp. $$\forall x,y\in B:x<y\Rightarrow f(x)\leq f(y).$$

Ak má funkcia $f$ niektorú z uvedených štyroch vlastností, nazýva sa monotónna na množine $B$; funkcia, ktorá je rastúca na množine $B$ alebo klesajúca na množine $B$, sa nazýva rýdzomonotónna na množine $B$.

Funkcia rastúca (klesajúca, nerastúca, neklesajúca, monotónna, rýdzomonotónna) na svojom definičnom obore sa nazýva rastúca (klesajúca, nerastúca, neklesajúca, monotónna, rýdzomonotónna).

\begin{enumerate}[resume]
  \item \useproblem[supremum-infimum]{supremum-infimum-40}
  \item \useproblem[supremum-infimum]{supremum-infimum-41}
  \item \useproblem[supremum-infimum]{supremum-infimum-42}
  \item \useproblem[supremum-infimum]{supremum-infimum-43}
\end{enumerate}

\textbf{Riešenie a):} Pre $x=0$ a $y=1$ platí $x<y\wedge f(x)>f(y)$; pre $x=2$ a $y=3$ platí $x<y\wedge f(x)<f(y)$. Preto funkcia $f$ nemôže byť nerastúca (a teda nemôže byť ani klesajúca), tomu totiž odporuje voľba $x=2,y=3$; súčasne $f$ nemôže byť neklesajúca (a teda nemôže byť ani rastúca), tomu odporuje voľba $x=0,y=1$. (Samozrejme, že pri výbere vhodných $x,y$ sme si pomáhali grafom funkcie $f$.)

\begin{enumerate}[resume]
  \item \useproblem[supremum-infimum]{supremum-infimum-44}
  \item \useproblem[supremum-infimum]{supremum-infimum-45}
  \item \useproblem[supremum-infimum]{supremum-infimum-46}
  \item \useproblem[supremum-infimum]{supremum-infimum-47}
\end{enumerate}

Funkcia $f$ sa nazýva periodická, ak existuje číslo $T>0$ tak, že platí
\begin{itemize}
\item $\forall a\in D(f):d(f)\cap \langle a+T,a+2T\rangle=\{x+T;x\in D(f)\cap \rangle a,a+T\rangle\}$;
\item  $\forall a\in D(f):f(a+T)=f(a)$.
\end{itemize}

Každé číslo $T$ s uvedenými vlastnosťami sa nazýva perióda funkcie $f$; ak existuje najmenšie také $T>0$, nazýva sa najmenšia perióda funkcie $f$ (možno sa stretnúť aj s terminológiou, v ktorej sa pojem perióda používa výlučne v zmysle tu zavedeného pojmu najmenšej periódy).

\begin{enumerate}[resume]
  \item \useproblem[supremum-infimum]{supremum-infimum-48}
  \item \useproblem[supremum-infimum]{supremum-infimum-49}
  \item \useproblem[supremum-infimum]{supremum-infimum-50}
  \item \useproblem[supremum-infimum]{supremum-infimum-51}
  \item \useproblem[supremum-infimum]{supremum-infimum-52}
\end{enumerate}

Funkcia $f:A \rightarrow\mathbb{R}$ sa nazýva prostá (injektívna, jednoznačná), ak platí $$\forall x,y\in A:x\neq y\Rightarrow f(x)\neq f(y).$$

Ak je funkcia $f:A \rightarrow\mathbb{R}$ prostá, tak funkcia s definičným oborom $f(A)$, ktorý každému číslu $x\in f(A)$ priradí to číslo $y\in A$, pre ktoré $f(y)=x$, sa nazýva inverzná funkcia k funkcii $f$ a označuje sa $f^{-1}$.

\begin{enumerate}[resume]
  \item \useproblem[supremum-infimum]{supremum-infimum-53}
\end{enumerate}

\textbf{Riešenie c):} Pre každé $x\in\mathbb{R}$ nájdime všetky tie $y\in D(f)$, pre ktoré $f(y)=x$, t.j. všetky tie $y\leq -1$, pre ktoré $\frac{2y}{(1-y^2)}=x$. Ak je $x$ dané, musia byť hľadané čísla $y$ riešeniami rovnice $$xy^2+2y-x=0;$$ tejto pre $x=0$ vyhovuje len $y=0$, pre $x\neq 0$ jej vyhovujú čísla $y_1=\frac{-1+\sqrt{1+x^2}}{x}$ a $y_2=\frac{-1-\sqrt{1+x^2}}{x}$. Pretože hľadáme len riešenia ležiace v intervale $(-\infty,-1\rangle$, musíme zistiť, kedy $y_1\leq -1$ a kedy $y_2\leq -1$. Prvá z týchto nerovností nie je splnená pre žiadnu $x\neq 0$, druhá platí pre každé $x>0$.

Zistili sme teda: pre žiadne $x\neq 0$ neexistuje $y\neq -1$ také, že $f(y)=x$; pre každé $x>0$ existuje práve jedno číslo $y\leq -1$, pre ktoré $f(y)=x$, toto číslo je určené vzťahom $y=-\frac{(1+\sqrt{1+x^2})}{x}$. To znamená:
\begin{itemize}
\item funkcia $f$ je prostá;
\item $f((-\infty,-1\rangle)=\langle 0,\infty )$;
\item inverzná funkcia $f^{-1}$ je definovaná na množine $f((-\infty,-1\rangle)$ predpisom $f^{-1}(x)=-\frac{(1+\sqrt{1+x^2})}{x}$.
\end{itemize}

\begin{enumerate}[resume]
  \item \useproblem[supremum-infimum]{supremum-infimum-54}
  \item \useproblem[supremum-infimum]{supremum-infimum-55}
  \item \useproblem[supremum-infimum]{supremum-infimum-56}
\end{enumerate}

\subsection{Cyklometrické a hyperbolické funkcie}

Inverzné funkcie k funkciám $\sin/\langle-\frac{\pi}{2},\frac{\pi}{2}\rangle \cos/\langle 0,\pi\rangle,\tan/(-\frac{\pi}{2},\frac{\pi}{2}),\cot/(0,\pi)$ sa nazývajú arkussínus, arkuskosínus, arkustangens a arkuskotangens a označujú sa $\arcsin,\arccos,\arctan,arccot$. Tieto funkcie majú spoločný názov cyklometrické.

\begin{enumerate}[resume]
  \item \useproblem[supremum-infimum]{supremum-infimum-57}
  \item \useproblem[supremum-infimum]{supremum-infimum-58}
  \item \useproblem[supremum-infimum]{supremum-infimum-59}
  \item \useproblem[supremum-infimum]{supremum-infimum-60}
\end{enumerate}

\textbf{Riešenie:} Pretože $\arccos x\in\langle 0,\pi\rangle$ pre každé $x\in\langle -1,1\rangle$ a rovnosť $\sin u=\sqrt{1-\cos^2 u}$ platí pre každú $u\in\langle 0,\pi\rangle$, je $\sin(\arccos x)=\sqrt{1-\cos^2(\arccos x)}=\sqrt{1-x^2}$.

\begin{enumerate}[resume]
  \item \useproblem[supremum-infimum]{supremum-infimum-61}
\end{enumerate}

\textbf{Riešenie a):} Pre $x\in\langle-1,0\rangle$ nemôže rovnosť platiť, pretože vtedy $\arccos\sqrt{1-x^2}\geq 0>\arcsin x$. Ak $x\in\langle 0,1\rangle$, ležia hodnoty $\arccos\sqrt{1-x^2}$ aj $\arcsin x$ v intervale $\langle 0,\frac{\pi}{2}$. Využijeme, že na tomto intervale je funkcia $\cos$ Prostá, t.j. že pre $\alpha,\beta\in\langle 0,\frac{\pi}{2}\rangle$ platí $$\alpha=\beta\Rightarrow\cos\alpha=\cos\beta.$$
 V našom prípade $\cos(\arccos\sqrt{1-x^2})=\sqrt{1-x^2}=\cos(\arcsin x)$ (pozri príklad $60$), teda pre $x\in\langle 0,1\rangle$ platí $\arccos\sqrt{1-x^2}=\arcsin x$.

\begin{enumerate}[resume]
  \item \useproblem[supremum-infimum]{supremum-infimum-62}
\end{enumerate}

Funkcie definované predpismi $sh x=\frac{e^x-e^{-x}}{2}$ (sínus hyberbolický), $ch x=\frac{e^x+e^{-x}}{2}$ (kosínus hyperbolický), $th x=\frac{sh x}{ch x}$ (tangens hyperbolický) a $cth x=\frac{ch x}{sh x}$ (kotangens hyperbolický) sa nazývajú hyperbolické funkcie.

\begin{enumerate}[resume]
  \item \useproblem[supremum-infimum]{supremum-infimum-63}
  \item \useproblem[supremum-infimum]{supremum-infimum-64}
\end{enumerate}

\subsection{Ďalšie príklady}
\begin{enumerate}[resume]
  \item \useproblem[supremum-infimum]{supremum-infimum-65}
  \item \useproblem[supremum-infimum]{supremum-infimum-66}
  \item \useproblem[supremum-infimum]{supremum-infimum-67}
  \item \useproblem[supremum-infimum]{supremum-infimum-68}
  \item \useproblem[supremum-infimum]{supremum-infimum-69}
  \item \useproblem[supremum-infimum]{supremum-infimum-70}
  \item \useproblem[supremum-infimum]{supremum-infimum-71}
  \item \useproblem[supremum-infimum]{supremum-infimum-72}
  \item \useproblem[supremum-infimum]{supremum-infimum-73}
  \item \useproblem[supremum-infimum]{supremum-infimum-74}
  \item \useproblem[supremum-infimum]{supremum-infimum-75}
  \item \useproblem[supremum-infimum]{supremum-infimum-76}
  \item \useproblem[supremum-infimum]{supremum-infimum-77}
  \item \useproblem[supremum-infimum]{supremum-infimum-78}
  \item \useproblem[supremum-infimum]{supremum-infimum-79}
  \item \useproblem[supremum-infimum]{supremum-infimum-80}
  \item \useproblem[supremum-infimum]{supremum-infimum-81}
  \item \useproblem[supremum-infimum]{supremum-infimum-82}
  \item \useproblem[supremum-infimum]{supremum-infimum-83}
  \item \useproblem[supremum-infimum]{supremum-infimum-84}
  \item \useproblem[supremum-infimum]{supremum-infimum-85}
  \item \useproblem[supremum-infimum]{supremum-infimum-86}
  \item \useproblem[supremum-infimum]{supremum-infimum-87}
  \item \useproblem[supremum-infimum]{supremum-infimum-88}
  \item \useproblem[supremum-infimum]{supremum-infimum-89}
\end{enumerate}

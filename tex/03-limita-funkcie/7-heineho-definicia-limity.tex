\begin{veta}
Nech $a \in \mathbb{R^*}$ je hromadný bod definičného oboru $D(f)$ funkcie $f$a
$b \in \mathbb{R^*}$. Potom sú nasledujúce výroky ekvivalentné:
\begin{itemize}
\item $\lim\limits_{x \rightarrow a} f(x)=b$
\item pre každú postupnosť ${\{a_n\}}_{n=1}^\infty$ prvkov z $D(f)\setminus \{
a\}$, ktorej limitou je $a$, platí $\lim\limits_{n \rightarrow \infty} f(a_n)=b$
\end{itemize}
\end{veta}

Limitu funkcie možno definovať aj bez použitia pojmu okolia; v takomto prípade
sa zavedie len pojem vlastnej a nevlastnej limity postupnosti a na definíciu
limity funkcie $f$ v bode $a$ sa použije vlastnosť $2$ z uvedenej vety.
Definícia limity funkcie v takejto podobe sa nazýva \textit{Heineho definíciou
limity}. Veta $15$ teda hovorí, že definícia limity pomocou okolia a Hejného
definícia limity sú ekvivalentné.

\begin{enumerate}[resume]
  \item \useproblem[limita-funkcie]{limita-158}
  \item \useproblem[limita-funkcie]{limita-159}
\end{enumerate}
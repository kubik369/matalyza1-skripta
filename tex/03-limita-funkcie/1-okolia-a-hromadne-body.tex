Nech $a \in \mathbb{R};$ každý interval
$\interval[open]{a - \varepsilon}{a + \varepsilon}$,
kde $\varepsilon > 0$, sa nazýva \textit{okolie bodu $a$}. Číslo $\varepsilon$
sa nazýva \textit{polomer okolia}
$\interval[open]{a - \varepsilon}{a + \varepsilon};$

ak chceme zdôrazniť, že dané okolie bodu $a$ má polomer $\varepsilon$, hovoríme
o $\varepsilon-$okolí bodu $a$. Okolím bodu $\infty$ sa nazýva každý interval
$\interval[open]{K}{\infty}$, kde $K \in \mathbb{R};$ okolím bodu $- \infty$ je
každý interval $(- \infty,K)$, kde $K \in \mathbb{R}.$ Okolie bodu $b \in
\mathbb{R^*}$ ($\mathbb{R^*}$ sa nazýva rozšírená množina reálnych čísel a
pozostáva zo všetkých reálnych čísel a symbolov
$\interval[open]{+\infty}{-\infty}$) budeme označovať $O(b)$, symbol
$O_\varepsilon (b)$ budeme používať pre $\varepsilon-$okolie bodu $b \in
\mathbb{R}$.

\textit{Poznámka:} Namiesto symbolu $+\infty$ sa často používa symbol $\infty$,
niektorí autori však zavádzajú symbol $\infty$ s iným významom, k tomu pozri
poznámku na konci odseku $2.5$

Bod $a \in \mathbb{R^*}$ sa nazýva hromadný bod množiny $M \subset \mathbb{R}$,
ak každé jeho okolie $O(a)$ obsahuje aspoň jeden prvok množiny $M$, rôzny od
bodu $a$, t.j. ak platí
\[
  (\forall O(a)): (O(a) \setminus \{ a \}) \cap M \neq \emptyset
\]
(tento výrok možno zapísať aj v tvare $(\forall O(a))(\exists x\in M); x \neq a
\wedge x \in O(a);$ je ekvivalentný s výrokom: každé okolie bodu $a$ obsahuje
nekonečne veľa prvkov množiny $M$).

Množinu všetkých hromadných bodov množiny $M$ budeme označovať $M'$.
\begin{enumerate}[resume]
  \item \useproblem[limita-funkcie]{limita-100}
  \item \useproblem[limita-funkcie]{limita-101}
  \item \useproblem[limita-funkcie]{limita-102}
  \item \useproblem[limita-funkcie]{limita-103}
  \item \useproblem[limita-funkcie]{limita-104}
\end{enumerate}

Nech $a \in \mathbb{R^*}$ je hromadný bod definičného oboru funkcie $f$. Bod $b
\in \mathbb{R^*}$ sa nazýva \textit{limita funkcie $f$ v bode $a$} ak platí
\[
  (\forall O(b)) (\exists O(a)) (\forall x \in (O(a)\setminus \{ a \}) \cap D(f) ):
  f(x) \in O(b)
\]
(Ak $b \in \mathbb{R}$, hovoríme o \textit{vlastnej (alebo konečnej) limite}; ak
$b=\infty$ alebo $b=-\infty$, o \textit{nevlastnej limite}; ak $a=\infty$ alebo
$a=-\infty$, používane názov \textit{limita v nevlastnom bode}.) Zapisujeme
$\lim_{x \rightarrow a} f(x)=b$ alebo $f(x) \rightarrow b$ pre $x \rightarrow
a$.

Všimnime si teraz jednotlivé prípady, ktoré zahŕňa uvedená definícia; začneme
postupnosťami:
\begin{itemize}
\item
  Nech $b \in \mathbb{R}$; hovoríme, že postupnosť ${\{a_n\}}_{n=1}^\infty$
  konverguje k (číslu) $b$, ak $\lim\limits_{n \rightarrow \infty} a_n=b$, t.j.
  ak platí
  \[
    (\forall \varepsilon > 0)
      (\exists n_0 \in \mathbb{N})
        (\forall n \in \mathbb{N}) (n > n_0): |a_n-b| < \varepsilon
  \]

  \textit{Poznámka:} Výroky
  \[
    (\forall \varepsilon > 0)
      (\exists n_0 \in \mathbb{N})
        (\forall n \in \mathbb{N})
          (n > n_0): |a_N - b| < \varepsilon
  \]
  a
  \[
    (\forall \varepsilon > 0)
      (\exists n_0 \in \mathbb{R})
        (\forall n \in \mathbb{N})
          (n > n_0): |a_N - b| < \varepsilon
  \]
  sú ekvivalentné. Pretože v definícii konečnej limity postupností je zvykom
  žiadať $n_0 \in \mathbb{N}$ (a nie $n_0 \in \mathbb{R}$), budeme ju v takej
  podobe používať aj my (hoci - ako uvidíme v príklade $105$ - sa tým niekedy
  komplikuje vyjadrenie závislosti čísla $n_0$ na čísle $\varepsilon$).
  Analogická poznámka sa vzťahuje aj na definície nevlastných limít postupností.
\item
  Hovoríme, že postupnosť ${\{a_n\}}_{n=1}^\infty$ diverguje k $+\infty$, ak
  $\lim\limits_{n \rightarrow \infty} a_N = +\infty$, t.j. ak platí
  \[
    (\forall K \in	\mathbb{R})
      (\exists n_0 \in \mathbb{N})
        (\forall n \in \mathbb{N})
        (n > n_0): a_n>K
  \]
\item
  Hovoríme, že postupnosť ${\{a_n\}}_{n=1}^\infty$ diverguje k $-\infty$, ak
  $\lim\limits_{n \rightarrow \infty} a_N = -\infty$, t.j. ak platí
  \[
    (\forall K \in \mathbb{R})
      (\exists n_0 \in \mathbb{N})
        (\forall n \in \mathbb{N})
          (n > n_0): a_n < K
  \]
  Postupnosti, ktoré majú vlastnú limitu, sa nazývajú \textit{konvergentné}; ak
  postupnosť nemá limitu alebo má nevlastnú limitu, nazýva sa
  \textit{divergentná}.
\end{itemize}

\begin{enumerate}[resume]
  \item \useproblem[limita-funkcie]{limita-105}
  \item \useproblem[limita-funkcie]{limita-106}
  \item \useproblem[limita-funkcie]{limita-107}
  \item \useproblem[limita-funkcie]{limita-108}
  \item \useproblem[limita-funkcie]{limita-109}
  \item \useproblem[limita-funkcie]{limita-110}
\end{enumerate}

Ak $x \in \mathbb{R^*}$, nastáva práve jedna z troch možností:
$x \in \mathbb{R}$, $x = +\infty$, $x = -\infty$.

Ak v definícii limity funkcie rozlíšime pre body $a, b \in \mathbb{R^*}$ tieto
možnosti, dostaneme nasledujúcich deväť špeciálnych prípadov (v nich už $a, b$
označujú len reálne čísla):

\begin{multicols}{2}
\begin{enumerate}
    \item $\lim\limits_{x \rightarrow a} f(x) = b$
    \item $\lim\limits_{x \rightarrow a} f(x) = +\infty$
    \item $\lim\limits_{x \rightarrow a} f(x) = -\infty$
    \item $\lim\limits_{x \rightarrow \infty} f(x) = b$
    \item $\lim\limits_{x \rightarrow \infty} f(x) = +\infty$
    \item $\lim\limits_{x \rightarrow \infty} f(x) = -\infty$
    \item $\lim\limits_{x \rightarrow -\infty} f(x) = b$
    \item $\lim\limits_{x \rightarrow -\infty} f(x) = +\infty$
    \item $\lim\limits_{x \rightarrow -\infty} f(x) = -\infty$
\end{enumerate}
\end{multicols}

\begin{enumerate}[resume]
  \item \useproblem[limita-funkcie]{limita-111}
\end{enumerate}

\begin{veta}[Cauchyho-Bolzanovo kritérium konvergencie]
Nech $a \in \mathbb{R}^*$ je hromadný bod definičného oboru funkcie $f$. Funkcia
$f$ má v bode $a$ konečnú limitu práve vtedy, keď platí:
\[
  (\forall \varepsilon > 0)
    (\exists O(a))
      (\forall x,y \in (O(a) \setminus \{ a \}) \cap D(f))
        : |f(x)-f(y)| < \varepsilon
\]
\end{veta}

\begin{enumerate}[resume]
  \item \useproblem[limita-funkcie]{limita-112}
  \item \useproblem[limita-funkcie]{limita-113}
  \item \useproblem[limita-funkcie]{limita-114}
  \item \useproblem[limita-funkcie]{limita-115}
\end{enumerate}

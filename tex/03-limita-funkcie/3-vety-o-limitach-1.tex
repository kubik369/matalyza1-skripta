\begin{veta}
\textbf{o limite skalárneho násobku, súčtu, rozdielu, súčinu a podielu}

Nech sú dané funkcie $f,g$, nech $a \in \mathbb{R^*}$ je hromadný bod množiny
$D(f) \cap D(g)$. Ak existujú konečné $\lim\limits_{x \rightarrow a} f(x)=A$,
$\lim\limits_{x \rightarrow a} g(x)=B$, tak existuje aj
$\lim\limits_{x \rightarrow a} cf(x)$ ($c \in \mathbb{R}$ je konštanta),
$\lim\limits_{x \rightarrow a} (f(x)+g(x)),
\lim\limits_{x \rightarrow a}(f(x)-g(x)),
\lim\limits_{x \rightarrow a} (f(x) \cdot g(x))$
a platí
\begin{multicols}{2}
\begin{enumerate}[label=]
    \item $\lim\limits_{x \rightarrow a} c \cdot f(x) = cA$
    \item $\lim\limits_{x \rightarrow a} (f(x)+g(x)) = A+B$
    \item $\lim\limits_{x \rightarrow a} (f(x)-g(x)) = A-B$
    \item $\lim\limits_{x \rightarrow a} (f(x) \cdot g(x)) = A \cdot B$
    \item $( = c \cdot \lim\limits_{x \rightarrow a} f(x))$
    \item $( = \lim\limits_{x \rightarrow a} f(x)+\lim\limits_{x \rightarrow a} g(x))$
    \item $( = \lim\limits_{x \rightarrow a} f(x)-\lim\limits_{x \rightarrow a} g(x))$
    \item $( = \lim\limits_{x \rightarrow a} f(x) \cdot \lim\limits_{x \rightarrow a} g(x))$
\end{enumerate}
\end{multicols}
Ak naviac $B \neq 0$, tak existuje aj $\lim\limits_{x \rightarrow a} \frac{f(x)}{g(x)}$
a platí
\begin{multicols}{2}
\begin{enumerate}[label=]
\item $\lim\limits_{x \rightarrow a} \frac{f(x)}{g(x)}=\frac{A}{B}$
\item $(=\frac{\lim\limits_{x \rightarrow a} f(x)}{\lim\limits_{x \rightarrow a} g(x)})$.
\end{enumerate}
\end{multicols}
\end{veta}

\begin{veta} \textbf{o limite zloženej funkcie}
Nech sú dané funkcie $f, g$ nech $a \in \mathbb{R^*}$ je hromadný bod množiny
$D(f \circ g)$. Ak $\lim\limits_{x \rightarrow a} g(x) = A$ $(A \in
\mathbb{R^*})$, pričom je splnená podmienka
\[
  (\exists O(a))
    (\forall x \in O(a)):
      x \neq a \Rightarrow g(x) \neq A
\]
a
\[
  \lim\limits_{x \rightarrow A} f(x) = B \quad (B \in \mathbb{R^*})
\]
tak
\[
  \lim\limits_{x \rightarrow a} f(g(x)) = B
\]
\end{veta}

\textit{Poznámka:}
Ak $A$ je hromadným bodom $D(f)$, ale $A \notin D(f)$, tak uvedená veta platí aj
vtedy, keď nie je splnená podmienka (*).

Niektoré limity možno nájsť len na základe definície, v ostatných prípadoch je
však oveľa efektívnejšie použiť vety o limitách. Pritom je potrebné osvojiť si
zdôvodňovanie jednotlivých krokov výpočtu, inak sa nenaučíme odlišovať správne
postupy od nesprávnych. Na ilustráciu podrobne popíšme nasledujúci výpočet

\[
  \begin{split}
    \lim_{x \rightarrow \infty} \frac{2x^2 + 3x + 5}{3x^2 - 7}
      &= \lim_{x \rightarrow \infty}
          \frac{
            2
            + \frac{3}{x}
            + \frac{5}{x^2}}{3 - \frac{7}{x^2}}
      = \frac{\lim\limits_{x \rightarrow \infty}
        (
          2
          + \frac{3}{x}
          + \frac{5}{x^2}
        )}{
          \lim\limits_{x \rightarrow \infty} (3 - \frac{7}{x^2})
        } \\
      &= \frac{
          \lim\limits_{x \rightarrow \infty}
          2
          + \lim\limits_{x \rightarrow \infty} \frac{3}{x}
          + \lim\limits_{x \rightarrow \infty} \frac{5}{x^2}
        }{
          \lim\limits_{x \rightarrow \infty} 3
          -
          \lim\limits_{x \rightarrow \infty}
            \frac{7}{x^2}
        } \\
      &= \frac{2+0+0}{3-0}
      = \frac{2}{3}.
  \end{split}
\]

\textbf{1. krok:}
Na intervale $\interval[open]{0}{\infty}$ iste platí
\[
  \frac{2x^2 + 3x + 5}{3x^2 - 7}
  = \frac{2 + \frac{3}{x} + \frac{5}{x^2}}{3 - \frac{7}{x^2}}
\]
(zlomok vľavo stačí rozšíriť výrazom $\frac{1}{x^2}$), preto: ak existuje limita
na pravej strane prvej rovnosti, tak existuje aj limita na jej ľavej strane a
tieto limity sa rovnajú.

\textit{Poznámka}: Táto elementárna, ale veľmi častá úvaha sa vo všeobecnosti
formuluje takto: Nech $a \in \mathbb{R^*}$ je hromadný bod $D(f)$, nech existuje
$O(a)$ tak, že:
\[
  D(f) \cap (O(a) \setminus \{ a\})=D(g) \cap (O(a) \setminus \{ a\})
\]
a $(\forall x \in D(f) \cap (O(a) \setminus \{ a\})$ platí $f(x)=g(x)$
Ak existuje $\lim\limits_{x \rightarrow a} g(x)=b$ $(\in \mathbb{R^*})$, tak
platí $\lim\limits_{x \rightarrow a} f(x)=b$.

Ďalej sa teda snažíme zistiť, či
existuje
\[
  \lim\limits_{x \rightarrow \infty}
    \frac{2+\frac{3}{x} + \frac{5}{x^2}}{3-\frac{7}{x^2}}
\]

\textbf{2. krok:}
tu sme použili vetu o limite podielu (zatiaľ ovšem len \enquote{na čestné
slovo}); presnejšie povedané: ak ukážeme, že limita v čitateli aj v menovateli
existujú a sú konečné, pričom limita v menovateli je nenulová, tak podľa vety o
limite podielu bude platiť druhá rovnosť;

\textbf{3. krok:}
v čitateli sme použili vetu o limite súčtu (tú možno indukciou rozšítiť na
ľubovoľný konečný počet sčítancov), v menovateli vetu o limite rozdielu (obidve
zatiaľ tiež len \enquote{na čestné slovo});

\textbf{4. krok:}
teraz už ľahko overíme, že druhá a tretia rovnosť skutočne platia, pretože
$\lim\limits_{x \rightarrow \infty} 2$,
$\lim\limits_{x \rightarrow \infty} \frac{3}{x}$,
$\lim\limits_{x \rightarrow \infty} \frac{5}{x^2}$,
$\lim\limits_{x \rightarrow \infty} 3$
a
$\lim\limits_{x \rightarrow \infty} \frac{7}{x^2}$
existujú a sú konečné (to ľahko dokážeme priamo z definície), bolo použitie viet
o limite súčtu a rozdielu v tretej rovnosti oprávnené (a preto $\lim\limits_{x
\rightarrow \infty} (2 + \frac{3}{x} + \frac{5}{x^2}) = 2$, $\lim\limits_{x
\rightarrow \infty} (3 - \frac{7}{x^2}) = 3$); rovnako oprávnené bolo použitie
vety o limite podielu v druhom kroku (limita v čitateli aj v menovateli - ako
sme sa práve presvedčili - skutočne existujú, sú konečné a limita v menovateli
je naviac nenulová).

Z platnosti prvej, druhej a tretej rovnosti vyplýva $\lim\limits_{x \rightarrow
\infty} \frac{2x^2 + 3x + 5}{3x^2 - 7} = \frac{2}{3}$.

Takéto zdôvodnenie (vykonané ovšem len v duchu alebo ústne) by malo byť súčasťou
výpočtu každej limity; po získaní istej prace budú zápisy aj argumentácia
podstatne stručnejšie

\begin{veta}
Ak $f$ je elementárna funkcia a bod $a \in D(f)$ je hromadný bod množiny $D(f)$
je hromadný bod množiny $D(f)$, tak $\lim\limits_{x \rightarrow a} f(x) = f(a)$.
(Toto tvrdenie veľmi úzko súvisí s pojmom spojitosti (pozri kap. 3).)
\end{veta}

\begin{enumerate}[resume]
  \item \useproblem[limita-funkcie]{limita-116}
  \item \useproblem[limita-funkcie]{limita-117}
  \item \useproblem[limita-funkcie]{limita-118}
  \item \useproblem[limita-funkcie]{limita-119}
  \item \useproblem[limita-funkcie]{limita-120}
  \item \useproblem[limita-funkcie]{limita-121}
  \item \useproblem[limita-funkcie]{limita-122}
  \item \useproblem[limita-funkcie]{limita-123}
  \item \useproblem[limita-funkcie]{limita-124}
  \item \useproblem[limita-funkcie]{limita-125}
\end{enumerate}

\begin{veta}
$\lim\limits_{x \rightarrow 0} \frac{\sin{x}}{x}=1$.
\end{veta}

\begin{enumerate}[resume]
  \item \useproblem[limita-funkcie]{limita-126}
  \item \useproblem[limita-funkcie]{limita-127}
  \item \useproblem[limita-funkcie]{limita-128}
  \item \useproblem[limita-funkcie]{limita-129}
  \item \useproblem[limita-funkcie]{limita-130}
\end{enumerate}

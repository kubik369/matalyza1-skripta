\begin{definicia}
Prstencovým okolím bodu $a$  sa nazýva množina $O(a) \setminus \{ a\}$, kde
$O(a)$ je okolie bodu $a$. (Teda každé okolie bodov $+\infty,-\infty$ je súčasne
aj ich prstencovým okolím.). Označujeme znakom $\mathcal{P}(a)$.
\end{definicia}

\begin{veta}
Nech sú dané sunkcie $f, g, h$ nech $a \in \mathbb{R}^*$ je hromadný bod množiny
$D(g)$ a nech pre niektoré jeho prstencové okolie $O^*(a)$ platí $O^*(a) \cap
D(f)=O^*(a)\cap D(g)=O^*(a) \cap D(h)=D$. Ak $\lim\limits_{x \rightarrow a}
f(x)=\lim\limits_{x \rightarrow a} h(x)=b$ $( \in \mathbb{R})$ a pre všetky $x
\in D$ je $f(x) \leq g(x) \leq h(x)$, tak existuje aj $\lim\limits_{x
\rightarrow a} g(x)$ a rovná sa $b$.
\end{veta}

\begin{veta}
Nech sú dané funkcie $f, g$, nech $a \in \mathbb{R}^*$ je hromadný bod množiny
$D(f) \cap D(g)$. Ak $\lim\limits_{x \rightarrow a} f(x) = 0$ a funkcia $g$ je
ohraničená v niektorom prstencovom okolí bodu $a$ (t.j. na niektorej z množín
$(O(a) \setminus \{ a\})\cap D(g)$, tak $\lim\limits_{x \rightarrow a} f(x)
\cdot g(x)=0$.
\end{veta}

\begin{enumerate}[resume]
  \item \useproblem[limita-funkcie]{limita-131}
  \item \useproblem[limita-funkcie]{limita-132}
  \item \useproblem[limita-funkcie]{limita-133}
  \item \useproblem[limita-funkcie]{limita-134}
  \item \useproblem[limita-funkcie]{limita-135}
  \item \useproblem[limita-funkcie]{limita-136}
\end{enumerate}

\begin{veta}
Nech sú dané funkcie $f,g$, nech $a \in \mathbb{R^*}$ je hromadný bod množiny
$D(f)$ a nech pre niektoré jeho rýche okolie $O^*(a)$ a platí $O^*(a)\cap
D(g)=:D$. Ak $\lim\limits_{x \rightarrow a} g(x)=\infty$  $(-\infty)$ a pre
všetky $x \in D$ platí $f(x)\geq g(x)$  $(f(x)\leq g(x))$ tak existuje aj
$\lim\limits_{x \rightarrow a} f(x)$ a rovná sa $\infty$  $(-\infty)$.
\end{veta}

\begin{veta}
Nech sú dané funkcie $f,g$, nech $a \in \mathbb{R^*}$ je hromadný bod množiny
$D(f)\cap D(g)$, nech existuje $\lim\limits_{x \rightarrow a} f(x)=:A$,
$\lim\limits_{x \rightarrow a} g(x)=:B$. Potom
\begin{enumerate}
\item ak $A=+\infty$, $B \in \mathbb{R}$ alebo $B=+\infty$, tak $\lim\limits_{x \rightarrow a}(f(x)+g(x))=+\infty$;
\item ak $A=+\infty$, $B >0$ alebo $B=+\infty$, tak $\lim\limits_{x \rightarrow a}(f(x)\cdot g(x))=+\infty$.
\end{enumerate}
\end{veta}

\begin{veta}
Nech $a \in \mathbb{R^*}$ je hromadný bod definičného oboru funkcie $f$. Ak
$\lim\limits_{x \rightarrow a} f(x)=+\infty$   $(-\infty)$, tak
$\lim\limits_{\frac{1}{f(x)}=0}$.
\end{veta}

\textit{Poznámka:}
Predchádzajúce tvrdenie (spolu s ďalšími, analogickými) si ľahko zapamätáme
pomocou nasledujúcich rovností (ktoré ovšem považujeme len za mnemotechnickú
pomôcku):
\begin{multicols}{2}
\begin{enumerate}
    \item $A +/- \infty = +/- \infty$;
    \item $+/- \infty +/- \infty = +/- \infty$;
    \item $\frac{1}{+/- \infty} = 0$;
    \item $A \cdot (+/- \infty) = \left\{ \begin{array}{r@{\quad}c}
    +/- \infty, & $ak$ A>0 \\
    -/+ \infty, & $ak$ A<0 \\ \end{array} \right.
    $;
    \item $\infty \cdot(+/- \infty)=+/- \infty$.
\end{enumerate}
\end{multicols}
($A$ označuje reálne číslo.) Všimnime si, že žiadna z uvedených viet sa
nevzťahuje na limity funkcií typu $+\inf\ -\infty, 0 \cdot ( +/-
\infty),\frac{+/- \infty}{ +/- \infty}, \frac{0}{0}$. Také funkcie budeme
nazývať neurčitými výrazmi práve im je venovaná väčšina príkladov na výpočet
limít.

\begin{veta}
Nech je daná funkcia $f$, nech $a \in \mathbb{R^*}$ je hromadný bod množiny
$D:=\{ x \in D(f): f(x)\neq 0 \}$, nech $\lim\limits_{x \rightarrow a} f(x)=0$.
Ak existuje také prstencové okolie $O^*(a)$ bodu $a$, že pre všetky $x \in D
\cap O^*(a)$ platí $f(x)>0$  $(f(x)<0)$, tak existuje $\lim\limits_{x
\rightarrow a}\frac{1}{f(x)}$ a rovná sa $+\infty)(-\infty)$
\end{veta}

\begin{enumerate}[resume]
  \item \useproblem[limita-funkcie]{limita-137}
  \item \useproblem[limita-funkcie]{limita-138}
  \item \useproblem[limita-funkcie]{limita-139}
  \item \useproblem[limita-funkcie]{limita-140}
  \item \useproblem[limita-funkcie]{limita-141}
\end{enumerate}

Nech je daná funkcia $f$, nech $a \in \mathbb{R}$ je hromadný bod množiny
$D^+:=D(f)\cap (a,\infty)$ (množiny $D^-:=D(f)\cap (-\infty,a)$); označme $f'$
zúženie funkcie na množinu $D^+$ (na množinu $D^-$). Ak existuje limita funkcie
$f'$ v bode $a$, nazývame ju \textit{limitou funkcie $f$ v bode $a$ sprava
(zľava)} a označujeme $\lim\limits_{x \rightarrow a+} f(x)$ $(\lim\limits_{x
\rightarrow a-} f(x))$. Pre limity sprava a zľava sa používa súhrnný názov
\textit{jednostranné limity}.

\begin{veta}
Nech je daná funkcia $f$, nech $a$ je hromadný bod množín $D(f) \cap
(-\infty,a)$ a $D(f) \cap (a,\infty)$. Potom $\lim\limits_{x \rightarrow a}
f(x)=\lim\limits_{x \rightarrow -a} f(x)$; pritom $\lim\limits_{x \rightarrow a}
f(x)$ sa rovná spoločnej hodnote týchto jednostranných limít.
\end{veta}

\begin{enumerate}[resume]
  \item \useproblem[limita-funkcie]{limita-142}
  \item \useproblem[limita-funkcie]{limita-143}
  \item \useproblem[limita-funkcie]{limita-144}
  \item \useproblem[limita-funkcie]{limita-145}
  \item \useproblem[limita-funkcie]{limita-146}
\end{enumerate}

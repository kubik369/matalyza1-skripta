\chapter{Dodatok $1$; Mohutnosť množín}

Neprázdna množina $A$ sa nazýva konečná, ak pre niektoré prirodzené číslo $n$ existuje bijekcia $\varphi:\{1,...,n\}\rightarrow A$ (t.j. $\varphi$ je prosté zobrazenie \footnote{Ak každému prvku a neprázdnej množiny $A$ priradíme práve jeden prvok neprázdnej množiny $B$, ktorý označíme $\varphi (a)$ (množiny $A,B$ nemusia byť číselné), tak hovoríme, že $\varphi$ je zobrazenie množiny $A$ do množiny $B$. Špeciálne, ak $A=\mathbb{N}$, hovoríme, že $\varphi$ je postupnosť prvkov z $B$. Zrejme pojmy funkcie a postupnosti uvedené v odseku $1.3$ sú špeciálnymi prípadmi tu uvedených pojmov zobrazenie a postupnosť.} a $\varphi(\{1,...,n\})=A$). Prázdnu množinu pokladáme za konečnú. Množina, ktorá nie je konečná, sa nazýva nekonečná.

Množina $A$ sa nazýva nekonečne spočítateľná, ak existuje bijekcia $\varphi:\mathbb{N}\rightarrow A$ (t.j. ak prvky množiny $A$ možno zoradiť do prostej postupnosti). Konečné a nekonečne spočítateľné množiny sa označujú spoločným názvom spočítateľné. Množina, ktorá nie je spočítateľná, sa nazýva nespočítateľná.

\begin{veta}
Každý nedegenerovaný interval je nespočítateľná množina (degenerovanými intervalmi sa nazývajú jednoprvkové množiny).
\end{veta}

\textbf{Poznámka:} Namiesto dvojice pojmov nekonečne spočítateľná-spočítateľná sa často v tom istom význame používa dvojica spočítateľná-najviac spočítateľná.

\begin{enumerate}[resume]
  \item \useproblem[dodatok-1]{dodatok-1-1}
  \item \useproblem[dodatok-1]{dodatok-1-2}
  \item \useproblem[dodatok-1]{dodatok-1-3}
  \item \useproblem[dodatok-1]{dodatok-1-4}
  \item \useproblem[dodatok-1]{dodatok-1-5}
  \item \useproblem[dodatok-1]{dodatok-1-6}
  \item \useproblem[dodatok-1]{dodatok-1-7}
  \item \useproblem[dodatok-1]{dodatok-1-8}
  \item \useproblem[dodatok-1]{dodatok-1-9}
  \item \useproblem[dodatok-1]{dodatok-1-10}
\end{enumerate}
\begin{defproblem}{diferencialny-pocet-1}
Výpočtom $\lim_{x \rightarrow a}\frac{f(x)-f(a)}{x-a}$ nájdite deriváciu funkcie $f$ v bode $a$, ak:
\begin{multicols}{2}
\begin{enumerate}
    \item $f(x)=x^2,a=0,1$;
    \item $f(x)=2\cdot \sin 3x,a=\frac{\pi}{6}$;
    \item $f(x)=1+2\cdot \ln x,a=1$;
    \item $f(x)=\arcsin x,a=0$;
    \item $f(x)=e^x,a=\ln 2$.
\end{enumerate}
\end{multicols}
\end{defproblem}

\begin{defproblem}{diferencialny-pocet-2}
Ukážte, že funkcia $f$ má v bode $a$ nevlastnú deriváciu, ak:
\begin{enumerate}
\item $f(x)=\sqrt[3]{x-1},a=1$;
\item $f(x)=sgn x,a=0$.
\end{enumerate}
\end{defproblem}

\begin{defproblem}{diferencialny-pocet-3}
Výpočtom $\lim_{h \rightarrow 0}\frac{f(x+h)-f)(x)}{h}$ nájdite $f'(x)$ a yistite definičný obor funkcie $f'$, ak:
\begin{multicols}{2}
\begin{enumerate}
   \item $f(x)=x^3+2x$;
   \item $f(x)=x\cdot \sqrt[3]{x}$;
   \item $f(x)=\frac{1}{1+x^2}$;
   \item $f(x)=2^{x+1}$;
   \item $f(x)=\ln x$;
   \item $f(x)=\cot x+2$;
   \item $f(x)=(x+1)^{\frac{2}{3}}$;
   \item $f(x)=sgn x$.
\end{enumerate}
\end{multicols}
\end{defproblem}

\begin{defproblem}{diferencialny-pocet-4}
Výpočtom príslušných limít nájdite $f'_+(a),f'_-(a)$, ak:
\begin{enumerate}
\item $f(x)=|\cos x|,a=\frac{\pi}{2}$;
\item $f(x)=[x]\cdot \sin \pi x,a=1$;
\item $f(x)=|x^2-5x+6|,a=2,a=3$.
\end{enumerate}
\end{defproblem}

\begin{defproblem}{diferencialny-pocet-5}
Porovnaním hodnôt $f'_+(a)$ a $f'_-(a)$ zistite, či existuj $f'(a)$, ak:
\begin{enumerate}
\item $f(x) = \left\{ \begin{array}{r@{\quad}c}
    x, & $ak $ x<0 \\
    \ln (1+x), &  $ak $ x\geq 0 \\ \end{array} \right.
    $, $a=0$;
\item $f(x) = \left\{ \begin{array}{r@{\quad}c}
    x^2-1, & $ak $ x\leq -1 \\
    -2x, & $ak $ x>-1 \\ \end{array} \right.
    $, $a=-1$;
\item $f(x)=x\cdot |x|,a=0$;
\item $f(x)=|\sin^3 x|,a=\pi$.
\end{enumerate}
\end{defproblem}

\begin{defproblem}{diferencialny-pocet-6}
\begin{enumerate}
\item Ak existuje vlastná derivácia funkcie $f$ v bode $0$, tak platí
$$f'(0)=\lim_{h \rightarrow 0}\frac{f(h)-f(-h)}{2h}.$$ Dokážte!
\item Rozhodnite, či z existencie $\lim_{h \rightarrow 0}\frac{f(h)-f(-h)}{2h}$ vyplýva existencia $f'(0)$; svoje tvrdenie dokážte!
\end{enumerate}
\end{defproblem}

\begin{defproblem}{diferencialny-pocet-7}
Ak $f':\mathbb{R}\rightarrow\mathbb{R}$ je derivácia nepárnej funkcie $f$, tak $f'$ je párna funkcia. Dokážte! (Definíciu párnej a nepánej funkcie pozri v príklade $146.$)
\end{defproblem}

\begin{defproblem}{diferencialny-pocet-8}
Nájdite derivácie nasledujúcich funkcií:
\begin{multicols}{2}
\begin{enumerate}
    \item $y=2+x-x^2$;
    \item $y=\frac{1+x-x^2}{1-x+x^2}$;
    \item $y=\frac{3}{5}\cdot x^{\frac{5}{3}}-x^{-\sqrt{5}}$;
    \item $y=\frac{\sqrt{x}}{2+\sqrt[3]{x^2}}$;
    \item $y=(3x-7)^{10}$;
    \item $y=x\cdot \sqrt{1+x^2}$;
    \item $y=\frac{(1-x)^p}{(1+x)^q},p>1,q>0$;
    \item $y=\frac{1}{\sqrt{1+x^2}\cdot (x+\sqrt{1+x^2})}$;
    \item $y=\sqrt[13]{9+7\cdot \sqrt[5]{2x}}$;
    \item $y=\sqrt{x+\sqrt{x+\sqrt{x}}}$.
\end{enumerate}
\end{multicols}
\end{defproblem}

\begin{defproblem}{diferencialny-pocet-9}
Nájdite $f'(a)$, ak $f(x)=(x-a)\varphi(x)$, kde $\varphi:\mathbb{R}\rightarrow\mathbb{R}$ je funkcia spojitá v bode $a$.
\end{defproblem}

\begin{defproblem}{diferencialny-pocet-10}
\begin{enumerate}
\item Ukážte, že funkcia $f(x)=|x-a|\varphi(x)$, kde $\varphi:\mathbb{R}\rightarrow\mathbb{R}$ je spojitá funkcia a $\varphi(a) \neq 0$, nemá deriváciu v bode $a$.
\item Ukážte, že funkcia $f(x)=|x-a|^{1+\varepsilon}\varphi(x)$, kde $\varphi:\mathbb{R}\rightarrow\mathbb{R}$  je ohraničená na $\mathbb{R}$ a $\varepsilon>0$, má v bode $a$ deriváciu $f'(a)=0$.
\end{enumerate}
\end{defproblem}

\begin{defproblem}{diferencialny-pocet-11}
Nájdite deriváciu funkcie
\begin{multicols}{2}
\begin{enumerate}
    \item $y=x\cdot\cos x$;
    \item $y=\frac{\sin x-x\cdot\cos x}{\cos x+x\cdot\sin x}$;
    \item $y=\frac{\sqrt{x}}{\tan x}$;
    \item $y=\sin^n x\cdot\cos nx$;
    \item $y=\sin (\cos^2 x)\cdot \cos (\sin^2 x)$;
    \item $y=\frac{\sin^2 x}{\sin x^2}$;
    \item $y=\frac{\sin^2 x}{1+\cot x}+\frac{\cos^2 x}{1+\tan x}$;
    \item $y=\sqrt{1+\tan (x^2+x^{-2})}$.
\end{enumerate}
\end{multicols}
\end{defproblem}

\begin{defproblem}{diferencialny-pocet-12}
Nájdite derivácie funkcií
\begin{multicols}{2}
\begin{enumerate}
    \item $y=(\sqrt{2})^x+(\sqrt{5})^{-x}$;
    \item $y=e^{-x^2}$;
    \item $y=e^x(x^2-2x+2)$;
    \item $y=2^{\sin x^2}$;
    \item $y=e^{\sqrt{\frac{1-x}{1+x}}}$;
    \item $y=e^{ax}\cdot\frac{a\cdot\sin bx - b\cdot\cos bx}{\sqrt{a^2+b^2}}$;
    \item $y=(\frac{a}{b})^x\cdot(\frac{b}{x})^a\cdot(\frac{x}{a})^b$;
    \item $y=(\frac{1-x^2}{2}\sin x -\frac{(1+x)^2}{2}\cos x)\cdot e^{-x}$.
\end{enumerate}
\end{multicols}
\end{defproblem}

\begin{defproblem}{diferencialny-pocet-13}
Možno tvrdiť, že neexistuje $(f+g)'(a)$ (funkcie $f,g$ sú definované v okolé bodu $a$),ak:
\begin{enumerate}
\item existuje vlastná $f'(a)$ a neexistuje $g'(a)$?
\item $f$ má v bode $a$ nevlastnú deriváciu a neexistuje $g'(a)$?
\item neexistuje $f'(a)$ ani $g'(a)$?
\end{enumerate}
\end{defproblem}

\begin{defproblem}{diferencialny-pocet-14}
Nech $f,g$ sú spojité funkcie definované na $\mathbb{R}$, $g'(a)=+\infty,f'(g(a))>0$. Potom existuje nevlastná derivácia funkcie $f \circ g$ v bode $a$. Dokážte!
\end{defproblem}

\begin{defproblem}{diferencialny-pocet-15}
\begin{multicols}{2}
\begin{enumerate}
    \item $y=\log_{3}(\tan \frac{x}{2})$;
    \item $y=\frac{1}{4}\ln \frac{x^2-1}{x^2+1}$;
    \item $y=\ln (x+\sqrt{x^2+1})$;
    \item $y=x\cdot\ln (x+\sqrt{1+x^2})-\sqrt{1+x^2}$;
    \item $y=\frac{1}{2} \ln (1+x)-\frac{1}{4} \ln (1+x^2)-\frac{1}{2\cdot(1+x)}$;
    \item $y=e^{\sqrt{\log_2(x^2+x+1)}}$;
    \item $y=\frac{1}{\sin a}\cdot\ln \frac{1+x}{1-x}-\cot a\cdot\ln \frac{1+x\cdot\cos a}{1-x\cdot\cos a}$;
    \item $y=x\cdot(\sin(\ln x)-\cos(\ln x))$;
    \item $y=\sqrt{x^2+1}-\ln (\frac{1}{x}+\sqrt{1+\frac{1}{x^2}})$;
    \item $y=\log_2 x\cdot\log_3 x\cdot\ln x$.
\end{enumerate}
\end{multicols}
\end{defproblem}

\begin{defproblem}{diferencialny-pocet-16}
\begin{multicols}{2}
\begin{enumerate}
    \item $y=\sqrt{x}-\arctan\sqrt{x}$;
    \item $y=x\cdot\arcsin x$;
    \item $y=\frac{\arccos x}{\arcsin xc}$;
    \item $y=\arctan \frac{x}{1+\sqrt{1-x^2}}$;
    \item $y=x\cdot\arcsin \sqrt{\frac{x}{1+x}}+\arctan \sqrt{x}-\sqrt{x}$;
    \item $y=\ln (\arccos \frac{1}{\sqrt{x}})$;
    \item $y=3^{\arctan(2x+\pi)}$;
    \item $y=\arcsin(\frac{\sin a\cdot \sin x}{1-\cos a\cdot\cos x})$;
    \item $y=x\cdot(\arcsin x)^2+2\sqrt{1-x^2}\cdot\arcsin x-2x$;
    \item $y=\frac{1}{\sqrt{2}\arcsin(\sqrt{\frac{2}{3}}\sin x)}$.
\end{enumerate}
\end{multicols}
\end{defproblem}

\begin{defproblem}{diferencialny-pocet-17}
\begin{multicols}{2}
\begin{enumerate}
    \item $y=\frac{1+x^2}{\sqrt[3]{x^4}\sin^7 x}$;
    \item $y=\frac{\sqrt{x-1}}{\sqrt[3]{(x+2)^2}}\cdot\sqrt{(x+3)^3}$;
    \item $y=\sqrt[3]{\frac{\sin 3x}{1-\sin 3x}},x\in(0,\frac{\pi}{6})$;
    \item $y=x^x,x>0$;
    \item $y=\sqrt[x]{x},x>0$;
    \item $y=x^{x^2},x>0$;
    \item $y=(\cos x)^{\sin x}+(\sin x)^{\cos x},x\in(0,\frac{\pi}{2})$.
\end{enumerate}
\end{multicols}
\end{defproblem}

\begin{defproblem}{diferencialny-pocet-18}
Uveďte príklad funkcií $f,g$ takých, že 
\begin{enumerate}
\item neexistujú $f'(a)$ ani $g'(a)$, ale existuje vlastná $(f\cdot g)(a)$;
\item neexistuje $g'(a)$, ale existujú vlastné $f'(a)$ a $(f\cdot g)'(a)$.
\end{enumerate}
(Inšpiráciou môžu byť príklady $287$ a $288$.)
\end{defproblem}

\begin{defproblem}{diferencialny-pocet-19}
Nech $f:\mathbb{R}\rightarrow\mathbb{R}$ je spojitá funkcia, nech v bode $a$ je $f(a)\neq 0$ a existuje nevlastná $f'(a)$. Potom funkcia $\frac{1}{f}$ má v bode $a$ nevlastnú deriváciu $-f'(a)$. Dokážte!
\end{defproblem}

\begin{defproblem}{diferencialny-pocet-20}
Nech funkcia $f$ je derivovaná v okolí $O(a)$ bodu $a$, pričom pre všetky $x\in O(a)$ platí $|f(x)-f(a)|<K\cdot |x-a|$, kde $K$ je kladná konštanta. Nech funkcia $g$ má deriváciu v bode $f(a),g'(f(a))=0$. Potom funkcia $g \circ f$ má deriváciu v bode $a$ rovnú $0$. Dokážte!
\end{defproblem}

\begin{defproblem}{diferencialny-pocet-21}
Nájdite derivácie funkcií:
\begin{enumerate}
\item $y=\ln \frac{b+a\cdot\cos x +\sqrt{b^2-a^2}\cdot \sin x}{a+b\cdot\cos x},x\in (0,\frac{\pi}{2}),0\leq a<b$;
\item $y=\ln (1+\sin^2 x)-2\cdot\sin x \cdot \arctan(\sin x)$;
\item $y=\frac{\arccos x}{x}+\frac{1}{2}\cdot \ln \frac{1-\sqrt{1-x^2}}{1+\sqrt{1-x^2}}$;
\item $y=x\cdot\ln^2(x+\sqrt{1+x^2})-2\sqrt{1+x^2}\cdot\ln(x+\sqrt{1+x^2})+2x$;
\item $\ln\frac{x+a}{\sqrt{x^2+b^2}}+\frac{a}{b}\arctan\frac{x}{b}$;
\item $y=\frac{1}{4\sqrt{2}}\ln\frac{x^2+x\sqrt{2}+1}{x^2-x\sqrt{2}+1}-\frac{1}{2\sqrt{2}}\arctan\frac{x\sqrt{2}}{x^2-1}$;
\item $y=x-\ln\sqrt{1+e^{2x}}+e^{-x}arccot e^{x}$;
\item $y=\frac{3-\sin x}{2}\sqrt{\cos^2 x-2\sin x}+2\arcsin x\cdot\frac{1+\sin x}{\sqrt{2}}$; 
\item $y=e^{\arcsin x}(\cos (m\cdot\arcsin x)+\sin (m\arcsin x)),|x|<1$;
\item $y=\ln\sqrt{x^2-2x\cdot\cos a +1}+\cot\cdot\arctan\frac{x-\cos a}{\sin a}$;
\item $y=|(x-1)^2\cdot(x+1)^3|$;
\item $y=[x]\cdot\sin^2 \pi x$;
\item $y = \left\{ \begin{array}{r@{\quad}c}
    1-x,& $ak $ x<1 \\
    (1-x)(2-x),& $ak $ x \in \langle 1,2 \rangle \\
    -(2-x), &  $ak $ x>2 \\ \end{array} \right.
    $ ;
\item $y = \left\{ \begin{array}{r@{\quad}c}
    \arctan x,& $ak $ |x|\leq 1 \\
    \frac{\pi}{4}sgn x+\frac{x-1}{2}, &  $ak $ |x|>1 \\ \end{array} \right.
    $ ;
\item $y=\ln (ch x)+\frac{1}{2\cdot ch^2 x}$;
\item $y=\frac{(\ln x)^x}{x^{\ln x}}$;
\item $y=(\arcsin \sin^2 x)^{\arctan x}$;
\item $y=x^{x^x}$;
\item $y=x^{\frac{2}{\ln x}}-2x^{\log_x e}\cdot e^{1+\ln x}+e^{1+\frac{2}{\log_x e}}$.
\end{enumerate}
\end{defproblem}

\begin{defproblem}{diferencialny-pocet-22}
Ako treba vybrať koeficienty $a,b$, aby funkcia
$f(x) = \left\{ \begin{array}{r@{\quad}c}
    x^2,& $ak $ x \leq x_0 \\
    ax+b, &  $ak $ x>x_0 \\ \end{array} \right.
    $ 
    bola spojitá a mala deriváciu v každom bode $x\in\mathbb{R}$?
\end{defproblem}

\begin{defproblem}{diferencialny-pocet-23}
Ak funkcie $f_1,...,f_n$ majú deriváciu v bode $a$, tak $(f_1\cdot f_2\cdot ...\cdot f_n)'(a)=f'_1(a)\cdot f_2(a)\cdot ... \cdot f_n(a)+f_1(a)\cdot f'_2(a)\cdot f_3(a)\cdot ... \cdot f_n(a)+...+f_1(a)\cdot ... \cdot f_{n-1}(a)\cdot f'_n(a)$. Dokážte! Na základe toho nájdite $f'(0)$, ak $f(x)=x\cdot(x-1)\cdot...\cdot (x-1000)$.
\end{defproblem}

\begin{defproblem}{diferencialny-pocet-24}
Nech nezáporné funkcie $f,g$ majú deriváciu v každom bode $x\in\mathbb{R}$. Nájdite deriváciu funkcie $y$, ak
\begin{multicols}{2}
\begin{enumerate}
    \item $y=\sqrt{f^2(x)+g^2(x)}$;
    \item $y=\arctan \frac{f(x)}{g(x)}$;
    \item $y=\log_{f(x)}g(x)$ $(f(x)>1)$;
    \item $y=f(x^2)$;
    \item $y=f(\sin^2 x)+g(\cos^2 x)$;
    \item $y=f(e^x)\cdot e^{f(x)}$.
\end{enumerate}
\end{multicols}
\end{defproblem}

\begin{defproblem}{diferencialny-pocet-25}
Nájdite vzorce pre súčty:
\begin{enumerate}
\item $P_n(x)=1+2x+3x^2+...+nx^{n-1}$;
\item $Q_n(x)=1+3x^2+5x^4+...+(2n-1)x^{2n-2},|x|\neq 1$;
\item $R_n(x)=1^2+2^2x+3^2x^2+...+n^2x^{n-1},x\neq 1$;
\item $T_n(x)=\cos x+2\cos 2x+...+n\cos nx,x\neq 2k\pi,(k\in\mathbb{Z})$.
\end{enumerate}
\end{defproblem}

\begin{defproblem}{diferencialny-pocet-26}
Ako treba voliť číslo $\alpha\in\mathbb{R}$ aby funkcia
$f(x) = \left\{ \begin{array}{r@{\quad}c}
    x^\alpha,& $ak $ x \neq 0 \\
    0, &  $ak $ x=0 \\ \end{array} \right.
    $ 
    bola
    \begin{enumerate}
    \item spojitá v bode $0$?
    \item mala deriváciu v bode $0$?
    \item mala deriváciu, ktorá je spojitá v bode $0$?
    \end{enumerate}
\end{defproblem}

\begin{defproblem}{diferencialny-pocet-27}
Možno použiť vetu o derivácii zloženej funkcie na výpočet derivácie funkcie $y=\sin^2(\sqrt[3]{x^2})$ v bode $0$? Existuje táto derivácia?
\end{defproblem}

\begin{defproblem}{diferencialny-pocet-28}
Zostrojte funkciu $f:\mathbb{R}\rightarrow \mathbb{R}$ tak, aby
\begin{enumerate}
\item definitným oborom jej derivácie bola množina $\{1,2\}$;
\item bola rastúca, mala vlastnú alebo nevlastnú deriváciu v každom bode $x\in\mathbb{R}$ a pre každé $n\in\mathbb{Z}$ platilo $f'(n)=+\infty$;
\item mala deriváciu v každom bode $x\in\mathbb{R}$, funkcia $f'$ bola nespokojná práve v bodoch množiny $\mathbb{N}\cup \{0\}$ a aby platilo $f'(n)=a$ ($a$ je dané reálne číslo) pre každé $n\in\mathbb{N}\cup \{0\}$.
\end{enumerate}
\end{defproblem}

\begin{defproblem}{diferencialny-pocet-29}
Nech funkcia $f$ má v bode $a$ obidve jednostranné derivácie. Potom $f$ je spojitá v bode $a$. Dokážte!
\end{defproblem}

\begin{defproblem}{diferencialny-pocet-30}
Uveďte príklad funkcie, ktorá je spojitá v každom bode množiny $\mathbb{R}$ a nemá vlastnú ani nevlastnú deriváciu v žiadnom bode množiny $\mathbb{N}$!
\end{defproblem}

\begin{defproblem}{diferencialny-pocet-31}
Uveďte príklad funkcie $f$ takej, že $f'(a)=-\infty$ a
\begin{enumerate}
\item $f$ je spojitá v bode $a$;
\item $a$ je bod nespojitosti $1.$ druhu funkcie $f$;
\item $a$ je bod nespojitosti $2.$ druhu funkcie $f$.
\end{enumerate}
\end{defproblem}

\begin{defproblem}{diferencialny-pocet-32}

\end{defproblem}

\begin{defproblem}{diferencialny-pocet-33}

\end{defproblem}

\begin{defproblem}{diferencialny-pocet-34}

\end{defproblem}

\begin{defproblem}{diferencialny-pocet-35}

\end{defproblem}

\begin{defproblem}{diferencialny-pocet-36}

\end{defproblem}

\begin{defproblem}{diferencialny-pocet-37}

\end{defproblem}

\begin{defproblem}{diferencialny-pocet-38}

\end{defproblem}

\begin{defproblem}{diferencialny-pocet-39}

\end{defproblem}

\begin{defproblem}{diferencialny-pocet-40}

\end{defproblem}

\begin{defproblem}{diferencialny-pocet-41}

\end{defproblem}

\begin{defproblem}{diferencialny-pocet-42}

\end{defproblem}

\begin{defproblem}{diferencialny-pocet-43}

\end{defproblem}

\begin{defproblem}{diferencialny-pocet-44}

\end{defproblem}

\begin{defproblem}{diferencialny-pocet-45}

\end{defproblem}
\begin{defproblem}{diferencialny-pocet-1}
Výpočtom $\lim_{x \rightarrow a}\frac{f(x)-f(a)}{x-a}$ nájdite deriváciu funkcie $f$ v bode $a$, ak:
\begin{multicols}{2}
\begin{enumerate}
    \item $f(x)=x^2,a=0,1$;
    \item $f(x)=2\cdot \sin 3x,a=\frac{\pi}{6}$;
    \item $f(x)=1+2\cdot \ln x,a=1$;
    \item $f(x)=\arcsin x,a=0$;
    \item $f(x)=e^x,a=\ln 2$.
\end{enumerate}
\end{multicols}
\end{defproblem}

\begin{defproblem}{diferencialny-pocet-2}
Ukážte, že funkcia $f$ má v bode $a$ nevlastnú deriváciu, ak:
\begin{enumerate}
\item $f(x)=\sqrt[3]{x-1},a=1$;
\item $f(x)=sgn x,a=0$.
\end{enumerate}
\end{defproblem}

\begin{defproblem}{diferencialny-pocet-3}
Výpočtom $\lim_{h \rightarrow 0}\frac{f(x+h)-f)(x)}{h}$ nájdite $f'(x)$ a yistite definičný obor funkcie $f'$, ak:
\begin{multicols}{2}
\begin{enumerate}
   \item $f(x)=x^3+2x$;
   \item $f(x)=x\cdot \sqrt[3]{x}$;
   \item $f(x)=\frac{1}{1+x^2}$;
   \item $f(x)=2^{x+1}$;
   \item $f(x)=\ln x$;
   \item $f(x)=\cot x+2$;
   \item $f(x)=(x+1)^{\frac{2}{3}}$;
   \item $f(x)=sgn x$.
\end{enumerate}
\end{multicols}
\end{defproblem}

\begin{defproblem}{diferencialny-pocet-4}
Výpočtom príslušných limít nájdite $f'_+(a),f'_-(a)$, ak:
\begin{enumerate}
\item $f(x)=|\cos x|,a=\frac{\pi}{2}$;
\item $f(x)=[x]\cdot \sin \pi x,a=1$;
\item $f(x)=|x^2-5x+6|,a=2,a=3$.
\end{enumerate}
\end{defproblem}

\begin{defproblem}{diferencialny-pocet-5}
Porovnaním hodnôt $f'_+(a)$ a $f'_-(a)$ zistite, či existuj $f'(a)$, ak:
\begin{enumerate}
\item $f(x) = \left\{ \begin{array}{r@{\quad}c}
    x, & $ak $ x<0 \\
    \ln (1+x), &  $ak $ x\geq 0 \\ \end{array} \right.
    $, $a=0$;
\item $f(x) = \left\{ \begin{array}{r@{\quad}c}
    x^2-1, & $ak $ x\leq -1 \\
    -2x, & $ak $ x>-1 \\ \end{array} \right.
    $, $a=-1$;
\item $f(x)=x\cdot |x|,a=0$;
\item $f(x)=|\sin^3 x|,a=\pi$.
\end{enumerate}
\end{defproblem}

\begin{defproblem}{diferencialny-pocet-6}
\begin{enumerate}
\item Ak existuje vlastná derivácia funkcie $f$ v bode $0$, tak platí
$$f'(0)=\lim_{h \rightarrow 0}\frac{f(h)-f(-h)}{2h}.$$ Dokážte!
\item Rozhodnite, či z existencie $\lim_{h \rightarrow 0}\frac{f(h)-f(-h)}{2h}$ vyplýva existencia $f'(0)$; svoje tvrdenie dokážte!
\end{enumerate}
\end{defproblem}

\begin{defproblem}{diferencialny-pocet-7}

\end{defproblem}

\begin{defproblem}{diferencialny-pocet-8}

\end{defproblem}

\begin{defproblem}{diferencialny-pocet-9}

\end{defproblem}

\begin{defproblem}{diferencialny-pocet-10}

\end{defproblem}

\begin{defproblem}{diferencialny-pocet-11}

\end{defproblem}

\begin{defproblem}{diferencialny-pocet-12}

\end{defproblem}

\begin{defproblem}{diferencialny-pocet-13}

\end{defproblem}

\begin{defproblem}{diferencialny-pocet-14}

\end{defproblem}

\begin{defproblem}{diferencialny-pocet-15}

\end{defproblem}

\begin{defproblem}{diferencialny-pocet-16}

\end{defproblem}

\begin{defproblem}{diferencialny-pocet-17}

\end{defproblem}

\begin{defproblem}{diferencialny-pocet-18}

\end{defproblem}

\begin{defproblem}{diferencialny-pocet-19}

\end{defproblem}

\begin{defproblem}{diferencialny-pocet-20}

\end{defproblem}

\begin{defproblem}{diferencialny-pocet-21}

\end{defproblem}
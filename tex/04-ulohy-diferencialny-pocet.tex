\begin{defproblem}{diferencialny-pocet-1}
Výpočtom $\lim_{x \rightarrow a}\frac{f(x)-f(a)}{x-a}$ nájdite deriváciu funkcie $f$ v bode $a$, ak:
\begin{multicols}{2}
\begin{enumerate}
    \item $f(x)=x^2,a=0,1$;
    \item $f(x)=2\cdot \sin 3x,a=\frac{\pi}{6}$;
    \item $f(x)=1+2\cdot \ln x,a=1$;
    \item $f(x)=\arcsin x,a=0$;
    \item $f(x)=e^x,a=\ln 2$.
\end{enumerate}
\end{multicols}
\end{defproblem}

\begin{defproblem}{diferencialny-pocet-2}
Ukážte, že funkcia $f$ má v bode $a$ nevlastnú deriváciu, ak:
\begin{enumerate}
\item $f(x)=\sqrt[3]{x-1},a=1$;
\item $f(x)=sgn x,a=0$.
\end{enumerate}
\end{defproblem}

\begin{defproblem}{diferencialny-pocet-3}
Výpočtom $\lim_{h \rightarrow 0}\frac{f(x+h)-f)(x)}{h}$ nájdite $f'(x)$ a yistite definičný obor funkcie $f'$, ak:
\begin{multicols}{2}
\begin{enumerate}
   \item $f(x)=x^3+2x$;
   \item $f(x)=x\cdot \sqrt[3]{x}$;
   \item $f(x)=\frac{1}{1+x^2}$;
   \item $f(x)=2^{x+1}$;
   \item $f(x)=\ln x$;
   \item $f(x)=\cot x+2$;
   \item $f(x)=(x+1)^{\frac{2}{3}}$;
   \item $f(x)=sgn x$.
\end{enumerate}
\end{multicols}
\end{defproblem}

\begin{defproblem}{diferencialny-pocet-4}
Výpočtom príslušných limít nájdite $f'_+(a),f'_-(a)$, ak:
\begin{enumerate}
\item $f(x)=|\cos x|,a=\frac{\pi}{2}$;
\item $f(x)=[x]\cdot \sin \pi x,a=1$;
\item $f(x)=|x^2-5x+6|,a=2,a=3$.
\end{enumerate}
\end{defproblem}

\begin{defproblem}{diferencialny-pocet-5}
Porovnaním hodnôt $f'_+(a)$ a $f'_-(a)$ zistite, či existuj $f'(a)$, ak:
\begin{enumerate}
\item $f(x) = \left\{ \begin{array}{r@{\quad}c}
    x, & $ak $ x<0 \\
    \ln (1+x), &  $ak $ x\geq 0 \\ \end{array} \right.
    $, $a=0$;
\item $f(x) = \left\{ \begin{array}{r@{\quad}c}
    x^2-1, & $ak $ x\leq -1 \\
    -2x, & $ak $ x>-1 \\ \end{array} \right.
    $, $a=-1$;
\item $f(x)=x\cdot |x|,a=0$;
\item $f(x)=|\sin^3 x|,a=\pi$.
\end{enumerate}
\end{defproblem}

\begin{defproblem}{diferencialny-pocet-6}
\begin{enumerate}
\item Ak existuje vlastná derivácia funkcie $f$ v bode $0$, tak platí
$$f'(0)=\lim_{h \rightarrow 0}\frac{f(h)-f(-h)}{2h}.$$ Dokážte!
\item Rozhodnite, či z existencie $\lim_{h \rightarrow 0}\frac{f(h)-f(-h)}{2h}$ vyplýva existencia $f'(0)$; svoje tvrdenie dokážte!
\end{enumerate}
\end{defproblem}

\begin{defproblem}{diferencialny-pocet-7}
Ak $f':\mathbb{R}\rightarrow\mathbb{R}$ je derivácia nepárnej funkcie $f$, tak $f'$ je párna funkcia. Dokážte! (Definíciu párnej a nepánej funkcie pozri v príklade $146.$)
\end{defproblem}

\begin{defproblem}{diferencialny-pocet-8}
Nájdite derivácie nasledujúcich funkcií:
\begin{multicols}{2}
\begin{enumerate}
    \item $y=2+x-x^2$;
    \item $y=\frac{1+x-x^2}{1-x+x^2}$;
    \item $y=\frac{3}{5}\cdot x^{\frac{5}{3}}-x^{-\sqrt{5}}$;
    \item $y=\frac{\sqrt{x}}{2+\sqrt[3]{x^2}}$;
    \item $y=(3x-7)^{10}$;
    \item $y=x\cdot \sqrt{1+x^2}$;
    \item $y=\frac{(1-x)^p}{(1+x)^q},p>1,q>0$;
    \item $y=\frac{1}{\sqrt{1+x^2}\cdot (x+\sqrt{1+x^2})}$;
    \item $y=\sqrt[13]{9+7\cdot \sqrt[5]{2x}}$;
    \item $y=\sqrt{x+\sqrt{x+\sqrt{x}}}$.
\end{enumerate}
\end{multicols}
\end{defproblem}

\begin{defproblem}{diferencialny-pocet-9}
Nájdite $f'(a)$, ak $f(x)=(x-a)\varphi(x)$, kde $\varphi:\mathbb{R}\rightarrow\mathbb{R}$ je funkcia spojitá v bode $a$.
\end{defproblem}

\begin{defproblem}{diferencialny-pocet-10}
\begin{enumerate}
\item Ukážte, že funkcia $f(x)=|x-a|\varphi(x)$, kde $\varphi:\mathbb{R}\rightarrow\mathbb{R}$ je spojitá funkcia a $\varphi(a) \neq 0$, nemá deriváciu v bode $a$.
\item Ukážte, že funkcia $f(x)=|x-a|^{1+\varepsilon}\varphi(x)$, kde $\varphi:\mathbb{R}\rightarrow\mathbb{R}$  je ohraničená na $\mathbb{R}$ a $\varepsilon>0$, má v bode $a$ deriváciu $f'(a)=0$.
\end{enumerate}
\end{defproblem}

\begin{defproblem}{diferencialny-pocet-11}
Nájdite deriváciu funkcie
\begin{multicols}{2}
\begin{enumerate}
    \item $y=x\cdot\cos x$;
    \item $y=\frac{\sin x-x\cdot\cos x}{\cos x+x\cdot\sin x}$;
    \item $y=\frac{\sqrt{x}}{\tan x}$;
    \item $y=\sin^n x\cdot\cos nx$;
    \item $y=\sin (\cos^2 x)\cdot \cos (\sin^2 x)$;
    \item $y=\frac{\sin^2 x}{\sin x^2}$;
    \item $y=\frac{\sin^2 x}{1+\cot x}+\frac{\cos^2 x}{1+\tan x}$;
    \item $y=\sqrt{1+\tan (x^2+x^{-2})}$.
\end{enumerate}
\end{multicols}
\end{defproblem}

\begin{defproblem}{diferencialny-pocet-12}
Nájdite derivácie funkcií
\begin{multicols}{2}
\begin{enumerate}
    \item $y=(\sqrt{2})^x+(\sqrt{5})^{-x}$;
    \item $y=e^{-x^2}$;
    \item $y=e^x(x^2-2x+2)$;
    \item $y=2^{\sin x^2}$;
    \item $y=e^{\sqrt{\frac{1-x}{1+x}}}$;
    \item $y=e^{ax}\cdot\frac{a\cdot\sin bx - b\cdot\cos bx}{\sqrt{a^2+b^2}}$;
    \item $y=(\frac{a}{b})^x\cdot(\frac{b}{x})^a\cdot(\frac{x}{a})^b$;
    \item $y=(\frac{1-x^2}{2}\sin x -\frac{(1+x)^2}{2}\cos x)\cdot e^{-x}$.
\end{enumerate}
\end{multicols}
\end{defproblem}

\begin{defproblem}{diferencialny-pocet-13}
Možno tvrdiť, že neexistuje $(f+g)'(a)$ (funkcie $f,g$ sú definované v okolé bodu $a$),ak:
\begin{enumerate}
\item existuje vlastná $f'(a)$ a neexistuje $g'(a)$?
\item $f$ má v bode $a$ nevlastnú deriváciu a neexistuje $g'(a)$?
\item neexistuje $f'(a)$ ani $g'(a)$?
\end{enumerate}
\end{defproblem}

\begin{defproblem}{diferencialny-pocet-14}
Nech $f,g$ sú spojité funkcie definované na $\mathbb{R}$, $g'(a)=+\infty,f'(g(a))>0$. Potom existuje nevlastná derivácia funkcie $f \circ g$ v bode $a$. Dokážte!
\end{defproblem}

\begin{defproblem}{diferencialny-pocet-15}
\begin{multicols}{2}
\begin{enumerate}
    \item $y=\log_{3}(\tan \frac{x}{2})$;
    \item $y=\frac{1}{4}\ln \frac{x^2-1}{x^2+1}$;
    \item $y=\ln (x+\sqrt{x^2+1})$;
    \item $y=x\cdot\ln (x+\sqrt{1+x^2})-\sqrt{1+x^2}$;
    \item $y=\frac{1}{2} \ln (1+x)-\frac{1}{4} \ln (1+x^2)-\frac{1}{2\cdot(1+x)}$;
    \item $y=e^{\sqrt{\log_2(x^2+x+1)}}$;
    \item $y=\frac{1}{\sin a}\cdot\ln \frac{1+x}{1-x}-\cot a\cdot\ln \frac{1+x\cdot\cos a}{1-x\cdot\cos a}$;
    \item $y=x\cdot(\sin(\ln x)-\cos(\ln x))$;
    \item $y=\sqrt{x^2+1}-\ln (\frac{1}{x}+\sqrt{1+\frac{1}{x^2}})$;
    \item $y=\log_2 x\cdot\log_3 x\cdot\ln x$.
\end{enumerate}
\end{multicols}
\end{defproblem}

\begin{defproblem}{diferencialny-pocet-16}
\begin{multicols}{2}
\begin{enumerate}
    \item $y=\sqrt{x}-\arctan\sqrt{x}$;
    \item $y=x\cdot\arcsin x$;
    \item $y=\frac{\arccos x}{\arcsin xc}$;
    \item $y=\arctan \frac{x}{1+\sqrt{1-x^2}}$;
    \item $y=x\cdot\arcsin \sqrt{\frac{x}{1+x}}+\arctan \sqrt{x}-\sqrt{x}$;
    \item $y=\ln (\arccos \frac{1}{\sqrt{x}})$;
    \item $y=3^{\arctan(2x+\pi)}$;
    \item $y=\arcsin(\frac{\sin a\cdot \sin x}{1-\cos a\cdot\cos x})$;
    \item $y=x\cdot(\arcsin x)^2+2\sqrt{1-x^2}\cdot\arcsin x-2x$;
    \item $y=\frac{1}{\sqrt{2}\arcsin(\sqrt{\frac{2}{3}}\sin x)}$.
\end{enumerate}
\end{multicols}
\end{defproblem}

\begin{defproblem}{diferencialny-pocet-17}
\begin{multicols}{2}
\begin{enumerate}
    \item $y=\frac{1+x^2}{\sqrt[3]{x^4}\sin^7 x}$;
    \item $y=\frac{\sqrt{x-1}}{\sqrt[3]{(x+2)^2}}\cdot\sqrt{(x+3)^3}$;
    \item $y=\sqrt[3]{\frac{\sin 3x}{1-\sin 3x}},x\in(0,\frac{\pi}{6})$;
    \item $y=x^x,x>0$;
    \item $y=\sqrt[x]{x},x>0$;
    \item $y=x^{x^2},x>0$;
    \item $y=(\cos x)^{\sin x}+(\sin x)^{\cos x},x\in(0,\frac{\pi}{2})$.
\end{enumerate}
\end{multicols}
\end{defproblem}

\begin{defproblem}{diferencialny-pocet-18}
Uveďte príklad funkcií $f,g$ takých, že 
\begin{enumerate}
\item neexistujú $f'(a)$ ani $g'(a)$, ale existuje vlastná $(f\cdot g)(a)$;
\item neexistuje $g'(a)$, ale existujú vlastné $f'(a)$ a $(f\cdot g)'(a)$.
\end{enumerate}
(Inšpiráciou môžu byť príklady $287$ a $288$.)
\end{defproblem}

\begin{defproblem}{diferencialny-pocet-19}
Nech $f:\mathbb{R}\rightarrow\mathbb{R}$ je spojitá funkcia, nech v bode $a$ je $f(a)\neq 0$ a existuje nevlastná $f'(a)$. Potom funkcia $\frac{1}{f}$ má v bode $a$ nevlastnú deriváciu $-f'(a)$. Dokážte!
\end{defproblem}

\begin{defproblem}{diferencialny-pocet-20}
Nech funkcia $f$ je derivovaná v okolí $O(a)$ bodu $a$, pričom pre všetky $x\in O(a)$ platí $|f(x)-f(a)|<K\cdot |x-a|$, kde $K$ je kladná konštanta. Nech funkcia $g$ má deriváciu v bode $f(a),g'(f(a))=0$. Potom funkcia $g \circ f$ má deriváciu v bode $a$ rovnú $0$. Dokážte!
\end{defproblem}

\begin{defproblem}{diferencialny-pocet-21}
Nájdite derivácie funkcií:
\begin{enumerate}
\item $y=\ln \frac{b+a\cdot\cos x +\sqrt{b^2-a^2}\cdot \sin x}{a+b\cdot\cos x},x\in (0,\frac{\pi}{2}),0\leq a<b$;
\item $y=\ln (1+\sin^2 x)-2\cdot\sin x \cdot \arctan(\sin x)$;
\item $y=\frac{\arccos x}{x}+\frac{1}{2}\cdot \ln \frac{1-\sqrt{1-x^2}}{1+\sqrt{1-x^2}}$;
\item $y=x\cdot\ln^2(x+\sqrt{1+x^2})-2\sqrt{1+x^2}\cdot\ln(x+\sqrt{1+x^2})+2x$;
\item $\ln\frac{x+a}{\sqrt{x^2+b^2}}+\frac{a}{b}\arctan\frac{x}{b}$;
\item $y=\frac{1}{4\sqrt{2}}\ln\frac{x^2+x\sqrt{2}+1}{x^2-x\sqrt{2}+1}-\frac{1}{2\sqrt{2}}\arctan\frac{x\sqrt{2}}{x^2-1}$;
\item $y=x-\ln\sqrt{1+e^{2x}}+e^{-x}arccot e^{x}$;
\item $y=\frac{3-\sin x}{2}\sqrt{\cos^2 x-2\sin x}+2\arcsin x\cdot\frac{1+\sin x}{\sqrt{2}}$; 
\item $y=e^{\arcsin x}(\cos (m\cdot\arcsin x)+\sin (m\arcsin x)),|x|<1$;
\item $y=\ln\sqrt{x^2-2x\cdot\cos a +1}+\cot\cdot\arctan\frac{x-\cos a}{\sin a}$;
\item $y=|(x-1)^2\cdot(x+1)^3|$;
\item $y=[x]\cdot\sin^2 \pi x$;
\item $y = \left\{ \begin{array}{r@{\quad}c}
    1-x,& $ak $ x<1 \\
    (1-x)(2-x),& $ak $ x \in \langle 1,2 \rangle \\
    -(2-x), &  $ak $ x>2 \\ \end{array} \right.
    $ ;
\item $y = \left\{ \begin{array}{r@{\quad}c}
    \arctan x,& $ak $ |x|\leq 1 \\
    \frac{\pi}{4}sgn x+\frac{x-1}{2}, &  $ak $ |x|>1 \\ \end{array} \right.
    $ ;
\item $y=\ln (ch x)+\frac{1}{2\cdot ch^2 x}$;
\item $y=\frac{(\ln x)^x}{x^{\ln x}}$;
\item $y=(\arcsin \sin^2 x)^{\arctan x}$;
\item $y=x^{x^x}$;
\item $y=x^{\frac{2}{\ln x}}-2x^{\log_x e}\cdot e^{1+\ln x}+e^{1+\frac{2}{\log_x e}}$.
\end{enumerate}
\end{defproblem}

\begin{defproblem}{diferencialny-pocet-22}
Ako treba vybrať koeficienty $a,b$, aby funkcia
$f(x) = \left\{ \begin{array}{r@{\quad}c}
    x^2,& $ak $ x \leq x_0 \\
    ax+b, &  $ak $ x>x_0 \\ \end{array} \right.
    $ 
    bola spojitá a mala deriváciu v každom bode $x\in\mathbb{R}$?
\end{defproblem}

\begin{defproblem}{diferencialny-pocet-23}
Ak funkcie $f_1,...,f_n$ majú deriváciu v bode $a$, tak $(f_1\cdot f_2\cdot ...\cdot f_n)'(a)=f'_1(a)\cdot f_2(a)\cdot ... \cdot f_n(a)+f_1(a)\cdot f'_2(a)\cdot f_3(a)\cdot ... \cdot f_n(a)+...+f_1(a)\cdot ... \cdot f_{n-1}(a)\cdot f'_n(a)$. Dokážte! Na základe toho nájdite $f'(0)$, ak $f(x)=x\cdot(x-1)\cdot...\cdot (x-1000)$.
\end{defproblem}

\begin{defproblem}{diferencialny-pocet-24}
Nech nezáporné funkcie $f,g$ majú deriváciu v každom bode $x\in\mathbb{R}$. Nájdite deriváciu funkcie $y$, ak
\begin{multicols}{2}
\begin{enumerate}
    \item $y=\sqrt{f^2(x)+g^2(x)}$;
    \item $y=\arctan \frac{f(x)}{g(x)}$;
    \item $y=\log_{f(x)}g(x)$ $(f(x)>1)$;
    \item $y=f(x^2)$;
    \item $y=f(\sin^2 x)+g(\cos^2 x)$;
    \item $y=f(e^x)\cdot e^{f(x)}$.
\end{enumerate}
\end{multicols}
\end{defproblem}

\begin{defproblem}{diferencialny-pocet-25}
Nájdite vzorce pre súčty:
\begin{enumerate}
\item $P_n(x)=1+2x+3x^2+...+nx^{n-1}$;
\item $Q_n(x)=1+3x^2+5x^4+...+(2n-1)x^{2n-2},|x|\neq 1$;
\item $R_n(x)=1^2+2^2x+3^2x^2+...+n^2x^{n-1},x\neq 1$;
\item $T_n(x)=\cos x+2\cos 2x+...+n\cos nx,x\neq 2k\pi,(k\in\mathbb{Z})$.
\end{enumerate}
\end{defproblem}

\begin{defproblem}{diferencialny-pocet-26}
Ako treba voliť číslo $\alpha\in\mathbb{R}$ aby funkcia
$f(x) = \left\{ \begin{array}{r@{\quad}c}
    x^\alpha,& $ak $ x \neq 0 \\
    0, &  $ak $ x=0 \\ \end{array} \right.
    $ 
    bola
    \begin{enumerate}
    \item spojitá v bode $0$?
    \item mala deriváciu v bode $0$?
    \item mala deriváciu, ktorá je spojitá v bode $0$?
    \end{enumerate}
\end{defproblem}

\begin{defproblem}{diferencialny-pocet-27}
Možno použiť vetu o derivácii zloženej funkcie na výpočet derivácie funkcie $y=\sin^2(\sqrt[3]{x^2})$ v bode $0$? Existuje táto derivácia?
\end{defproblem}

\begin{defproblem}{diferencialny-pocet-28}
Zostrojte funkciu $f:\mathbb{R}\rightarrow \mathbb{R}$ tak, aby
\begin{enumerate}
\item definitným oborom jej derivácie bola množina $\{1,2\}$;
\item bola rastúca, mala vlastnú alebo nevlastnú deriváciu v každom bode $x\in\mathbb{R}$ a pre každé $n\in\mathbb{Z}$ platilo $f'(n)=+\infty$;
\item mala deriváciu v každom bode $x\in\mathbb{R}$, funkcia $f'$ bola nespokojná práve v bodoch množiny $\mathbb{N}\cup \{0\}$ a aby platilo $f'(n)=a$ ($a$ je dané reálne číslo) pre každé $n\in\mathbb{N}\cup \{0\}$.
\end{enumerate}
\end{defproblem}

\begin{defproblem}{diferencialny-pocet-29}
Nech funkcia $f$ má v bode $a$ obidve jednostranné derivácie. Potom $f$ je spojitá v bode $a$. Dokážte!
\end{defproblem}

\begin{defproblem}{diferencialny-pocet-30}
Uveďte príklad funkcie, ktorá je spojitá v každom bode množiny $\mathbb{R}$ a nemá vlastnú ani nevlastnú deriváciu v žiadnom bode množiny $\mathbb{N}$!
\end{defproblem}

\begin{defproblem}{diferencialny-pocet-31}
Uveďte príklad funkcie $f$ takej, že $f'(a)=-\infty$ a
\begin{enumerate}
\item $f$ je spojitá v bode $a$;
\item $a$ je bod nespojitosti $1.$ druhu funkcie $f$;
\item $a$ je bod nespojitosti $2.$ druhu funkcie $f$.
\end{enumerate}
\end{defproblem}

\begin{defproblem}{diferencialny-pocet-32}
Nájdite deriváciu funkcie $f^{-1}$ v bode $b$, ak
\begin{enumerate}
\item $f(x)=x+\frac{1}{5}x^5,\frac{\alpha}{b}=0,\frac{\beta}{b}=\frac{6}{5}$;
\item $f(x)=2x-\frac{1}{2}\cos x,b=-\frac{1}{2}$;
\item $f(x)=0,1x+e^{0,1x},b=1$;
\item $f(x)=2x^2-x^4,x>1,b=0$;
\item $f(x)=2x^2-x^4,x\in (0,1),b=\frac{3}{4}$.
\end{enumerate}
\end{defproblem}

\begin{defproblem}{diferencialny-pocet-33}
Odvoďte vzorec pre deriváciu inverznej funkcie z vety o derivácii zloženej funkcie a stanovte predpoklady, za ktorých možno toto odvodenie vykonať!
\end{defproblem}

\begin{defproblem}{diferencialny-pocet-34}
Na základe vety o derivácii inverznej funkcie nájdite deriváciu funkcií
\begin{multicols}{2}
\begin{enumerate}
    \item $f(x)=\arcsin x$;
    \item $f(x)=\arccos x$;
    \item $f(x)=\arctan x$;
    \item $f(x)=arccot x$;
    \item $f(x)=\ln x$.
\end{enumerate}
\end{multicols}
\end{defproblem}

\begin{defproblem}{diferencialny-pocet-35}
Nech $f$ je prostá funkcia definovaná na intervale $(-\varepsilon,\varepsilon)$, kde $\varepsilon$ je dané kladné číslo, nech $f'(0)=\infty,f(0)=0$. Potom funkcia $f^{-1}$ má v bode $0$ deriváciu rovnú $0$. Dokážte!
\end{defproblem}

\begin{defproblem}{diferencialny-pocet-36}
Nájdite dotyčnicu v bode $A$ ku krivke $y=f(x)$, ak:
\begin{enumerate}
\item $f(x)=(1+x)\sqrt[3]{3-x},\frac{\alpha}{A}=(-1,0),\frac{\beta}{A}=(2,3)$;
\item $f(x)=|x-1|\ln (x+\sqrt{x^2+1}),A=(0,1)$;
\item $f(x)=\sqrt[3]{x-1},A=(1,0)$;
\item $f(x)=\frac{1}{1+x^2},A$ je priesečník krivky $y=f(x)$ s hyperbolou $y=\frac{1}{1+x}$.
\end{enumerate}
\end{defproblem}

\begin{defproblem}{diferencialny-pocet-37}
Nájdite rovnicu tej dotyčnice k parabole $y=x^2-2x+3$, ktorá je
\begin{enumerate}
\item rovnobežná s priamkou $3x-y+5=0$;
\item kolmá na priamku $x+y-1=0$.
\end{enumerate}
\end{defproblem}

\begin{defproblem}{diferencialny-pocet-38}
\begin{enumerate}
\item Aký musí byť vzťah medzi koeficientami $a,b,c$, aby sa parabola $y=ax^2+bx+c$ dotýkala priamky $y=0$?
\item Pre akú hodnotu parametra $a$ sa parabola $y=ax^2$ dotýka krivky $y=\ln x$?
\end{enumerate}
\end{defproblem}

\begin{defproblem}{diferencialny-pocet-39}
Pomocou derivácie a diferenciálu nezávislej premennej zapíšte $df(a)$, ak funkcia $f$ je daná predpisom:
\begin{multicols}{2}
\begin{enumerate}
    \item $f(x)=\frac{\ln x}{\sqrt{x}}$;
    \item $f(x)=\sqrt{A^2+x^2}$;
    \item $f(x)=\ln (1-x^2)$;
    \item $f(x)=\frac{\sin x}{2\cos^2 x}+\frac{1}{2}\ln |\tan (\frac{x}{2})+\frac{\pi}{4}|$.
\end{enumerate}
\end{multicols}
\end{defproblem}

\begin{defproblem}{diferencialny-pocet-40}
Nech $u,v,w$ sú kladné diferencovateľné funkcie, nájdite diferenciál funkcie $y$ v bode $a$, ak
\begin{multicols}{2}
\begin{enumerate}
    \item $y=u\cdot v\cdot w$;
    \item $y=\frac{u}{v^2}$;
    \item $y=\arctan \frac{u}{w}$;
    \item $y=\ln \sqrt{u^2+v^2}$.
\end{enumerate}
\end{multicols}
\end{defproblem}

\begin{defproblem}{diferencialny-pocet-41}
Pomocou diferenciálu odvoďte približné vzťahy pre výpočet nasledujúcich výrazov:
\begin{enumerate}
\item $\sqrt{a^2+x^2},(a>0,x$ dostatočne malé); 
\item $\ln (x+\sqrt{1+x^2})$ ($x$ blízke $0$);
\item $\arctan (1+x)$ ($x$ blízke $0$);
\item $\sqrt[n]{a^n+x}$ ($a>0$,$x$ blízke $0$);
\item$\ln (\tan \frac{x}{2})$ ($x$ blízke číslu $\frac{\pi}{2}$).
\end{enumerate}
\end{defproblem}

\begin{defproblem}{diferencialny-pocet-42}
Nech funkcia $f$ je definovaná v okolí bodu $a\in\mathbb{R}$. Potom sú nasledujúce dva výroky ekvivalentné:
\begin{enumerate}
\item funkcia $f$ je diferencovateľná v bode $a$;
\item existuje funkcia $\varphi$ (s rovnakým definičným oborom ako $f$), ktorá je spojitá v bode $a$, pričom platí $f(x)=f(a)+\varphi(x)(x-a)$.
(Výrok $(b)$ sa nazýva Čechova definícia diferenciálu.)
\end{enumerate}
\end{defproblem}

\begin{defproblem}{diferencialny-pocet-43}
Nájdite $y''$, ak
\begin{multicols}{2}
\begin{enumerate}
    \item $y=x\cdot \sqrt{1+x^2}$;
    \item $y=\frac{x}{\sqrt{1-x^2}}$;
    \item $y=e^{-x^2}$;
    \item $y=(1+x^2)\cdot \arctan x$;
    \item $y=x\cdot \ln x$;
    \item $y=x\cdot(\sin \ln x+\cos \ln x)$.
\end{enumerate}
\end{multicols}
\end{defproblem}

\begin{defproblem}{diferencialny-pocet-44}
Nájdite určené derivácie nasledujúcich funkcií:
\begin{enumerate}
\item $y=x\cdot (2x-1)^2(x+3)^3,y^{VI},y^{VII}$;
\item $y=(2x-7)^2(x+7)^3,y^{V},y^{VI}$;
\item $y=\sqrt{x},y^{(10)}$;
\item $y=\frac{1-x}{1+x},y^{(22)}$;
\item $y=\frac{2x+1}{x^2+x-2},y^{(13)}$;
\item $y=\sin 2x\cdot\cos 4x,y^{(15)}$;
\item $y=\sin x\cdot \sin 2x \cdot \sin 3x,y^{(10)}$.
\end{enumerate}
\end{defproblem}

\begin{defproblem}{diferencialny-pocet-45}
Pomocou Leibnitzovho vzorca nájdite určené derivácie nasledujúcich funkcií:
\begin{enumerate}
\item $y=\frac{1+x}{\sqrt{1-x}},y^{100}$;
\item $y=x\cdot\ln x,y^{V}$;
\item $y=x^2\sin 2x,y^{(50)}$;
\item $y=y=\frac{\cos 3x}{\sqrt[3]{1-3x}},y'''$;
\item $y=(x^2-2x)\cos 3x,y^{(101)}$;
\item $y=(x-\sin x)^2,y^{(16)}$;
\item $y=x\cdot\sin x\cdot\cos 2x,y^{(100)}$.
\end{enumerate}
\end{defproblem}

\begin{defproblem}{diferencialny-pocet-46}
AK funkcia $f$ má derivácie až do rádu $n$, tak platí
$$(f(ax+b))^{(n)}=a^nf^{(n)}(ax+b).$$ Dokážte!
\end{defproblem}

\begin{defproblem}{diferencialny-pocet-47}
Nájdite $y^{(n)}$, ak:
\begin{multicols}{2}
\begin{enumerate}
    \item $y=\ln (ax+b)$;
    \item $y=\frac{1}{x\cdot(1-x)}$;
    \item $y=\frac{1}{x^2-3x+2}$;
    \item $y=\frac{1}{\sqrt{1-2x}}$;
    \item $y=\sin^2 x$;
    \item $y=\sin3x \cdot \cos 5x$;
    \item $y=\sin^4 x+\cos^4 x$;
    \item $y=\sin^2 ax \cdot \cos bx$.
\end{enumerate}
\end{multicols}
\end{defproblem}

\begin{defproblem}{diferencialny-pocet-48}
Nájdite $y^{(n)}$, ak:
\begin{multicols}{2}
\begin{enumerate}
    \item $y=(x-1)\cdot 2^{x-1}$;
    \item $y=\frac{x}{\sqrt[3]{1+x}}$;
    \item $y=x^2\sin bx$;
    \item $y=(x^2+2x+2)\cdot e^{-x}$;
    \item $y=x\cdot\log_{2}(1-3x)$;
    \item $y=(3-2x)^2\cdot e^{2-3x}$.
\end{enumerate}
\end{multicols}
\end{defproblem}

\begin{defproblem}{diferencialny-pocet-49}
Nájdite $f^{(n)}(0)$, ak:
\begin{multicols}{2}
\begin{enumerate}
    \item $y=\frac{1}{(1-2x)(1+x)}$;
    \item $y=x^2e^{2x}$;
    \item $y=\frac{x}{\sqrt{1-x}}$;
    \item $y=\frac{x}{1-x^2}$;
    \item $y=\arcsin x$;
    \item $y=\arctan x$;
    \item $y=\frac{1}{(1-x^2)^2}$.
\end{enumerate}
\end{multicols}
\end{defproblem}

\begin{defproblem}{diferencialny-pocet-50}
Dokážte, že funkcia definovaná predpisom
$f(x) = \left\{ \begin{array}{r@{\quad}c}
   x^{2n \sin \frac{1}{x}},& $ak $ x \neq 0 \\
    0, &  $ak $ x=0 \\ \end{array} \right.$
    
    $n\in\mathbb{N}$ má v bode $0$ derivácie až do rádu $n$, ale nemá 
    
    $(n+1)$-vú deriváciu v tomto bode.
\end{defproblem}

\begin{defproblem}{diferencialny-pocet-51}
Nech funkcia $f$ je spojitá na intervale $(a,b)$, kde $a<b,a,b\in\mathbb{R^*}$, nech pre každé $x\in (a,b)$ existuje vlastná alebo nevlastná nenulová $f'(x)$. Potom $f$ je prostá na intervale $(a,b)$. Dokážte!
\end{defproblem}

\begin{defproblem}{diferencialny-pocet-52}
Nech funkcia $f$ je $n$-krát diferencovateľná na intervale $\langle a,b \rangle$ (to znamená, že existujú $f^{(1)},...,f^{(n)}$ definované na intervale $\langle a,b \rangle$) a má tam nulové hodnoty v $n+1$ bodoch. Potom exostuje také $o\in (a,b)$, že $f^{(n)}(c)=0$. Dokážte!
\end{defproblem}

\begin{defproblem}{diferencialny-pocet-53}
Ak sú všetky korene polynómu $P_n(x)=a_0x^n+...+a_n$ $(a_0,...,a_n\in\mathbb{R},a_0\neq 0)$ reálne, tak aj jeho derivácie $P'_n,...,P_n^{(n-1)}$ majú len reálne korene.

(Komplexné číslo $a$ sa nazýva $m$-násobný koreň polynómu $P_n$($n\geq m,n,m\in\mathbb{N}$), ak možno $P_n$ písať v tvare $P_n(x)=(x-a)^mQ_{n-m}(x)$, pričom číslo $a$ nie je koreňom polynómu $Q_{n-m}$. Základná veta algebry hovorí, že súčet násobností navzájom rôznych komplexných koreňov polynómu sa rovná jeho stupňu. Teda polynóm $P_n$ stupňa $n$ má všetky korene reálne práve vtedy, keď súčet násobností jeho navzájom rôznych reálnych koreňov je $n$.)
\end{defproblem}

\begin{defproblem}{diferencialny-pocet-54}
Ukážte, že rovnica $x^3-3x^2+6x-1=0$ má len jeden koreň, pričom tento koreň je prostý (prostými sa nazývajú jednonásobné korene polynómu).
\end{defproblem}

\begin{defproblem}{diferencialny-pocet-55}
Nech funkcia $f:(a,b)\rightarrow\mathbb{R}$, kde $a<b,a,b\in\mathbb{R^*}$, je spojitá, má vlastnú alebo nevlastnú deriváciu v každom bode $x\in (a,b)$ a nech $\lim_{x\rightarrow a}f(x)=\lim_{x\rightarrow a}f(x)$. Potom existuje $c\in (a,b)$, v ktorom $f'(c)=0$. Dokážte!
\end{defproblem}

\begin{defproblem}{diferencialny-pocet-56}
Zostrojte funkciu $f$ definovanú na intervale $\langle a,b \rangle$, pre ktorú $f(a)=f(b)$, pre každé $x\in (a,b)$ existuje vlastná $f'(x)$ a pre všetky $x\in (a,b)$ platí $f'(x)\neq 0$.
\end{defproblem}

\begin{defproblem}{diferencialny-pocet-57}
Funkcia $y=1-\sqrt[3]{x^2}$ nadobúda nulové hodnoty pre $a=-1,b=1$, ale napriek tomu nemá v žiadnom bode intervalu $(-1,1)$ nulovú deriváciu. Je to v rozpore s tvrdením Rolleho vety?
\end{defproblem}

\begin{defproblem}{diferencialny-pocet-58}
\begin{enumerate}
\item Nech funkcia $f$ je definovaná na intervale $I$, pričom v každom bode $x\in I$ platí $f'(x)=0$. Potom funkcia $f$ je konštantná na intervale $I$. Dokážte!
\item Uveďte príklad takej funkcie definovanej na otvorenej množine $M$, že $f'=0$ na $M$ a $f$ nie je konštantná na množine $M$!
\end{enumerate}
\end{defproblem}

\begin{defproblem}{diferencialny-pocet-59}
Ukážte, že funkcie $f(x)=\arctan \frac{1+x}{1-x}$ a $g(x)=\arctan x$ majú rovnaké derivácie v každom bode množiny $(-\infty,1)\cup (1,\infty)$. Na základe toho odvoďte vtťah medzi funkciami $f$ a $g$!
\end{defproblem}

\begin{defproblem}{diferencialny-pocet-60}
Dokážte rovnosti:
\begin{enumerate}
\item $\arctan x=\frac{\pi}{2}-\arctan x,x\in\mathbb{R}$;
\item $\arcsin x+\arccos x=\frac{\pi}{2},x\in\langle -1,1 \rangle$;
\item $2\arctan x+\arcsin \frac{2x}{1+x^2}=\pi sgn x,|x|\geq 1$;
\item $3\arccos x-\arccos (3x-4x^3)=\pi ,|x|\leq \frac{1}{2}$;
\item $\arcsin x=\arctan \frac{x}{\sqrt{1-x^2}},x\in (-1,1)$.
\end{enumerate}
\end{defproblem}

\begin{defproblem}{diferencialny-pocet-61}
Nech funkcia $f$ je definovaná na intervale $I$, nech pre každé $x\in I$ platí $f'(x)=k$ ($k$ je reálna konštanta). Potom $f(x)=kx+b$. Dokážte!
\end{defproblem}

\begin{defproblem}{diferencialny-pocet-62}
Dokážte, že funkcia $sgn x$ nie je deriváciou žiadnej funkcia!
\end{defproblem}

\begin{defproblem}{diferencialny-pocet-63}
Dokážte nasledujúce nerovnosti:
\begin{enumerate}
\item $|\sin x-\sin y|\leq |x-y|,x,y\in\mathbb{R}$;
\item $p\cdot y^{p-1}(x-y)\leq x^p-y^p \leq p\cdot x^{p-1}(x-y)$, ak $0<y<x,p>1$;
\item $|\arctan a-\arctan b|\leq |a-b|,a,b\in\mathbb{R}$;
\item $\frac{a-b}{a}<\ln \frac{a}{b}<\frac{a-b}{b}$, ak $0<b<a$.
\end{enumerate}
\end{defproblem}

\begin{defproblem}{diferencialny-pocet-64}
\begin{enumerate}
\item Nech funkcia $f$ má na ohraničenom alebo neohraničenom intervale $(a,b)$ ohraničenú deriváciu $f'$. Potom $f$ je rovnomerne spojitá na $(a,b)$. Dokážte!
\item Ukážte, že funkcia $f(x)=x\cdot \sin \frac{1}{x}$ je rovnomerne spojitá na intervale $(0,1)$, ale nemá tam ohraničenú deriváciu!
\end{enumerate}
\end{defproblem}

\begin{defproblem}{diferencialny-pocet-65}
Nech funkcia $f$, ktorá má v každom bode ohraničeného intervalu $(a,b)$ deriváciu, nie je ohraničená na $(a,b)$. Potom funkcia $f'$ nie je ohraničená na $(a,b)$. Dokážte!
\end{defproblem}

\begin{defproblem}{diferencialny-pocet-66}
Nech $f,F$ sú funkcie definované na otvorenom intervale $I \subset \mathbb{R},f$ je spojitá a pre každé $x,y\in I,x<y$, existuje $y\in (x,y)$ tak, že $F(x)-F(y)=(x-y)\cdot f(z)$. Potom na intervale $I$ platí $F'=f$. Dokážte!
\end{defproblem}

\begin{defproblem}{diferencialny-pocet-67}
Nech derivácia funkcie $f$ je spojitá na uzavretom intervale $\langle a,b \rangle$. Možno tvrdiť, že pre každý bod $c\in (a,b)$ existuje interval $\langle \alpha,\beta \rangle \subset (a,b)$ taký, že $f'(c)\cdot (\beta-\alpha)=f(\beta)-f(\alpha)$?
\end{defproblem}

\begin{defproblem}{diferencialny-pocet-68}
Nájdite chybu v nasledujúcom "dôkaze" Cauchyho vety: "Nech funkcie $f,g$ vyhovujú prvému, druhému a štvrtému predpokladu a nech $g'(x)\neq 0$ pre $x\in (a,b)$. Potom každá z funkcií $f,g$ vyhovuje predpokladom Lagrangeovej vety o strednej hodnote, a teda platí $f(b)-f(a)=f'(c)(a-b),g(b)-g(a)=g'(c)(b-a)$ pre niektoré $c\in (a,b)$. Ak vydelíme prvú z týchto rovností druhou, dostaneme 
$$\frac{f(b)-f(a)}{g(b)-g(a)}=\frac{f'(c)}{g'(c)}.$$"
\end{defproblem}

\begin{defproblem}{diferencialny-pocet-69}
Nech funkcia $f$ a jej derivácia sú definované na intervale $\langle 1,2 \rangle$. Potom existuje $c\in (1,2)$ tak, že $f(2)-f(1)=\frac{c^2}{2}f'(c)$. Dokážte!
\end{defproblem}

\begin{defproblem}{diferencialny-pocet-70}
Nech funkcia $f$ a jej derivácia $f'$ sú definované na intervle $\langle a,b \rangle$, pričom $a\cdot b>0$. Dokážte, že potom existuje $c\in (a,b)$ tak, že 
$$\frac{1}{a-b}
\begin{vmatrix}
a & b \\ 
f(a) & f(b)
\end{vmatrix}
=f(c)-cf'(c). $$
\end{defproblem}

\begin{defproblem}{diferencialny-pocet-71}
Zistite, na ktorých intervaloch sú nasledujúce funkcie monotónne:
\begin{multicols}{2}
\begin{enumerate}
    \item $y=3x-x^3$;
	\item $y=\frac{2x}{1+x^2};$
	\item $y=x+\sin x$;
	\item $y=x+|\sin 2x|$;
	\item $y=\cos \frac{\pi}{x}$;
	\item $y=\frac{x^2}{2^x}$;
	\item $y=x^2-\ln x^2$;
	\item $y= \left\{ \begin{array}{r@{\quad}c}
   x(\sqrt{\frac{3}{2}}+\sin \ln x),& $ak $ x > 0 \\
    0, &  $ak $ x=0 \\ \end{array} \right.$
\end{enumerate}
\end{multicols}
\end{defproblem}

\begin{defproblem}{diferencialny-pocet-72}
Nech funkcia $f: I \rightarrow\mathbb{R}$ Je rastúca na otvorenom intervale $I$ a diferencovateľná v každom bode $x\in I$. Potom $f'(x)\geq 0$ pre Každé $x\in I$. Dokážte!
\end{defproblem}

\begin{defproblem}{diferencialny-pocet-73}
Pre každý polynóm $P_n$ stupňa $n\geq 1$ možno nájsť číslo $a>0$ tak, že $P_n$ je rýdzomonotónny na intervale $(-\infty,-a)$ a rýdzomonotónny na intervale $(a,\infty)$. Dokážte!
\end{defproblem}

\begin{defproblem}{diferencialny-pocet-74}
Dokážte nasledujúce nerovnosti:
\begin{enumerate}
\item $e^x>1+x$ pre $x\neq 0$;
\item $x-\frac{x^3}{6}<\sin x<x$ pre $x>0$;
\item $\tan x>x$ pre $0<x<\frac{\pi}{2}$;
\item $\tan x>x+\frac{x^3}{3}$ pre $0<x<\frac{\pi}{2}$;
\item $x^{\alpha}-1>\alpha\cdot(x-1)$ pre $\alpha\geq 2,x>1$;
\item $\sqrt[n]{x}-\sqrt[n]{a}<\sqrt[n]{x-a}$ pre $n>1,x>a>0$;
\item $1+2\cdot\ln x\leq x^2$ pre $x>0$.
\end{enumerate}
\end{defproblem}

\begin{defproblem}{diferencialny-pocet-75}
Dokážte nerovnosť $\frac{x}{1+x}<\ln (1+x)<x$ pre $x>-1$ a na jej základ erovnosť $\lim_{n \rightarrow \infty}(\frac{1}{n}+\frac{1}{n+1}+...+\frac{1}{2n})=\ln 2$. 
\end{defproblem}

\begin{defproblem}{diferencialny-pocet-76}
Nech $M$ je konečná podmnožina intervalu $I=(a,b)$, kde $a<b,a,b\in\mathbb{R^*}$, nech funkcia $f$ je spojitá na $I$ a má kladnú deriváciu v každom bode $x\in I \setminus M$. Potom $f$ je rastúca na $I$. Dokážte!

(Všimnime si, že v predpokladoch tohto tvrdenia sa nehovorí nič o existencii derivácie v bodoch množiny $M$; teda funkcia $f'$ môže byť definovaná v niektorých (alebo aj vo všetkých) bodoch množiny $M$, ale pripúšťa sa aj možnosť, že definičným oborom funkcie $f'$ je len množina $I \setminus M$.)
\end{defproblem}

\begin{defproblem}{diferencialny-pocet-77}
Nech funkcie $f,g$ sú diferencovateľné na intervale $\langle a,\infty),f(a)>g(a)$ a Nech pre všetky $x\in (a,\infty)$ platí $f'(x)\geq g'(x)$. Potom pre všetky $x\in \langle a,\infty)$ platí $f(x)>g(x)$. Dokážte!
\end{defproblem}

\begin{defproblem}{diferencialny-pocet-78}
Musí byť derivácia monotónnej funkcie monotónna?
\end{defproblem}

\begin{defproblem}{diferencialny-pocet-79}
Zistite, na ktorých intervaloch sú nasledujúce funkcie rýdzo konvexné, resp. rýdzo konkávne:
\begin{multicols}{2}
\begin{enumerate}
    \item $y=3x^2-x^3$;
	\item $y=x+x^{\frac{5}{3}}$;
	\item $y=x+\sin x$;
	\item $y=\ln (1+x^2)$;
	\item $y=x\cdot\sin (\ln x)$;
	\item $y=\sqrt[3]{|x|}$.
\end{enumerate}
\end{multicols}
\end{defproblem}

\begin{defproblem}{diferencialny-pocet-80}
\begin{enumerate}
\item Pre ktoré hodnoty $p\in\mathbb{N},q\in\mathbb{N}$ je bod $x=0$ inflexným bodom funkcie $y=x^{\frac{p}{q}}$ (predpokladáme, že $p,q$ sú nesúdeliteľné čísla)?
\item Akým podmienkam musia vyhovovať koeficienty $a\neq 0,b,c$, aby krivka $y=ax^4+bx^3+cx^2+dx+e$ mala inflexné body?
\item Pre aké hodnoty parametra $a$ je krivka $y=x^4+ax^3+\frac{3}{2}x^2+1$ konvexná na $\mathbb{R}$?
\end{enumerate}
\end{defproblem}

\begin{defproblem}{diferencialny-pocet-81}
Nech funkcia $f$ je diferencovateľná v každom bode intervalu $(a,b)$, pričom $f'$ je na $(a,b)$ rastúca. Potom $f$ je rýdzo konvexná na $(a,b)$. Dokážte!
\end{defproblem}

\begin{defproblem}{diferencialny-pocet-82}
Dokážte nasledujúce nerovnosti:
\begin{enumerate}
\item $\frac{1}{2}(x^n+y^n)>(\frac{x+y}{2})^n$ pre $x>0,y>0,x\neq y,n>1$;
\item  $\frac{1}{2}(e^x+e^y)>e^{\frac{x+y}{2}}$ pre $x\neq y$;
\item $x\cdot\ln x+\ln y>(x+y)\ln \frac{x+y}{2}$ pre $x>0,y>0,x\neq y$;
\item $\frac{2}{\pi}x<\sin x$ pre $x\in (0,\frac{\pi}{2})$;
\item $\arctan x+\arctan y>2\arctan \frac{x+y}{2}$ pre $x\neq y$;
\item $x-1<\log_2 x$ pre $x\in (1,2)$.
\end{enumerate}
\end{defproblem}

\begin{defproblem}{diferencialny-pocet-83}
Ukážte, že neexistuje funkcia, ktorá je kladná na $\mathbb{R}$ a má v každom bode $x\in\mathbb{R}$ zápornú druhú deriváciu.
\end{defproblem}

\begin{defproblem}{diferencialny-pocet-84}
\begin{enumerate}
\item Nech $f$ je rýdzo konvexná a diferencovateľná na intervale $I$, nech $a\in I$. Potom na množine $I\setminus \{a\}$ leží graf funkcie $f$ nad svojou dotyčnicou v bode $a$. Dokážte!
\item V inflexnom bode prechádza graf funkcie z jednej strany dotyčnice na druhú, t.j. ak $a$ je inflexný bod funkcie $f$, tak existuje $\varepsilon >0$ tak, že na jednej z množín $(a-\varepsilon,a),(a,a+\varepsilon)$ leží graf funkcie $f$ nad svojou dotyčnicou v bode $(a,f(a))$, na druhej z týchto množín leží pod ňou. Dokážte!
\end{enumerate}
\end{defproblem}

\begin{defproblem}{diferencialny-pocet-85}
Nech
$f(x)= \left\{ \begin{array}{r@{\quad}c}
   x^3(2+\cos \frac{1}{x}),& $ak $ x \neq 0 \\
    0, &  $ak $ x=0 \\ \end{array} \right.$
    Ukážte, že
    \begin{enumerate}
    \item $f$ má deriváciu v bode $0$;
    \item graf funkcie $f$ prechádza z jednej strany v bode $(0,0)$ na jej druhú stranu;
    \item bod $0$ napriek tomu nie je inflexný bod funkcie $f$.
    \end{enumerate}
\end{defproblem}

\begin{defproblem}{diferencialny-pocet-86}

\end{defproblem}

\begin{defproblem}{diferencialny-pocet-87}

\end{defproblem}

\begin{defproblem}{diferencialny-pocet-88}

\end{defproblem}

\begin{defproblem}{diferencialny-pocet-89}

\end{defproblem}

\begin{defproblem}{diferencialny-pocet-90}

\end{defproblem}

\begin{defproblem}{diferencialny-pocet-91}

\end{defproblem}

\begin{defproblem}{diferencialny-pocet-92}

\end{defproblem}

\begin{defproblem}{diferencialny-pocet-93}

\end{defproblem}

\begin{defproblem}{diferencialny-pocet-94}

\end{defproblem}

\begin{defproblem}{diferencialny-pocet-95}

\end{defproblem}

\begin{defproblem}{diferencialny-pocet-96}

\end{defproblem}

\begin{defproblem}{diferencialny-pocet-97}

\end{defproblem}

\begin{defproblem}{diferencialny-pocet-98}

\end{defproblem}

\begin{defproblem}{diferencialny-pocet-99}

\end{defproblem}

\begin{defproblem}{diferencialny-pocet-100}

\end{defproblem}

\begin{defproblem}{diferencialny-pocet-101}

\end{defproblem}

\begin{defproblem}{diferencialny-pocet-102}

\end{defproblem}

\begin{defproblem}{diferencialny-pocet-103}

\end{defproblem}

\begin{defproblem}{diferencialny-pocet-104}

\end{defproblem}

\begin{defproblem}{diferencialny-pocet-105}

\end{defproblem}

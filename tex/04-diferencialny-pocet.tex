\chapter{Diferenciálny počet funkcií jednej premennej}%\label{chapter:gramatiky}

\section{Definícia derivácie}
Hovoríme, že funkcia $f$ (definovaná v okolí bodu $a$ \footnote{t.j. pre niektoré $\varepsilon>0$ platí $(a-\varepsilon,a+\varepsilon) \subset D(f)$; pojmy derivácie a nevlastnej derivácie a nevlastnej derivácie v bode $a$ by bolo možné zaviesť aj za slabšieho predpokladu "$a \in D(f)$ je hromadný bod množiny $D(f)$" (odstránili by sa tým aj niektoré ťažkosti so zavedením pojmu derivácie ako funkcie), definícia predpokladajúca, že $a$ je vnútorný bod množiny $D(f)$, je však v literatúre najčastejšia}) má deriváciu v bode $a,(a \in \mathbb{R})$, ak existuje konečná $\lim_{x \rightarrow a}\frac{f(x)-f(a)}{x-a}$. Ak existuje nevlastná $\lim_{x \rightarrow a}\frac{f(x)-f(a)}{x-a}$, hovoríme, že funkcia $f$ má nevlastnú (alebo nekonečnú) deriváciu v bode $a$. Hodnotu $\lim_{x \rightarrow a}\frac{f(x)-f(a)}{x-a}$ v obidvoch týchto prípadoch označujeme $f'(a)$. Ak nás nezaujíma, či je $\lim_{x \rightarrow a}\frac{f(x)-f(a)}{x-a}$ konečná alebo nekonečná. používame spoločný názov vlastná alebo nevlastná derivácia v bode $a$.
Pojmy derivácie sprava a nevlastnej derivácie sprava, resp. derivácie zľava a nevlastnej derivácie zľava dostaneme, ak v predchádzajúcich definíciách $\lim_{x \rightarrow a}\frac{f(x)-f(a)}{x-a}$ nahradíme limitou $\lim_{x \rightarrow a+}\frac{f(x)-f(a)}{x-a}$, resp. $\lim_{x \rightarrow a-}\frac{f(x)-f(a)}{x-a}$ (v takom prípade stačí predpokladať, že definičný obor funkcie $f$ obsahuje interval $\langle a,a+\varepsilon), resp. (a-\varepsilon,a\rangle$ pre niektoré $\varepsilon>0$). Hodnoty $\lim_{x \rightarrow a+}\frac{f(x)-f(a)}{x-a}$, resp. $\lim_{x \rightarrow a-}\frac{f(x)-f(a)}{x-a}$ označujeme $f'_+(a)$,resp. $f'_-(a)$.

\begin{veta}
Funkcia $f$ (definovaná v okolí bodu $a \in \mathbb{R}$) má vlastnú alebo nevlastnú deriváciu $f'(a)$ práve vtedy, keď existujú $f'_+(a),f'_-(a)$ a platí $f'_+(a)=f'_-(a)$. Hodnota $f'(a)$ sa pritom rovná spoločnej hodnote $f'_+(a)$ a $f'_-(a)$.
\end{veta}

Pojem derivácie ako funkcie sa vo všeobecnosti definuje nasledovne: Nech $M$ je množina všetkých bodov definičného oboru $D(f)$, v ktorých má funkcia $f$ deriváciu. Funkcia $f':M \rightarrow \mathbb{R}$, ktorá každému bodu $a$  $M$ priradí hodnotu $f'(a)$ derivácie funkcie $f$ v bode $a$, sa nazýva derivácia funkcie $f$. V špeciálnych prípadoch, ktoré ilustruje nasledujúca poznámka, sa niekedy pojem derivácie ako funkcie chápe širšie: ak definičným oborom funkcie $f$ je niektorý z intervalov $\langle a,b \rangle, \langle a,b),(a,b \rangle$, pričom v jeho koncovom bode existuje jednostranná derivácia (v prvom prípade prichádzajú do úvahy body $a,b$, v druhom bod $a$, v trečom bod $b$), tak funkciu $f'$ považujeme za definovanú aj v tomto bode, jej funkčnou hodnotou je hodnota príslušnej jednostrannej derivácie.

\begin{enumerate}[resume]
	\item \useproblem[diferencialny-pocet]{diferencialny-pocet-1}
	\item \useproblem[diferencialny-pocet]{diferencialny-pocet-2}
	\item \useproblem[diferencialny-pocet]{diferencialny-pocet-3}
	\item \useproblem[diferencialny-pocet]{diferencialny-pocet-4}
	\item \useproblem[diferencialny-pocet]{diferencialny-pocet-5}
	\item \useproblem[diferencialny-pocet]{diferencialny-pocet-6}
	\item \useproblem[diferencialny-pocet]{diferencialny-pocet-7}
\end{enumerate}

\begin{veta}
Ak funkcie $f,g$ majú derivácie v bode $a$, tak aj funkcie $c \cdot f$ ($c$ je reálna konštanta), $f+g,f-g,f\cdot g$ majú v bode $a$ derivácie a platí:
\begin{enumerate}
\item $(c\cdot f)'(a)=c\cdot f'(a)$,
\item $(f+g)'(a)=f'(a)+g'(a)$,
\item $(f-g)'(a)=f'(a)-g'(a)$,
\item $(f\cdot g)'(a)=f'(a)\cdot g'(a)$.
\end{enumerate}
Ak naviac $g(a)\neq 0$ majú v bode $a$ deriváciu aj funkcie $\frac{1}{g},\frac{f}{g}$ a platí
\begin{enumerate}
\item $(\frac{1}{g})'(a)=-\frac{g'(a)}{g^2(a)}$,
\item $(\frac{f}{g})'(a)=\frac{f'(a)\cdot g(a)-f(a)\cdot g'(a)}{g^2 (a)}$.
\end{enumerate}
\end{veta}

\begin{veta}
Ak funkcia $f$ má deriváciu v bode $a$, funkcia $g$ v bode $f(a)$ a zložená funkcia $h=g \circ f$ je definovaná v okolí bodu $a$, tak $h$ má v bode $a$ deriváciu a platí $$h'(a)=f'(a)\cdot g'(f(a)).$$
(Analogické vety možno dokázať aj pre jednostranné derivácie.)
\end{veta}

V nasledujúcej tabuľke sú derivácie základných elementárnych funkcií ($c$ je reálna konštanta):
\begin{itemize}
\item $c'=0$,
\item $(x^n)'=n\cdot x^{n-1},(n \in \mathbb{R} \setminus \{0\})$,
\item $(\sin x)'=\cos x$,
\item $(\cos x)'=-\sin x$,
\item $(\tan x)'=\frac{1}{\cos^2 x}$,
\item $(a^x)'=a^x \cdot \ln a$, špeciálne $(e^x)'=e^x$,
\item $(\log_{a} x)'=\frac{1}{x\cdot \ln a},x>0$, špeciálne $(\ln x)'=\frac{1}{x},x>0$,
\item $(\arcsin x)'=\frac{1}{\sqrt{1-x^2}}$,
\item $(\arccos x)'=-\frac{1}{\sqrt{1-x^2}}$,
\item $(\arctan x)'=\frac{1}{1+x^2}$,
\item $(arcctan x)'=-\frac{1}{1+x^2}$.
\end{itemize} 

Pre deriváciu hyperbolických funkcií platia nasledujúce vzorce
\begin{multicols}{2}
\begin{itemize}
   \item $(sh x)'=ch x$,
   \item $(ch x)'=sh x$,
   \item $(th x)'=\frac{1}{ch^2 x}$,
   \item $(cth x)'=-\frac{1}{sh^2 x}$.
\end{itemize}
\end{multicols}

Ak pri hľadaní derivácie funkcie využívame len znalosť derivácií základných elementárnych funkcií a vety o derivácii súčtu, rozdielu, súčinu, podielu a zloženej funkcie, nazýva sa taký postup tabuľkovým derivovaním.

\begin{enumerate}[resume]
	\item \useproblem[diferencialny-pocet]{diferencialny-pocet-8}
	\item \useproblem[diferencialny-pocet]{diferencialny-pocet-9}
	\item \useproblem[diferencialny-pocet]{diferencialny-pocet-10}
	\item \useproblem[diferencialny-pocet]{diferencialny-pocet-11}
	\item \useproblem[diferencialny-pocet]{diferencialny-pocet-12}
	\item \useproblem[diferencialny-pocet]{diferencialny-pocet-13}
	\item \useproblem[diferencialny-pocet]{diferencialny-pocet-14}
\end{enumerate}

Nájdite derivácie funkcií:

\begin{enumerate}[resume]
	\item \useproblem[diferencialny-pocet]{diferencialny-pocet-15}
	\item \useproblem[diferencialny-pocet]{diferencialny-pocet-16}
	\item \useproblem[diferencialny-pocet]{diferencialny-pocet-17}
\end{enumerate}

\textit{Riešenie (a):}
Výpočet sa zjednoduší nasledujúcou úvahou: ak funkcia $f$ má v bode $a$ deriváciu a $f(a)\neq 0$, tak $(\ln |f|)'(a)=\frac{1}{f(a)}\cdot f'(a)$; odtiaľ možno vyjadriť 
$$f'(a)=f(a)\cdot (\ln |f|)'(a).$$
V našom prípade $y=\frac{1+x^2}{\sqrt[3]{x^4}\cdot\sin^7 x}\cdot (\ln |y|)'=(\ln |1+x^2|-\frac{4}{3}\ln |x|-7\cdot\ln |\sin x|)'=\frac{2x}{1+x^2}-\frac{4}{3x}-7\cdot \cot x=\frac{2x^2-4}{3x\cdot (1+x^2)}-7\cot x$. Teda $y'=\frac{1+x^2}{\sqrt[3]{x^4}\cdot \sin^7 x}\cdot (\frac{2x^2-4}{3x\cdot (1+x^2)}-7\cdot\cot x)$.

\textit{Riešenie (d):}
Funkcie tvaru $f^g$ (o funkcii $f$ predpokladáme, že je nezáporná) možno derivovať na základe úvahy z riešenia príkladu $295.(a)$ alebo (čo je vlastne to isté) prepísať ich predpis do podoby $e^{g\cdot\ln f}$ a takto zapísanú funkciu potom derivovať na základe viet o derivácii súčinu a zloženej funkcie.

V našom prípade $(x^x)'=(e^{x\cdot\ln x})'=e^{x\cdot\ln x}\cdot(x\cdot\ln x)'=x^x\cdot(\ln x +1)$.
\begin{enumerate}[resume]
	\item \useproblem[diferencialny-pocet]{diferencialny-pocet-18}
	\item \useproblem[diferencialny-pocet]{diferencialny-pocet-19}
	\item \useproblem[diferencialny-pocet]{diferencialny-pocet-20}
	\item \useproblem[diferencialny-pocet]{diferencialny-pocet-21}
	\item \useproblem[diferencialny-pocet]{diferencialny-pocet-22}
	\item \useproblem[diferencialny-pocet]{diferencialny-pocet-23}
	\item \useproblem[diferencialny-pocet]{diferencialny-pocet-24}
	\item \useproblem[diferencialny-pocet]{diferencialny-pocet-25}
	\item \useproblem[diferencialny-pocet]{diferencialny-pocet-26}
	\item \useproblem[diferencialny-pocet]{diferencialny-pocet-27}
	\item \useproblem[diferencialny-pocet]{diferencialny-pocet-28}
\end{enumerate}

\begin{veta}
Ak funkcia $f$ má v bode $a$ deriváciu, tak $f$ je v tomto bode spojitá.
\end{veta}

\begin{enumerate}[resume]
	\item \useproblem[diferencialny-pocet]{diferencialny-pocet-29}
	\item \useproblem[diferencialny-pocet]{diferencialny-pocet-30}
	\item \useproblem[diferencialny-pocet]{diferencialny-pocet-31}
\end{enumerate}

\section{Derivácia inverznej funkcie}
\begin{veta}
Nech funkcia $f:\mathbb{I}\rightarrow\mathbb{R}$, rýdzomonotónna na intervale $I$, má deriváciu v bode $a$. Potom inverzná funkcia $f^{-1}$ má vlastnú alebo nevlastnú deriváciu v bode $f(a)$, pričom platí
\begin{enumerate}
\item ak $f'(a)\neq 0$, tak $(f^{-1})'(f(a))=\frac{1}{f'(a)}$; 
\item ak $f'(a)=0$ a $f$ je rastúca, tak $(f^{-1})'(f(a))=+\infty$;
\item ak $f'(a)=0$ a $f$ je klesajúca, tak $(f^{-1})'(f(a))=-\infty$.
\end{enumerate}
\end{veta}

\begin{enumerate}[resume]
	\item \useproblem[diferencialny-pocet]{diferencialny-pocet-32}
\end{enumerate}

\textit{Riešenie (a):}
Pretože $b=f(0)$ a $f'(x)=1+x^4$, je podľa vety $5$ $(f^{-1})'(b)=\frac{1}{f'(0)}=1$. (Ako sa presvedčíme o rýdzomonotónnej funkcie $f$, ukážeme neskôr-pozri vetu $11.$)

\begin{enumerate}[resume]
	\item \useproblem[diferencialny-pocet]{diferencialny-pocet-33}
	\item \useproblem[diferencialny-pocet]{diferencialny-pocet-34}
	\item \useproblem[diferencialny-pocet]{diferencialny-pocet-35}
\end{enumerate}
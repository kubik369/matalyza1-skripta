\chapter{Diferenciálny počet funkcií jednej premennej}%\label{chapter:gramatiky}

\section{Definícia derivácie}
Hovoríme, že funkcia $f$ (definovaná v okolí bodu $a$ \footnote{t.j. pre niektoré $\varepsilon>0$ platí $(a-\varepsilon,a+\varepsilon) \subset D(f)$; pojmy derivácie a nevlastnej derivácie a nevlastnej derivácie v bode $a$ by bolo možné zaviesť aj za slabšieho predpokladu "$a \in D(f)$ je hromadný bod množiny $D(f)$" (odstránili by sa tým aj niektoré ťažkosti so zavedením pojmu derivácie ako funkcie), definícia predpokladajúca, že $a$ je vnútorný bod množiny $D(f)$, je však v literatúre najčastejšia}) má deriváciu v bode $a,(a \in \mathbb{R})$, ak existuje konečná $\lim_{x \rightarrow a}\frac{f(x)-f(a)}{x-a}$. Ak existuje nevlastná $\lim_{x \rightarrow a}\frac{f(x)-f(a)}{x-a}$, hovoríme, že funkcia $f$ má nevlastnú (alebo nekonečnú) deriváciu v bode $a$. Hodnotu $\lim_{x \rightarrow a}\frac{f(x)-f(a)}{x-a}$ v obidvoch týchto prípadoch označujeme $f'(a)$. Ak nás nezaujíma, či je $\lim_{x \rightarrow a}\frac{f(x)-f(a)}{x-a}$ konečná alebo nekonečná. používame spoločný názov vlastná alebo nevlastná derivácia v bode $a$.
Pojmy derivácie sprava a nevlastnej derivácie sprava, resp. derivácie zľava a nevlastnej derivácie zľava dostaneme, ak v predchádzajúcich definíciách $\lim_{x \rightarrow a}\frac{f(x)-f(a)}{x-a}$ nahradíme limitou $\lim_{x \rightarrow a+}\frac{f(x)-f(a)}{x-a}$, resp. $\lim_{x \rightarrow a-}\frac{f(x)-f(a)}{x-a}$ (v takom prípade stačí predpokladať, že definičný obor funkcie $f$ obsahuje interval $\langle a,a+\varepsilon), resp. (a-\varepsilon,a\rangle$ pre niektoré $\varepsilon>0$). Hodnoty $\lim_{x \rightarrow a+}\frac{f(x)-f(a)}{x-a}$, resp. $\lim_{x \rightarrow a-}\frac{f(x)-f(a)}{x-a}$ označujeme $f'_+(a)$,resp. $f'_-(a)$.

\begin{veta}
Funkcia $f$ (definovaná v okolí bodu $a \in \mathbb{R}$) má vlastnú alebo nevlastnú deriváciu $f'(a)$ práve vtedy, keď existujú $f'_+(a),f'_-(a)$ a platí $f'_+(a)=f'_-(a)$. Hodnota $f'(a)$ sa pritom rovná spoločnej hodnote $f'_+(a)$ a $f'_-(a)$.
\end{veta}

Pojem derivácie ako funkcie sa vo všeobecnosti definuje nasledovne: Nech $M$ je množina všetkých bodov definičného oboru $D(f)$, v ktorých má funkcia $f$ deriváciu. Funkcia $f':M \rightarrow \mathbb{R}$, ktorá každému bodu $a$  $M$ priradí hodnotu $f'(a)$ derivácie funkcie $f$ v bode $a$, sa nazýva derivácia funkcie $f$. V špeciálnych prípadoch, ktoré ilustruje nasledujúca poznámka, sa niekedy pojem derivácie ako funkcie chápe širšie: ak definičným oborom funkcie $f$ je niektorý z intervalov $\langle a,b \rangle, \langle a,b),(a,b \rangle$, pričom v jeho koncovom bode existuje jednostranná derivácia (v prvom prípade prichádzajú do úvahy body $a,b$, v druhom bod $a$, v trečom bod $b$), tak funkciu $f'$ považujeme za definovanú aj v tomto bode, jej funkčnou hodnotou je hodnota príslušnej jednostrannej derivácie.

\begin{enumerate}[resume]
	\item \useproblem[diferencialny-pocet]{diferencialny-pocet-279}
	\item \useproblem[diferencialny-pocet]{diferencialny-pocet-280}
	\item \useproblem[diferencialny-pocet]{diferencialny-pocet-281}
	\item \useproblem[diferencialny-pocet]{diferencialny-pocet-282}
	\item \useproblem[diferencialny-pocet]{diferencialny-pocet-283}
	\item \useproblem[diferencialny-pocet]{diferencialny-pocet-284}
	\item \useproblem[diferencialny-pocet]{diferencialny-pocet-285}
\end{enumerate}

\begin{veta}
Ak funkcie $f,g$ majú derivácie v bode $a$, tak aj funkcie $c \cdot f$ ($c$ je reálna konštanta), $f+g,f-g,f\cdot g$ majú v bode $a$ derivácie a platí:
\begin{enumerate}
\item $(c\cdot f)'(a)=c\cdot f'(a)$,
\item $(f+g)'(a)=f'(a)+g'(a)$,
\item $(f-g)'(a)=f'(a)-g'(a)$,
\item $(f\cdot g)'(a)=f'(a)\cdot g'(a)$.
\end{enumerate}
Ak naviac $g(a)\neq 0$ majú v bode $a$ deriváciu aj funkcie $\frac{1}{g},\frac{f}{g}$ a platí
\begin{enumerate}
\item $(\frac{1}{g})'(a)=-\frac{g'(a)}{g^2(a)}$,
\item $(\frac{f}{g})'(a)=\frac{f'(a)\cdot g(a)-f(a)\cdot g'(a)}{g^2 (a)}$.
\end{enumerate}
\end{veta}

\begin{veta}
Ak funkcia $f$ má deriváciu v bode $a$, funkcia $g$ v bode $f(a)$ a zložená funkcia $h=g \circ f$ je definovaná v okolí bodu $a$, tak $h$ má v bode $a$ deriváciu a platí $$h'(a)=f'(a)\cdot g'(f(a)).$$
(Analogické vety možno dokázať aj pre jednostranné derivácie.)
\end{veta}

V nasledujúcej tabuľke sú derivácie základných elementárnych funkcií ($c$ je reálna konštanta):
\begin{itemize}
\item $c'=0$,
\item $(x^n)'=n\cdot x^{n-1},(n \in \mathbb{R} \setminus \{0\})$,
\item $(\sin x)'=\cos x$,
\item $(\cos x)'=-\sin x$,
\item $(\tan x)'=\frac{1}{\cos^2 x}$,
\item $(a^x)'=a^x \cdot \ln a$, špeciálne $(e^x)'=e^x$,
\item $(\log_{a} x)'=\frac{1}{x\cdot \ln a},x>0$, špeciálne $(\ln x)'=\frac{1}{x},x>0$,
\item $(\arcsin x)'=\frac{1}{\sqrt{1-x^2}}$,
\item $(\arccos x)'=-\frac{1}{\sqrt{1-x^2}}$,
\item $(\arctan x)'=\frac{1}{1+x^2}$,
\item $(arcctan x)'=-\frac{1}{1+x^2}$.
\end{itemize}

Pre deriváciu hyperbolických funkcií platia nasledujúce vzorce
\begin{multicols}{2}
\begin{itemize}
   \item $(sh x)'=ch x$,
   \item $(ch x)'=sh x$,
   \item $(th x)'=\frac{1}{ch^2 x}$,
   \item $(cth x)'=-\frac{1}{sh^2 x}$.
\end{itemize}
\end{multicols}

Ak pri hľadaní derivácie funkcie využívame len znalosť derivácií základných elementárnych funkcií a vety o derivácii súčtu, rozdielu, súčinu, podielu a zloženej funkcie, nazýva sa taký postup tabuľkovým derivovaním.

\begin{enumerate}[resume]
	\item \useproblem[diferencialny-pocet]{diferencialny-pocet-286}
	\item \useproblem[diferencialny-pocet]{diferencialny-pocet-287}
	\item \useproblem[diferencialny-pocet]{diferencialny-pocet-288}
	\item \useproblem[diferencialny-pocet]{diferencialny-pocet-289}
	\item \useproblem[diferencialny-pocet]{diferencialny-pocet-290}
	\item \useproblem[diferencialny-pocet]{diferencialny-pocet-291}
	\item \useproblem[diferencialny-pocet]{diferencialny-pocet-292}
\end{enumerate}

Nájdite derivácie funkcií:

\begin{enumerate}[resume]
	\item \useproblem[diferencialny-pocet]{diferencialny-pocet-293}
	\item \useproblem[diferencialny-pocet]{diferencialny-pocet-294}
	\item \useproblem[diferencialny-pocet]{diferencialny-pocet-295}
\end{enumerate}

\textit{Riešenie (a):}
Výpočet sa zjednoduší nasledujúcou úvahou: ak funkcia $f$ má v bode $a$ deriváciu a $f(a)\neq 0$, tak $(\ln |f|)'(a)=\frac{1}{f(a)}\cdot f'(a)$; odtiaľ možno vyjadriť
$$f'(a)=f(a)\cdot (\ln |f|)'(a).$$
V našom prípade $y=\frac{1+x^2}{\sqrt[3]{x^4}\cdot\sin^7 x}\cdot (\ln |y|)'=(\ln |1+x^2|-\frac{4}{3}\ln |x|-7\cdot\ln |\sin x|)'=\frac{2x}{1+x^2}-\frac{4}{3x}-7\cdot \cot x=\frac{2x^2-4}{3x\cdot (1+x^2)}-7\cot x$. Teda $y'=\frac{1+x^2}{\sqrt[3]{x^4}\cdot \sin^7 x}\cdot (\frac{2x^2-4}{3x\cdot (1+x^2)}-7\cdot\cot x)$.

\textit{Riešenie (d):}
Funkcie tvaru $f^g$ (o funkcii $f$ predpokladáme, že je nezáporná) možno derivovať na základe úvahy z riešenia príkladu $295.(a)$ alebo (čo je vlastne to isté) prepísať ich predpis do podoby $e^{g\cdot\ln f}$ a takto zapísanú funkciu potom derivovať na základe viet o derivácii súčinu a zloženej funkcie.

V našom prípade $(x^x)'=(e^{x\cdot\ln x})'=e^{x\cdot\ln x}\cdot(x\cdot\ln x)'=x^x\cdot(\ln x +1)$.
\begin{enumerate}[resume]
	\item \useproblem[diferencialny-pocet]{diferencialny-pocet-296}
	\item \useproblem[diferencialny-pocet]{diferencialny-pocet-297}
	\item \useproblem[diferencialny-pocet]{diferencialny-pocet-298}
	\item \useproblem[diferencialny-pocet]{diferencialny-pocet-299}
	\item \useproblem[diferencialny-pocet]{diferencialny-pocet-300}
	\item \useproblem[diferencialny-pocet]{diferencialny-pocet-301}
	\item \useproblem[diferencialny-pocet]{diferencialny-pocet-302}
	\item \useproblem[diferencialny-pocet]{diferencialny-pocet-303}
	\item \useproblem[diferencialny-pocet]{diferencialny-pocet-304}
	\item \useproblem[diferencialny-pocet]{diferencialny-pocet-305}
	\item \useproblem[diferencialny-pocet]{diferencialny-pocet-306}
\end{enumerate}

\begin{veta}
Ak funkcia $f$ má v bode $a$ deriváciu, tak $f$ je v tomto bode spojitá.
\end{veta}

\begin{enumerate}[resume]
	\item \useproblem[diferencialny-pocet]{diferencialny-pocet-307}
	\item \useproblem[diferencialny-pocet]{diferencialny-pocet-308}
	\item \useproblem[diferencialny-pocet]{diferencialny-pocet-309}
\end{enumerate}

\section{Derivácia inverznej funkcie}
\begin{veta}
Nech funkcia $f:\mathbb{I}\rightarrow\mathbb{R}$, rýdzomonotónna na intervale $I$, má deriváciu v bode $a$. Potom inverzná funkcia $f^{-1}$ má vlastnú alebo nevlastnú deriváciu v bode $f(a)$, pričom platí
\begin{enumerate}
\item ak $f'(a)\neq 0$, tak $(f^{-1})'(f(a))=\frac{1}{f'(a)}$;
\item ak $f'(a)=0$ a $f$ je rastúca, tak $(f^{-1})'(f(a))=+\infty$;
\item ak $f'(a)=0$ a $f$ je klesajúca, tak $(f^{-1})'(f(a))=-\infty$.
\end{enumerate}
\end{veta}

\begin{enumerate}[resume]
	\item \useproblem[diferencialny-pocet]{diferencialny-pocet-310}
\end{enumerate}

\textit{Riešenie (a):}
Pretože $b=f(0)$ a $f'(x)=1+x^4$, je podľa vety $5$ $(f^{-1})'(b)=\frac{1}{f'(0)}=1$. (Ako sa presvedčíme o rýdzomonotónnej funkcie $f$, ukážeme neskôr-pozri vetu $11.$)

\begin{enumerate}[resume]
	\item \useproblem[diferencialny-pocet]{diferencialny-pocet-311}
	\item \useproblem[diferencialny-pocet]{diferencialny-pocet-312}
	\item \useproblem[diferencialny-pocet]{diferencialny-pocet-313}
\end{enumerate}

\section{Diferenciál}
Hovoríme, že funkcia $f$ (definovaná v okolí bodu $a$ \footnote{rovnako ako deriváciu v bode $a$ možno aj diferenciál v bode $a$ definovať za slabšieho predpokladu "$a\in D(f)$ je hromadný bod množiny $D(f)$"}) má v bode $a$ diferenciál (je diferencovateľná v bode $a$), ak existuje reálna konštanta $A$ taká, že pre funkciu $\omega$, definovanú vzťahom
$$f(x)=f(a)+A(x-a)+\omega(x)$$ platí $\lim_{x \rightarrow a}\frac{\omega(x)}{x-a}=0$.
Funkcia definovaná predpisom $y=A(x-a)$ sa v takom prípade označuje $df(a)$ a nazýva sa diferenciál funkcie $f$ v bode $a$. Funkcia $df(a)$ sa zvyčajnezapisuje v tvare $df(a)=A$ $dx(a)$ \footnote{písmeno $a$ sa v zápisoch často vynecháva,preto sa možno stretnúť aj so zápisom $df=A$ $dx$}, kde symbol $dx(a)$ (označujúci diferenciál funkcie $g(x)=x$ v bode $a$,t.j. funkciu danú predpisom $y=x-a$) sa nazýva diferenciál nezávislej premennej.
\begin{veta}
Funkcia $f$ je diferencovateľná v bode $a$ práve vtedy, keď $f$ má v bode $a$ deriváciu; pritom platí $A=f'(a)$, kde $A$ je konštanta z definície diferenciálu funkcie $f$ v bode $a$.
\end{veta}

Graf funkcie $y=f(a)+f'(a)(x-a)$ je dotyčnicou v bodu $(a,f(a))$ ku grafu funkcir $f$. Ak je funkcia $f$ spojitá v bode $a$, pričom $f'(a)$ je nevlastná, je dotyčnicou v bode $(a,f(a))$ ku grafu funkcie $f$ priamka $x=a$.

\begin{enumerate}[resume]
	\item \useproblem[diferencialny-pocet]{diferencialny-pocet-314}
	\item \useproblem[diferencialny-pocet]{diferencialny-pocet-315}
	\item \useproblem[diferencialny-pocet]{diferencialny-pocet-316}
	\item \useproblem[diferencialny-pocet]{diferencialny-pocet-317}
	\item \useproblem[diferencialny-pocet]{diferencialny-pocet-318}
	\item \useproblem[diferencialny-pocet]{diferencialny-pocet-319}
\end{enumerate}

\textit{Riešenie (e):}
Funkcia daná predpisom $f(x)=\ln (\tan \frac{x}{2})$ má v bode $\frac{\pi}{2}$ deriváciu $f'(\frac{\pi}{2})=1$, preto je podľa vety $6$ diferencovateľná a možno ju teda písať v tvare $f(x)=f(\frac{\pi}{2})+f'(\frac{\pi}{2})\cdot (x-\frac{\pi}{2})+\omega(x)=(x-\frac{\pi}{2})+\omega(x)$, kde $\lim_{x \rightarrow \frac{\pi}{2}}\frac{\omega(x)}{x-\frac{\pi}{2}}=0$. Zanedbaním funkcie $\omega(x)$ dostaneme približný vzorec $\ln (\tan \frac{x}{2})\approx (x-\frac{\pi}{2})$ pre $x$ blízke číslu $\frac{\pi}{2}$. (Geometricky to znamená, že hodnoty funkcie $f$ nahrádzame funkčnými hodnotami jej dotyčnice v bode $(\frac{\pi}{2},0)$, $\omega(x)$ vyjadruje chybu, ktorej sa pri takomto nahradení dopúšťame.)

\begin{enumerate}[resume]
	\item \useproblem[diferencialny-pocet]{diferencialny-pocet-320}
\end{enumerate}

\section{Derivácie vyšších rádov}
Derivácie vyšších rádov definujeme rekurentne: Nech funkcia $f$ je definovaná v okolí bodu $a\in\mathbb{R}$; označme $f^{(0)}:=f$. Hovoríme,že funkcia $f$ má n-tú deriváciu v bode $a$, reps. nevlastnú n-tú deriváciu v bode $a$, ak existuje derivácia, resp. nevlastná derivácia funkcie $f^{(n-1)}$ v bode $a$. n-tú deriváciu v bode $a$ aj nevlastnú n-tú deriváciu v bode $a$   označujeme $f^{(n)}(a)$. (Analogicky sa definujú vlastné a nevlastné jednostranné n-té derivácie v bode $a$.) Derivácia funkcie $f^{(n-1)}$ sa nazýva n-tá derivácia funkcie $f$ a označuje sa $f^{(n)}$. Okrem označení $f^{(1)},f^{(2)},f^{(3)},f^{(4)},f^{(5)},...$ sa používajú aj označenia $f',f'',f''',f''''$ (alebo $f^{VI}$),$f^{V},...$ .
Platia nasledujúce vzťahy:
\begin{multicols}{2}
\begin{enumerate}
    \item $(x^m)^{(n)}=\left\{ \begin{array}{r@{\quad}c}
    \frac{m!x^{m-n}}{(m-n)!},& $ak $ n\leq m \\
    0, &  $ak $ n>m \\ \end{array} \right.
    $,$(m\in\mathbb{N})$;
    \item $(\sin x)^{(n)}=\sin (x+\frac{n\pi}{2})$;
    \item $(\cos x)^{(n)}=\cos (x+\frac{n\pi}{2})$;
    \item $(a^x)^{(n)}=a^x\ln^n a$;
    \item $(e^x)^{(n)}=e^x$;
    \item $(\log_a x)^{(n)}=\frac{(-1)^{n-1}(n-1)!}{x^n\ln^n a},x>0$;
    \item $(\ln x)^{(n)}=\frac{(-1)^{n-1}(n-1)!}{x^n},x>0$.
\end{enumerate}
\end{multicols}

\begin{veta}
\textit{Leibnitzov vzorec}
Ak funkcie $f,g$ majú n-tú deriváciu v bode $a$, tak existuje $(f\cdot g)^{(n)}(a)$ a platí
$$(f\cdot g)^{(n)}(a)=\sum_{k=0}^n {n \choose k} f^{k}(a)\cdot g^{(n-k)}(a). $$
\end{veta}

\begin{enumerate}[resume]
	\item \useproblem[diferencialny-pocet]{diferencialny-pocet-321}
	\item \useproblem[diferencialny-pocet]{diferencialny-pocet-322}
\end{enumerate}

\textit{Riešenie (a):}
Funkcia $y$ je polynóm $6.$ stupňa, teda $y=a_0x^6+a_1x^5+...+a_6$ nie je ťažké vypočítať, že $a_0=4$. Potom $y^{VI}=(a_0x^6+a_1x^5+...+a_6)^{VI}=a_0(x^6)^{VI}+a_1(x^5)^{VI}+...+a_5x^{VI}=6!;a_0=6;4=2880$. $y^{VII}=0$ (rád derivácie je vyšší ako stupeň polynómu).

\textit{Riešenie (e):}
Funkciu $y$ možno zapísať v podobe, ktorá je pre derivovanie výhodnejšia:
$$y=\frac{2x+1}{x^2+x-2}=\frac{2x+1}{(x+2)(x-1)}=\frac{(x+2)+(x-1)}{(x+2)(x-1)}=\frac{1}{x-1}+\frac{1}{x+2}.$$
Potom $y^{(13)}=((x-1)^{-1})^{(13)}+((x+2)^{-1})^{(13)}=(-1)^{13}\cdot 13!(x-1)^{-14}+(-1)^{13}13!(x+2)^{-14}=-(13!)(\frac{1}{(x-1)^{14}}+\frac{1}{(x+2)^{14}})$.

\begin{enumerate}[resume]
	\item \useproblem[diferencialny-pocet]{diferencialny-pocet-323}
	\item \useproblem[diferencialny-pocet]{diferencialny-pocet-324}
	\item \useproblem[diferencialny-pocet]{diferencialny-pocet-325}
	\item \useproblem[diferencialny-pocet]{diferencialny-pocet-326}
	\item \useproblem[diferencialny-pocet]{diferencialny-pocet-327}
\end{enumerate}

\textit{Riešenie (f):}
Pretože $f'(x)=\frac{1}{1+x^2}$, platí $(1+x^2)f'(x)=1$ pre všetky $x \in\mathbb{R}$. To znamená, že funkcia $(1+x^2)f'(x)$ je na $\mathbb{R}$ konštantná, preto jej derivácie všetkých rádov sú rovné $0$:
$$[(1+x^2)f'(x)]^{(k)}=0,(k\in\mathbb{N}).$$
Položme $k=n-1$; pre $n=2$ má uvedená rovnosť tvar $(*)$
$$(1+x^2)f''(x)+2x\cdot f'(x)=0$$
a pre $n\geq 3$ použitím Leibnitzovho vzorca dostaneme
$$(1+x^2)f^{(n)}(x)+2(n-1)x\cdot f^{(n-1)}(x)+(n-1)(n-2)f^{(n-2)}(x)=0.$$
(Až pre $k/geq 2$,t.j. $n\geq 3$, obsahuje Leibnitzov vzorec všetky nenulové derivácie funkcie $1+x^2$, zápis prvej derivácie funkcie $(1+x^2)f'(x)$ sa teda odlišuje od zápisu k-tej derivácie tejto funkcie pre $k\geq 2$, preto treba prípad $k=1$ robiť samostatne.)

Ak v poslednej rovnosti položíme $x=0$, dostávame pre $n\geq 3$
$$f^{(n)}(0)=-(n-1)(n-2)f^{(n-2)}(0),$$
čo je rekurentný vzťah pre vyjadrenie $f^{(n)}(0)$. Dosadením do predpisu pre $f'$ dostaneme $f'(0)=1$, z $(*)$ vychádza $f''(0)=0$. Na základe toho môžeme matematickou indukciou dokázať. že
$f^{(2k+2)}(0)=0,f^{(2k+1)}(0)=(-1)^k(2k)$; pre $k=0,1,2,...$ .

\textit{Poznámka:}
Prechádzajúca úvaha má (našťastie len zdanlivo) jeden nedostatok: ak chceme použiť Leibnitzov vzorec na výpočet $(n-1)$-vej derivácie súčinu $(1+x^2)\cdot f'(x)$, treba sa najprv presvedčiť, že existujú $(n-1)$-vé derivácie funkcií $(1+x^2)$ a $f'(x)$ (t.j. $(1+x^2)^{(n-1)}$ a $f^{(n)}(x)$); to sme ale neurobili.

Dokázať, že existuje $(1+x^2)^{(n-1)}$, je ľahké; zostávateda dokázať existenciu $(\arctan x)^{(n)}$. (Návod: možno postupovať indrukciou; z predpokladu, že $f^{(k-1)}$ je podielom polynómov $P_{k-1}$ a $Q_{k-1}$ nemá reálne korene, vyplýva na základe vety o derivácii podielu, že $f^{(k)}$ existuje a možno ju zapísať v tvare podielu polynómov $P_k$ a $Q_k$, pričom $Q_k$ nemá reálne korene.)

\begin{enumerate}[resume]
	\item \useproblem[diferencialny-pocet]{diferencialny-pocet-328}
\end{enumerate}

\section{Základné vety diferenciálneho počtu}
\begin{veta}
\textit{(Rolle).}
Nech funkcia $f:\langle a,b \rangle\rightarrow \mathbb{R}$ vyhovuje nasledujúcim podmienkam:
\begin{itemize}
\item $f$ je spojitá na intervale $\langle a,b \rangle$;
\item v každom bode $x\in (a,b)$ existuje vlastná alebo nevlastná $f'(x)$;
\item $f(a)=f(b)$.
\end{itemize}
Potom existuje bod $c\in (a,b)$, v ktorom $f'(c)=0$.
\end{veta}

\begin{enumerate}[resume]
	\item \useproblem[diferencialny-pocet]{diferencialny-pocet-329}
	\item \useproblem[diferencialny-pocet]{diferencialny-pocet-330}
	\item \useproblem[diferencialny-pocet]{diferencialny-pocet-331}
	\item \useproblem[diferencialny-pocet]{diferencialny-pocet-332}
	\item \useproblem[diferencialny-pocet]{diferencialny-pocet-333}
	\item \useproblem[diferencialny-pocet]{diferencialny-pocet-334}
	\item \useproblem[diferencialny-pocet]{diferencialny-pocet-335}
\end{enumerate}

\begin{veta}
\textit{(Lagrangeova veta o strednej hodnote.)}
Nech funkcia $f:\langle a,b \rangle\rightarrow\mathbb{R}$ vyhovuje nasledujúcim podmienkam:
\begin{itemize}
\item $f$ je spojitá;
\item v každom bode $x\in (a,b)$ existuje vlastná alebo nevlastná $f'(x)$.
Potom existuje bod $c\in (a,b)$, v ktorom $f'(c)=\frac{f(b)-f(a)}{b-a}.$
\end{itemize}
\end{veta}

\begin{enumerate}[resume]
	\item \useproblem[diferencialny-pocet]{diferencialny-pocet-336}
	\item \useproblem[diferencialny-pocet]{diferencialny-pocet-337}
\end{enumerate}

\textit{Riešenie:}
Derivovaním dostaneme $f'(x)=\frac{1}{1+x^2},x\in\mathbb{R}\setminus \{1\},g'(x)=\frac{1}{1+x^2}$. Označme $h:=f-g$; potom $h'(x)=0$ pre $x\in (1,\infty)$. Podľa výsledku príkladu $336$ je teda funkcia $h$ konštantná na intervale $(-\infty,1)$ a konštantná na intervale $(1,\infty)$. Hodnotu konštanty na intervale $(-\infty,1)$ nájdeme ako funkčnú hodnotu v niektorom vhodne zvolenom bode, napr. $k_1=h(0)=\arctan 1- \arctan 0=\frac{\pi}{4}$. Pretože výpočet $h(c)$ v niektorom $c\in (1,\infty)$ by bol komplikovanejší, pomôžeme si nasledujúcou úvahou: pretože $h(x)=k_2$ na intervale $(1,\infty)$, platí $k_2=\lim_{x\rightarrow 1+}h(x)=-\frac{\pi}{2}-\frac{\pi}{4}=-\frac{3}{4}\pi$. Celkovo teda

$\arctan \frac{1+x}{1-x}-\arctan x = \left\{ \begin{array}{r@{\quad}c}
   \frac{\pi}{4},& $ak $ x<1 \\
    -\frac{3}{4}\pi, &  $ak $ x>1 \\ \end{array} \right.$,

    čo môžeme prepísať do podoby $\arctan \frac{1+x}{1-x}-\arctan x=-\frac{\pi}{2} sgn (x-1)-\frac{\pi}{4},x\neq 1$.

\begin{enumerate}[resume]
	\item \useproblem[diferencialny-pocet]{diferencialny-pocet-338}
	\item \useproblem[diferencialny-pocet]{diferencialny-pocet-339}
	\item \useproblem[diferencialny-pocet]{diferencialny-pocet-340}
	\item \useproblem[diferencialny-pocet]{diferencialny-pocet-341}
\end{enumerate}

\textit{Riešenie (a):}
Pre $x=y$ Uvedená nerovnosť zrejme platí. Ak $x<y$ (dôkaz pre $x>y$ by bol rovnaký), tak na intervale $\langle x,y \rangle$ funkcia $f(x)=\sin x$ vyhovuje predpokladom Lagrangeovej vety o strednej hodnote, podľa ktorej $\sin s=\sin y=(x-y)\cos c$ pre niektoré $c\in (x,y)$. Pretože $|\cos c|\leq 1$ pre ľubovoľné $c\in\mathbb{R}$, je $|\sin x-\sin y|=|\cos c|\cdot|x-y|\leq |x-y|$.

\begin{enumerate}[resume]
	\item \useproblem[diferencialny-pocet]{diferencialny-pocet-342}
	\item \useproblem[diferencialny-pocet]{diferencialny-pocet-343}
	\item \useproblem[diferencialny-pocet]{diferencialny-pocet-344}
	\item \useproblem[diferencialny-pocet]{diferencialny-pocet-345}
\end{enumerate}

\begin{veta}
\textit{(Cauchyho veta o strednej hodnote).}
Nech funkcie $f: \langle a,b \rangle \rightarrow \mathbb{R}$ a $g: \langle a,b \rangle \rightarrow \mathbb{R}$ spĺňajú nasledujúce podmienky:
\begin{itemize}
\item $f$ a $g$ sú spojité na intervale $\langle a,b \rangle$;
\item v každom bode $x\in (a,b)$ existujú vlastná alebo nevlastná $f'(x)$ a vlastná $g'(x)$.
\end{itemize}
Potom existuje bod $c\in (a,b)$, v ktorom platí
$$(f(b)-f(a))g'(c)=(g(b)-g(a))f'(c).$$
Ak sú naviac splnené predpoklaady
\begin{itemize}
\item $(f'(x))^2+(g'(x))^2>0$ Pre každé $x\in (a,b)$;
\item $g(b)\neq g(a)$;
\end{itemize}
možnouvedenú rovnosť písať v tvare
$$\frac{f(b)-f(a)}{g(b)-g(a)}=\frac{f'(c)}{g'(c)}.$$
\end{veta}

\begin{enumerate}[resume]
	\item \useproblem[diferencialny-pocet]{diferencialny-pocet-346}
	\item \useproblem[diferencialny-pocet]{diferencialny-pocet-347}
	\item \useproblem[diferencialny-pocet]{diferencialny-pocet-348}
\end{enumerate}

\section{Vyšetrovanie niektorých vlastností funkcií pomocou diferenciálneho počtu}

\subsection{Monotónnosť}
\begin{veta}
Nech funkcia $f: I \rightarrow\mathbb{R}$ je spojitá na intervale $I$ a má deriváciu v každom jeho vnútornom bode. Potom
\begin{itemize}
\item ak $f'>0,(f'\geq 0)$ vnútri intervalu $I$ (t.j. ak pre každý vnútorný bod $x$ intervalu $I$ platí $f'(x)>0,(f'(x)\geq 0)$), tak $f$ je rastúca (neklesajúca) na $I$;
\item ak $f'<0,(f'\leq 0)$ vnútri intervalu $I$, tak $f$ je klesajúca (nerastúca) na $I$.
\end{itemize}
\end{veta}

\begin{enumerate}[resume]
	\item \useproblem[diferencialny-pocet]{diferencialny-pocet-349}
	\item \useproblem[diferencialny-pocet]{diferencialny-pocet-350}
	\item \useproblem[diferencialny-pocet]{diferencialny-pocet-351}
\end{enumerate}

\begin{veta}
Nech funkcie $f,g$ sú $n$-krát diferencovateľné na intervale $I$, nech v bode $a\in I$ platí $f(a)=g(a),f'(a)=g'(a),...,f^{(n-1)}(a)=g^{(n-1)}(a)$ (teda ak $n=1$, predpokladáme len $f(a)=g(a)$). Potom
\begin{enumerate}
\item ak $f^{(n)}(x)>g^{(n)}(x)$, pre všetky $x\in I \cap (a,\infty)$, tak $f(x)>g(x)$ Pre všetky $x\in I \cap (a,\infty)$ (pritom samozrejme predpokladáme, že $I \cap (a,\infty)\neq \emptyset$);
\item
\begin{itemize}
\item ak $f^{(n)}(x)>g^{(n)}(x)$ pre všetky $x\in I \cap (-\infty,a)$ a $n$ je párne, tak $f(x)>g(x)$ pre všetky $x\in I \cap (-\infty,a)$;
\item ak $f^{(n)}(x)>g^{(n)}(x)$ pre všetky $x\in I \cap (-\infty,a)$ a $n$ je nepárne, tak $f(x)<g(x)$ pre všetky $x\in I \cap (-\infty,a)$ (pritom predpokladáme $I \cap (-\infty,a)\neq \emptyset$).
\end{itemize}
\end{enumerate}
\end{veta}

\begin{enumerate}[resume]
	\item \useproblem[diferencialny-pocet]{diferencialny-pocet-352}
	\item \useproblem[diferencialny-pocet]{diferencialny-pocet-353}
	\item \useproblem[diferencialny-pocet]{diferencialny-pocet-354}
	\item \useproblem[diferencialny-pocet]{diferencialny-pocet-355}
	\item \useproblem[diferencialny-pocet]{diferencialny-pocet-356}
\end{enumerate}

\subsection{Konvexnosť a konkávnosť. Inflexné body}
Funkcia $f$ sa nazýva rýdzo konvexná (konvexná) na intervale $I \subset D(f)$, ak platí $(*)$
$$\forall x,y\in I,x\neq y \forall p,q>0,p+q=1;f(px+qy)<pf(x)+qf(y)$$
$$(\forall x,y\in I,x\neq y \forall p,q>0,p+q=1;f(px+qy)\leq pf(x)+qf(y)).$$
Funkcia $f$ sa nazýva rýdzo konkávna (konkávna) na intervale $I \subset D(f)$, ak platí
$$\forall x,y\in I,x\neq y \forall p,q>0,p+q=1;f(px+qy)>pf(x)+qf(y)$$
$$(\forall x,y\in I,x\neq y \forall p,q>0,p+q=1;f(px+qy)\geq pf(x)+qf(y)).$$
(Geometricky možno výrok $(*)$ interpretovať takto: pre ľubovoľné $2$ čísla $x,y\in I,x<y$, leží úsečka spájajúca body $(x,f(x))$ nad grafom funkcie $f/(x,y)$.)

\begin{veta}
Nech funkcia $f$ je spojitá na intervale $I$ a dvakrát diferencovateľná v každom jeho vnútornom bode. Potom
\begin{itemize}
\item $f''>0,(f''\geq 0)$ vnútri intervalu $I$, tak $f$ je rýdzo konvexná (konvexná) na $I$;
\item $f''<0,(f''\leq 0)$ vnútri intervalu $I$, tak $f$ je rýdzo konkávna (konkávna) na $I$.
\end{itemize}
Vnútorný bod $a$ množiny $D(f)$ sa nazýva inflexný bod funkcie $f$, ak $f$ má v bode $a$ deriváciu a existuje $\varepsilon >0$ tak, že funkcia $f$ je rýdzo konvexná na jednej z množín $(a-\varepsilon,a\rangle$, $\langle a,a+\varepsilon )$ a rýdzo konkávna na druhej z nich.
\end{veta}

\begin{veta}
Nech funkcia $f$ je triktár diferencovateľná v bode $a$ a dvakrát diferencovateľná v niektorom jeho okolí. Ak $f''(a)=0,f''(a)\neq 0$, tak $a$ je inflexný bod funkcie $f$.
\end{veta}

\textit{Poznámka:}
Existujú aj iné definície rýdzej konvexnosti, rýdzej konkávnosti a inflexného bodu, ktoré nie sú ekvivalentné tu uvedeným. Všetky v matematickej literatúre používané definície týchto pojmov sú však volené tak, že vety $13$ a $14$ zostanú v platnosti.

\begin{enumerate}[resume]
	\item \useproblem[diferencialny-pocet]{diferencialny-pocet-357}
	\item \useproblem[diferencialny-pocet]{diferencialny-pocet-358}
	\item \useproblem[diferencialny-pocet]{diferencialny-pocet-359}
	\item \useproblem[diferencialny-pocet]{diferencialny-pocet-360}
\end{enumerate}

\textit{Riešenie (b):}
Pretože $(e^x)''=e^x>0$ pre všetky $x\in\mathbb{R}$, je funkcia $f(x)=e^x$ rýdzo konvexná na $\mathbb{R}$: teda platí
$$\forall x,y\in \mathbb{R},x\neq y \forall p,q>0,p+q=1;f(px+qy)<pf(x)+qf(y).$$
Ak špeciálne zvolíme $p=q=\frac{1}{2}$, dostaneme
$$e^{\frac{x+y}{2}}=f(\frac{1}{2}x+\frac{1}{2}y)<\frac{1}{2}f(x)+\frac{1}{2}f(y)=\frac{1}{2}(e^x+e^y),x\neq y.$$

\begin{enumerate}[resume]
	\item \useproblem[diferencialny-pocet]{diferencialny-pocet-361}
	\item \useproblem[diferencialny-pocet]{diferencialny-pocet-362}
	\item \useproblem[diferencialny-pocet]{diferencialny-pocet-363}
\end{enumerate}

\textit{Poznámka:}
Na vlastnostiach odvodených v príklade $362$ sa zakladajú nasledujúce definície, odlišné od definícií používaných v tomto odstavci:
\begin{itemize}
\item Funkcia $f$ diferencovateľná na intervale $I$ sa nazýva rýdzo konvexná (rýdzo konkávna) na $I$, ak pre každé $a\in I$ platí: na množine $I \setminus \{a\}$ leží graf funkcie $f$ nad (pod) svojou dotyčnicou v bode $(a,f(a))$.
\item Bod $a$ sa nazýva inflexný bod funkcie $f$, ak existuje konečná $f'(a)$ a graf funkcie $f$ prechádza v bode $a$ z jednej strany svojej dotyčnice na druhú.
\end{itemize}
Prvá z týchto definícií predstavuje užšie chápanie pojmov rýdzo konvexná a rýdzo konkávna funkcia (to sme tu nedokazovali, ale môžete to skúsiť sami), druhá naopak čirčie chápanie pojmu inflexný bod (to ukazujú príklady $362$ a $363$).

\subsection{Extrémy}
Nech definičným oborom funkcie $f$ je interval $I$. Hovoríme, že funkcia $f$ má v bode $a$ $I$ lokálne maximum (lokálne minimum), ak existuje okolie $O(a)$ bodu $a$ tak, že platí
$$\forall x\in (O(a)\setminus \{a\})\cap I:f(x)\leq f(a),(\forall x\in (O(a)\setminus \{a\})\cap I:f(x)\geq f(a)).$$
Definíciu ostrého lokálneho maxima (ostrého lokálneho minima) dostaneme, ak v predchádzajúcej definícii zameníme znak $\leq$ ($\geq$) znakom $<(>)$. Lokálne maximá a lokálne minimá sa súhrnne nazývajú lokálnymi extrémami.

\begin{veta}
Ak funkcia $f$ má lokálny extrém vo vnútornom bode $a$ svojho definičného oboru, tak buď neexistuje vlastná ani nevlastná $f'(a)$, alebo $f'(a)=0$.
Bod $a$ sa nazýva stacionálny bod funkcie $f$, ak $f'(a)=0$.
Pri hľadaní lokálnych extrémov funkcie $f:I\rightarrow\mathbb{R}$, treba teda vyšetriť:
\begin{enumerate}
\item všetky jej stacionárne body;
\item všetky body $a\in I$, v ktorých neexistuhe $f'(a)$;
\item všetky body $a\in I$, ktoré nie sú vnútornými bodmi intervalu $I$.
\end{enumerate}
\end{veta}

\begin{veta}
Ak funkcia $f$ je dvakrát diferencovateľná vo vnútornom bode $a$ množiny $D(f)$ a platí $f'(a)=0,f''(a)>0$  $(f'(a)=0,f''(a)<0)$, tak $f$ má v bode $a$ ostré lokálne minimum (ostré lokálne maximum).
\end{veta}

\begin{veta}
Nech funkcia $f$ je $n$-krát $(n\geq 2)$ diferencovateľná vo vnútornom bode $a$ množiny $D(f)$, nech $f'(a)=...=f^{(n-1)}(a)=0,f^{(n)}\neq 0$.

Ak $n$ je párne a $f^{(n)}(a)>0$  $(f^{(n)}(a)<0)$, tak funkcia $f$ má v bode $a$ ostré lokálne minimum (ostré lokálne maximum).

Ak $n$ je nepárne, nemá funkcia $f$ v bode $a$ lokálny extrém.
\end{veta}

\begin{enumerate}[resume]
	\item \useproblem[diferencialny-pocet]{diferencialny-pocet-364}
	\item \useproblem[diferencialny-pocet]{diferencialny-pocet-365}
\end{enumerate}

\textit{Návod:}
Stačí použiť úvahy analogické tejto: ak $f$ na $(a,b),f'(x)>0$ pre všetky $x\in (a,c)$ (t.j. $f$ rastie na $(a,c\rangle$), $f'(x)<0$ pre všetky $x\in (c,b)$ (t.j. $f$ klesá na $\langle c,b)$), tak $f$ Má v bode $c$ lokálne maximum.

\begin{enumerate}[resume]
	\item \useproblem[diferencialny-pocet]{diferencialny-pocet-366}
	\item \useproblem[diferencialny-pocet]{diferencialny-pocet-367}
	\item \useproblem[diferencialny-pocet]{diferencialny-pocet-368}
\end{enumerate}

\textit{Návod:}
Ak funkcia $f$ Je spojitá na uzavretom ohraničenom intervale $I$, tak podľa vety $5$ z kapitoly $3$ existujú $\max_{x\in I}f(x),\min_{x\in I}f(x)$; tieto čísla sú zrejme aj lokálnymi extrémami funkcie $f/I$. Preto ak chceme nájsť globálne extrémy funkcie $f$ na intervale $I$, stačí zistiť funkčné hodnoty vo všetkých bodoch, v ktorých môže mať funkcia $f/I$ lokálny extrém (t.j. v stacionárnych bodoch, v krajných bodoch intervalu $I$ a v tých bodoch, v ktorých neexistuje derivácia); najväčšie z týchto čísel je potom $\max_{x\in I}f(x)$, najmenšie z nich je $\min_{x\in I}f(x)$.

V prípade spojitej funkcie a nekompaktného intervalu $I$ možno o existencii čísel $\max_{x\in I}f(x),\min_{x\in I}f(x)$ často rozhodnúť na základe rastu a klesania funkcie $f$ alebo jednoduchých úvah tohto typu: ak $f(c)=\max_{x\in \langle a,b \rangle}f(x)$ a $c\in (a,b)$, tak $f(x)=\max_{x\in (a,b)}f(x)$.

\begin{enumerate}[resume]
	\item \useproblem[diferencialny-pocet]{diferencialny-pocet-369}
\end{enumerate}

\textit{Riešenie (a):}
Pretože $\max_{x\in \langle -2,2 \rangle}(3x-x^3)=2$ a $\min_{x\in \langle -2,2 \rangle}(3x-x^3)=-2$, platia pre všetky $x\in \langle -2,2 \rangle$ nerovnosti $-2\leq 3x-x^3\leq 2$, t.j. $|3x-x^3|\leq 2$. (Funkcia $3x-x^3,x\in \langle-2,2 \rangle$ je diferencovateľná v každom bode intervalu $(-2,2)$, má stacionárne body $1$ a $-1$; preto jej maximum (minimum) nájdeme ako najvňčšie (najmenšie) z čísel $f(-1)=2,f(1)=-2,f(-2)=2,f(2)=-2$.)

\begin{enumerate}[resume]
	\item \useproblem[diferencialny-pocet]{diferencialny-pocet-370}
	\item \useproblem[diferencialny-pocet]{diferencialny-pocet-371}
	\item \useproblem[diferencialny-pocet]{diferencialny-pocet-372}
	\item \useproblem[diferencialny-pocet]{diferencialny-pocet-373}
	\item \useproblem[diferencialny-pocet]{diferencialny-pocet-374}
	\item \useproblem[diferencialny-pocet]{diferencialny-pocet-375}
	\item \useproblem[diferencialny-pocet]{diferencialny-pocet-376}
\end{enumerate}

\subsection{L' Hospitalovo pravidlo}
\begin{veta}
Nech
\begin{enumerate}
\item funkcia $f,g$ sú doferencovateľné v niektorom prstencovom okolí $O^*(a)$ bodu $a\in\mathbb{R^*}$;
\item $\forall x\in O^*(a):g'(x)\neq 0$;
\item $\lim_{x\rightarrow a}f(x)=\lim_{x\rightarrow a}g(x)=0$ alebo $\lim_{x\rightarrow a}|g(x)|=+\infty$;
\item existuje vlastná alebo nevlastná $\lim_{x\rightarrow a0\frac{f'(x)}{g'(x)}}$. Potom existuje aj $\lim_{x\rightarrow a}\frac{f(x)}{g(x)}$ a platí $\lim_{x\rightarrow a}\frac{f(x)}{g(x)}=\lim_{x\rightarrow a}\frac{f'(x)}{g'(x)}$.

(Analogické tvrdenia možno sformulovať aj pre jednostranné limity.)
\end{enumerate}
\end{veta}

\begin{enumerate}[resume]
	\item \useproblem[diferencialny-pocet]{diferencialny-pocet-377}
	\item \useproblem[diferencialny-pocet]{diferencialny-pocet-378}
\end{enumerate}

\textit{Riešenie (a):}
$\lim_{x\rightarrow 0}\frac{\arcsin 2x-2\arcsin x}{x^3}=\lim_{x\rightarrow 0}\frac{\frac{2}{\sqrt{1-4x^2}}-\frac{2}{\sqrt{1-x^2}}}{3x^2}=\lim_{x\rightarrow 0}\frac{2}{3}\cdot\frac{\frac{\sqrt{1-x^2}-\sqrt{1-4x^2}}{\sqrt{1-4x^2}\sqrt{1-x^2}}}{x^2}=\frac{2}{3}\lim_{x\rightarrow 0}\frac{1}{\sqrt{1-4x^2}\sqrt{1-x^2}}\cdot \lim_{x\rightarrow 0}\frac{\sqrt{1-x^2}-\sqrt{1-4x^2}}{x^2}=\frac{2}{3}\cdot \lim_{x\rightarrow 0}\frac{\frac{4x}{\sqrt{1-4x^2}}-\frac{x}{\sqrt{1-x^2}}}{2x}=\frac{2}{1}\cdot\frac{1}{2}\cdot \lim_{x\rightarrow 0}(\frac{4}{\sqrt{1-4x^2}}-\frac{1}{\sqrt{1-x^2}})=1$.

\textit{Prvá úprava:}
ide o neurčitý výraz typu $\frac{0}{0}$, skúsime preto použiť l'Hospitalovo pravidlo (t.j. vetu $18$); prvé tri predpoklady sú zrejme splnené, splnenie štvrtého predpokladu preveríme až ďalším výpočtom, prvá rovnosť má teda zatiaľ podmienený charakter: ak ukážeme existenciu limity na jej pravej strane, tak bude existovať aj limita vľavo a bude platiť prvá rovnosť (ak zistíme, že limita vpravo neexistuje, neboli sme oprívnení použiť l'Hospitalovo pravidlo);

\textit{Druhá úprava:}
výraz na ľavej strane je opäť typu $\frac{0}{0}$, skôr než však skúsime znovu použiť vetu $18$, upravíme limitovanú funkciu ("bezhlavým" používaním l'Hospitalovho pravidla sa výpočet často viac skomplikuje než zjednoduší), využijeme, že $\lim_{x\rightarrow 0}\sqrt{1-x^2}\sqrt{1-4x^2}=1$, a vetu $18$ použijeme len pri výpočte druhej z limít vpravo (predpoklady $1,2,3$ sú opäť zrejme splnené);

\textit{Tretia úprava:}
táto rovnosť je podmienená podobne ako prvá rovnosť

\textit{Štvrtá úprava:}
z existencie tejto limity vyplýva, že použitie l'Hospitalovho pravidla bolo v obidvoch prípadoch oprávnené, a teda všetky uvedené rovnosti skutočne platia.

\begin{enumerate}[resume]
	\item \useproblem[diferencialny-pocet]{diferencialny-pocet-379}
\end{enumerate}

\textit{Návod:}
Ak súčin $f\cdot g$, ktorý je neurčitým výrazom typu $0\cdot\infty$, prepíšeme do tvaru podielu $\frac{f}{\frac{1}{g}}$ alebo $\frac{g}{\frac{1}{f}}$, dostaneme neurčitý výraz typu $\frac{0}{0}$ alebo $\frac{\infty}{\infty}$, ktorého limitu môžeme skúsiť vypočítať pomocou l'Hospitalovho pravidla.

\begin{enumerate}[resume]
	\item \useproblem[diferencialny-pocet]{diferencialny-pocet-380}
\end{enumerate}

\textit{Návod}
Funkciu $f^g$, ktorá je neurčitým výrazom typu $0^0,(+\infty)^0$ alebo $1^{\infty}$, môžeme prepísať do tvaru $e^{g\cdot\ln f}$; exponent $g\cdot\ln f$ je potom neurčitým výrazom typu $0\cdot\infty$, pri výpočte jeho limity môžeme postupovať ako v príklade $379$ (v prípade neurčitých výrazov typu $1^{\infty}$ je však často výhodnejšie prepísať ich do tvaru $[(1+(f-1))^{\frac{1}{f-1}}]^{(f-1)\cdot g}$ a postup z príkladu $379$ použiť pri výpočte limity funkcie $(f-1)\cdot g$, ktorá je neurčitým výrazom typu $0\cdot\infty$).

\begin{enumerate}[resume]
	\item \useproblem[diferencialny-pocet]{diferencialny-pocet-381}
\end{enumerate}

\textit{Návod:}
Ak chceme použiť l'Hospitalovo pravidlo pri výpočte limít neurčitých výrazov typu $\infty,-\infty$, musíme predovšetkým funkciu $f-g$ napísať v tvare podielu. Vo všeobecnosti to možno dosiahnúť nahradením funkcie $f-g$ funkciou $\frac{\frac{1}{g}-\frac{1}{f}}{\frac{1}{g\cdot f}}$ (tá je neurčitým výrazom typu $\frac{0}{0}$), v jednotlivých prípadoch však často možno nájsť jednoduchší postup.

\begin{enumerate}[resume]
	\item \useproblem[diferencialny-pocet]{diferencialny-pocet-382}
	\item \useproblem[diferencialny-pocet]{diferencialny-pocet-383}
	\item \useproblem[diferencialny-pocet]{diferencialny-pocet-384}
\end{enumerate}

\subsection{Taylorov polynóm}
Nech funkcia $f$ je $n$-krát diferencovateľná v bode $a\in\mathbb{R}$. Potom Taylorovým polynómom stupňa $n$ funkcie $f$ v bode $a$ sa nazýva polynóm (v premennej $x$)
$$f(a)+\frac{f'(a)}{1!}(x-a)+...+\frac{f^{(n)}(a)}{n!}(x-a)^n.$$
Ak špeciálne $a=0$, používa sa namiesto názvu Taylorov polynóm označenie Maclaurinov polynóm.

\begin{veta}
Ak funkcia $f$ je $n$-krát diferencovateľná v bode $a\in\mathbb{R}$ a $T_n$ je jej Taylorov polynóm stupňa $n$ v bode $a$, tak $$\lim_{x\rightarrow a}\frac{f(x)-T_n(x)}{(x-a)^n}=0.$$
(Rozdiel $f-T_n$ sa nazýva zvyšok Taylorovho polynómu stupňa $n$ funkcie $f$ v bode $a$.)
\end{veta}

Nech funkcia $g$ je definovaná v niektorom rýdzom okolí bodu $a\in\mathbb{R^*}$ a nenadobúda tam nulové hodnoty. Znakom $c(g)$ budeme označovať triedu všetkých funkcií $f$ takých, že
\begin{itemize}
\item ich definičný obor obsahuje niektoré rýdze okolie bodu $a$;
\item platí $\lim_{x\rightarrow a}\frac{f(x)}{g(x)}=0$.
\end{itemize}

Namiesto zápisu $f\in c(g)$ sa používa zápis $f=c(g)$. (Pokiaľ by z kontextu nebolo jasné, ktorého $a\in\mathbb{R^*}$ sa vzťah $f=c(g)$ týka, používa sa zápis $f=c(g)$ $(x\rightarrow a).$) Zápis $f=h+c(g)$ treba chápať nasledovne: funkcia $f$ je súčtom funkcie $h$ a niektorej funkcie z triedy $c(g)$.

Ak funkcia $f$ je $n$-krát diferencovateľná v bode $a\in\mathbb{R}$, patrí podľa vety $19$ zvyšok jej Taylorovho polynómu stupňa $n$ v bode $a$ do triedy $c((x-a)^n)$, možno teda písať
$$f(x)=f(a)+\frac{f'(a)}{1!}(x-a)+...+\frac{f^{(n)}(a)}{n!}c((x-a)^n);$$
tento zápis sa nazýva Taylorovým vzorcom so zvyškom v Peanovom tvare.
V nasledujúcich príkladoch budeme používať tieto tvrdenia (v zápisoch všade vynechávame $x\rightarrow 0;m,n$ sú prirodzené čísla);
\begin{itemize}
\item ak $f(x)=c(x^n)$ a $g(x)=c(x^n)$, tak $f(x)+g(x)=c(x^n)$;
\item ak $m>n$ a $f(x)=c(x^m)$, tak $f(x)=c(x^n)$;
\item ak $f(x)=c(x^n)$, tak $f^m(x)=c(x^{m\cdot n})$;
\item ak $f(x)=c(x^n)$, tak $x^m\cdot f(x)=c(x^{m+n})$;
\item ak $f(x)=c(x^n)$ a $g(x)=c(x^m)$, tak $f(x)\cdot g(x)=c(x^{m+n})$.
\end{itemize}

Uvedené implikácie budeme zapisovať nasledovne:
\begin{itemize}
\item $c(x^n)+c(x^n)=c(x^n)$;
\item $c(x^m)=c(x^n)$ pre $m>n$;
\item $c^m(x^n)=c(x^{m\cdot n})$;
\item $x^m\cdot c(x^n)=c(x^{m+n})$;
\item $c(x^n)\cdot c(x^m)=c(x^{m+n})$.
\end{itemize}

Používanie týchto zápisov vyžaduje istú opatrnosť, pretože ide o symbolické vyjadrenie implikácií, možno (na rozdiel od skutočných rovností) tieto "rovnosti" čítať sprava doľava.

\begin{enumerate}[resume]
	\item \useproblem[diferencialny-pocet]{diferencialny-pocet-385}
	\item \useproblem[diferencialny-pocet]{diferencialny-pocet-386}
\end{enumerate}

V nasledujúcich príkladoch využijeme tvrdenie z príkladu $386$ a znalosť MAclaurinových polynómov týchto funkcií (možno ich zostrojiť výpočtom príslušných derivácií):
\begin{itemize}
\item $e^x=1+x+\frac{x^2}{2!}+...+\frac{x^n}{n!}+c(x^n)$;
\item $\sin x=x-\frac{x^3}{3!}+...+(-1)^{n-1}\frac{x^{2n-1}}{(2n-1)!}+c(x^{2n})$\footnote{pretože $\sin^{(2n)}(0)=0$, majú Maclaurinov polynóm stupňa $2n-1$ a Maclaurinov polynóm stupňa $2n$ rovnaký tvar, preto aj zvyšok Maclaurinovho polynómu stupňa $2n-1$ funkcie $\sin$ patrí do triedy $c(x^{2n})$; podobná poznáma platí o funkcii $\cos$};
\item $\cos x=1-\frac{x^2}{2!}+...+(-1)^{n}\frac{x^{2n}}{(2n)!}+c(x^{2n+1})$;
\item $\ln (1+x)=x-\frac{x^2}{2}+...+(-1)^{n-1}\frac{x^n}{n}+c(x^n)$;
\item $\frac{1}{1-x}=1+x+x^2+...+\frac{x^n}{n!}+c(x^n)$;
\item $(1+x)^{\alpha}=1+x+\frac{\alpha(\alpha-1)}{2!}x^2+...+\frac{(\alpha-1)...(\alpha-n+1)}{n!}x^n+c(x^n),(\alpha\in\mathbb{R})$.
\end{itemize}

\begin{enumerate}[resume]
	\item \useproblem[diferencialny-pocet]{diferencialny-pocet-387}
\end{enumerate}

\textit{Riešenie (b):}
Pretože funkcia $e^{x^2}$ má derivácie všetkých rádov v bode $0$ (vyplýva to matematickou indukciou z viet o derivácii súčtu, súčinu a zloženej funkcie), existuje jej Maclaurinov polynóm ľubovoľného stupňa. Jeho koeficienty nebudeme hľadať derivovaním funkcie $e^{x^2}$, namiesto toho použijeme tento postup:
Ak označíme $R_n(x)$ zvyšok Maclaurinovho polynómu stupňa $n$ funkcie $e^x$, tak pre každé $x\in\mathbb{R}$ platí $(*)$
$$e^x=1+x+\frac{x^2}{2!}+...+\frac{x^n}{n!}+R_n(x),$$
pritom podľa vety $19$ je $\lim_{x\rightarrow 0}\frac{R_n(x)}{x^n}=0$, t.j. pre každé $x\in\mathbb{R}$ platí
$$e^{x^2}=1+x^2+\frac{x^2}{2!}+...+\frac{x^{2n}}{n!}+R_n(x^2),$$
pritom $\lim_{x\rightarrow 0}\frac{R_n(x^2)}{x^{2n}}=0$ (vyplýva to z vety o limite zloženej funkcie a z toho, že $\lim_{x\rightarrow 0}\frac{R_n(z)}{z^{2n}}=0$), teda $R_n(x^2)=c(x^{2n}).$ Podľa tvrdenia z príkladu $386$ je preto $1+x^2+...+\frac{x^{2n}}{n!}$ Maclaurinov polynóm stupňa $2n$ funkcie $e^{x^2}$.
(Priamo z definície Maclaurinovho polynómu vyplýva: ak z Maclaurinovho polynómu stupňa $k$ $(x\geq 2)$ funkcie $f$ vynecháme člen obsahujúci $x^k$, dostaneme Maclaurinov polynóm stupňa $k-1$ funkcie $f$. Teda Maclaurinov polynóm stupňa $2n-1$ funkcie $e^{x^2}$ je $1+x^2+...+\frac{x^{2n-2}}{(n-1)}$!)

\textit{Riešenie (d):}
Ak rovnosť
$$\ln (1+x)=x-\frac{x^2}{2}+...+(-1)^{n-3}\frac{x^{n-2}}{n-2}+(-1)^{n-2}\frac{x^{n-1}}{n-1}+(-1)^{n-1}\frac{x^n}{n}+c(x^n)$$
 vynásobím $x^2$, dostaneme
 $$x^2\ln (1+x)=x^3-\frac{x^4}{2}+...+(-1)^{n-3}\frac{x^{n}}{n-2}+(-1)^{n-2}\frac{x^{n+1}}{n-1}+(-1)^{n-1}\frac{x^{n+2}}{n}+x^2\cdot c(x^n).$$
 Funkcia $(-1)^{n-2}\frac{x^{n+1}}{n-1}+(-1)^{n-1}\frac{x^{n+2}}{n}+x^2\cdot c(x^n)$ patrí do triedy $c(x^n)$, preto môžeme písať:
 $$x^2\cdot\ln (1+x)=x^3-\frac{x^4}{2}+...+(-1)^{n-3}\frac{x^n}{n-2},$$ čo podľa tvrdenia z príkladu $386$ znamená, že $x^3-\frac{x^4}{2}+...+(-1)^{n-3}\frac{x^n}{n-2}$ je Maclaurinov polynóm stupňa $n,(n\geq 3)$ funkcie $x^2\cdot\ln (1+x)$. Zrejme Maclaurinovými polynómami stupňa $1$ a $2$ sú funkcie identicky rovné $0$.

\begin{enumerate}[resume]
	\item \useproblem[diferencialny-pocet]{diferencialny-pocet-388}
\end{enumerate}

\textit{Riešenie (a):}
Ak do rovnosti
$$e^z=1+z+\frac{z^2}{2!}+R(z)$$
($R$ označuje zvyšok Maclaurinovho polynómu) dosadíme $z=\sin^2 x$, dostaneme $(*)$
$$e^{\sin^2 x}=1+\sin^2 x+\frac{\sin^4 x}{2!}+R(\sin^2 x),$$
pritom $R(\sin^2 s)=c(x^4)$ (pretože $R(z^2)=c(z^2)$, je $\lim_{x\rightarrow 0}\frac{R(\sin^2 s)}{x^4}=\lim_{x\rightarrow 0}\frac{\sin^4 x}{x^4}\cdot\lim_{x\rightarrow 0}\frac{R(\sin^2 x)}{\sin^4 x}=1\cdot\lim_{x\rightarrow 0}\frac{R(z)}{z^2}=0)$.

Pomocou rovností $\sin x=x-\frac{x^3}{3!}+c(x^3),\sin x=x+c(x)$ môžeme $(*)$ prepísať a upraviť nasledovne $(**)$:
$e^{\sin^2 x}=1+(x-\frac{x^3}{3!}+c(x^3))^2+\frac{(x+c(x))^4}{2}+c(x^4)=1+(x^2+\frac{x^6}{3!\cdot 3!}+c^2 (x^3)-\frac{2x^4}{3!}+2xc(x^3)-\frac{2x^3c(x^3)}{3!})+(x^4+4x^3c(x)+6x^2c^2(x)+4xc(x)+c^4(x))+c(x^4)$.

Všetky sčítance okrem vyznačených patria do $c(x^4)$, patrí tam teda aj ich súčet. Rovnosť $(**)$ preto môžeme písať
$$e^{\sin^2 x}=1+x^2+(1-\frac{2}{3!})x^4+c(x^4),$$
čo podľa tvrdenia z príkladu $386$ znamená, že $1+x^2+\frac{2}{3}x^4$ je Maclaurinov polynóm stupňa $4$ funkcie $e^{\sin^2 x}$.

\textit{Poznámka:}
Zo záveru uvedeného postupu vyplýva, že stupne Maclaurinových polynómov funkcií $e^x$ a $\sin^2 x$ sme volili tak, aby v $(**)$ všetky sčítance, ktoré nemajú tvar $\alpha x^m,m=0.1....$ (medzi ne patrí aj funkcia $R(\sin^2 x)$) boli z triedy $c(x^4)$.

Všimnite si tiež, že hoci strana rovnosti $(**)$ obsahuje dokonca polynóm $6.$ stupňa, nemôžeme tvrdiť, že tento polynóm je Maclaurinovým polynómom $6.$ stupňa funkcie $e^{\sin^2 x}$; z nášho postupu totiž nevyplýva, že by súčet zvyšných členov na pravej strane rovnosti $(**)$ patril do $c(x^6)$.

\begin{enumerate}[resume]
	\item \useproblem[diferencialny-pocet]{diferencialny-pocet-389}
	\item \useproblem[diferencialny-pocet]{diferencialny-pocet-390}
\end{enumerate}

\textit{Riešenie (b):}
Funkciu v čitateli napíšeme v tvare $\alpha x^m+c(x^m)$, kde $\alpha\neq 0$ (t.j. z tých jej Maylaurinových polynómov, ktoré nie sú identicky rovné $0$, vyberieme polynóm najnižšieho stupňa); to isté urobíme v menovateli.

V našom prípade

$\cos x -e^{-\frac{x^2}{2}}=(1-\frac{x^2}{2}+\frac{x^4}{24}+c(x^4))-(1-\frac{x^2}{2}+\frac{x^4}{8}+c(x^4))=-\frac{1}{12}x^4+c(x^4),\sin^4 x=x^4+c(x^4)$.

Potom
$$\lim_{x\rightarrow 0}\frac{\cos x -e^{-\frac{x^2}{2}}}{\sin x^4}=\lim_{x\rightarrow 0}\frac{-\frac{1}{12}x^4+c(x^4)}{x^4+c(x^4)}=\lim_{x\rightarrow 0}\frac{{-\frac{1}{12}}+\frac{c(x^4)}{x^4}}{1+\frac{c(x^4)}{x^4}}=-\frac{1}{12}$$

(rovnosť $\lim_{x\rightarrow 0}\frac{c(x^4)}{x^4}=0$ vyplýva z definície symboli $c(x^4)$).

\begin{enumerate}[resume]
	\item \useproblem[diferencialny-pocet]{diferencialny-pocet-391}
\end{enumerate}

\begin{veta}
Nech je daná funkcia $f$, nech funkcia $f^{(n)}$ je definovaná a spojitá v niektorom okolí $O(a)$ bodu $a\in\mathbb{R}$, nech pre každé $x\in O(a)\setminus \{a\}$ existuje vlastná $f^{(n+1)}(x)$. Nech $T_n$ je Taylorov polynóm stupňa $n$ funkcie $f$ v bode $a$. Potom
\begin{itemize}
\item pre každé $x\in O(a),x>a$ $(x\in O(a),x<a)$ existuje číslo $\vartheta (x)\in (a,x)$  $(\vartheta (x)\in (x,a))$ také, že
$$f(x)-T_n(x)=\frac{f^{(n+1)}(\vartheta (x))}{(n+1)!}(x-a)^{n+1}$$
(tzv. Lagrangeov tvar zvyšku);
\item pre každé $x\in O(a),x>a$  $(x\in O(a),x<a)$ existuje číslo $\vartheta (x)\in (a,x)$  $\vartheta (x)\in (a,x)$  $(\vartheta (x)\in (x,a))$ také, že
$$f(x)-T_n(x)=\frac{f^{(n+1)}(\vartheta (x))}{n!}(x-a)(x-\vartheta (x))^n$$
(tzv. Cauchyho tvar zvyšku).
\end{itemize}
\end{veta}

\begin{enumerate}[resume]
	\item \useproblem[diferencialny-pocet]{diferencialny-pocet-392}
	\item \useproblem[diferencialny-pocet]{diferencialny-pocet-393}
\end{enumerate}

\subsection{Použitie diferenciálneho počtu pri zostrojovaní grafov funkcií}
Pri zostrojovaní grafu funkcie $f$ postupujeme spravidla nasledovne:
\begin{itemize}
\item určíme $D(f)$;
\item nájdeme všetky hodnoty $x$, pre ktoré $f(x)=0$ (t.j. priesečníky grafu funkcie $f$ s osou $Ox$);
\item vyšetríme spojitosť funkcie $f$ a jej správanie sa v bodoch nespokojnosti;
\item zistíme, na ktorých intervaloch je $f$ monotónna, a nájdeme body, v ktorých nadobúda lokálne extrémy;
\item vyšetríme konvexnosť a konkávnosť funkcie $f$, nájdeme inflexné body;
\item nájdeme asymptoty grafu funkcie (definíciu asymptoty pozri ďalej).
\end{itemize}
Zostrojenie grafu funkcie $f$ môžu uľahčiť niektoré jej špeciálne vlastnosti: pri párnej alebo nepárnej funkcii stačí zostrojiť graf funkcie $f/D(f)\cap \langle 0,+\infty)$, v prípade periodickej funkcie $f$ graf funkcie $f/\langle a,a+T\rangle cap D(f)$ a $T$ je niektorá perióda funkcie $f$.

(Nech je daná funkcia $f$, nech $a\in\mathbb{R}$ je hromadný bod množiny $D(f)\cap (a,+\infty)$) (množiny $D(f)\cap (-\infty,a)$). Ak existuje $\lim_{x\rightarrow a+}f(x)$  $(\lim_{x\rightarrow a-}f(x))$ a je nevlastná, nazýva sa priamka $x=ay$ asymptotou bez smernice grafu funkcie $f$.

Nech bod $+\infty$ je hromadný bod definičného oboru funkcie $f$. Priamka $y=kx+q$ sa nazýva asymptota so smernicou grafu funkcie $f$ v bode $+\infty$, ak $\lim_{x\rightarrow \infty}(f(x)-kx-q)=0$.

Analogicky sa definuje asymptota so smernicou grafu funkcie $f$ v bode $-\infty$.

\begin{veta}
Nech bod $+\infty$ je hromadný bod definičného oboru funkcie $f$. Priamka $y=kx+q$ je asymptotou so smernicou grafu funkcie $f$ v bode $+\infty$ práve vtedy, keď:
\begin{itemize}
\item existuje konečná $\lim_{x\rightarrow \infty} \frac{f(x)}{x}$ a platí $\lim_{x\rightarrow \infty} \frac{f(x)}{x}=k$;
\item existuje konečná $\lim_{x\rightarrow \infty} (f(x)-kx)$ a rovná sa číslu $q$.

(Analogická veta platí pre asymptotu so smernicou grafu funkcie $f$ v bode $-\infty$.)
\end{itemize}
\end{veta}

Zostrojte grafy nasledujúcich funkcií:
\begin{enumerate}[resume]
	\item \useproblem[diferencialny-pocet]{diferencialny-pocet-394}
	\item \useproblem[diferencialny-pocet]{diferencialny-pocet-395}
	\item \useproblem[diferencialny-pocet]{diferencialny-pocet-396}
	\item \useproblem[diferencialny-pocet]{diferencialny-pocet-397}
	\item \useproblem[diferencialny-pocet]{diferencialny-pocet-398}
	\item \useproblem[diferencialny-pocet]{diferencialny-pocet-399}
	\item \useproblem[diferencialny-pocet]{diferencialny-pocet-400}
	\item \useproblem[diferencialny-pocet]{diferencialny-pocet-401}
	\item \useproblem[diferencialny-pocet]{diferencialny-pocet-402}
	\item \useproblem[diferencialny-pocet]{diferencialny-pocet-403}
	\item \useproblem[diferencialny-pocet]{diferencialny-pocet-404}
	\item \useproblem[diferencialny-pocet]{diferencialny-pocet-405}
	\item \useproblem[diferencialny-pocet]{diferencialny-pocet-406}
	\item \useproblem[diferencialny-pocet]{diferencialny-pocet-407}
	\item \useproblem[diferencialny-pocet]{diferencialny-pocet-408}
	\item \useproblem[diferencialny-pocet]{diferencialny-pocet-409}
	\item \useproblem[diferencialny-pocet]{diferencialny-pocet-410}
	\item \useproblem[diferencialny-pocet]{diferencialny-pocet-411}
\end{enumerate}

\subsection{Ďalšie príklady}
\begin{enumerate}[resume]
	\item \useproblem[diferencialny-pocet]{diferencialny-pocet-412}
	\item \useproblem[diferencialny-pocet]{diferencialny-pocet-413}
	\item \useproblem[diferencialny-pocet]{diferencialny-pocet-414}
	\item \useproblem[diferencialny-pocet]{diferencialny-pocet-415}
	\item \useproblem[diferencialny-pocet]{diferencialny-pocet-416}
	\item \useproblem[diferencialny-pocet]{diferencialny-pocet-417}
	\item \useproblem[diferencialny-pocet]{diferencialny-pocet-418}
	\item \useproblem[diferencialny-pocet]{diferencialny-pocet-419}
	\item \useproblem[diferencialny-pocet]{diferencialny-pocet-420}
	\item \useproblem[diferencialny-pocet]{diferencialny-pocet-421}
	\item \useproblem[diferencialny-pocet]{diferencialny-pocet-422}
	\item \useproblem[diferencialny-pocet]{diferencialny-pocet-423}
	\item \useproblem[diferencialny-pocet]{diferencialny-pocet-424}
	\item \useproblem[diferencialny-pocet]{diferencialny-pocet-425}
	\item \useproblem[diferencialny-pocet]{diferencialny-pocet-426}
	\item \useproblem[diferencialny-pocet]{diferencialny-pocet-427}
	\item \useproblem[diferencialny-pocet]{diferencialny-pocet-428}
	\item \useproblem[diferencialny-pocet]{diferencialny-pocet-429}
	\item \useproblem[diferencialny-pocet]{diferencialny-pocet-430}
	\item \useproblem[diferencialny-pocet]{diferencialny-pocet-431}
	\item \useproblem[diferencialny-pocet]{diferencialny-pocet-432}
	\item \useproblem[diferencialny-pocet]{diferencialny-pocet-433}
	\item \useproblem[diferencialny-pocet]{diferencialny-pocet-434}
	\item \useproblem[diferencialny-pocet]{diferencialny-pocet-435}
	\item \useproblem[diferencialny-pocet]{diferencialny-pocet-436}
	\item \useproblem[diferencialny-pocet]{diferencialny-pocet-437}
	\item \useproblem[diferencialny-pocet]{diferencialny-pocet-438}
	\item \useproblem[diferencialny-pocet]{diferencialny-pocet-439}
	\item \useproblem[diferencialny-pocet]{diferencialny-pocet-440}
	\item \useproblem[diferencialny-pocet]{diferencialny-pocet-441}
	\item \useproblem[diferencialny-pocet]{diferencialny-pocet-442}
	\item \useproblem[diferencialny-pocet]{diferencialny-pocet-443}
	\item \useproblem[diferencialny-pocet]{diferencialny-pocet-444}
	\item \useproblem[diferencialny-pocet]{diferencialny-pocet-445}
	\item \useproblem[diferencialny-pocet]{diferencialny-pocet-446}
	\item \useproblem[diferencialny-pocet]{diferencialny-pocet-447}
	\item \useproblem[diferencialny-pocet]{diferencialny-pocet-448}
	\item \useproblem[diferencialny-pocet]{diferencialny-pocet-449}
	\item \useproblem[diferencialny-pocet]{diferencialny-pocet-450}
	\item \useproblem[diferencialny-pocet]{diferencialny-pocet-451}
	\item \useproblem[diferencialny-pocet]{diferencialny-pocet-452}
	\item \useproblem[diferencialny-pocet]{diferencialny-pocet-453}
	\item \useproblem[diferencialny-pocet]{diferencialny-pocet-454}
	\item \useproblem[diferencialny-pocet]{diferencialny-pocet-455}
	\item \useproblem[diferencialny-pocet]{diferencialny-pocet-456}
	\item \useproblem[diferencialny-pocet]{diferencialny-pocet-457}
	\item \useproblem[diferencialny-pocet]{diferencialny-pocet-458}
	\item \useproblem[diferencialny-pocet]{diferencialny-pocet-459}
	\item \useproblem[diferencialny-pocet]{diferencialny-pocet-460}
	\item \useproblem[diferencialny-pocet]{diferencialny-pocet-461}
	\item \useproblem[diferencialny-pocet]{diferencialny-pocet-462}
	\item \useproblem[diferencialny-pocet]{diferencialny-pocet-463}
	\item \useproblem[diferencialny-pocet]{diferencialny-pocet-464}
	\item \useproblem[diferencialny-pocet]{diferencialny-pocet-465}
	\item \useproblem[diferencialny-pocet]{diferencialny-pocet-466}
	\item \useproblem[diferencialny-pocet]{diferencialny-pocet-467}
	\item \useproblem[diferencialny-pocet]{diferencialny-pocet-468}
	\item \useproblem[diferencialny-pocet]{diferencialny-pocet-469}
	\item \useproblem[diferencialny-pocet]{diferencialny-pocet-470}
	\item \useproblem[diferencialny-pocet]{diferencialny-pocet-471}
	\item \useproblem[diferencialny-pocet]{diferencialny-pocet-472}
	\item \useproblem[diferencialny-pocet]{diferencialny-pocet-473}
	\item \useproblem[diferencialny-pocet]{diferencialny-pocet-474}
	\item \useproblem[diferencialny-pocet]{diferencialny-pocet-475}
	\item \useproblem[diferencialny-pocet]{diferencialny-pocet-476}
	\item \useproblem[diferencialny-pocet]{diferencialny-pocet-477}
	\item \useproblem[diferencialny-pocet]{diferencialny-pocet-478}
\end{enumerate}
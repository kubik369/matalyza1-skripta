\chapter{Diferenciálny počet funkcií jednej premennej}%\label{chapter:gramatiky}

\section{Definícia derivácie}
Hovoríme, že funkcia $f$ (definovaná v okolí bodu $a$ \footnote{t.j. pre niektoré $\varepsilon>0$ platí $(a-\varepsilon,a+\varepsilon) \subset D(f)$; pojmy derivácie a nevlastnej derivácie a nevlastnej derivácie v bode $a$ by bolo možné zaviesť aj za slabšieho predpokladu "$a \in D(f)$ je hromadný bod množiny $D(f)$" (odstránili by sa tým aj niektoré ťažkosti so zavedením pojmu derivácie ako funkcie), definícia predpokladajúca, že $a$ je vnútorný bod množiny $D(f)$, je však v literatúre najčastejšia}) má deriváciu v bode $a,(a \in \mathbb{R})$, ak existuje konečná $\lim_{x \rightarrow a}\frac{f(x)-f(a)}{x-a}$. Ak existuje nevlastná $\lim_{x \rightarrow a}\frac{f(x)-f(a)}{x-a}$, hovoríme, že funkcia $f$ má nevlastnú (alebo nekonečnú) deriváciu v bode $a$. Hodnotu $\lim_{x \rightarrow a}\frac{f(x)-f(a)}{x-a}$ v obidvoch týchto prípadoch označujeme $f'(a)$. Ak nás nezaujíma, či je $\lim_{x \rightarrow a}\frac{f(x)-f(a)}{x-a}$ konečná alebo nekonečná. používame spoločný názov vlastná alebo nevlastná derivácia v bode $a$.
Pojmy derivácie sprava a nevlastnej derivácie sprava, resp. derivácie zľava a nevlastnej derivácie zľava dostaneme, ak v predchádzajúcich definíciách $\lim_{x \rightarrow a}\frac{f(x)-f(a)}{x-a}$ nahradíme limitou $\lim_{x \rightarrow a+}\frac{f(x)-f(a)}{x-a}$, resp. $\lim_{x \rightarrow a-}\frac{f(x)-f(a)}{x-a}$ (v takom prípade stačí predpokladať, že definičný obor funkcie $f$ obsahuje interval $\langle a,a+\varepsilon), resp. (a-\varepsilon,a\rangle$ pre niektoré $\varepsilon>0$). Hodnoty $\lim_{x \rightarrow a+}\frac{f(x)-f(a)}{x-a}$, resp. $\lim_{x \rightarrow a-}\frac{f(x)-f(a)}{x-a}$ označujeme $f'_+(a)$,resp. $f'_-(a)$.

\begin{veta}
Funkcia $f$ (definovaná v okolí bodu $a \in \mathbb{R}$) má vlastnú alebo nevlastnú deriváciu $f'(a)$ práve vtedy, keď existujú $f'_+(a),f'_-(a)$ a platí $f'_+(a)=f'_-(a)$. Hodnota $f'(a)$ sa pritom rovná spoločnej hodnote $f'_+(a)$ a $f'_-(a)$.
\end{veta}

Pojem derivácie ako funkcie sa vo všeobecnosti definuje nasledovne: Nech $M$ je množina všetkých bodov definičného oboru $D(f)$, v ktorých má funkcia $f$ deriváciu. Funkcia $f':M \rightarrow \mathbb{R}$, ktorá každému bodu $a$  $M$ priradí hodnotu $f'(a)$ derivácie funkcie $f$ v bode $a$, sa nazýva derivácia funkcie $f$. V špeciálnych prípadoch, ktoré ilustruje nasledujúca poznámka, sa niekedy pojem derivácie ako funkcie chápe širšie: ak definičným oborom funkcie $f$ je niektorý z intervalov $\langle a,b \rangle, \langle a,b),(a,b \rangle$, pričom v jeho koncovom bode existuje jednostranná derivácia (v prvom prípade prichádzajú do úvahy body $a,b$, v druhom bod $a$, v trečom bod $b$), tak funkciu $f'$ považujeme za definovanú aj v tomto bode, jej funkčnou hodnotou je hodnota príslušnej jednostrannej derivácie.

\begin{enumerate}[resume]
	\item \useproblem[diferencialny-pocet]{diferencialny-pocet-1}
	\item \useproblem[diferencialny-pocet]{diferencialny-pocet-2}
	\item \useproblem[diferencialny-pocet]{diferencialny-pocet-3}
	\item \useproblem[diferencialny-pocet]{diferencialny-pocet-4}
	\item \useproblem[diferencialny-pocet]{diferencialny-pocet-5}
	\item \useproblem[diferencialny-pocet]{diferencialny-pocet-6}
	\item \useproblem[diferencialny-pocet]{diferencialny-pocet-7}
\end{enumerate}

\begin{veta}
Ak funkcie $f,g$ majú derivácie v bode $a$, tak aj funkcie $c \cdot f$ ($c$ je reálna konštanta), $f+g,f-g,f\cdot g$ majú v bode $a$ derivácie a platí:
\begin{enumerate}
\item $(c\cdot f)'(a)=c\cdot f'(a)$,
\item $(f+g)'(a)=f'(a)+g'(a)$,
\item $(f-g)'(a)=f'(a)-g'(a)$,
\item $(f\cdot g)'(a)=f'(a)\cdot g'(a)$.
\end{enumerate}
Ak naviac $g(a)\neq 0$ majú v bode $a$ deriváciu aj funkcie $\frac{1}{g},\frac{f}{g}$ a platí
\begin{enumerate}
\item $(\frac{1}{g})'(a)=-\frac{g'(a)}{g^2(a)}$,
\item $(\frac{f}{g})'(a)=\frac{f'(a)\cdot g(a)-f(a)\cdot g'(a)}{g^2 (a)}$.
\end{enumerate}
\end{veta}

\begin{veta}
Ak funkcia $f$ má deriváciu v bode $a$, funkcia $g$ v bode $f(a)$ a zložená funkcia $h=g \circ f$ je definovaná v okolí bodu $a$, tak $h$ má v bode $a$ deriváciu a platí $$h'(a)=f'(a)\cdot g'(f(a)).$$
(Analogické vety možno dokázať aj pre jednostranné derivácie.)
\end{veta}

V nasledujúcej tabuľke sú derivácie základných elementárnych funkcií ($c$ je reálna konštanta):
\begin{itemize}
\item $c'=0$,
\item $(x^n)'=n\cdot x^{n-1},(n \in \mathbb{R} \setminus \{0\})$,
\item $(\sin x)'=\cos x$,
\item $(\cos x)'=-\sin x$,
\item $(\tan x)'=\frac{1}{\cos^2 x}$,
\item $(a^x)'=a^x \cdot \ln a$, špeciálne $(e^x)'=e^x$,
\item $(\log_{a} x)'=\frac{1}{x\cdot \ln a},x>0$, špeciálne $(\ln x)'=\frac{1}{x},x>0$,
\item $(\arcsin x)'=\frac{1}{\sqrt{1-x^2}}$,
\item $(\arccos x)'=-\frac{1}{\sqrt{1-x^2}}$,
\item $(\arctan x)'=\frac{1}{1+x^2}$,
\item $(arcctan x)'=-\frac{1}{1+x^2}$.
\end{itemize} 

Pre deriváciu hyperbolických funkcií platia nasledujúce vzorce
\begin{multicols}{2}
\begin{itemize}
   \item $(sh x)'=ch x$,
   \item $(ch x)'=sh x$,
   \item $(th x)'=\frac{1}{ch^2 x}$,
   \item $(cth x)'=-\frac{1}{sh^2 x}$.
\end{itemize}
\end{multicols}

Ak pri hľadaní derivácie funkcie využívame len znalosť derivácií základných elementárnych funkcií a vety o derivácii súčtu, rozdielu, súčinu, podielu a zloženej funkcie, nazýva sa taký postup tabuľkovým derivovaním.

\begin{enumerate}[resume]
	\item \useproblem[diferencialny-pocet]{diferencialny-pocet-8}
	\item \useproblem[diferencialny-pocet]{diferencialny-pocet-9}
	\item \useproblem[diferencialny-pocet]{diferencialny-pocet-10}
	\item \useproblem[diferencialny-pocet]{diferencialny-pocet-11}
	\item \useproblem[diferencialny-pocet]{diferencialny-pocet-12}
	\item \useproblem[diferencialny-pocet]{diferencialny-pocet-13}
	\item \useproblem[diferencialny-pocet]{diferencialny-pocet-14}
\end{enumerate}

Nájdite derivácie funkcií:

\begin{enumerate}[resume]
	\item \useproblem[diferencialny-pocet]{diferencialny-pocet-15}
	\item \useproblem[diferencialny-pocet]{diferencialny-pocet-16}
	\item \useproblem[diferencialny-pocet]{diferencialny-pocet-17}
\end{enumerate}

\textit{Riešenie (a):}
Výpočet sa zjednoduší nasledujúcou úvahou: ak funkcia $f$ má v bode $a$ deriváciu a $f(a)\neq 0$, tak $(\ln |f|)'(a)=\frac{1}{f(a)}\cdot f'(a)$; odtiaľ možno vyjadriť 
$$f'(a)=f(a)\cdot (\ln |f|)'(a).$$
V našom prípade $y=\frac{1+x^2}{\sqrt[3]{x^4}\cdot\sin^7 x}\cdot (\ln |y|)'=(\ln |1+x^2|-\frac{4}{3}\ln |x|-7\cdot\ln |\sin x|)'=\frac{2x}{1+x^2}-\frac{4}{3x}-7\cdot \cot x=\frac{2x^2-4}{3x\cdot (1+x^2)}-7\cot x$. Teda $y'=\frac{1+x^2}{\sqrt[3]{x^4}\cdot \sin^7 x}\cdot (\frac{2x^2-4}{3x\cdot (1+x^2)}-7\cdot\cot x)$.

\textit{Riešenie (d):}
Funkcie tvaru $f^g$ (o funkcii $f$ predpokladáme, že je nezáporná) možno derivovať na základe úvahy z riešenia príkladu $295.(a)$ alebo (čo je vlastne to isté) prepísať ich predpis do podoby $e^{g\cdot\ln f}$ a takto zapísanú funkciu potom derivovať na základe viet o derivácii súčinu a zloženej funkcie.

V našom prípade $(x^x)'=(e^{x\cdot\ln x})'=e^{x\cdot\ln x}\cdot(x\cdot\ln x)'=x^x\cdot(\ln x +1)$.
\begin{enumerate}[resume]
	\item \useproblem[diferencialny-pocet]{diferencialny-pocet-18}
	\item \useproblem[diferencialny-pocet]{diferencialny-pocet-19}
	\item \useproblem[diferencialny-pocet]{diferencialny-pocet-20}
	\item \useproblem[diferencialny-pocet]{diferencialny-pocet-21}
	\item \useproblem[diferencialny-pocet]{diferencialny-pocet-22}
	\item \useproblem[diferencialny-pocet]{diferencialny-pocet-23}
	\item \useproblem[diferencialny-pocet]{diferencialny-pocet-24}
	\item \useproblem[diferencialny-pocet]{diferencialny-pocet-25}
	\item \useproblem[diferencialny-pocet]{diferencialny-pocet-26}
	\item \useproblem[diferencialny-pocet]{diferencialny-pocet-27}
	\item \useproblem[diferencialny-pocet]{diferencialny-pocet-28}
\end{enumerate}

\begin{veta}
Ak funkcia $f$ má v bode $a$ deriváciu, tak $f$ je v tomto bode spojitá.
\end{veta}

\begin{enumerate}[resume]
	\item \useproblem[diferencialny-pocet]{diferencialny-pocet-29}
	\item \useproblem[diferencialny-pocet]{diferencialny-pocet-30}
	\item \useproblem[diferencialny-pocet]{diferencialny-pocet-31}
\end{enumerate}

\section{Derivácia inverznej funkcie}
\begin{veta}
Nech funkcia $f:\mathbb{I}\rightarrow\mathbb{R}$, rýdzomonotónna na intervale $I$, má deriváciu v bode $a$. Potom inverzná funkcia $f^{-1}$ má vlastnú alebo nevlastnú deriváciu v bode $f(a)$, pričom platí
\begin{enumerate}
\item ak $f'(a)\neq 0$, tak $(f^{-1})'(f(a))=\frac{1}{f'(a)}$; 
\item ak $f'(a)=0$ a $f$ je rastúca, tak $(f^{-1})'(f(a))=+\infty$;
\item ak $f'(a)=0$ a $f$ je klesajúca, tak $(f^{-1})'(f(a))=-\infty$.
\end{enumerate}
\end{veta}

\begin{enumerate}[resume]
	\item \useproblem[diferencialny-pocet]{diferencialny-pocet-32}
\end{enumerate}

\textit{Riešenie (a):}
Pretože $b=f(0)$ a $f'(x)=1+x^4$, je podľa vety $5$ $(f^{-1})'(b)=\frac{1}{f'(0)}=1$. (Ako sa presvedčíme o rýdzomonotónnej funkcie $f$, ukážeme neskôr-pozri vetu $11.$)

\begin{enumerate}[resume]
	\item \useproblem[diferencialny-pocet]{diferencialny-pocet-33}
	\item \useproblem[diferencialny-pocet]{diferencialny-pocet-34}
	\item \useproblem[diferencialny-pocet]{diferencialny-pocet-35}
\end{enumerate}

\section{Diferenciál}
Hovoríme, že funkcia $f$ (definovaná v okolí bodu $a$ \footnote{rovnako ako deriváciu v bode $a$ možno aj diferenciál v bode $a$ definovať za slabšieho predpokladu "$a\in D(f)$ je hromadný bod množiny $D(f)$"}) má v bode $a$ diferenciál (je diferencovateľná v bode $a$), ak existuje reálna konštanta $A$ taká, že pre funkciu $\omega$, definovanú vzťahom
$$f(x)=f(a)+A(x-a)+\omega(x)$$ platí $\lim_{x \rightarrow a}\frac{\omega(x)}{x-a}=0$.
Funkcia definovaná predpisom $y=A(x-a)$ sa v takom prípade označuje $df(a)$ a nazýva sa diferenciál funkcie $f$ v bode $a$. Funkcia $df(a)$ sa zvyčajnezapisuje v tvare $df(a)=A$ $dx(a)$ \footnote{písmeno $a$ sa v zápisoch často vynecháva,preto sa možno stretnúť aj so zápisom $df=A$ $dx$}, kde symbol $dx(a)$ (označujúci diferenciál funkcie $g(x)=x$ v bode $a$,t.j. funkciu danú predpisom $y=x-a$) sa nazýva diferenciál nezávislej premennej.
\begin{veta}
Funkcia $f$ je diferencovateľná v bode $a$ práve vtedy, keď $f$ má v bode $a$ deriváciu; pritom platí $A=f'(a)$, kde $A$ je konštanta z definície diferenciálu funkcie $f$ v bode $a$.
\end{veta}

Graf funkcie $y=f(a)+f'(a)(x-a)$ je dotyčnicou v bodu $(a,f(a))$ ku grafu funkcir $f$. Ak je funkcia $f$ spojitá v bode $a$, pričom $f'(a)$ je nevlastná, je dotyčnicou v bode $(a,f(a))$ ku grafu funkcie $f$ priamka $x=a$.

\begin{enumerate}[resume]
	\item \useproblem[diferencialny-pocet]{diferencialny-pocet-36}
	\item \useproblem[diferencialny-pocet]{diferencialny-pocet-37}
	\item \useproblem[diferencialny-pocet]{diferencialny-pocet-38}  
	\item \useproblem[diferencialny-pocet]{diferencialny-pocet-39}
	\item \useproblem[diferencialny-pocet]{diferencialny-pocet-40}
	\item \useproblem[diferencialny-pocet]{diferencialny-pocet-41}
\end{enumerate}

\textit{Riešenie (e):}
Funkcia daná predpisom $f(x)=\ln (\tan \frac{x}{2})$ má v bode $\frac{\pi}{2}$ deriváciu $f'(\frac{\pi}{2})=1$, preto je podľa vety $6$ diferencovateľná a možno ju teda písať v tvare $f(x)=f(\frac{\pi}{2})+f'(\frac{\pi}{2})\cdot (x-\frac{\pi}{2})+\omega(x)=(x-\frac{\pi}{2})+\omega(x)$, kde $\lim_{x \rightarrow \frac{\pi}{2}}\frac{\omega(x)}{x-\frac{\pi}{2}}=0$. Zanedbaním funkcie $\omega(x)$ dostaneme približný vzorec $\ln (\tan \frac{x}{2})\approx (x-\frac{\pi}{2})$ pre $x$ blízke číslu $\frac{\pi}{2}$. (Geometricky to znamená, že hodnoty funkcie $f$ nahrádzame funkčnými hodnotami jej dotyčnice v bode $(\frac{\pi}{2},0)$, $\omega(x)$ vyjadruje chybu, ktorej sa pri takomto nahradení dopúšťame.)

\begin{enumerate}[resume]
	\item \useproblem[diferencialny-pocet]{diferencialny-pocet-42}
\end{enumerate}

\section{Derivácie vyšších rádov}
Derivácie vyšších rádov definujeme rekurentne: Nech funkcia $f$ je definovaná v okolí bodu $a\in\mathbb{R}$; označme $f^{(0)}:=f$. Hovoríme,že funkcia $f$ má n-tú deriváciu v bode $a$, reps. nevlastnú n-tú deriváciu v bode $a$, ak existuje derivácia, resp. nevlastná derivácia funkcie $f^{(n-1)}$ v bode $a$. n-tú deriváciu v bode $a$ aj nevlastnú n-tú deriváciu v bode $a$   označujeme $f^{(n)}(a)$. (Analogicky sa definujú vlastné a nevlastné jednostranné n-té derivácie v bode $a$.) Derivácia funkcie $f^{(n-1)}$ sa nazýva n-tá derivácia funkcie $f$ a označuje sa $f^{(n)}$. Okrem označení $f^{(1)},f^{(2)},f^{(3)},f^{(4)},f^{(5)},...$ sa používajú aj označenia $f',f'',f''',f''''$ (alebo $f^{VI}$),$f^{V},...$ . 
Platia nasledujúce vzťahy:
\begin{multicols}{2}
\begin{enumerate}
    \item $(x^m)^{(n)}=\left\{ \begin{array}{r@{\quad}c}
    \frac{m!x^{m-n}}{(m-n)!},& $ak $ n\leq m \\
    0, &  $ak $ n>m \\ \end{array} \right.
    $,$(m\in\mathbb{N})$;
    \item $(\sin x)^{(n)}=\sin (x+\frac{n\pi}{2})$;
    \item $(\cos x)^{(n)}=\cos (x+\frac{n\pi}{2})$;
    \item $(a^x)^{(n)}=a^x\ln^n a$;
    \item $(e^x)^{(n)}=e^x$;
    \item $(\log_a x)^{(n)}=\frac{(-1)^{n-1}(n-1)!}{x^n\ln^n a},x>0$;
    \item $(\ln x)^{(n)}=\frac{(-1)^{n-1}(n-1)!}{x^n},x>0$.
\end{enumerate}
\end{multicols}

\begin{veta}
\textit{Leibnitzov vzorec}
Ak funkcie $f,g$ majú n-tú deriváciu v bode $a$, tak existuje $(f\cdot g)^{(n)}(a)$ a platí
$$(f\cdot g)^{(n)}(a)=\sum_{k=0}^n {n \choose k} f^{k}(a)\cdot g^{(n-k)}(a). $$
\end{veta}

\begin{enumerate}[resume]
	\item \useproblem[diferencialny-pocet]{diferencialny-pocet-43}
	\item \useproblem[diferencialny-pocet]{diferencialny-pocet-44}
\end{enumerate}

\textit{Riešenie (a):}
Funkcia $y$ je polynóm $6.$ stupňa, teda $y=a_0x^6+a_1x^5+...+a_6$ nie je ťažké vypočítať, že $a_0=4$. Potom $y^{VI}=(a_0x^6+a_1x^5+...+a_6)^{VI}=a_0(x^6)^{VI}+a_1(x^5)^{VI}+...+a_5x^{VI}=6!;a_0=6;4=2880$. $y^{VII}=0$ (rád derivácie je vyšší ako stupeň polynómu).

\textit{Riešenie (e):}
Funkciu $y$ možno zapísať v podobe, ktorá je pre derivovanie výhodnejšia:
$$y=\frac{2x+1}{x^2+x-2}=\frac{2x+1}{(x+2)(x-1)}=\frac{(x+2)+(x-1)}{(x+2)(x-1)}=\frac{1}{x-1}+\frac{1}{x+2}.$$
Potom $y^{(13)}=((x-1)^{-1})^{(13)}+((x+2)^{-1})^{(13)}=(-1)^{13}\cdot 13!(x-1)^{-14}+(-1)^{13}13!(x+2)^{-14}=-(13!)(\frac{1}{(x-1)^{14}}+\frac{1}{(x+2)^{14}})$.

\begin{enumerate}[resume]
	\item \useproblem[diferencialny-pocet]{diferencialny-pocet-45}
	\item \useproblem[diferencialny-pocet]{diferencialny-pocet-46}
	\item \useproblem[diferencialny-pocet]{diferencialny-pocet-47}
	\item \useproblem[diferencialny-pocet]{diferencialny-pocet-48}
	\item \useproblem[diferencialny-pocet]{diferencialny-pocet-49}
\end{enumerate}

\textit{Riešenie (f):}
Pretože $f'(x)=\frac{1}{1+x^2}$, platí $(1+x^2)f'(x)=1$ pre všetky $x \in\mathbb{R}$. To znamená, že funkcia $(1+x^2)f'(x)$ je na $\mathbb{R}$ konštantná, preto jej derivácie všetkých rádov sú rovné $0$:
$$[(1+x^2)f'(x)]^{(k)}=0,(k\in\mathbb{N}).$$
Položme $k=n-1$; pre $n=2$ má uvedená rovnosť tvar $(*)$
$$(1+x^2)f''(x)+2x\cdot f'(x)=0$$
a pre $n\geq 3$ použitím Leibnitzovho vzorca dostaneme
$$(1+x^2)f^{(n)}(x)+2(n-1)x\cdot f^{(n-1)}(x)+(n-1)(n-2)f^{(n-2)}(x)=0.$$
(Až pre $k/geq 2$,t.j. $n\geq 3$, obsahuje Leibnitzov vzorec všetky nenulové derivácie funkcie $1+x^2$, zápis prvej derivácie funkcie $(1+x^2)f'(x)$ sa teda odlišuje od zápisu k-tej derivácie tejto funkcie pre $k\geq 2$, preto treba prípad $k=1$ robiť samostatne.)

Ak v poslednej rovnosti položíme $x=0$, dostávame pre $n\geq 3$
$$f^{(n)}(0)=-(n-1)(n-2)f^{(n-2)}(0),$$
čo je rekurentný vzťah pre vyjadrenie $f^{(n)}(0)$. Dosadením do predpisu pre $f'$ dostaneme $f'(0)=1$, z $(*)$ vychádza $f''(0)=0$. Na základe toho môžeme matematickou indukciou dokázať. že 
$f^{(2k+2)}(0)=0,f^{(2k+1)}(0)=(-1)^k(2k)$; pre $k=0,1,2,...$ .

\textit{Poznámka:}
Prechádzajúca úvaha má (našťastie len zdanlivo) jeden nedostatok: ak chceme použiť Leibnitzov vzorec na výpočet $(n-1)$-vej derivácie súčinu $(1+x^2)\cdot f'(x)$, treba sa najprv presvedčiť, že existujú $(n-1)$-vé derivácie funkcií $(1+x^2)$ a $f'(x)$ (t.j. $(1+x^2)^{(n-1)}$ a $f^{(n)}(x)$); to sme ale neurobili.

Dokázať, že existuje $(1+x^2)^{(n-1)}$, je ľahké; zostávateda dokázať existenciu $(\arctan x)^{(n)}$. (Návod: možno postupovať indrukciou; z predpokladu, že $f^{(k-1)}$ je podielom polynómov $P_{k-1}$ a $Q_{k-1}$ nemá reálne korene, vyplýva na základe vety o derivácii podielu, že $f^{(k)}$ existuje a možno ju zapísať v tvare podielu polynómov $P_k$ a $Q_k$, pričom $Q_k$ nemá reálne korene.)

\begin{enumerate}[resume]
	\item \useproblem[diferencialny-pocet]{diferencialny-pocet-50}
\end{enumerate}

\section{Základné vety diferenciálneho počtu}
\begin{veta}
\textit{(Rolle).}
Nech funkcia $f:\langle a,b \rangle\rightarrow \mathbb{R}$ vyhovuje nasledujúcim podmienkam:
\begin{itemize}
\item $f$ je spojitá na intervale $\langle a,b \rangle$;
\item v každom bode $x\in (a,b)$ existuje vlastná alebo nevlastná $f'(x)$; 
\item $f(a)=f(b)$.
\end{itemize}
Potom existuje bod $c\in (a,b)$, v ktorom $f'(c)=0$.
\end{veta}

\begin{enumerate}[resume]
	\item \useproblem[diferencialny-pocet]{diferencialny-pocet-51}
	\item \useproblem[diferencialny-pocet]{diferencialny-pocet-52}
	\item \useproblem[diferencialny-pocet]{diferencialny-pocet-53}
	\item \useproblem[diferencialny-pocet]{diferencialny-pocet-54}
	\item \useproblem[diferencialny-pocet]{diferencialny-pocet-55}
	\item \useproblem[diferencialny-pocet]{diferencialny-pocet-56}
	\item \useproblem[diferencialny-pocet]{diferencialny-pocet-57}
\end{enumerate}

\begin{veta}
\textit{(Lagrangeova veta o strednej hodnote.)}
Nech funkcia $f:\langle a,b \rangle\rightarrow\mathbb{R}$ vyhovuje nasledujúcim podmienkam:
\begin{itemize}
\item $f$ je spojitá;
\item v každom bode $x\in (a,b)$ existuje vlastná alebo nevlastná $f'(x)$.
Potom existuje bod $c\in (a,b)$, v ktorom $f'(c)=\frac{f(b)-f(a)}{b-a}.$
\end{itemize}
\end{veta}

\begin{enumerate}[resume]
	\item \useproblem[diferencialny-pocet]{diferencialny-pocet-58}
	\item \useproblem[diferencialny-pocet]{diferencialny-pocet-59}
\end{enumerate}

\textit{Riešenie:}
Derivovaním dostaneme $f'(x)=\frac{1}{1+x^2},x\in\mathbb{R}\setminus \{1\},g'(x)=\frac{1}{1+x^2}$. Označme $h:=f-g$; potom $h'(x)=0$ pre $x\in (1,\infty)$. Podľa výsledku príkladu $336$ je teda funkcia $h$ konštantná na intervale $(-\infty,1)$ a konštantná na intervale $(1,\infty)$. Hodnotu konštanty na intervale $(-\infty,1)$ nájdeme ako funkčnú hodnotu v niektorom vhodne zvolenom bode, napr. $k_1=h(0)=\arctan 1- \arctan 0=\frac{\pi}{4}$. Pretože výpočet $h(c)$ v niektorom $c\in (1,\infty)$ by bol komplikovanejší, pomôžeme si nasledujúcou úvahou: pretože $h(x)=k_2$ na intervale $(1,\infty)$, platí $k_2=\lim_{x\rightarrow 1+}h(x)=-\frac{\pi}{2}-\frac{\pi}{4}=-\frac{3}{4}\pi$. Celkovo teda

$\arctan \frac{1+x}{1-x}-\arctan x = \left\{ \begin{array}{r@{\quad}c}
   \frac{\pi}{4},& $ak $ x<1 \\
    -\frac{3}{4}\pi, &  $ak $ x>1 \\ \end{array} \right.$,
    
    čo môžeme prepísať do podoby $\arctan \frac{1+x}{1-x}-\arctan x=-\frac{\pi}{2} sgn (x-1)-\frac{\pi}{4},x\neq 1$.
    
\begin{enumerate}[resume]
	\item \useproblem[diferencialny-pocet]{diferencialny-pocet-60}
	\item \useproblem[diferencialny-pocet]{diferencialny-pocet-61}
	\item \useproblem[diferencialny-pocet]{diferencialny-pocet-62}
	\item \useproblem[diferencialny-pocet]{diferencialny-pocet-63}
\end{enumerate}

\textit{Riešenie (a):}
Pre $x=y$ Uvedená nerovnosť zrejme platí. Ak $x<y$ (dôkaz pre $x>y$ by bol rovnaký), tak na intervale $\langle x,y \rangle$ funkcia $f(x)=\sin x$ vyhovuje predpokladom Lagrangeovej vety o strednej hodnote, podľa ktorej $\sin s=\sin y=(x-y)\cos c$ pre niektoré $c\in (x,y)$. Pretože $|\cos c|\leq 1$ pre ľubovoľné $c\in\mathbb{R}$, je $|\sin x-\sin y|=|\cos c|\cdot|x-y|\leq |x-y|$.

\begin{enumerate}[resume]
	\item \useproblem[diferencialny-pocet]{diferencialny-pocet-64}
	\item \useproblem[diferencialny-pocet]{diferencialny-pocet-65}
	\item \useproblem[diferencialny-pocet]{diferencialny-pocet-66}
	\item \useproblem[diferencialny-pocet]{diferencialny-pocet-67}
\end{enumerate}
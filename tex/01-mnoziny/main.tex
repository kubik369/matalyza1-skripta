\chapter{Množiny}

\loadallproblems[realne-cisla]{./tex/01-mnoziny/ulohy-realne-cisla.tex}
\loadallproblems[supremum-infimum]{./tex/01-mnoziny/ulohy-supremum-infimum.tex}
\loadallproblems[mohutnost]{./tex/01-mnoziny/ulohy-mohutnost.tex}
\loadallproblems[kompaktne]{./tex/01-mnoziny/ulohy-otvorene-uzavrete-kompaktne.tex}

Ďalej budeme používať tieto označenia:
\begin{itemize}[label=]
  \item $\mathbb{N}$ množina všetkých prirodzených čísel ($= \{ 1, 2, 3, \ldots \}$)
  \item $\mathbb{Z}$ množina všetkých celých čísel ($= \{ 0, -1, 1, 2, -2, \ldots \}$)
  \item $\mathbb{Q}$ množina všetkých racionálnych čísel
    ($= \{ \frac{p}{q}; p \in \mathbb{Z} \land q \in \mathbb{N} \}$)
  \item $\mathbb{R}$ množina všetkých reálny čísel
  \item $\mathbb{R}^+$ množina všetkých kladných reálny čísel ($= ( 0, \infty )$)
  \item $\mathbb{R}^+_0$ množina všetkých nezáporných reálny čísel
    ($= \interval[open right]{0}{\infty}$)
\end{itemize}
Ak pre niektorý prvok $a$ neprázdnej množiny $A \subset \mathbb{R}$ platí
\[
  (\forall x \in A) : x \leq a ((\forall x \in A): q \geq a)
\]
nazývame tento prvok \emph{maximum (minimum) množiny $A$} a označujeme ho
$\max(A)$ ($\min(A)$)

\section{Reálne čísla}
\begin{enumerate}[resume]
  \foreachproblem[realne-cisla]{\item\thisproblem}
\end{enumerate}


\section{Ohraničené množiny, supremum a infimum}
Neprázdna množina $A \subset \mathbb{R}$ sa nazýva \textit{zhora (zdola)
ohraničená}, ak platí
\[
  (\exists K \in  \mathbb{R})
    (\forall x \in A):
      x \leq K
\]
\[
  ((\exists K \in \mathbb{R})
    (\forall x \in A):
      x \geq K)
\]

Číslo $K$ s uvedenou vlastnosťou sa nazýva \textit{horné (dolné)
ohraničenie množiny $A$.} $\emptyset$ považujeme za ohraničenú zhora aj zdola.

Množina, ktorá je zhora aj zdola ohraničená, sa nazýva \textit{ohraničená}.
Množina, ktorá nie je ohraničená sa nazýva \textit{neohraničená}.

\begin{enumerate}[resume]
\item \useproblem[supremum-infimum]{supremum-infimum-14}
\item \useproblem[supremum-infimum]{supremum-infimum-15}
\item \useproblem[supremum-infimum]{supremum-infimum-16}
\end{enumerate}


Číslo $\alpha \in \mathbb{R}$ sa nazýva \textit{supremum množiny $A$}
$\subset \mathbb{R}, A \neq \emptyset$, ak
\[
  (\forall x \in A): x \leq \alpha \label{eq:supremum-i}
\]
\[
  (\forall \varepsilon > 0) (\exists \: x_\varepsilon \in A):
  x_\varepsilon > \alpha - \varepsilon
\]

Podľa \ding{34} je $\alpha$ horné ohraničenie množiny $A$; \ding{37} je negácia
výroku ``pre niektoré $\varepsilon > 0$ je číslo $\alpha - \varepsilon$ horným
ohraničením množiny $A$'', hovorí teda, že neexistuje horné ohraničenie množiny
$A$, ktoré by bolo menšie než $\alpha$. Teda $\alpha$ je najmenšie horné
ohraničenie množiny $A$. Supremum množiny $A$ označujeme $\sup \{ A \}$.

Číslo $\beta \in \mathbb{R}$ sa nazýva \textit{infimum množiny} $A \subset
\mathbb{R}, A \neq \emptyset$, ak:

\begin{align*}
\forall x \in A&: x \geq \beta \tag{\ding{34}} \\
(\forall \varepsilon > 0) (\exists \: x_\varepsilon \in A)&:
x_\varepsilon < \beta + \varepsilon \tag{\ding{37}}
\end{align*}

To znamená, že $\beta$ je najväčšie z dolných ohraniční množiny $A$. Infimum
množiny $A$ označujeme $\inf\{A\}$

Ak usporiadané pole $\mathbb{R}$ reálnych čísel konštruujeme z poľa
$\mathbb{Q}$ racionálnych čísel pomocou Dedekindových rezov, môžeme dokázať
nasledujúce dve ekvivalentné tvrdenia:

\begin{veta}
Každá neprázdna zhora ohraničená množina reálnych čísel má supremum.
\end{veta}

\begin{veta}
Každá neprázdna zdola ohraničená množina reálnych čísel má infimum.
\end{veta}

Ak usporiadané pole $\mathbb{R}$ zavádzame axiomaticky, považujeme prvé z
uvedených tvrdení za axiómu, z nej možno odvodiť vetu o existencií infima.

\showanswers
\begin{enumerate}[resume]
\item \useproblem[supremum-infimum]{supremum-infimum-17}
\hideanswers
\item \useproblem[supremum-infimum]{supremum-infimum-18}
\item \useproblem[supremum-infimum]{supremum-infimum-19}
\item \useproblem[supremum-infimum]{supremum-infimum-20}
\item \useproblem[supremum-infimum]{supremum-infimum-21}
\end{enumerate}


\section{Mohutnosť množín}
Neprázdna množina $A$ sa nazýva konečná, ak pre niektoré prirodzené číslo $n$
existuje bijekcia $\varphi:\{1,...,n\}\rightarrow A$ (t.j. $\varphi$ je prosté
zobrazenie
%\footnote{Ak každému prvku a neprázdnej množiny $A$ priradíme práve
%jeden prvok neprázdnej množiny $B$, ktorý označíme $\varphi (a)$ (množiny $A,B$
%nemusia byť číselné), tak hovoríme, že $\varphi$ je zobrazenie množiny $A$ do
%množiny $B$. Špeciálne, ak $A=\mathbb{N}$, hovoríme, že $\varphi$ je postupnosť
%prvkov z $B$. Zrejme pojmy funkcie a postupnosti uvedené v odseku $1.3$ sú
%špeciálnymi prípadmi tu uvedených pojmov zobrazenie a postupnosť.}
a $\varphi(\{1,...,n\})=A$). Prázdnu množinu pokladáme za konečnú. Množina,
ktorá nie je konečná, sa nazýva nekonečná.

Množina $A$ sa nazýva nekonečne spočítateľná, ak existuje bijekcia
$\varphi:\mathbb{N}\rightarrow A$ (t.j. ak prvky množiny $A$ možno zoradiť do
prostej postupnosti). Konečné a nekonečne spočítateľné množiny sa označujú
spoločným názvom spočítateľné. Množina, ktorá nie je spočítateľná, sa nazýva
nespočítateľná.

\begin{veta}
Každý nedegenerovaný interval je nespočítateľná množina (degenerovanými
intervalmi sa nazývajú jednoprvkové množiny).
\end{veta}

\textbf{Poznámka:} Namiesto dvojice pojmov nekonečne spočítateľná - spočítateľná
sa často v tom istom význame používa dvojica spočítateľná - najviac
spočítateľná.

\begin{enumerate}[resume]
  \item \useproblem[mohutnost]{mohutnost-90}
  \item \useproblem[mohutnost]{mohutnost-91}
  \item \useproblem[mohutnost]{mohutnost-92}
  \item \useproblem[mohutnost]{mohutnost-93}
  \item \useproblem[mohutnost]{mohutnost-94}
  \item \useproblem[mohutnost]{mohutnost-95}
  \item \useproblem[mohutnost]{mohutnost-96}
  \item \useproblem[mohutnost]{mohutnost-97}
  \item \useproblem[mohutnost]{mohutnost-98}
  \item \useproblem[mohutnost]{mohutnost-99}
\end{enumerate}


\section{Otvorené, uzavreté a kompaktné množiny}
Bod $a \in \mathbb{R}$ sa nazýva \textit{vnútorný bod množiny $a \subset
\mathbb{R}$}, ak existuje také jeho okolie $O(a)$, že platí $O(a) \subset A$.

Neprázdna množina $a \subset \mathbb{R}$ sa nazýva \textit{otvorená}, ak každý
jej prvok je jej vnútorným bodom. Ďalej sa dohodneme, že prázdnu množinu budeme
pokladať za otvorenú. Množina $A \subset \mathbb{R}$ sa nazýva
\textit{uzavretá}, ak je $\mathbb{R} \setminus A$ otvorená.

\begin{veta}
Pre ľubovoľnú množinu $A \subset \mathbb{R}$ sú nasledujúce tvrdenia
ekvivalentné:
\begin{enumerate}
\item
    $A$ je uzavretá množina
\item
    ak $a \in \mathbb{R}$ je hromadný bod množiny $A$, tak $a \in A$
\item
    ak všetky členy konvergentnej postupnosti ${\{a_n\}}_{n=1}^\infty$ sú
    prvkami množiny $A$, tak $\lim\limits_{n \rightarrow \infty} a_n \in A$
\end{enumerate}
\end{veta}

\textit{Poznámka:}
Často sa pojem uzavretej množiny definuje pomocou vlastnosti $2.$ alebo $3.$.
Otvorená množina sa niekedy definuje pomocou uzavretej, a to tak, že sa najprv
pomocou niektorej z vlastností $2.$,$3.$ zavedie pojem uzavretej množiny a za
otvorené sa potom vy.hlásia tie množiny $A$, ktorých doplnky $\mathbb{R}
\setminus A$ uzavreté.
\begin{enumerate}[resume]
    \item \useproblem[kompaktne]{kompaktne-208}
    \item \useproblem[kompaktne]{kompaktne-209}
    \item \useproblem[kompaktne]{kompaktne-210}
    \item \useproblem[kompaktne]{kompaktne-211}
    \item \useproblem[kompaktne]{kompaktne-212}
    \item \useproblem[kompaktne]{kompaktne-213}
\end{enumerate}

Systém $\{A_t: t \in I\}$ množín reálnych čísel (neprázdna množina I - nie nutne
číselná - sa nazýva indexová) sa nazýva \textit{pokrytie množiny $A \subset
\mathbb{R}$}, ak
\[
    A \subset \bigcup_{t \in I} A_t (:= \{x \in \mathbb{R};
    (\exists t \in I): x \in A_t\})
\]
Ak je naviac každá z množín $A_t$, $t \in I$,
otvorená, nazýva sa tento systém \textit{otvorené pokrytie množiny A}; termín
\textit{konečné pokrytie} používame, ak je množina I konečná.

Množina $A \subset \mathbb{R}$ sa nazýva \textit{kompaktná množina (kompakt)},
ak z každého jej otvoreného pokrytia $\{a_t: t \in I\}$ možno vybrať konečné
pokrytie (t.j. existuje konečná neprázdna množina $I' \subset I$ tak, že $A
\subset \bigcup_{t \in I} A_t$).

\begin{veta}
Pre ľubovoľnú množinu $A \subset \mathbb{R}$ sú nasledujúce tvrdenia
ekvivalentné:
\begin{enumerate}
\item $A$ je kompaktná množina
\item $A$ je uzavretá a ohraničená množina
\item
    ak všetky členy postupnosti ${\{a_n\}}_{n=1}^\infty$ sú prvkami množiny
    $A$, tak z ${\{a_n\}}_{n=1}^\infty$ možno vybrať konvergentnú podpostupnosť,
    ktorej milita je prvkom množiny $A$

\textit{Poznámka:}
Často sa kompaktná množina definuje pomocou vlastnosti $3.$.
\end{enumerate}
\end{veta}

\begin{enumerate}[resume]
    \item \useproblem[kompaktne]{kompaktne-214}
    \item \useproblem[kompaktne]{kompaktne-215}
    \item \useproblem[kompaktne]{kompaktne-216}
    \item \useproblem[kompaktne]{kompaktne-217}
    \item \useproblem[kompaktne]{kompaktne-218}
    \item \useproblem[kompaktne]{kompaktne-219}
    \item \useproblem[kompaktne]{kompaktne-220}
\end{enumerate}


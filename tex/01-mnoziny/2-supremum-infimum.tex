Neprázdna množina $A \subset \mathbb{R}$ sa nazýva \textit{zhora (zdola)
ohraničená}, ak platí
\[
  (\exists K \in  \mathbb{R})
    (\forall x \in A):
      x \leq K
\]
\[
  ((\exists K \in \mathbb{R})
    (\forall x \in A):
      x \geq K)
\]

Číslo $K$ s uvedenou vlastnosťou sa nazýva \textit{horné (dolné)
ohraničenie množiny $A$.} $\emptyset$ považujeme za ohraničenú zhora aj zdola.

Množina, ktorá je zhora aj zdola ohraničená, sa nazýva \textit{ohraničená}.
Množina, ktorá nie je ohraničená sa nazýva \textit{neohraničená}.

\begin{enumerate}[resume]
\item \useproblem[supremum-infimum]{supremum-infimum-14}
\item \useproblem[supremum-infimum]{supremum-infimum-15}
\item \useproblem[supremum-infimum]{supremum-infimum-16}
\end{enumerate}


Číslo $\alpha \in \mathbb{R}$ sa nazýva \textit{supremum množiny $A$}
$\subset \mathbb{R}, A \neq \emptyset$, ak
\[
  (\forall x \in A): x \leq \alpha \label{eq:supremum-i}
\]
\[
  (\forall \varepsilon > 0) (\exists \: x_\varepsilon \in A):
  x_\varepsilon > \alpha - \varepsilon
\]

Podľa \ding{34} je $\alpha$ horné ohraničenie množiny $A$; \ding{37} je negácia
výroku ``pre niektoré $\varepsilon > 0$ je číslo $\alpha - \varepsilon$ horným
ohraničením množiny $A$'', hovorí teda, že neexistuje horné ohraničenie množiny
$A$, ktoré by bolo menšie než $\alpha$. Teda $\alpha$ je najmenšie horné
ohraničenie množiny $A$. Supremum množiny $A$ označujeme $\sup \{ A \}$.

Číslo $\beta \in \mathbb{R}$ sa nazýva \textit{infimum množiny} $A \subset
\mathbb{R}, A \neq \emptyset$, ak:

\begin{align*}
\forall x \in A&: x \geq \beta \tag{\ding{34}} \\
(\forall \varepsilon > 0) (\exists \: x_\varepsilon \in A)&:
x_\varepsilon < \beta + \varepsilon \tag{\ding{37}}
\end{align*}

To znamená, že $\beta$ je najväčšie z dolných ohraniční množiny $A$. Infimum
množiny $A$ označujeme $\inf\{A\}$

Ak usporiadané pole $\mathbb{R}$ reálnych čísel konštruujeme z poľa
$\mathbb{Q}$ racionálnych čísel pomocou Dedekindových rezov, môžeme dokázať
nasledujúce dve ekvivalentné tvrdenia:

\begin{veta}
Každá neprázdna zhora ohraničená množina reálnych čísel má supremum.
\end{veta}

\begin{veta}
Každá neprázdna zdola ohraničená množina reálnych čísel má infimum.
\end{veta}

Ak usporiadané pole $\mathbb{R}$ zavádzame axiomaticky, považujeme prvé z
uvedených tvrdení za axiómu, z nej možno odvodiť vetu o existencií infima.

\showanswers
\begin{enumerate}[resume]
\item \useproblem[supremum-infimum]{supremum-infimum-17}
\hideanswers
\item \useproblem[supremum-infimum]{supremum-infimum-18}
\item \useproblem[supremum-infimum]{supremum-infimum-19}
\item \useproblem[supremum-infimum]{supremum-infimum-20}
\item \useproblem[supremum-infimum]{supremum-infimum-21}
\end{enumerate}

\begin{defproblem}{kompaktne-208}
Rozhodnite o uzavretosti a otvorenosti nasledujúcich množín:
\begin{tasks}(2)
    \task
        $\interval[open]{0}{1} \cup \interval{2}{3}$
    \task
        $\interval[open]{0}{1} \setminus \{\frac{1}{2^n}; n \in \mathbb{N}\}$
    \task $\interval{0}{1} \setminus \{\frac{1}{n}; n \in \mathbb{N}\}$
    \task $\mathbb{N}$
    \task $\mathbb{Q}$
    \task $\zeta(\mathbb{R})$
    \task
        $f(\mathbb{R})$, kde $f$ je Riemannova funkcia (definíciu Riemannovej
        funkcie pozri v príklade $230$)
    \task $\bigcup_{n \in \mathbb{N}} (\frac{1}{2n+1},\frac{1}{2n})$
    \task!
        $\bigcup_{n \in \mathbb{N}} \interval{\frac{1}{2n+1}}{\frac{1}{2n}}$
        $(= \interval{\frac{1}{3}}{\frac{1}{2}}
        \cup \interval{\frac{1}{5}}{\frac{1}{4}} \cup ...)$
\end{tasks}
\end{defproblem}

\begin{defproblem}{kompaktne-209}
Dokážte nasledujúce tvrdenia:
\begin{tasks}
\task
    ak $A,B$ sú otvorené množiny, tak aj množiny $A \cup B$,$A \cap B$ sú
    otvorené
\task
    ak $A,B$ sú uzavreté množiny, tak aj množiny $A \cup B$,$A \cap B$ sú
    uzavreté
\task
    ak $A$ je otvorená a $B$ uzavretá množina, tak $A \setminus B$ je otvorená
    $B \setminus A$ je uzavretá množina
\task
    ak $\{A_{\alpha}: \alpha \in I\}$ je systém otvorených množín (I je
    neprázdna indexová množina), tak $\bigcup_{\alpha \in I} A_\alpha (:=\{x \in
    \mathbb{R}; \exists \alpha \in I: x \in A_\alpha\})$ je otvorená množina
    (teda slovne: zjednotenie ľubovoľného systému otvorených množín je otvorená
    množina)
\end{tasks}
\end{defproblem}

\begin{defproblem}{kompaktne-210}
Ak $\emptyset \neq A \subset \mathbb{R}$ je otvorená množina a $B \subset
\mathbb{R}$ je ľubovoľná neprázdna množina, tak $A+B$ je otvorená množina
(definíciu množiny $A+B$ pozri v príklade $21$).
\end{defproblem}

\begin{defproblem}{kompaktne-211}
Ukážte, že $1.$ interval $(a,b)$ možno písať ako zjednotenie uzavretých
nedegenerovaných intervalov (degenerovanými intervalmi sa nazývajú jednoprvkové
množiny), ale $2.$ interval $\interval{a}{b}$ nemožno písať v tvare zjednotenia
otvorených intervalov.
\end{defproblem}

\begin{defproblem}{kompaktne-212}
Nech $\emptyset \neq A \subset \mathbb{R}$ je otvorená množina a $B \subset
\mathbb{R}$ je ľubovoľná neprázdna množina. Potom množina $A:=\{|x-y|; x \in A,
y \in B\}$ je buď otvorená alebo je zjednotením otvorenej množiny s
jednoprvkovou množinou $\{0 \}$.
\end{defproblem}

\begin{defproblem}{kompaktne-213}
Ak množina $A \subset \mathbb{R}$ je súčasne otvorená aj uzavretá, tak
$A=\emptyset$ alebo $A=\mathbb{R}$. Dokážte!
\end{defproblem}

\begin{defproblem}{kompaktne-214}
Dokážte, že:
\begin{tasks}
\task uzavretá podmnožina kompaktu je kompakt
\task zjednotenie konečného počtu kompaktov je kompakt
\task prienik konečného počtu kompaktov je kompakt
\end{tasks}
\end{defproblem}

\begin{defproblem}{kompaktne-215}
Je daná spočítateľná kompaktná množina
$E=\{0,1,\frac{1}{2},...,\frac{1}{2^n},...\}$
ktorá je pokrytá systémom intervalov
\[
\{
    (-\varepsilon, \varepsilon),
    (1-\varepsilon,1+\varepsilon),
    (\frac{1-\varepsilon}{2},\frac{1+\varepsilon}{2}),
    ...,
    (\frac{1-\varepsilon}{2^n},
    \frac{1+\varepsilon}{2^n}),
    ...
\}
\]
kde $\varepsilon$ je dané kladné číslo, $\varepsilon<\frac{1}{2}$. Nájdite
konečné pokrytie vybrané z tohto otvoreného pokrytia.
\end{defproblem}

\begin{defproblem}{kompaktne-216}
Je daná spočítateľná množina
$E=\{0,1,\frac{1}{2},...,\frac{1}{2^n},... \}$
pokrytá systémom otvorených intervalov
$$
\{
    (-\varepsilon, \varepsilon),
    (1-\varepsilon,1+\varepsilon),
    (\frac{1-\varepsilon}{2},\frac{1+\varepsilon}{2}),
    ...,
    (\frac{1-\varepsilon}{2^n},\frac{1+\varepsilon}{2^n}),
    ...
\}
$$
kde $\varepsilon$ je dané kladné číslo, $\varepsilon<\frac{1}{2}$. Možno z tohto
pokrytia vybrať konečné podpokrytie?
\end{defproblem}

\begin{defproblem}{kompaktne-217}
Množina $\interval[open left]{0}{1}$ nie je kompaktná. Nájdite také jej otvorené
pokrytie, z ktorého nemožno vybrať konečné podpokrytie!
\end{defproblem}

\begin{defproblem}{kompaktne-218}
\begin{tasks}
\task
    Nech $A_1,...,A_n,...$ sú neprázdne kompaktné množiny, nech
    \[A_1 \supset A_2 \supset ... \supset A_n \supset ...\]
    Dokážte, že množina
    \[
        \bigcap_{n \in \mathbb{N}} A_n (:=\{x \in \mathbb{R};
        (\forall n \in \mathbb{N}): x \in A_n\})
    \]
    je neprázdna!
\task
    Nech ${\{A_n\}}_{n=1}^\infty$ je postupnosť kompaktov taká, že prienik
    ľubovoľného konečného počtu jej členov je neprázdny. Potom
    \[ \bigcap_{n \in \mathbb{N}} A_n \neq \emptyset \]
    Dokážte!
\end{tasks}
\end{defproblem}

\begin{defproblem}{kompaktne-219}
Rozhodnite, či tvrdenie \enquote{$A \subset \mathbb{R}$ je kompaktná množina} je
ekvivalentné s niektorou z podmienok:
\begin{tasks}
\task
    otvorenými intervalmi
\task
    uzavretými množinami
\task
    uzavretými nedegenerovanými intervalmi
\task
    intervalmi typu $\interval[open right]{a}{b}$ možno vybrať konečné
    podpokrytie
\end{tasks}
\end{defproblem}

\begin{defproblem}{kompaktne-220}
Ak sú $A,B \subset \mathbb{R}$ neprázdne kompaktné množiny, tak aj množiny
$A+B,A \cdot B$ sú kompaktné. Dokážte! $(A \cdot B :=\{a \cdot b; a \in A, b \in
B\})$
\end{defproblem}

\begin{defproblem}{supremum-infimum-1}
Zistite, či sú dané množiny zhora, resp. zdola ohraničené:

\begin{enumerate}
  \item $A = \{ \sqrt{a + \sqrt{b}} ; \: a, b \in \mathbb{N}, a < b \}$ ;
  \item $B = \{ \frac{1}{x + \frac{1}{x}} ; \: x \in (0, \inf) \}$ ;
  \item $C = \{ sin(n!) ; \: n \in \mathbb{N} \}$ ;
  \item $D = \{ \frac{\sqrt{x}}{\sqrt[3]{x} + \sqrt[4]{x}}; \:
              x \in \mathbb{Q} \cap (2, 3) \}$ ;
  \item $E = \{ x \in \mathbb{R}; \: \exists \: a, b, c \in \mathbb{Q}:
                a \neq 0 \land ax^2 + bx +c = 0 \}$

        (teda $E$ je množina koreňov všetkých polynómov druhého stupňa s
        racionálnymi koeficientami).
\end{enumerate}
\end{defproblem}

\begin{defproblem}{supremum-infimum-2}
Ak $A \cap \mathbb{R}$ je neohraničená množina, tak platí
$$
(\forall a \in A) (\forall \beta > 0) (\exists \: b \in A): |a - b| > \beta .
$$
Dokážte!
\end{defproblem}

\begin{defproblem}{supremum-infimum-3}
Nech $A, B \subset \mathbb{R}$ sú neprázdne množiny, pričom $B$ je
neohraničená. Ak existuje $\beta > 0$ také, že platí:
$$
(\forall x \in B) (\exists \: y \in A): |x - y| < \beta
$$
tak $A$ je neohraničená množina. Dokážte!
\end{defproblem}

\begin{defproblem}{supremum-infimum-4}
Dokážte alebo vyvráťte nasledujúce rovnosti:

\begin{enumerate}
  \item $1 = \inf \: \{ \frac{x^2 + 2}{x^2 + 1}; x \in \mathbb{R} \}$;
  \item $1 = \sup \: \{ \frac{2x^2}{2x^2 + 1}; x \in \mathbb{R} \}$;
  \item $-2 = \inf \: \{ 2x^2 + 8x + 1; x \in \mathbb{R} \}$;
  \item $12 = \inf \: \{ 1 + 6x - x^2; x \in \mathbb{R} \}$.
\end{enumerate}

\begin{solution}
  Príklad (a): Musíme zistiť, či číslo 1 vyhovuje podmienkam \ding{34} a
  \ding{37} z definície infíma:

  \begin{itemize}
    \item[\ding{34}] pretože $\frac{x^2 + 2}{x^2 + 1} = 1 + \frac{1}{1 + x^2}$ a číslo
        $\frac{1}{1 + x^2}$ je kladné číslo pre každé $x \in \mathbb{R}$, platí
        $\frac{x^2 + 2}{x^2 + 1} > 1$ pre všetky $x \in \mathbb{R}$; teda číslo
        1 vyhovuje podmienke \ding{34};
    \item[\ding{37}] podmienka má v tomto prípade tvar
          $$(\forall \varepsilon > 0) (\exists a \in \mathbb{R}):
            \frac{a^2 + 2}{a^2 + 1} < 1 + \varepsilon$$
          čo je ekvivalentné s podmienkou
          $$(\forall \varepsilon > 0) (\exists a \in \mathbb{R}):
            a^2 > \frac{1}{\varepsilon} - 1$$
  \end{itemize}

  Odtiaľ už vidíme, že pre každé dané $\varepsilon > 0$ také číslo
  $a \in \mathbb{R}$ skutočne existuje (pre $\varepsilon > 1$ vyhovuje
  uvedenej nerovnosti dokonca každé reálne číslo $a$, pre
  $\varepsilon \in (0, 1>$ stačí za $a$ zvoliť ľubovoľné číslo, pre ktoré
  platí $|a| > \sqrt{\frac{1}{\varepsilon} - 1}$); teda pre číslo 1 je splnená
  aj podmienka \ding{37}.

  Pretože číslo 1 vyhovuje obidvom podmienkam z definície infima, platí
  $1 = \inf \{ \frac{x^2 + 2}{x^2 + 1}; x \in \mathbb{R} \}$.
\end{solution}
\end{defproblem}

\begin{defproblem}{supremum-infimum-5}
Najdite supremum a infimum nasledujúcich množín (nezabudnite, že svoje tvrdenia)
musíte dokázať podobne ako v pr. 17):

\begin{enumerate}[label=\arabic*.]
  \item $A = \{ x \in (2, 3);$ zápis čísla $x$ v desiatkovej sústave má konečný
        počet cifier za desationnou čiarkou$\}$;
  \item $B = \{ x \in \left[0, 2\right);$ zápis čísla $x$ v desiatkovej sústave obsahuje
        len cifry $0, 1 \}$;
  \item $C = \{ \cos \pi(n!); n \in \mathbb{N} \}$.
\end{enumerate}
\end{defproblem}

\begin{defproblem}{supremum-infimum-6}
Nech $A \subset B \subset \mathbb{R}$, pričom $A$ je neprázdna a $B$ je zhora
ohraničená množina. Potom $A$ je zhora ohraničená množina a platí
$\sup\{A\} \leq \sup\{B\}$. Dokážte! Sformulujte analogické tvrdenie pre infíma!
\end{defproblem}

\begin{defproblem}{supremum-infimum-7}
Nech $A$ je neprázdna ohraničená množina; definujme množina $-A$ nasledovne:
$$-A := \{ -z; z \in A\}$$
Potom
$$\sup\{-A\} = -\inf\{A\}$$
$$\inf\{-A\} = -\sup\{A\}$$
Dokážte!

\begin{solution}
  Dokážeme prvú z uvedených rovností. Označme $\beta := \inf\{A\}$ (pre číslo
  $\beta$ teda platí:
  \begin{align*}
    (\forall z \in A)&: z \geq \beta \\
    (\forall \varepsilon > 0) \: (\exists z_\varepsilon \in A)&:
            z_\varepsilon < \beta + \varepsilon
  \end{align*}
  máme ukázať, že číslo $-\beta$ je supremom množiny $-A$, t.j. že vyhovuje
  podmienkam \ding{34} a \ding{37} z definície suprema.

  \begin{itemize}
    \item[\ding{34}]
      Podmienka má v tomto prípade podobu
      $$\forall x \in -A: x \leq -\beta$$
      čo je ekvivalentné s podmienkou
      $$\forall z \in A: -z \leq -\beta$$
      t.j.
      $$\forall z \in A: z \geq \beta$$
      posledné tvrdenie je pravdivé, pretože $\beta$ vyhovuje
      podmienke (1).
    \item[\ding{37}]
      Podmienka má tvar
      $$(\forall \varepsilon > 0) (\exists x_\varepsilon \in -A):
        x_\varepsilon > -\beta - \varepsilon$$
      to je ekvivalentné s výrokom
      $$(\forall \varepsilon > 0) (\exists z_\varepsilon \in A):
        -z_\varepsilon > -\beta - \varepsilon$$
      t.j.
      $$(\forall \varepsilon > 0) (\exists z_\varepsilon \in A):
        z_\varepsilon < \beta + \varepsilon$$
      to je ale podmienka (2), ktorá je podľa predpokladov splnená.
  \end{itemize}
\end{solution}

\end{defproblem}

\begin{defproblem}{supremum-infimum-8}
Nech $A, B$ sú neprázdne ohraničené množiny, definujme množinu $A + B$
nasledovne: $A + B := \{ a + b ; a \in A, b \in B \}$. Potom platí:
$$\sup\{A + B\} = \sup\{A\} + \sup\{B\}$$
Dokážte! Sformulujte a dokážte analogické tvrdenie pre infíma!
\end{defproblem}

\begin{defproblem}{supremum-infimum-9}
Zložením ktorých základných elementárnych funkcií vzniknú funkcie dané predpismi
\begin{multicols}{2}
\begin{enumerate}
    \item $\sin^3 x$;
    \item $\sin (x^3)$;
    \item $5^{\tan^2 x}$;
    \item $\log_3 \sin^2 \sqrt{b^x}$;
    \item $\sqrt{\cos (2^{\sin x})}$?
\end{enumerate}
\end{multicols}
\end{defproblem}

\begin{defproblem}{supremum-infimum-10}
Nájdite $D(f)$, ak funkcia $f$ je daná predpisom 
\begin{multicols}{2}
\begin{enumerate}
    \item $f(x)=\sqrt{3x-x^3}$;
    \item $f(x)=\log(x^2-4)$;
    \item $f(x)=\sqrt{\sin \sqrt{x}}$;
    \item $f(x)=\ln (\cos (\ln x))$;
    \item $f(x)=\sqrt{\sin 2x}+\sqrt{\sin 3x}$;
    \item $f(x)=\sqrt{\cos x^2}$;
    \item $f(x)=\sin (\ln \frac{1}{3x+1})$;
    \item $f(x)=\ln (3\sin^2 x -4)$;
    \item $f(x)=\sqrt{9-x^2}+\ln \frac{x+1}{x-2}$;
    \item $f(x)=\sqrt{\log \frac{5x-x^2}{4}}$;
    \item $f(x)=\log_\frac{1}{2}\log_3 \log_\frac{1}{4} x$;
    \item $f(x)=\sqrt{-\sin^2 \zeta x}$;
    \item $f(x)=\sqrt[3]{\frac{x+2}{\log\cos x}}$;
    \item $f(x)=\sqrt{\log_3 \frac{2x-3}{x-1}}$;
    \item $f(x)=\ln (1-\log (x^2-5x+16))$;
    \item $f(x)=\sqrt{\frac{\sin x +\cos x}{\sin x -\cos x}}$;
    \item $f(x)=\frac{\sqrt{\cos x -\frac{1}{2}}}{\sqrt{6-35x-6x^2}}$?
\end{enumerate}
\end{multicols}
\end{defproblem}

\begin{defproblem}{supremum-infimum-11}
Napíšte predpid a určte definičný obor funkcií $f\circ f,f\circ g,g\circ f$ a $g\circ g$, ak
\begin{align*}
f(x)=x^2, & g(x)=2^x \\
f(x) = sgn x \footnote{táto funkcia sa nazýva signum}= \left\{ \begin{array}{r@{\quad}c}
    0, $ak $ x=0 \\
    1, $ak $ x>0 \\
    -1, $ak $ x<0 \\ \end{array} \right.
    , & g(x)=\frac{1}{x} \\
f(x) = \left\{ \begin{array}{r@{\quad}c}
    0, $ak $ x\leq 0 \\
    x, $ak $ x>0 \\ \end{array} \right.
    , &  g(x)= \left\{ \begin{array}{r@{\quad}c}
    0, $ak $ x<0 \\
    -x^2, $ak $ x\geq0 \\ \end{array} \right.\\
f(x)= \left\{ \begin{array}{r@{\quad}c}
    x^2, $ak $ x\in \langle 0,1 \rangle \\
    3x, $ak $ x\notin \langle 0,1 \rangle \\ \end{array} \right., & g(x)= \left\{ \begin{array}{r@{\quad}c}
    2x, $ak $ x\in \langle 0,1 \rangle \\
    4x-2, $ak $ x\notin \langle 0,1 \rangle \\ \end{array} \right. \\
f(x)=\zeta(x)\footnote{táto funkcia sa nazýva Dirichletova funkcia} := \left\{ \begin{array}{r@{\quad}c}
    0, $ak $ x\in\mathbb{R}\ \mathbb{Q} \\
    1, $ak $ x\in \mathbb{Q} \\ \end{array} \right., & g(x)=\frac{1}{x^2} \\
f(x)= \left\{ \begin{array}{r@{\quad}c}
    0, $ak $ |x|\neq 1 \\
    1, $ak $ |x|>1\\ \end{array} \right. , & g(x)= \left\{ \begin{array}{r@{\quad}c}
    x^2-2, $ak $ |x|\leq 2 \\
   -1, $ak $ |x|>2 \\ \end{array} \right. 
\end{align*}
\end{defproblem}

\begin{defproblem}{supremum-infimum-12}
Nech je daná funkcia $f$; označme $f_1:=f,f_2:=f\circ f_1,f_{n+1}:=f\circ f_n$. Nájdite predpis pre $f_n$, ak
\begin{enumerate}
\item $f(x)=\frac{x}{\sqrt{1+x^2}}$;
\item $f(x)=a+bx$;
\item $f(x)=\frac{x}{ax+b},(b\neq 1)$.
\end{enumerate}
\end{defproblem}

\begin{defproblem}{supremum-infimum-13}
Zistite, či sa rovnajú funkcie $f$ a $g$. Ak nie, nájdite najväčšiu množinu $A\subset\mathbb{R}$, pre ktorú platí $\frac{f}{A}=\frac{g}{A}$.
\begin{align*}
f(x)=\sqrt{x(x-1)}, & g(x)=\sqrt{x}\sqrt{x-1};\\
f(x)=\ln (x^2-4), & g(x)=\ln (x-2)+\ln (x+2);\\
f(x)=\cot x, & g(x)=\frac{1}{\tan x};\\
f(x)=\ln |\frac{\sqrt{x^2+1}-x}{\sqrt{x^2+1}+x}|, & g(x)=-2\ln |x+\sqrt{x^2+1}|;\\
f(x)=\sqrt{x^2+4x+4}-\sqrt{x^2-8x+16}, & g(x)= \left\{ \begin{array}{r@{\quad}c}
    -6, $ak $ x<-2 \\
    2x-2, $ak $ x\in\langle -2,4 \rangle \\
   6, $ak $ x>4 \\ \end{array} \right. .
\end{align*}
\end{defproblem}

\begin{defproblem}{supremum-infimum-14}
Zostrojte grafy nasledujúcich funkcií:
\begin{multicols}{2}
\begin{enumerate}
    \item $y=\ln (1+x)$;
    \item $y=1+e^{-x}$;
    \item $y=\sin \frac{x}{2}$;
    \item $y=3+2\cos 3x$;
    \item $y=x^2+4x+2$;
    \item $y=\frac{x^2}{2}+x+1$;
    \item $y=\frac{1+x}{1-x}$;
    \item $y=\frac{2+3x}{1-4x}$;
    \item $y=-\sqrt{-x-2}$;
    \item $y=\sin 2(x+3)$;
    \item $y=\sin (2x+3)$;
    \item $y=\frac{1}{3}2^{1-3x}+2$;
    \item $y=\tan (2x-\frac{\pi}{2})$;
    \item $y=3-0,5\sqrt[3]{3x-2}$;
    \item $y=\log_3 (0,5x+2)$.
\end{enumerate}
\end{multicols}
\end{defproblem}

\begin{defproblem}{supremum-infimum-15}
Zostrojte grafy nasledujúcich funkcií:
\begin{multicols}{2}
\begin{enumerate}
    \item $y=\ln |x|$;
    \item $y=\sin |x|$;
    \item $y=|\sin x|$;
    \item $y=|x^2-2x-1|$;
    \item $y=2\cos |x-2|+4$;
    \item $y=x|x+2|$;
    \item $y=x+\sqrt{x^2}$;
    \item $y=|\log_{\frac{1}{2}}(|x|-2)|$;
    \item $y=2|x-2|-|x+1|+x$;
    \item $y= \left\{ \begin{array}{r@{\quad}c}
    \sin x, $ak $ x\in \langle -\pi,0\rangle \\
    2, $ak $ x\in (0,1\rangle \\
    \frac{1}{(x-1)}, $ak $ x\in (1,4\rangle \\ \end{array} \right.$.    
\end{enumerate}
\end{multicols}
\end{defproblem}

\begin{defproblem}{supremum-infimum-16}
Nech graf funkcie $g:\mathbb{R}\rightarrow\mathbb{R}$ je symetričký s grafom funkcie $f$
\begin{enumerate}
\item podľa priamky $x=x_0$;
\item podľa priamky $y=y_0$;
\item podľa bodu $(x_0,y_0)$.
\end{enumerate}
Vyjadrite funkčné hodnoty funkcie $g$ pomocou funkčných hodnôt funkcie $f$!
\end{defproblem}

\begin{defproblem}{supremum-infimum-17}
Načrtnite približne grafy funkcií:
\begin{multicols}{2}
\begin{enumerate}
    \item $y=\sin x^2$;
    \item $y=\sin \frac{1}{x}$;
    \item $y=x\sin x$;
    \item $y=e^x\cos x$;
    \item $y=\frac{\sin x}{x}$;
    \item $y=\frac{\cos x}{1+x^2}$.
\end{enumerate}
\end{multicols}
\end{defproblem}

\begin{defproblem}{supremum-infimum-18}
Zostrojte grafy nasledujúcich funkcií tak, že predpis $y=a\cos x +b\sin x$ upravíte na tvar $y=A\sin (x-x_0)$:
\begin{enumerate}
\item $y=\sqrt{3}\cos x +\sin x$;
\item $y=\cos x -\sin x$;
\item $y=6\cos x +8\sin x$.
\end{enumerate}
\end{defproblem}

\begin{defproblem}{supremum-infimum-19}

\end{defproblem}

\begin{defproblem}{supremum-infimum-20}

\end{defproblem}

\begin{defproblem}{supremum-infimum-21}

\end{defproblem}

\begin{defproblem}{supremum-infimum-22}

\end{defproblem}
\begin{defproblem}{supremum-infimum-1}
Zistite, či sú dané množiny zhora, resp. zdola ohraničené:

\begin{enumerate}
  \item $A = \{ \sqrt{a + \sqrt{b}} ; \: a, b \in \mathbb{N}, a < b \}$ ;
  \item $B = \{ \frac{1}{x + \frac{1}{x}} ; \: x \in (0, \inf) \}$ ;
  \item $C = \{ sin(n!) ; \: n \in \mathbb{N} \}$ ;
  \item $D = \{ \frac{\sqrt{x}}{\sqrt[3]{x} + \sqrt[4]{x}}; \:
              x \in \mathbb{Q} \cap (2, 3) \}$ ;
  \item $E = \{ x \in \mathbb{R}; \: \exists \: a, b, c \in \mathbb{Q}:
                a \neq 0 \land ax^2 + bx +c = 0 \}$
        
        (teda $E$ je množina koreňov všetkých polynómov druhého stupňa s
        racionálnymi koeficientami).
\end{enumerate}
\end{defproblem}

\begin{defproblem}{supremum-infimum-2}
Ak $A \cap \mathbb{R}$ je neohraničená množina, tak platí
$$
(\forall a \in A) (\forall \beta > 0) (\exists \: b \in A): |a - b| > \beta .
$$
Dokážte!
\end{defproblem}

\begin{defproblem}{supremum-infimum-3}
Nech $A, B \subset \mathbb{R}$ sú neprázdne množiny, pričom $B$ je
neohraničená. Ak existuje $\beta > 0$ také, že platí:
$$
(\forall x \in B) (\exists \: y \in A): |x - y| < \beta
$$
tak $A$ je neohraničená množina. Dokážte!
\end{defproblem}

\begin{defproblem}{supremum-infimum-x}
\end{defproblem}

Nech sú dané funkcie $f$, $g$.

\begin{itemize}
  \item
    Ak je množina $D_1:=D(f)\cap D(g)$ neprázdna, nazývajú sa funkcie $p, q, r$
    definované na množine $D_1$ predpismi
    \begin{align*}
      p(x) &= f(x)+g(x) \\
      q(x) &= f(x)-g(x) \\
      r(x) &= f(x)\cdot g(x)
    \end{align*}
    súčet, rozdiel a súčin funkcií $f, g$ a označujú sa $f+g$, $f-g$, $f\cdot g$.
  \item
    Ak je množina $D_2:=D(f)\cap \{x\in D(g);g(x)\neq 0\}$ neprázdna, nazýva sa
    funkcia $s:D_2 \rightarrow \mathbb{R}$ definovaná predpisom
    \[
      a(x)=\frac{f(x)}{g(x)}
    \]
    podiel funkcií $f$, $g$ a označuje sa $\frac{f}{g}$.
  \item
    Ak je množina $D_3:=\{x\in D(f);f(x)\in D(g)\}$ neprázdna, nazýva sa funkcia
    $t:D_3\rightarrow\mathbb{R}$ daná predpisom
    $$t(x)=g(f(x))$$
    zložená funkcia z funkcií $f$, $g$ (superpozícia funkcií $f$ a $g$) a označuje
    sa $g \circ f$. Funkcia $f$ sa nazýva vnútorná zložka, funkcia $g$ vonkajšia
    zložka funkcie $g \circ f$.
\end{itemize}

Základnými elementárnymi funkciami nazývame nasledujúce funkcie:

\noindent
% FIXME odkazy na footnotes v tabulke
\begin{tabular}{|p{0.21\textwidth}|p{0.22\textwidth}|p{0.45\textwidth}|}
  \hline
  \textbf{Názov} & \textbf{Predpis} & \textbf{Definičný obor} \\
  \hline
  konštantné &
  $\underset{a \in \mathbb{R}}{f(x)}\equiv a$  & $\mathbb{R}$ \\
  \hline
  mocninové &
  $\underset{a \in \mathbb{R} \setminus {0}}{f(x)}\equiv a$ &
  \begin{enumerate}[leftmargin=0.5cm]
    \item ak $a>0$:
      \begin{enumerate}[leftmargin=0.5cm]
        \item ak $a=\frac{p}{q},p,q\in\mathbb{N},$ $p$ je párne alebo $p$ aj $q$
              sú nepárne: $\mathbb{R}$
        \item vo všetkých ostatných prípadoch: $\interval[open right]{0}{\infty}$
      \end{enumerate}
    \item ak $a<0$:
      \begin{enumerate}[leftmargin=0.5cm]
        \item ak $a=-\frac{p}{q},p,q\in\mathbb{N}$, $p$ je párne lebo $p$ aj $q$
              sú nepárne: $\mathbb{R} \setminus \{0\}$
        \item vo všetkých ostatných prípadoch: $\interval[open]{0}{\infty}$
      \end{enumerate}
  \end{enumerate}
  \\
  \hline
  exponenciálne
    & $f(x)=a^x, $\newline$ a>0,a\neq 1$
    & $\mathbb{R}$ \\
  \hline
  logaritmické
    & $f(x)=\log_a x,$\newline$ a>0,a\neq 1$ & $\mathbb{R}^{+}$ \\
  \hline
  goniometrické & $f(x)=\sin x$ & $\mathbb{R}$ \\
    & $f(x)=\cos x$ & $\mathbb{R}$ \\
    & $f(x)=\tan x$ & $\mathbb{R}\ \{\frac{\pi}{2}+k\pi;k\in\mathbb{Z}\}$ \\
    & $f(x)=\cot x$ & $\mathbb{R}\ \{k\pi;k\in\mathbb{Z}\}$ \\
  \hline
  cyklometrické
    & $f(x)=\arcsin x$ & $\interval{-1}{1}$ \\
    & $f(x)=\arccos x$ & $\interval{-1}{1}$ \\
    & $f(x)=\arctan x$ & $\mathbb{R}$ \\
    & $f(x)=\arccot x$ & $\mathbb{R}$ \\
  \hline
\end{tabular}

Funkcie, ktoré vzniknú zo základných elementárnych funkcií len použitím
operácií súčtu, rozdielu, súčinu, podielu a superpozície funkcií, sa nazývajú
elementárne funkcie.

Všimnime si, že definičný obor funkcie, ktorá je súčtom, rozdielom, súčinom,
podielom alebo superpozíciou daných funkcií $f$ a $g$, je jednoznačne určený
množinami $D(f)$ a $D(g)$. Teda ak funkcia $h$ vznikne z funkcií
$f_1,\ldots,f_n$ len použitím operácií súčtu, rozdielu, súčinu, podielu a
superpozície funkcií, je množina $D(h)$ jednoznačne určená množinami
$D(f_1),\ldots,D(f_n)$. Preto ak napíšeme predpis takejto funkcie $h$ bez toho,
aby sme výslovne určili jej definičný obor, považujeme funkciu $h$ za
definovanú práve na tej množine, ktorá je určená množinami
$D(f_1),\ldots,D(f_n)$ na základe definícií súčtu, rozdielu, súčinu, podielu a
superpozície funkcií. (Teda trocha nepresne povedané, za definičný obor takejto
funkcie $h$ považujeme množinu všetkých tých $x\in\mathbb{R}$, pre ktoré má
predpis funkcie $h$ \enquote{zmysel}.)

\begin{enumerate}[resume]
  \item \useproblem[funkcie]{funkcie-22}
  \item \useproblem[funkcie]{funkcie-23}
  \item \useproblem[funkcie]{funkcie-24}
  \item \useproblem[funkcie]{funkcie-25}
\end{enumerate}

Ak $f: A \rightarrow \mathbb{R}, g: B \rightarrow \mathbb{R}$ sú funkcie, $A
\subset B$ a pre všetky $x \in A$ platí $f(x) = g(x)$, hovoríme, že funkcia $f$
je zúženie funkcie $g$ na množinu $A$ ($f$ je funkcia $g$ zúžená na množinu
$A$) a označujeme $f = \frac{g}{A}$.

% má tam byť číslovanie v príklade
\begin{enumerate}[resume]
  \item \useproblem[funkcie]{funkcie-26}
\end{enumerate}

Inverzné funkcie k funkciám
$\sin / \interval{-\frac{\pi}{2}}{\frac{\pi}{2}}$,
$\cos / \interval{0}{\pi}$,
$\tan / \interval[open]{-\frac{\pi}{2}}{\frac{\pi}{2})}$,
$\cot / \interval[open]{0}{\pi}$
sa nazývajú arkussínus, arkuskosínus, arkustangens a arkuskotangens a označujú
sa $\arcsin, \arccos, \arctan, \arccot$. Tieto funkcie majú spoločný názov
cyklometrické.

\begin{enumerate}[resume]
  \item \useproblem[funkcie]{funkcie-57}
  \item \useproblem[funkcie]{funkcie-58}
  \item \useproblem[funkcie]{funkcie-59}
  \showanswers
  \item \useproblem[funkcie]{funkcie-60}
  \item \useproblem[funkcie]{funkcie-61}
  \hideanswers
  \item \useproblem[funkcie]{funkcie-62}
\end{enumerate}

Funkcie definované predpismi
$\sinh x=\frac{e^x-e^{-x}}{2}$ (sínus hyberbolický),
$\cosh x=\frac{e^x+e^{-x}}{2}$ (kosínus hyperbolický),
$\tanh x=\frac{\sinh x}{\cosh x}$ (tangens hyperbolický) a
$\coth x=\frac{\cosh x}{\sinh x}$ (kotangens hyperbolický) sa
nazývajú hyperbolické funkcie.

\begin{enumerate}[resume]
  \item \useproblem[funkcie]{funkcie-63}
  \item \useproblem[funkcie]{funkcie-64}
\end{enumerate}

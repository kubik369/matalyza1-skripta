Nech je v rovine daná pravouhlá súradnicová sústava, pričom jednotky dĺžky na
súradnicových osiach $Ox$ a $Oy$ sú rovnaké. Množina
\[
  \{(x,f(x)); x \in A\}
\]
bodov roviny, kde $f:A \rightarrow \mathbb{R}$ je daná funkcia, sa nazýva
\emph{graf funkcie $f$}, $((x,f(x))$ je zápis bodu roviny pomocou jeho súradníc
v danej súradnicovej sústave).

V nasledujúcej tabuľke sú opísané elementárne transformácie grafov funkcií:

\begin{center}
  \begin{tabular}{|p{0.2\textwidth}|p{0.6\textwidth}|}
    \hline
    {\bf funkcia $y=g(x)$} & {\bf transformácia grafu funkcie $y=f(x)$}  \\
    \hline
    $y = f(x) + c$     & posunutie o $c$ v smere osi $Oy$ \\
    $y = f(x - c)$     & posunutie o $c$ v smere osi $Ox$ \\
    $y = f(-x)$        & symetria podľa osi $Oy$ \\
    $y = -f(x)$        & symetria podľa osi $Ox$ \\
    $y = a \cdot f(x)$ & vynásobenie každej $y$-ovej súradnice číslom $a$ \\
    $y = f(ax)$        & vydelenie každej $x$-ovej súradnice číslom $a$, $(a \neq 0)$ \\
    \hline
  \end{tabular}
\end{center}

\begin{enumerate}[resume]
  \showanswers
  \item \useproblem[funkcie]{funkcie-27}
  \hideanswers
  \item \useproblem[funkcie]{funkcie-28}
  \item \useproblem[funkcie]{funkcie-29}
  \showanswers
  \item \useproblem[funkcie]{funkcie-30}
  \item \useproblem[funkcie]{funkcie-31}
  \hideanswers
  \item \useproblem[funkcie]{funkcie-32}
  \item \useproblem[funkcie]{funkcie-33}
  \item \useproblem[funkcie]{funkcie-34}
\end{enumerate}

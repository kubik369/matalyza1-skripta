Nech $A\subset \mathbb{R}$ je neprázdna množina. Ak je každému číslu $x\in A$
priradené práve jedno číslo $y\in\mathbb{R}$, ktoré označíme $f(x)$, hovoríme,
že $f$ je funkcia (funkcia definovaná na množine $A$). Číslo $f(x)$ sa nazýva
funkčná hodnota (v bode $x$), množina $A$ definičný obor funkcie $f$ (túto
množinu budeme označovať $D(f)$). Na označenie funkcií budeme používať písmená
latinskej a gréckej abecedy. Ak chceme zdôraznič, že definičným oborom funkcie
$f$ je množina $A$, použijeme zápis $f:A \rightarrow \mathbb{R}$ (prípadne $f:A
\rightarrow B$, ak pre každé $x\in A$ platí $f(x)\in B$) alebo $f(x)$, $x\in A$.
Okrem označení \enquote{funkcia $f$}, \enquote{funkcia $f:A \rightarrow
\mathbb{R}^n$} sa možno stretnúť aj so spojeniami \enquote{funkcia $y=f(x)$}
alebo \enquote{funkcia $f(x)$} (istá nepresnosť posledných dvoch spojení spočíva
v tom, že symbol $f(x)$ sa zvykne označovať funkčná hodnota v danom bode $x$;
mnohí autori preto rozlišujú označenie $f(x)$ pre funkčnú hodnotu a $f(.)$ pre
funkciu).

Hovoríme, že funkcie $f$ a $g$ sa rovnajú, ak $D(f)=D(g)$ a pre každé $x\in
D(f)$ platí $f(x)=g(x)$ (teda funkcia je jednoznačne určená predpisom
priradenia a definičným oborom).

Funkciu $a$, ktorej definičným oborom je množina $N$, nazývame postupnosť a
označujeme ju spravidla $\{a_n\}_{n=1}^\infty$; funkčná hodnota v bode $n$ sa
nazýva $n$-tý člen postupnosti $\{a_n\}_{n=1}^\infty$ a označuje sa $a_n$.

Ak $f:A \rightarrow \mathbb{R}$ je funkcia a $B \subset A$ neprázdna množina,
tak množina $f(B):=\{f(x);x\in B \}$ sa nazýva obraz množiny $B$ (pri zobrazení
$f$). Špeciálne množina $f(A)$ sa nazýva obor hodnôt funkcie $f$.

Nech $A,B\subset\mathbb{R}$ sú neprázdne množiny, $B\subset A$. Funkcia
$f:A\rightarrow\mathbb{R}$ sa nazýva ohraničená na množine $B$, ak je ohraničená
množina $f(B)$. Funkcia ohraničená na svojom definičnom obore sa nazýva
ohraničená. Analogicky sa zavádza pojem funkcie ohraničenej zhora, resp. zdola a
funkcie ohraničenej zhora, resp. zdola na množine $B$.

Ak je množina $f(B)$ zhora (zdola) ohraničená, tak jej supremum (infimum) sa
nazýva supremum (infimum) funkcie $f$ na množine $B$ a označuje sa
$\sup\limits_{x\in B} f(x)$ ($\inf\limits_{x\in B} f(x)$). Ak existuje maximum
(minimum) množiny $f(B)$, nazýva sa toto číslo maximum (minimum) funkcie $f$ na
množine $B$ (často sa používa názov globálne maximum (globálne minimum) funkcie
$f$ na množine $B$) a označuje sa $\max\limits_{x\in B}f(x)$ ($\min\limits_{x\in
B}f(x)$).

%Ak je množina $f(B)$ zdola ohraničená, tak jej infimum sa nazýva infimum funkcie
%$f$ na množine $B$ a označuje sa $\inf_{x\in B} f(x)$. Ak existuje minimum
%množiny $f(B)$, nazýva sa toto číslo minimum funkcie $f$ na množine $B$ (často
%sa používa názov globálne minimum funkcie $f$ na množine $B$) a označuje sa
%$\min_{x\in B}f(x)$.

%Ak je množina $f(B)$ zhora ohraničená, tak jej supremum sa nazýva supremum
%funkcie $f$ na množine $B$ a označuje sa $\sup_{x\in B} f(x)$. Ak existuje
%maximum množiny $f(B)$, nazýva sa toto číslo maximum funkcie $f$ na množine $B$
%(často sa používa názov globálne maximum funkcie $f$ na množine $B$) a označuje
%sa $\max_{x\in B}f(x)$.

\begin{enumerate}[resume]
  \item \useproblem[funkcie]{funkcie-35}
  \item \useproblem[funkcie]{funkcie-36}
  \item \useproblem[funkcie]{funkcie-37}
  \item \useproblem[funkcie]{funkcie-38}
  \item \useproblem[funkcie]{funkcie-39}
\end{enumerate}

Na $A,B\subset\mathbb{R}$ sú neprázdne množiny, $B\subset A$. Funkcia
$f:A\rightarrow\mathbb{R}$ sa nazýva rastúca na množine $B$, resp. neklesajúca
na množine $B$, ak platí
\[
  (\forall x,y \in B):
    x < y \Rightarrow f(x) < f(y)
\]
resp.
\[
  (\forall x,y \in B):
    x < y \Rightarrow f(x) \leq f(y)
\]

Ak má funkcia $f$ niektorú z uvedených štyroch vlastností, nazýva sa monotónna
na množine $B$; funkcia, ktorá je rastúca na množine $B$ alebo klesajúca na
množine $B$, sa nazýva rýdzomonotónna na množine $B$.

Funkcia rastúca (klesajúca, nerastúca, neklesajúca, monotónna, rýdzomonotónna)
na svojom definičnom obore sa nazýva rastúca (klesajúca, nerastúca, neklesajúca,
monotónna, rýdzomonotónna).

\begin{enumerate}[resume]
  \item \useproblem[funkcie]{funkcie-40}
  \item \useproblem[funkcie]{funkcie-41}
  \item \useproblem[funkcie]{funkcie-42}
  \showanswers
  \item \useproblem[funkcie]{funkcie-43}
  \hideanswers
  \item \useproblem[funkcie]{funkcie-44}
  \item \useproblem[funkcie]{funkcie-45}
  \item \useproblem[funkcie]{funkcie-46}
  \item \useproblem[funkcie]{funkcie-47}
\end{enumerate}

Funkcia $f$ sa nazýva periodická, ak existuje číslo $T>0$ tak, že platí
\begin{itemize}
\item
  $
    (\forall a\in D(f)):
      d(f)\cap \interval{a+T}{a+2T}=\{x+T;x\in D(f)\cap \interval{a}{a + T}\}
  $
\item  $(\forall a \in D(f)):f(a+T)=f(a)$
\end{itemize}

Každé číslo $T$ s uvedenými vlastnosťami sa nazýva perióda funkcie $f$; ak
existuje najmenšie také $T>0$, nazýva sa najmenšia perióda funkcie $f$ (možno sa
stretnúť aj s terminológiou, v ktorej sa pojem perióda používa výlučne v zmysle
tu zavedeného pojmu najmenšej periódy).

\begin{enumerate}[resume]
  \item \useproblem[funkcie]{funkcie-48}
  \item \useproblem[funkcie]{funkcie-49}
  \item \useproblem[funkcie]{funkcie-50}
  \item \useproblem[funkcie]{funkcie-51}
  \item \useproblem[funkcie]{funkcie-52}
\end{enumerate}

Funkcia $f:A \rightarrow\mathbb{R}$ sa nazýva prostá (injektívna, jednoznačná),
ak platí:
\[
  (\forall x,y\in A):
    x \neq y \Rightarrow f(x) \neq f(y)
\]

Ak je funkcia $f:A \rightarrow \mathbb{R}$ prostá, tak funkcia s definičným
oborom $f(A)$, ktorý každému číslu $x\in f(A)$ priradí to číslo $y\in A$, pre
ktoré $f(y)=x$, sa nazýva inverzná funkcia k funkcii $f$ a označuje sa $f^{-1}$.

\begin{enumerate}[resume]
  \showanswers
    \item \useproblem[funkcie]{funkcie-53}
  \hideanswers
  \item \useproblem[funkcie]{funkcie-54}
  \item \useproblem[funkcie]{funkcie-55}
  \item \useproblem[funkcie]{funkcie-56}
\end{enumerate}

$\boxed{90.}$
$ \mathbb{Z} $ možno zoradiť do postupnosti $ 0,1,-1,2,-2,3,-3,..... $;
$\boxed{91.}$
zoraďme $ B $ do prostej postupnosti $ b_{1}, b_{2}..... $; "vyškrtaním" prnkov nepatriacich do $ A $ dostaneme prostú postupnosť, ktorá obsahuje všetky prvky z $ A $; pri konkrétnom zápise tejto "voľnej" úvahy využite princíp konštrukcie postupnosti matematickou indukciou: a/ určí sa prvý člen (nech $ k_{1}:= min \lbrace n \in \mathbb{N} ;  b_{n} \in A \rbrace $ , potom $ a_{1}=b_{k_{1}} $); b/ udá sa spôsob, akým určiť člen $ a_{n+1} $, ak sú už určené členy $ a_{1},...,a_{n} $ (ak $ a_{n}=b_{k_{n}} $, potom $ b_{n+1}:= \min \lbrace n \in \mathbb{N}; n>k_{n} \, \land b_{n} \in A \rbrace $, potom $ a_{n+1}:=b_{k_{n+1}} $);
$\boxed{92.}$
pre $ A\cap B=\varnothing $: všimnite si, ako sme "zoradili" množiny $ \lbrace1,2,3,...\rbrace $ a $\lbrace 0,-1,-2,...\rbrace $ v pr. 90; pre $ A \cap B \ne \varnothing $ použite ten istý postup na $ A $ a $ A-B $ (pozor  $ A-B $ môže byť aj nekonečná);
$\boxed{93.}$
ak $ A_{i}\cap B_{i}=\varnothing $ pre $ i \ne j $, $ i,j \in \mathbb{N} $, zoraďte $ A $ podľa schémy

\newpage


vo všeobecnom prípade využite rovnosť $ A=A_{1} \cup (A_{2}-A_{1}) \cup (A_{3}- (A_{1}\cup A_{2})... $;
$\boxed{94.}$
využite pr. 92 a 93 $ ( \mathbb{Q}-\mathbb{Q^{-} \cup \mathbb{Q^{+}_{O}}} $, kde $\mathbb{Q^{+}_{O}} =\bigcup\limits_{n \in \mathbb{R}} Q_{n} $, $  Q_{n}=\lbrace \frac{m}{n}; \, m\in \lbrace 0 \rbrace \cup \mathbb{N}\rbrace )$; 
$\boxed{95.}$
nepriamo, ak $ E-\langle-c,c\rangle $ je spočitateľná pre každé $ c $, tak $ E= \bigcup\limits_{n \in \mathbb{R}} (\langle-n,n\rangle \cap E)$  je spočitateľná;
$\boxed{96.}$
nepriamo, ak $ E_{1} $ je nespočitateľná, tak existuje $ n \in \mathbb{N} $ tak, že $ E_{1} \cap ( \frac{1}{n}, \infty) $ je nespočitateľná, teda nekonečná; potom $ S $ nie je zhora ohraničená;
$\boxed{97.}$
áno; $ A:= \lbrace x-y;  x,y \in E \rbrace $ je spočitateľná; ak $ a \in \mathbb{R}-A $   ($ \mathbb{R}-A $ je neprázdna množina, lebo $ \mathbb{R} $ je nespočitateľná), tak $ E+a \cap E = \varnothing $;
$\boxed{99.}$
z každého intervalu z $ \varphi $ vyberme jedno racionálne číslo (zrejme čísla vybrané z rôznych intervalov sú navzájom rôzne), takto vytvorená množina čísel je podmnožinou množiny $ \mathbb{Q} $, preto je spočitateľná;
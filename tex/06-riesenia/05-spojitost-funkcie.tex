$\boxed{221.}$ $\exists \varepsilon > 0 \, \forall \delta > 0 \, \exists  x_{\delta} \in D(f): \vert x_{\delta} -a \vert < \delta \land \vert f(x_{\delta}) - f(a) \vert \geq \varepsilon $;
$\boxed{222.}$ $\boldsymbol{1.}$ $0$ je hromadný bod množiny $D(f)$, $\lim_{x \to 0} f(x) = 1 = f(0)$, teda $f$ je spojitá v bode $0$;
$\boldsymbol{2.}$  nie je spojitá v bode $0$ ($0$  je hromadný bod $D(f)$  a   $\lim_{x \to 0} f(x) $ neexistuje);
$\boldsymbol{3.}$ nie je spojitá v bode $-1$;
$\boldsymbol{4.}$ nie je spojitá v bode $0$;
$\boldsymbol{5.}$ je spojitá v bode $0$;
$\boxed{223.}$ $\boldsymbol{1.}$ $f(0)$ treba položiť rovné $\frac{3}{2} $ ($= \lim_{x \to 0} f(x)$);
$\boldsymbol{2.}$ $f(0):=\sqrt{e}$;
$\boldsymbol{3.}$ $f(0):=0$;
$\boldsymbol{4.}$ $f(1):=1$;
$\boldsymbol{5.}$ $f(-3):=0$;
$\boxed{224.}$ $\boldsymbol{1.}$ spojitá (vyplýva to z viet 1 a 3);
$\boldsymbol{3.}$ spojitá  v každom bode množiny $\mathbb{R}- \lbrace -1,0,1 \rbrace $, nespojitá v bodoch $-1,0,1 $;
$\boldsymbol{4.}$ spojitá len  v bodoch množiny $(\mathbb{R}- \mathbb{Z}) \cup \lbrace 0 \rbrace $;
$\boldsymbol{5.}$ $f(x)=1$ pre $x \in \langle0,1)$, $f(x):=\frac{1}{2}$  pre $x =1$, $f(x):=0$  pre $x > 1$; $f$ je spojitá práve v bodoch množiny $\langle 0, +\infty) -\lbrace 1 \rbrace $;
$\boldsymbol{6.}$ $f(x)=1$ pre $ \vert x \vert \leq 1$, $f(x)=x^{2}$  pre $ \vert x \vert >1$; $f$ je spojitá;
$\boldsymbol{7.}$ (množina $D(f)= \mathbb{Z}$ pozostáva len z izolovaných bodov);
$\boxed{227.}$ nemožno; napr. 1. ak $x_{0}$ je izolovaný bod množiny $D(f)$, je funkcia $g\circ f $ spojitá v bode $x_{0}$ ($x_{0}$ je potom totiž aj izolovaný bod množiny  $D(g\circ f)$); alebo 2. $f(x)= const$, $x \in  \mathbb{R}$  ; $ g $ je ľubovoľná funkcia nespojitá v bode $ f(a)$;
$\boxed{228.}$ pri dôkaze spojitosti funkcie $\vert f \vert$ v bode $a$ využite nerovnosť $\vert \vert r \vert - \vert s \vert \vert \leq \vert r-s \vert$ ($r,s \in \mathbb{R}$) a spojitosť funkcie  $f$  v bode $a$; ďalej platí $\max \lbrace f,g \rbrace =\frac{f+g+\vert f-g\vert}{2}$, analogicky možno vyjadriť $\min \lbrace f,g \rbrace $;
$\boxed{229.}$ funkcia $\chi $ nie je spojitá v žiadnom bode  množiny $\mathbb{R}$;
$\boxed{230.}$ nech $a \in \mathbb{R} - \mathbb{Q} $, nech $\varepsilon  >0$ je dané; nech $\overline{n} \in \mathbb{N} $ je také, že $\frac{1}{\overline{n}}< \varepsilon $; pre každé $n \in \lbrace1,...,\overline{n} \rbrace $ existuje $\delta_{n}>0 $ tak, že v $O_{\delta_{n}}(a)$ neležia body s funkčnými hodnotami  $\frac{1}{n} $; pre dané  $\varepsilon $ potom stačí položiť  $\delta = \min \lbrace \delta_{1},...,\delta_{\overline{n}} \rbrace$; dôkaz druhého tvrdenia z pr. 230 prenechávame čitateľovi;
$\boxed{232.}$ $\boldsymbol{1.}$ $-1$ je bod neodstrániteľnej nespojitosti, bod nespojitosti 2. druhu;
$\boldsymbol{2.}$ $1$ a $-2$ sú body neodstrániteľnej nespojitosti, body nespojitosti 2. druhu;
$\boldsymbol{3.}$ $0$ je bod odstrániteľnej nespojitosti;
$\boldsymbol{4.}$ $1$ a $0$ sú body odstrániteľnej nespojitosti, $-1$ je bod neodstrániteľnej nespojitosti, bod nespojitosti 2. druhu;
$\boldsymbol{5.}$ $2$ a $-2$ sú body odstrániteľnej nespojitosti;
$\boldsymbol{6.}$ $0$ je bod neodstrániteľnej nespojitosti, bod nespojitosti 2. druhu;
$\boldsymbol{7.}$ $0$ je bod odstrániteľnej nespojitost;
$\boldsymbol{8.}$ $0$ je bod odstrániteľnej nespojitost, $1$ je bod neodstrániteľnej nespojitosti, bod nespojitosti 2. druhu;
$\boldsymbol{9.}$ $-1$ je bod neodstrániteľnej nespojitost, bod nespojitosti 1. druhu;
$\boldsymbol{10.}$ každý prvok množiny $\lbrace \frac{1}{n}; \, n \in \mathbb{Z} -\lbrace 0 \rbrace \rbrace$ je bod neodstrániteľnej nespojitosti, bod nespojitosti 1. druhu;  $0$ je bod odstrániteľnej nespojitost  $\lim_{x \to 0} f(x) =1$; platí totiž $f \langle ( \frac{1}{n+1}, \frac{1}{n} \rangle ) $ pre  $n \in \mathbb{N}$,  $f (( \frac{1}{n}, -\frac{1}{n+1} \rangle ) = \langle 1, \frac{n+1}{n})$ pre  $n \in \mathbb{N}$;
$\boldsymbol{11.}$ $f(x) = x$, $D(f) = \mathbb{R} - \lbrace 0,1 \rbrace$; $0$ a $1$ sú body odstrániteľnej nespojitosti;
$\boldsymbol{12.}$ každý prvok  $a \in\mathbb{R}- \mathbb{Z} $ je bod neodstrániteľnej nespojitosti, bod nespojitosti 2. druhu;
$\boldsymbol{13.} $ platí $g(\varphi (x)) = g(x)$;  $0$ a  $2$ sú body odstrániteľnej nespojitost (všimnite si napriek tomu, že ;$\varphi $ je spojitá len v bode $1$, je $g \circ \varphi $ spojitá fubkcia);
$\boxed{234.}$ $\boldsymbol{2.}$ pre spojitú funkciu  $f(x) = x- \cos x$ platí  $f(0)<0$, $f(\frac{\pi}{2})>0$;
$\boldsymbol{3.}$ $\lim_{x \to \infty} P(x) $ a $\lim_{x \to -\infty} P(x) $ sú nevlastné a opačných znamienok, preto musia existovať $a,b \in \mathbb{R}$ tak, že $P(a)> 0$, $P(b)< 0 $ ;
$\boxed{235.}$ využite nerovnosti $\min \lbrace f(x_{1}),..., f(x_{n}) \rbrace \leq \frac{1}{n}(f(x_{1})+...+f(x_{n}))\leq \max \lbrace f(x_{1}),..., f(x_{n}) \rbrace$;
$\boxed{236.}$ ak $f(0)=0$ alebo   $f(1)=1$, je dôkaz skončený; ak  $f(0)\ne 0$, $f(1)\ne 1$, tak  pre  spojitú  funkciu  $g(x) = f(x)-x $ platí  $fg(0)>0$, $fg(1)< 0$;
$\boxed{238.}$ stačí dokázať tvrdenie: každá nekonštantná funkcia s uvedenou vlastnosťou je spojitá;
$\boxed{240.}$ $f$ nadobúda na $\langle a,b \rangle$ minimum, ktoré (pretože je funkčnou hodnotou), je kladné;
$\boxed{241.}$ ak $a,b \in \mathbb{R}$, tak funkcia $g$ určená predpisom $g(a) = \lim_{x \to a^{+} }f(x), \, g(b) = \lim_{x \to b^{-}} f(x), \, g(x)=f(x)$ pre $x \in (a,b)$ je spojitá, a teda ohraničená na kompakte $\langle a,b \rangle$; $f$ je potom ohraničená ako zúženie ohraničenej funkcie; vo všeobecnosti (tj. ak $a, b \in \mathbb{R^{*}}$) možno dokazovať nasledovne: $f$ je ohraničená na niektorom okolí $O(a)$ (to vyplýva z existencie vlastnej  $\lim_{x \to a} f(x)$)  a na niektorom okolí $O(b)$; množina $A:=(a,b)-(O(a) \cup O(b)) $ môže byť prázdna, jednoprvková (v týchto prípadoch je dôkaz skončený) alebo kompaktný interval, na tom je $f$ ohraničená;
$\boxed{242.}$ ak  $\lim_{x \to \infty }P(x) = \lim_{x \to -\infty }P(x)= +\infty $, tak existuje $a>0$ tak, že $\forall x \in \mathbb{R}, \, \vert x \vert > a: \, f(x) > f(0) $; na kompakte $\langle -a,a \rangle$ nadobúda $f$ minimum, ktoré je globálnym minimom na $ \mathbb{R} $;
$\boxed{245.}$ dôkaz je analogický dôkazu veta 5;
$\boxed{246.}$ napr. $f(x)=e^{x} \sin \frac{1}{x}$, $ x \in (0,1 \rangle$; 
$\boxed{247.}$ ($\forall x \in (a,b) \quad \forall\varepsilon  >0 \quad   \exists \delta >0 \,  \forall y \in (a,b): \, \vert x-y \vert < \delta \, \Rightarrow \, \vert f(x)-f(y) \vert < \varepsilon$)  $\land$ ( $\exists \varepsilon _{0} >0 \quad \forall \delta  >0 \, \exists x_{\delta}, y_{\delta}  \in (a,b): \vert  x_{\delta}- y_{\delta}\vert <  \delta $  $\land$  $\vert f(x_{\delta})-f(y_{\delta}) \vert \geq \varepsilon_{0} $);
$\boxed{248.}$  $\boldsymbol{1.}$ áno;
$\boldsymbol{3.}$  nie;
$\boldsymbol{4.}$ áno;
$\boldsymbol{6.}$ áno (Ťažkosti spôsobuje bod $0$, keby sme boli na intervale $\langle \eta , +\infty)$, kde $\eta >0$ , mohli by sme použiť odhad $ \vert \sqrt{x} - \sqrt{y} \vert =\vert \frac{x-y}{\sqrt{x} + \sqrt{y}}\vert \leq \frac{\vert x-y \vert}{2 \sqrt{\eta}} $. Možmo uvažovať nasledovne: nech je da né $\varepsilon  >0 $; ak $0 <  x <  \varepsilon ^{2} $, $0 <  y <  \varepsilon ^{2} $, tak  $ \vert \sqrt{x} - \sqrt{y} \vert < \varepsilon$. Zostáva sa ešte zaoberať prípadom $x\geq \frac{\varepsilon^{2}}{2} $, $y\geq \frac{\varepsilon^{2}}{2} $; potom $ \vert \sqrt{x} - \sqrt{y} \vert \leq \frac{\vert x-y \vert}{2 \sqrt{\eta}}$; teda pre dané $\varepsilon$ a $x,y \geq \frac{\varepsilon^{2}}{2}$ stačí položiť $\delta <  \sqrt{2}\varepsilon^{2} $, tj. napr. $\delta =\frac{\varepsilon^{2}}{2}$. Celkovo sme dokázali: ak  $ \vert x-y  \vert < \frac{\varepsilon^{2}}{2}$, tak $ \vert \sqrt{x} - \sqrt{y} \vert < \varepsilon$.);
 $\boldsymbol{7.}$ áno  (využite vzorec $ \sin x - \sin y = 2 \sin \frac{x-y}{2} \cos \frac{x+y}{2}$  a nerovnosť $\sin \alpha < \vert \alpha\vert$);
$\boxed{249.}$ "$\Rightarrow$ ": k číslu $\frac{\varepsilon}{2}$	nájdime $\delta$ z definície rovnomernej spojitosti, nech pre  body $a=x_{0 }< x_{1}<...< x_{n}=b$ platí $\vert x_{i} -x_{i-1}\vert < \delta$ ($i=1,...,n $), potom graf funkcie $\varphi $ je lomená čiara s vrcholmi  $(x_{0 }, f(x_{0})),(x_{1 }, f(x_{1})), ..., (x_{n }, f(x_{n}))$; pri dôkaze implikácie "$\Leftarrow$" využite, že každá spojitá po častiach lineárna funkcia definovaná na ohraničenom intervale $I$ je rovnomerne spojitá (to možno dokázať priamo z definície), $\vert f(x)-f(y) \vert \leq  \vert f(x)-\varphi(x) \vert + \vert \varphi(x)-\varphi(y) \vert + \vert \varphi(y)-f(y) \vert$;
$\boxed{250.}$ návodom je dôkaz implikácie "$\Rightarrow$ " v pr. 249;
$\boxed{251.}$ $\boldsymbol{1.}$ áno ($f$ je spojitá, a teda podľa vety 6 aj rovnomerne spojitá na kompakte $\langle -1,1 \rangle $);
$\boldsymbol{3.}$ áno (zvoľme $a>0$ pevné; rovnomernú spojitosť na $\langle a,+ \infty) $ možno dokázať z definície, rovnomerná spojitosť na $\langle 0,a \rangle $ vyplýva z vety 6; ak je funkcia rovnomerne spojitá na $\langle 0,a \rangle $ a na  $\langle a,+ \infty) $, tak je rovnomerne spojitá na ich zjednotení);
$\boldsymbol{4.}$ áno ( $f$ je spojitá na kompakte $\langle 0,\pi \rangle$);
$\boldsymbol{5.}$ nie (rodeľme každý z intervalov $\langle \frac{\pi}{2}+k \pi,\frac {3 \pi}{2}+ k\pi \rangle $, $k=0,1,2,...,$ na $n$ rovnakých častí bodmi $x_{0}^{(k)}:= \frac{\pi}{2}+k\pi$,  $x_{1}^{(k)}:= \frac{\pi}{2}+k\pi + \frac{\pi}{n}$,...,$x_{n-1}^{(k)}:= \frac{\pi}{2}+k\pi + \frac{(n-1)\pi}{n}$, $x_{n}^{(k)}:= \frac{3\pi}{2}+k\pi $. Pretože $2\pi({1+k})=$ $\vert f(x_{0}^{(k)})-f(x_{n}^{(k)})  \vert\leq $ $\vert f(x_{0}^{(k)})-f(x_{1}^{(k)})  \vert+...+\vert f(x_{n-1}^{(k)})-f(x_{n}^{(k)})  \vert $, musí byť aspoň jedno z čísel na pravej strane tejto  nerovnosti väčšie ako $\frac{2\pi(1+k)}{n} $. Teda pre ľubovoľné $n \in \mathbb{N}$ existuje v každom z intervalov $\langle \frac{\pi}{2}+k\pi,\frac{3\pi}{2}+k\pi\rangle $ dvojica ($x_{k},y_{k} $) taká, že $\vert x_{k}-y_{k}\vert \leq  \frac{\pi}{n}$ a $\vert f(x_{k})-f(y_{k})\vert > \frac{2\pi(1+k)}{n}$, pritom $\frac{2\pi(1+k)}{n} \rightarrow +\infty$ pre $k \rightarrow +\infty$);
$\boldsymbol{6.}$ áno ($\lim_{x \to 1}f(x)=\lim_{x \to -1}f(x)=0 $,ďalej pozri riešenie pr. 251.2);
$\boldsymbol{7.}$ áno (Ak $x,y\geq 1 $, $0< x-y < \eta  $, tak  $\frac{x}{y}=\frac{x-y+y}{y} < 1+\eta $.
Pretože $x>y\geq 1 $ je nerovnosť  $\vert\ln x - \ln y \vert < \varepsilon $ ekvivalentná s nerovnosťou $\frac{x}{y}< e^{\varepsilon}=1+(e^{\varepsilon}-1) $, stačí pre dané $\varepsilon>0$ položiť $\delta= \varepsilon-1 $.);
$\boxed{252.}$ ak $a:=\lim_{x \to \infty }f(x)$, tak k danému  $\varepsilon >0$ existuje $O(+\infty) $ tak, že $f(O(+\infty)) \subset (a-\frac{\varepsilon}{2}, a+\frac{\varepsilon}{2}) $; pre $ x,y \in O(+\infty)$ potom platí $\vert  f(x)-f(y) \vert < \varepsilon$; existuje kompaktný interval $I$ obsahujúci množinu $ A:= \langle 0,+\infty)-O(+\infty) $ (pozn.: v prípade $A=\varnothing$ netreba už vlastne postupovať ďalej, pre dané $\varepsilon>0$ stačí za $\delta$
 zvoliť ľubovoľné kladné číslo; úvahu s intervalom   $I$  možno ale použiť aj na tento prípad, vyhneme sa tým predlžovaniu dôkazu spôsobenému rozlišovaním jednotlivých možmostí) na  $I$ je  $f$ rovnomerne spojitá, a preto k číslu  $\varepsilon$ možno nájsť príslušné  $\delta$ z definície rovnomernej spojitosti (na  $I$; zvyšok úvah, ako nájsť  $\delta$ vyhovujúce definícii rovnomernej spojitosti na celom intervale  $\langle 0, +\infty)$, prenecháme čitateľovi; 
$\boxed{254.}$ $\boldsymbol{1.}$ sporom; nech $f$ je rovnomerne spojitá na $(a,b)$, zvoľme $\varepsilon >0$ pevné, napr.   $ \varepsilon=1$ a nájdime k nemu príslušné $\delta$ z definície rovnomernej spojitosti; potom pre ľubovoľné $x \in (b-\delta, b) $ je $\vert x-(b-\delta) \vert < \delta$, preto $\vert f(x)-f(b-\delta)) \vert < \varepsilon$; funkcia $f$ je teda ohraničená na $(b-\delta, b) $, čo je spor s tým, že $\lim_{x \to b^{-} }f(x)=+\infty$;
$\boldsymbol{2.}$ ak neexistuje  $\lim_{x \to b^{-} }f(x)$, tak existujú postupnosti $\lbrace a_{n} \rbrace _{n=1}^{\infty} $, $\lbrace b_{n} \rbrace _{n=1}^{\infty} $ ( $a_{n}, b_{n} \in (a,b))$ tak, že $\lim_{n \to \infty }a_{n}=\lim_{n \to \infty }b_{n}=b$, ($A:=)$  $\lim_{n \to \infty }f(a_{n}) \ne \lim_{n \to \infty }f(b_{n})$ ($B:=)$ (pozri pr. 199); ak $A, B \in \mathbb{R}, \, A > B $ (ostatné prípady sú analogické), tak v ľubovoľne malom okolí bodu  $b$ existujú body s funkčnými hodnotami väčšími ako  $ A-\frac{A-B}{3} $ aj body s funkčnými hodnotami menšími ako  $ B+\frac{A-B}{3} $; (iný možný dôkaz tvrdení z pr.254: ak  $f $ je rovnomerne spojitá na  $(a,b) $, tak v bode  $b $ je splnené Cauchyho-Bolzanovo kritérium konvergencie);
$\boxed{255.}$ inšpiráciou pre dôkaz  $" \Leftarrow "$ je pr. 251.2, pri dôkaze $" \Rightarrow "$ využite pr. 254;
$\boxed{256.}$ $\boldsymbol{1.}$ využite pr. 255 a 241;
$\boldsymbol{2.}$ pozri pr. 248.3;
$\boxed{257.}$ $\boldsymbol{1.}$ každý prvok množiny $\lbrace (-1)^{k}\sqrt{n}; \, k \in \lbrace -1,1 \rbrace, n \in \mathbb{N}\rbrace $ je bod neodstrániteľnej nespojitosti, bod  nespojitosti 1. druhu; $0$ je bod neodstrániteľnej nespojitosti, bod  nespojitosti 2. druhu;
$\boldsymbol{2.}$ každý prvok množiny $\lbrace \frac{(-1)^{k}}{n}; \, k \in \lbrace -1,1 \rbrace, n \in \mathbb{N}\rbrace $ je bod neodstrániteľnej nespojitosti, bod  nespojitosti 1. druhu; $0$ je bod neodstrániteľnej nespojitosti, bod  nespojitosti 2. druhu;
$\boldsymbol{3.}$  $0$ je bod neodstrániteľnej nespojitosti, bod  nespojitosti 2. druhu;
$\boldsymbol{4.}$ každý prvok množiny $\lbrace \frac{k \pi}{2}; \, k \in  \mathbb{Z}- \lbrace 0 \rbrace\rbrace $ je bod neodstrániteľnej nespojitosti, bod  nespojitosti 1. druhu; 
$\boldsymbol{5.}$ $f$ je spojitá funkcia;
$\boldsymbol{6.}$ $f$ je spojitá funkcia; 
$\boxed{258.}$ (pozri pr. 192) $\boldsymbol{1.}$ $0$;
$\boldsymbol{2.}$ $1$;
$\boldsymbol{3.}$ $0$; 
$\boxed{259.}$ (možno využiť pr. 228) ($F(x)= \min \lbrace c, \max \lbrace f(x), -c \rbrace \rbrace $); 
$\boxed{260.}$ obrátená implikácia neplatí (položme napr. $ a_{n}=f(q_{n})$, kde $f(x)=0$ pre $x \in \mathbb{Q}-\mathbb{N}, \, f(x)=1 $ pre $x \in \mathbb{N}$ ;
$\boxed{261.}$ $f$ je spojitá v každom bode množiny $A= \lbrace (\mathbb{R}-\mathbb{Q} \cap (0+\infty))$ a nespojitá v bodoch množiny $\mathbb{R} -A$;
$\boxed{262.}$  $f$ je spojitá;
$\boxed{263.}$ $\boldsymbol{1.}$ nevyplýva (vhodný protipríklad už nájdite sami); uvedená podmienka je ekvivalentná s podmienkou:  $f$ je ohraničená v každom okolí bodu  $x_{0}$;
$\boldsymbol{2.}$ nevyplýva (uvažujte napr. $f(x) = x$ pre $x<0$,   $f(x) = x+1$ pre $x\geq 0$); uvedená podmienka je ekvivalentná s podmienkou:  $f$ má spojitú inverznú funkciu;
$\boxed{264.}$ $\boldsymbol{1.}$ nie (možno využiť pr. 232.13);
$\boldsymbol{2.}$ áno (pri dôkaze možno využiť pr.239);
$\boxed{266.}$ napr. $f(x)=x$ pre $x \in \mathbb{Q}- \lbrace 0,1 \rbrace $, $f(x)=-x$ pre $x \in \mathbb{R}- \mathbb{Q} $, $f(0)=1$, $f(1)=0$;
$\boxed{267.}$ Sporom. Zvoľme $x_{0}$, platí $1_{0}:=\lim_{x \to x_{0} }f(x) \ne f(x_{0})$, preto $\exists \varepsilon_{0} > 0 $  $ \exists\delta_{0 } > 0: f(x_{0}) \notin \langle 1_{0}-\varepsilon_{0}, 1_{0}+\varepsilon_{0} \rangle \, (=:B_{0}) \, \land f(\langle x_{0}+\delta_{0}/3, x_{0}+\delta_{0}\rangle ) \subset B_{0}$. Pre bod $x_{1} :=x_{0}+2\delta_{0}/3$ platí $l_{1}:= \lim_{x \to x_{1} }f(x) \ne f(x_{1}) $, preto $\exists 0< \varepsilon_{1} < \varepsilon_{0}/2 $  $ \exists 0<\delta_{1 } < \delta_{0}/3: f(x_{1}) \notin \langle 1_{1}-\varepsilon_{1}, 1_{1}+\varepsilon_{1} \rangle \, (=:B_{0}) \, \land f(\langle x_{1}+\delta_{1}/3, x_{1}+\delta_{1}\rangle ) \subset \langle l_{1}-\varepsilon_{1},l_{1}+\varepsilon_{1} \rangle$; teraz položme  $x_{2} :=x_{1}+2\delta_{1}/3$ atď. Tak dostaneme postupnosť prvkov $\lbrace x_{n} \rbrace _{n=0}^{\infty}$, postupnosť uzavretých intervalov  $\lbrace B_{n} \rbrace _{n=0}^{\infty} = \lbrace \langle l_{n}-\varepsilon_{n} , l_{n}+\varepsilon_{n}\rangle \rbrace _{n=0}^{\infty}$ a postupnosť uzavretých intervalov  $\lbrace A_{n} \rbrace _{n=0}^{\infty}= \lbrace \langle x_{n}+ \delta_{n}/3, x_{n}+\delta_{n} \rangle \rbrace _{n=0}^{\infty}$ s vlastnosťami: $A_{0}\supset A_{1} \supset  A_{2} \supset A_{3} \supset ..., B_{0}\supset B_{1} \supset B_{2} \supset...,$  $\delta_{n+1}< \delta_{n}/3 $; $\varepsilon_{n+1}< \varepsilon_{n}/2$,  $x_{n+1}\in A_{n}- A_{n+1} $, $f(x_{n+1})\in B_{n} $, $f(A_{n})\subset B_{n} $. Označme $a,b$ tie prvky, pre ktoré platí $\lbrace a\rbrace = \bigcap\limits_{n=0}^\infty A_{n} $, $\lbrace b\rbrace = \bigcap\limits_{n=0}^\infty B_{n} $; potom platí  $ a= \lim_{n \to \infty}x_n , \,  b= \lim_{n \to \infty}f(x_n)$, preto $\lim_{n \to \infty}f(x=b)$. Súčasne ale (pretože $a \in A_{n} $ pre každé $n=0,1,...) $ je $f(a) \in B_{n} (n=0,1,...)$, preto $f(a)=b $, čo je spor.
$\boxed{268.}$ $\boldsymbol{1.}$ každá z množín $ A_{n}:=\lbrace x \in \mathbb{R}; \, f(x)>\frac{1}{n}\rbrace $,  $ B_{n}:=\lbrace x \in \mathbb{R}; \, f(x)<\frac{1}{n}\rbrace $ ($ n \in \mathbb{N}$ je konečná a $A = \bigcup\limits_{n=1}^\infty (A_{n} \cup  B_{n} $;
$\boldsymbol{2.}$ množina  $ M=\lbrace x \in \mathbb{R};f(x)=0 \rbrace  $ je konečná, funkcia  $g(x) = \frac{1}{f(x)}, x \in \mathbb{R}-M $, vhodne dodefinovaná v bodoch množiny $M$, vyhovuje predpokladom pr. 268.1;
$\boxed{271.}$ existujú postupnosti $\lbrace a_{n}\rbrace_{n=1}^{\infty} $, $\lbrace b_{n}\rbrace_{n=1}^{\infty}$ tak, že $  \lim_{n \to \infty}a_n =\lim_{n \to \infty}b_n = +\infty$, $(a:=\lim_{n \to \infty}f(a_n)<\lim_{n \to \infty}f(b_n) = +\infty (=:b$, $a,b \in \mathbb{R}$ (pozri pr. 199); potom stačí položiť napr. $A=\frac{a+b}{2} $;
$\boxed{273.}$ sporom; ak $g(x):=f(x+ \frac{T}{2})-f(x) \ne 0$ pre každé $x\in \mathbb{R}$, tak $g$ nemení znamienko na $ \mathbb{R}$, potom ale (ak napr. $g(x) >0 $ na  $\mathbb{R} $ je  $f(x+T) - f(x)= (f(x+T)-f(x+T/2))+(f(x+T/2)-f(x)) >0$;
$\boxed{274.}$ $\boldsymbol{2.}$ napr. $f(x)=0$ pre  $x\in \mathbb{R}-(\mathbb{N} \cup \lbrace \frac{1}{n}; n\in \mathbb{N} \rbrace$), $f(n)=\frac{1}{n} \in \mathbb{R}$;
$\boldsymbol{3.}$ napr. $f(x)=x$ pre  $x\in \langle 0,1\rangle $, $f(x)=x-1$ pre  $x\in (2,3\rangle $;
$\boxed{276.}$ $\boldsymbol{1.}$ pri dôkaze rovnomernej spojitosti funkcie $f.g$ využite, že $f$ aj $g$ sú na $(a,b)$ ohraničené (pozri pr. 256.1);
$\boxed{277.}$ tvrdenie je nepravdivé;
$\boxed{278.}$ $\boldsymbol{1,2.}$ každá funkcia rovnomerne spojitá na $\mathbb{R}$;
$\boldsymbol{3.}$ každá konštantná funkcia;

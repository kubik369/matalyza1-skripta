$\boxed{100.}$
$\boldsymbol{1.}$
$ A=\lbrace 1/k \pi; \, k \in \mathbb{Z}- \lbrace 0 \rbrace \rbrace $, treba dokázať $ \forall O_{\varepsilon} (0) $ $ \exists x \in A; x \ne 0 \land x \in O_{\varepsilon} (0) $, čo možno zapísať v tvare $ \forall \varepsilon > 0 $ $ \exists k \in \mathbb{Z}- \lbrace0\rbrace : \vert 1/k \pi \vert < \varepsilon $, existencia riešenia poslednej nerovnosti vyplýva z Archimedovho princípu;
$\boldsymbol{3.}$
využite, že do $ A $ patria čísla $ 0,1;0,11;0,111...... $;
$\boxed{101.}$
$\boldsymbol{1.}$
$ A^{)} \langle0,1\rangle $ (nestačí len dokázať, že každý prvok z $ \langle0,1\rangle $ je hromadným bodom množiny $ A $, treba aj dokázať, že žiadny prvok z $ \mathbb{R}^{\chi} - \langle0,1\rangle $ už nepatrí do $ A^{)} $);
$\boldsymbol{2.}$
$ B^{)}= \lbrace0,\infty \rbrace $;
$\boldsymbol{3.}$
$ C ^{)}=\langle-2,+ \infty) \cup \lbrace+\infty\rbrace $;
$\boldsymbol{4.}$
$ D- \mathbb{Q} \cap (0,1) $, $ D^{)} = \langle0,1\rangle$ (pri dôkaze poslednej rovnosti využite fakt, že (nedegenerovaný) interval obsahuje aspoň jedno racionálne číslo);
$\boldsymbol{5.}$
$ E= \mathbb{Q}-\lbrace1\rbrace $, $ E^{)}=\mathbb{R}^{\chi} $;
$\boldsymbol{6.}$
$ F^{)}=\langle1,2\rangle $;
$\boxed{102.}$
$\boldsymbol{1.}$
$ \exists O(+ \infty):M \cap (+ \infty)=\varnothing $, to možno zapísať aj v tvare $ \exists x \in \mathbb{R} \quad \forall x \in M; x\leq K $;
$\boldsymbol{2.}$
$ \forall a \in \mathbb{R}^{\chi} \exists O(a); \quad (O(a)-\lbrace a \rbrace ) \cap M= \varnothing \ $ to možno zapísať aj v tvare $ \forall a \in \mathbb{R}^{\chi} \quad  \exists O(a) \quad \forall x \in M: \quad x=a \lor x \notin O(a)  $;
$\boxed{103.}$
$\boldsymbol{1.}$
ľubovoľná konečná neprázdna množina má túto vlastnosť;
$\boldsymbol{2.}$
napr. $ A= \mathbb{N} \cup \lbrace \frac{n+1}{n}; n\in \mathbb{N} \rbrace$;
$\boldsymbol{3.}$
napr. $ A= \mathbb{Z} $;
$\boldsymbol{4.}$
na $ A $ stačí zvoliť ľubovoľný interval;
$\boldsymbol{5.}$
napr. $ A= \lbrace m+ \frac{1}{n}; \, n \in \mathbb{N} \rbrace $, potom $ A^{)}= \mathbb{N} \cup \lbrace \infty \rbrace $;
$\boxed{104.}$
pri dôkaze využite druhú vlastnosť suprema;
$\boxed{105.}$
$\boldsymbol{1.}$
nerovnosti $ \vert \frac{3n^{2}+1}{5n^{2}-1} - \frac{3}{5} \vert < \varepsilon $ (pre dané $ \varepsilon> 0 $) vyhovujú všetky tie $ n \in \mathbb{N} $, pre ktoré platí $ n > \sqrt{\frac{5\varepsilon +8}{25 \varepsilon}} $; stačí teda položiť napr. $ n_{0}= [  \sqrt{\frac{5\varepsilon +8}{25 \varepsilon}}]   $ pre $ \varepsilon \in (0, \frac{2}{5}\rangle $ a $ n_{0}= 1 $ pre $ \varepsilon > \frac{2}{5} $ (pre $ \varepsilon > \frac{2}{5} $ je totiž $ [  \sqrt{\frac{5\varepsilon +8}{25 \varepsilon}}] =0 $ keby  sme v definícii limity postupnosti namiesto $n \in \mathbb{N}  $ požadovali $ n \in \mathbb{R} $ stačilo by pre dané $ \varepsilon > 0 $ položiť $ n_{0} \geq   \frac{5\varepsilon +8}{25 \varepsilon }$); a/ pre všetky $  n \in \mathbb{N} $; b/ pre všetky $ n\geq 8 $; c/ pre všetky $ n\geq 80 $;
$\boldsymbol{3.}$
nerovnosti $ \frac{n^{2}}{n+8}>K $ vyhovujú pre $ K > 0 $ všetky čísla $ n \in \mathbb{N} $ pre $ K \geq 0 $ všetky tie $ n \in \mathbb{N} $, pre ktoré platí $ n > \frac{1}{2} (K+ \sqrt{K^{2}+32K)} $ stačí teda položiť napr. $ n_{0}=1 $ pre $ K < 0 $, $ n_{0}= [ \frac{1}{2} (K+ \sqrt{K^{2}+32K)}] +1 $ pre $ K\geq 0 $ pripočítaním čísla 1 zaručíme, že aj v prípade $ [ \frac{1}{2} (K+ \sqrt{K^{2}+32K)}]=0 $ bude platiť $ n_{0} \in \mathbb{N}; \frac{n^{2}}{n+8}> 10^{3} $ platí pre všetky $ n > 1007 $;
$\boldsymbol{4.}$
nerovnosti $ \frac{5}{n}-n < K $ vyhovujú všetky tie $ n \in \mathbb{N} $, pre ktoré $ n > \frac{1}{2} ( \sqrt{K^{2}+20} -K)  $, stačí teda pre dané $ K \in \mathbb{R} $ položiť $ n_{0} = [ \frac{1}{2} ( \sqrt{K^{2}+20}-K )]+1 $; iné riešenie : pretože  $ \frac{5}{n}-n \leq 5-n $
pre $ n \in \mathbb{N} $ je každé riešenie nerovnice $ \frac{5}{n}-n < K $, teda nerovnici $ \frac{5}{n}-n < K $ iste vyhovujú všetky tie $ n $, pre ktoré $ a> 5-K $ (podobnými únahami možno zjednodušiť aj riešenia pr. 105.1-3);
$\boldsymbol{5.}$
ak $ q=0 $ , stačí položiť $ n_{0}=1 $ pre každé $ \varepsilon  > 0 $, ak $ q\neq 0 $, $ \vert q \vert < 1 $, sú riešeniami nerovnice $ \vert q^{n} \vert < \varepsilon $ všetky tie $ n \in \mathbb{N} $, pre ktoré platí $ n  > \frac{\ln \varepsilon}{\ln \vert q \vert} $, stačí teda položiť $ n_{0}=1 $ pre $ \varepsilon\geq 1 $, $ n_{0}= [  \frac{\ln \varepsilon}{\ln \vert q \vert} ]+1 $ pre $ \varepsilon \in (0,1) $;
$\boxed{106.}$
$\boldsymbol{1.}$
$ \lim_{n \to \infty}a_n=1$; nerovnosti $ \vert a_{n} -1 \vert < \varepsilon $ vyhovujú všetky tie párne $ n \in \mathbb{N} $, pre ktoré $ n > \frac{1}{\varepsilon} $ a všetky tie nepárne $ n \in \mathbb{N} $, pre ktoré $ n > \frac{1}{\sqrt{\varepsilon}} $, pre dané $ \varepsilon > 0 $ stačí teda položiť napr. $ n_{0}= [ \max \lbrace \frac{1}{\varepsilon} , \frac{1}{\sqrt{\varepsilon}} \rbrace ]+1 $;
$\boldsymbol{2.}$
$ \lim_{n \to \infty}a_n=0$;
$\boxed{108.}$
$\boldsymbol{1.}$
neexistujú, nerovnica $\vert a_{n}  -b \vert < \varepsilon   $ nemá totiž riešenie pre $ \varepsilon < 0 $;
$\boldsymbol{2.}$
len konštantná postupnosť $ b,b,b,b,...(b \in \mathbb{R}) $;
$\boldsymbol{3.}$
na overenie pravdivosti tvrdenia  $ \exists  \, \varepsilon > 0 \quad \exists \, n_{0 } \in \mathbb{N} \quad \forall n \in \mathbb{N} , \, n> n_{0}: \vert 2-7 \vert < \varepsilon $ stačí položiť $ \varepsilon = 6 $;
$\boldsymbol{4.}$
len konštantná postupnosť $ b,b,b,b,...(b \in \mathbb{R}) $;
$\boxed{109.}$
áno; z tvrdenia $ \exists N^{*}\in \mathbb{N} \quad \forall \, \varepsilon > 0 $ $ \forall n \in \mathbb{N} , \, n> N^{*} $: $ \vert a_{n} -b \vert < \varepsilon $ vyplýva totiž tvrdenie $ \forall \varepsilon > 0 $, $ \exists n_{0 }\in \mathbb{N} $, $ \forall n \in \mathbb{N}, \, n>n_{0} $: $ \vert a_{n} -b \vert < \varepsilon $; pre každé $ \varepsilon > 0  $ stačí položiť $ n_{0} = N^{*} $ ( z tvrdenia   $ \exists N^{*}\in \mathbb{N} \quad \forall \, \varepsilon > 0 $ $ \forall n \in \mathbb{N} , \, n> N^{*} $: $ \vert a_{n} -b \vert < \varepsilon $ vyplýva aj to, že pre  $ n> N^{*}$  je $ a_{n}=b $);
$\boxed{110.}$
$\boldsymbol{1.}$
všetky ohraničené postupnosti; 
$\boldsymbol{2.}$
len postupnosť   $ 0,0,0,... $;
$\boldsymbol{3.}$
všetky  ohraničené  postupnosti (každú z uvedených odpovedí treba samozrejme podrobne zdôvodniť);
$\boxed{111.}$
$\boldsymbol{2.}$
$\forall K \in \mathbb{R} \quad \exists \delta > 0  \,  \forall x \in D(f), \, x \ne a: \, \vert x-a \vert < \delta  $  $\Rightarrow   \, f(x)  > K  $; 
$\boldsymbol{3.}$
$\forall K \in \mathbb{R} \quad \exists \delta  >0  \, \forall x \in D(f), \, x \ne a: \, \vert x-a \vert < \delta  $  $\Rightarrow   \, f(x)  < K  $;
$\boldsymbol{4.}$
$ \forall \varepsilon >0  $  $\exists L \in \mathbb{R} \quad \forall x \in D(f): \, x>L $ $\Rightarrow \, \vert f(x) - b \vert  < \varepsilon   $;
$\boldsymbol{5.}$
$ \forall K \in \mathbb{R} \quad \exists L \in \mathbb{R} \quad \forall x \in D(f): \, x>L \, \Rightarrow \, f(x)>K $;
$\boldsymbol{6.}$
$ \forall K \in \mathbb{R} \quad \exists L \in \mathbb{R} \quad \forall x \in D(f): \, x>L \, \Rightarrow \, f(x)<K $;
$\boldsymbol{7.}$
$ \forall \varepsilon >0  $  $\exists L \in \mathbb{R} \quad \forall x \in D(f), \, x<L : \, \vert f(x) - b \vert  < \varepsilon   $;
$\boldsymbol{8.}$
$ \forall K \in \mathbb{R} \quad \exists L \in \mathbb{R} \quad \forall x \in D(f), \, x<L \, : \, f(x)>K $;
$\boldsymbol{9.}$
$ \forall K \in \mathbb{R} \quad \exists L \in \mathbb{R} \quad \forall x \in D(f), \, x<L \, : \, f(x)<K $;
$\boxed{112.}$
$\boldsymbol{2.}$
ak $ \vert x-1 \vert<\delta, \, x\ne 1 $, tak  $\frac{1}{(1-x)^{2}}> \frac{1}{ \delta ^{2}}$; pre dané $ K \in \mathbb{R} $ stačí položiť napr.  $ \delta = \sqrt{1+ \vert K \vert} $ (definíciu limity používame v takej podobe, ako je v pr. 111.2);
$\boldsymbol{4.}$
ak $x\geq 0 $ a $\vert x-8 \vert < \delta $, tak $\vert \sqrt[3]{x}-2\vert = \frac{\vert x-8 \vert}{\vert \sqrt[3]{x^{2}}+2 \sqrt[3]{x}+4 \vert}< \frac{\delta}{4}  $ (pre $x\geq 0  $ je $ \sqrt[3]{x^{2}}+2 \sqrt[3]{x}+4 \geq 4 $, pre dané $\varepsilon > 0  $ stačí položiť  $\delta = \min \lbrace 4 \varepsilon, 8 \rbrace   $ (podmienka  $\delta \leq 8 $ zaručuje, že pre všetky  $x \in O_{\delta}(8) $ platí nerovnosť  $  x > 0$, ktorú sme potrebovali pri odhadovaní výrazu $ \vert \sqrt[3]{x}-2 \vert $);
$\boldsymbol{5.}$
ak $ x<0 $ a $\vert x+2\vert< \delta $ , tak  $ \vert x^{2}-4 \vert = \vert x-2 \vert \vert x+2 \vert  <2 \delta $; pre dané $\varepsilon > 0$ stačí položiť  $\delta= \min \lbrace \frac{\varepsilon}{2},2 \rbrace $;
$\boxed{113.}$
$\boldsymbol{1.}$
$\exists \varepsilon  > 0 \quad \forall \delta > 0 \quad \exists x \in D(f): \, x \ne 0 \land \vert x \vert < \delta \land \, \vert f(x)-4 \vert > \varepsilon $;
$\boldsymbol{2.}$
$\forall a \in \mathbb {R^{*}} \quad \exists O(a) \quad \forall O_{\delta} (O) \quad \exists x \in D(f), \, x\in O_{ \delta}(O), \, x \ne 0: \, f(x)\notin O(a) $;
$\boxed{114.}$
z nerovnosti $ \vert \vert f(x) \vert - \vert b \vert \vert \leq \vert f(x) - b \vert$ vyplýva: ak pre všetky $ x \in O(a)$ platí $ \vert f(x) - b \vert < \varepsilon $ (pre dané $\varepsilon > 0  $ existenciu takého $O(a) $ zaručuje predpoklad $\lim_{x \to a}f(x)=b$), tak pre všetky $ x \in O(a) $ platí $ \vert \vert f(x) \vert - \vert b \vert \vert < \varepsilon  $; opačná implikácia neplatí (napr. $f(x)= \sgn (x-a) $; neexistenciu limity v bode $a$ môžeme dokázať rovnako ako v pr. 115.2);
$\boxed{115.}$
$\boldsymbol{1.}$
ak pre všetky $ z\in O(a )$ platí $ \vert f(z) - b \vert < \varepsilon $ a ak $ x, y \in O(a ) $, tak $ \vert f(x) - f(y) \vert \leq \vert f(x) - b \vert + \vert b - f(y) \vert < 2\varepsilon $;
$\boxed{116.}$
$\boldsymbol{1.}$
$ \frac{1}{2}$;
$\boldsymbol{2.}$
$0$;
$\boldsymbol{3.}$
$5^{-5}$ (uvedenú limitu  možno napr. zapísať ako súčin piatich limít, z ktorých každá je rovná $ \frac{1}{5}$);
$\boldsymbol{4.}$
$\frac{2}{9}$ (najprv dať na spoločného menovateľa;
$\boldsymbol{5.}$
$x+ \frac{a}{2}$ (limitovaný výraz má tvar $ \frac{1}{n} \quad [(n-1)x+ \frac{a}{n} (1+2+...+(n-1))]= \frac{n-1}{n}x+\frac{a}{n^{2}}. \frac{(n-1)n}{2}$);
$\boldsymbol{6.}$
$\frac{1}{3}$ (zlomok rozšíriť $3^{-n}$ a využiť pr. 105.5);
$\boxed{117.}$
$\boldsymbol{2.}$
$\frac{2}{3}$;	
$\boldsymbol{3.}$
$1$ (stačí dosadiť, daná funkcia je elementárna a definovaná v bode 0);
$\boldsymbol{4.}$
$1$;
$\boldsymbol{5.}$
$0$;
$\boldsymbol{6.}$
$(\frac{3}{2})^{10}$;
$\boldsymbol{7.}$
$\frac{m}{n}$;
$\boldsymbol{8.}$
$-\frac{1}{2}$ (spoločný menovateľ je $ x(x-1)(x-2))$;
$\boldsymbol{9.}$
$\frac{n(n+1)}{2}$ (delenie výrazom $ x^{-1} $ si uľahčíme, ak čitateľ napíšeme v tvare $ (x-1)+(x^{2}-1)+...(x^{n}-1))$;
$\boldsymbol{10.}$
$\frac{49}{24}(x^{100}-2x+1=(x^{100}-1)-2(x-1)$; $x^{50}-2x+1=(x^{50}-1)-2(x-1)$; tento prepis uľahčí delenie výrazom $x-1$);
$\boxed{118.}$
$\boldsymbol{1.,2.}$
napr. $f(x)=\chi (x), \, g(x)=1-\chi (x) $ (tým sme vlastne ukázali, že implikácie  vo vetách o limite súčtu a limite súčinu nemožno obrátiť);
$\boxed{119.}$
áno; napr. $ s_{n}=(-1)^{n}\frac{1}{n}$ možno dokázať, že pre každú postupnosť  $\lbrace a_{n}\rbrace_{n=1}^\infty $ vyhovujúcu pr. 119 platí $\lim_{n \to \infty}a_n =0$);
$\boxed{120.}$
$\boldsymbol{2.}$
$\frac{4}{3}$;
$\boldsymbol{3.}$
$\frac{1}{\sqrt{2a}}$ (= $\lim_{x \to a}( \frac{\sqrt{x-a}}{\sqrt{x+a}(x+a)}+\frac{1}{\sqrt{x+a}})$) ;
$\boldsymbol{4.}$
$-\frac{1}{16}$;
$\boldsymbol{5.}$
$\frac{2}{27}$ na úpravu čitateľa sa použije vzťah $A^{3}-B^{3}=(A-B)(A^{2}+AB+B^{2}) $, v menovateli stačí vyňať pred zátvorku premennú $ x $);
$\boldsymbol{6.}$
$\frac{1}{n!}$;
$\boxed{121.}$
$\boldsymbol{2.}$
$\frac{112}{27}$;
$\boldsymbol{3.}$
$\frac{7}{36}$;
$\boxed{122.}$
$\boldsymbol{1.}$
$\frac{5}{3}$ (subst. $t=\sqrt[15]{x} $);
$\boldsymbol{3.}$
$\frac{n}{m}$ (subst. $t=\sqrt[m.n]{x} $);
$\boldsymbol{4.}$
$\frac{1}{2}$;
$\boxed{122.}$
$\boldsymbol{1.}$
1 (zlomok rozšíriť výrazom $\frac{1}{\sqrt{x}}$; uvedomte si, že pri výpočte limity v čitateli aj v menovateli takto získaného zlomku sa používa veta o limite zloženej funkcie);
$\boldsymbol{2.}$
$\frac{1}{\sqrt{2}}$ (zlomok rozšíriť výrazom $\frac{1}{\sqrt{x}}$;
$\boldsymbol{3.}$
-2 (pre $x>0$ je $\vert x \vert=x $);
$\boldsymbol{4.}$
$\frac{a+b}{2}$ (rozšíriť výrazom $\sqrt{(x+a)(x+b)}+x $);
$\boldsymbol{5.}$
$\frac{2}{3}$;
$\boldsymbol{6.}$
$\frac{a_{1}+...+a_{n}}{n}$;
$\boldsymbol{7.}$
$-\frac{1}{4}$ (rozšíriť výrazom $ x+\sqrt{x^{2}+2x}+2\sqrt{x^{2}+x}$, potom napísať v tvare súčinu tak, aby jeden zo súčiniteľov bol $ \frac{2x}{x+\sqrt{x^{2}+2x}+2\sqrt{x^{2}+x}}$, jeho limitu už vieme vypočítať; z druhého súčiniteľa vyňať pred zátvorku $ x $ potom rozšíriť výrazom $\sqrt{x^{2}+2x}+x+1 $);
$\boldsymbol{8.}$
$\frac{1}{3}$;
$\boxed{125.}$
áno, podkladom pre konštrukciu funkcií $f,g $ môže byť rozbor dôkazov vety o limite zloženej funkcie a tvrdenia z pr. 124; $g $ nesmie spĺňať podmienku (*) z uvedenej vety, pre $f$ nesmie platiť $\lim_{x \to 2}f(x) = f(2)$);
$\boxed{126.}$
$\boldsymbol{1.}$ 5;
$\boldsymbol{2.}$
$\sin 1$ (stačí dosadiť);
$\boldsymbol{3.}$
$\frac{m}{n}$ (rozšíriť výrazom $\frac{1}{x}$);
$\boldsymbol{4.}$ 2 (rozšírte výrazom $x^{3}+2x$ a využite, že  $\lim_{x \to 0}\frac{\sin (x^{3}+2x)}{x^{3}+2x} = 1$ podľa vety o limite zloženej funkcie);
$\boldsymbol{5.}$
$\frac{1}{3}$;
$\boldsymbol{6.}$ $x$ (subst. $t= \frac{x}{2^{n}} $);
$\boxed{127.}$
$\boldsymbol{1.}$
$\frac{1}{2} $ (rozšíriť výrazom $1+ \cos x $ alebo použiť vzorec $1-\cos x =2 \sin ^{2}\frac{x}{2}$);
$\boldsymbol{2.}$
$\frac{1}{2} $;
$\boldsymbol{3.}$ 2 (stačí rozšíriť  výrazom $\frac{1}{x} $);
$\boldsymbol{4.}$
$\frac{1}{2} $;
$\boldsymbol{5.}$ 4 (v čitateli stačí odpočítať a pripočítať číslo 1 a použiť výsledok pr. 127.1);
$\boldsymbol{6.}$
$\frac{1}{p} $ (rozšíriť výrazom $\frac{1}{x} $; využiť, že $\lim_{x \to 0}\frac{1-\cos x}{x} = \lim_{x \to 0}\frac{1-\cos x}{x^{2}}x=0$, pozri pr. 127.1);
$\boxed{128.}$
$\boldsymbol{1.}$ $\cos a $ (použite vzorec $\sin x- \sin a = 2 \sin \frac{x-a}{2} \cos \frac{x+a}{2} $);
$\boldsymbol{2.}$
$-\frac{1}{\sin ^{2}a} $;
$\boldsymbol{3.}$
$- \cos a$;
$\boldsymbol{4.}$
$\frac{3}{2} \sin 2a $ (v čitateli napr. odpočítať a pripočítať výraz $\sin ( a+x) \sin a$);
$\boxed{129.}$
$\boldsymbol{1.}$ 
$\frac{1}{2} $;
$\boldsymbol{2.}$ -3 (subst. $ t= \sin x$);
$\boldsymbol{3.}$
$\frac{3}{4} $;
$\boldsymbol{5.}$
$\frac{2}{\pi} $;
$\boldsymbol{6.}$
1 (subst. $t = \arcsin x $;
$\boxed{130.}$
$\boldsymbol{1.}$ 
$\frac{1}{4} $;
$\boldsymbol{2.}$ 
$\frac{4}{3} $;
$\boldsymbol{3.}$ 
$-\frac{1}{12} $;
$\boldsymbol{4.}$ 
$\sqrt{2} $ (limitovaná funkcia je definovaná len pre $ x>0$, na jej definičnom obore teda platí $x= \sqrt{x^{2}} $);
$\boldsymbol{5.}$ 
$-\frac{1}{20} $;
$\boxed{131.}$
$\boldsymbol{1.}$ 
0 ($\lim_{x \to \infty}  \frac{1}{x}=0$, $\sin  $ je ohroničená funkcia);
$\boldsymbol{2.}$ 
$\frac{1}{2} $ (rozšíriť výrazom $\frac{1}{x^{2}} $);
$\boldsymbol{3.}$ 0;
$\boldsymbol{4.}$ 3;
$\boldsymbol{5.}$
$\frac{1}{2}$;
$\boxed{132.}$
$\boldsymbol{2.}$ 
$\frac{n}{2^{n}}= \frac{n}{(3/2)^{n}}.(3/4)^{n}\leq (3/4)^{n}$;
$\boldsymbol{3.}$
$( \frac{n}{\sqrt[3]{25}^{n}})^{3/2}\leq (\frac{n}{2^{n}})^{3/2}$ (rovnosti z pr. 132 možno dokázať aj na základe pr. 133.1;
$\boxed{133.}$
$\boldsymbol{1.}$ 
 $ 0 \leq a_{n} \leq \frac{a_{1}}{q}.q^{n}  $;
$\boldsymbol{2.}$ neplatí, stačí zvoliť  $ a_{n}= \frac{n+1}{2n}$;
$\boldsymbol{3a/.}$ $0$ (použiť pr. $133.1$ a nerovnosť $(1+ \frac{1}{n})^{n}\geq 2) $;$\boldsymbol{3b/.}$ $0$ (treba si uvedomiť, že tvrdenie z príkladu $133.1$ zostane v platnosti, aj keď $\frac{a_{n+1}}{a_{n}}\leq q$ bude platiť len pre všetky $n\geq K$, kde $K\in \mathbb{N}$ je dané);
$\boxed{134.}$
$\boldsymbol{1.}$ $0$ $((\frac{7}{n})^{n}\leq (\frac{7}{8})^{n}$ pre $n\geq 8$);$\boldsymbol{2.}$ $0$;
$\boxed{135.}$
$\boldsymbol{2.}$
pre $\omega(n):=\sqrt[n]{n}-1$ platí $0\leq \omega(n) \leq \sqrt{\frac{2}{n}}$;
$\boxed{136.}$
$\boldsymbol{1.}$  $0$ (rozšíriť výrazom $\frac{1}{n!}$);
$\boldsymbol{2.}$  $1$ (pre $n\geq 19 $ je $\sqrt[n]{4}\leq \sqrt[n]{\frac{5n+1}{n+5}} \leq \sqrt[n]{6} $; alebo všeobecnejšie: $\lim_{n\rightarrow\infty} \frac{5n+1}{n+5} =5$, preto pre dané $\varepsilon \in (0,5) $ existuje $O (\infty )$, v ktorom platí $5- \varepsilon \leq \frac{5n+1}{n+5} \leq 5+ \varepsilon $, a teda aj $\sqrt[n]{5- \varepsilon} \leq  \sqrt[n]{\frac{5n+1}{n+5}}\leq \sqrt[n]{5+ \varepsilon} $)
$\boldsymbol{3.}$  
$3$ $ (3^{n}-2^{n} = (3-2).(3^{n-1}+3^{n-2}.2+...+2^{n-1}) 3^{n-1}$; iná možnosť: $ \sqrt[n]{3^{n}-2^{n}}= 3 \sqrt[n]{1-(\frac{2}{3})^{n}}$, pritom $\lim_{n\rightarrow\infty} (1- ( \frac{2}{3})^{n})=1$, preto pre dané $\varepsilon \in (0,1) $  existuje $O( \infty )$, v ktorom platí $1- \varepsilon \leq 1- (\frac{2}{3})^{n} \leq 1+ \varepsilon $, a teda aj $ \sqrt[n]{1- \varepsilon } \leq \sqrt[n ]{1- (\frac{2}{3})^{n}} \leq \sqrt[n]{1+ \varepsilon }$; ďalšou možnosťou výpočtu limít z pr. 136.2-4. je použitie postupov uvedených v odseku 2.5);
$\boldsymbol{4.}$  $1 $  ($\frac{1}{2^{n}}\leq \frac{1}{2n} $, iná možnosť:  $\sqrt[n]{\frac{1}{n}-\frac{1}{2^{n}}}= \sqrt[n]{\frac{1}{n}}\sqrt[n]{1-\frac{n}{2^{n}}} $, ďalej využite analogicky ako v pr. 136.2,3 fakt, že $\lim_{n\rightarrow\infty} (1-\frac{n}{2^{n}})=1$);
$\boxed{137.}$
$\boldsymbol{1.}$  $-\infty$  ($\lim_{x\rightarrow-\infty} \sqrt{(x+a)(x+b)}=+\infty$, $\lim_{x\rightarrow-\infty} -x = + \infty$, preto $\lim_{x\rightarrow-\infty}( \sqrt{(x+a)(x+b)}-x)=+\infty$;
$\boldsymbol{2.}$  $-\infty$;
$\boldsymbol{3.}$  $+\infty$;
$\boldsymbol{4.}$  $+\infty$ (rozšíriť výrazom $\frac{1}{x^{2}} $; $x^{2}-\frac{5}{x}\rightarrow \, +\infty $, $1/(1-\frac{3}{x}+\frac{1}{x^{2}}) \rightarrow \, 1 $ pre  $x\rightarrow -\infty $);
$\boldsymbol{5.}$  $+\infty$;
$\boldsymbol{6.}$  $+\infty$;
$\boldsymbol{7.}$  $+\infty$ ($2+ \sin x \geq 1 $ pre $x \in \mathbb{R} $, preto $(2+ \sin x).x\geq x $ pre $x\geq0 $);
$\boxed{138.}$
$\boldsymbol{1.}$  $+\infty$;
$\boldsymbol{2.}$  $+\infty$ ; ($\frac{\sin x}{x}\rightarrow 1$,  $\frac{1}{x^{2}}\rightarrow +\infty$ pre $x\rightarrow 0 $);
$\boldsymbol{3.}$  $+\infty$ (rozšíriť výrazom $\frac{1}{x^{2}} $  alebo $\frac{1}{x^{4}} $);
$\boldsymbol{4.}$  $+\infty$;
$\boldsymbol{5.}$  $+\infty$;
$\boldsymbol{6.}$  $+\infty$;
$\boxed{139.}$  $\boldsymbol{3.}$ napr.  $f(x)= \frac{1}{(x-1)^{2}}+ \chi (x) $, $g(x)= -\frac{1}{(x-1)^{2}} $;
$\boxed{140.}$
$\boldsymbol{4.}$ napr. $a_{n}=n, \, b_{n}=(-1)^{n}. \frac{1}{n} $ ( príklady 139 a 140 ukazujú, že možno sformulovať všeobecné tvrdenia, ktoré by umožňovali výpočet limít neurčitých výrazov $ +\infty$ $ -\infty $, resp. $0.(+\infty) $);
$\boxed{141.}$ 
$\lim_{x\rightarrow\infty} R(x) =$ $ f(n)=\left\{\begin{matrix} +\infty, & \mbox{ak }n>m, \frac{a_{0}}{b_{0}}>0 \\ -\infty, & \mbox{ak }n>m, \frac{a_{0}}{b_{0}}<0 \\ \frac{a_{0}}{b_{0}} & \mbox {ak }n=m, \\ 0 & \mbox {ak }n<m\end{matrix}\right.$
najprv dokážte - napr. matematickou indukciou - že $\lim_{x\rightarrow-\infty}(c_{0}x^{s}+c_{1}x^{s-1}+...+c_{s})=$ 
$\left\{\begin{matrix} +\infty, & \mbox{ak } c_{0}>0 \\ - \infty, & \mbox{ak} c_{0}<0\end{matrix}\right. $);
$\boxed{142.}$
$\boldsymbol{1.}$
$\forall \varepsilon >0 \quad \exists \delta >0 \quad \forall x \in D(f): \, 0< x-a < \delta \, \Rightarrow \vert f(x) - b \vert < \varepsilon$;
$\boldsymbol{2.}$
$\forall \varepsilon >0 \quad \exists \delta >0 \quad \forall x \in D(f): \, -\delta< x-a < 0 \, \Rightarrow \vert f(x) - b \vert < \varepsilon$ (pritom predpokladáme, že $a$ je hromadný bod množiny $ D(f) \cap (a, \infty )$ resp. množiny $D(f) \cap (-\infty, a)$);
$\boxed{143.}$
$\boldsymbol{1.}$
$\lim_{x\rightarrow 1^{+}} f(x) =2$, $\lim_{x\rightarrow 1^{-}} f(x) =-2$;
$\boldsymbol{2.}$
$\lim_{x\rightarrow 0^{+}} f(x) =\sqrt{2}$, $\lim_{x\rightarrow 0^{-}} f(x) =-\sqrt{2}$ ($1-\cos 2x = 2\sin ^{2}x $);
$\boldsymbol{3.}$
$\lim_{x\rightarrow 2^{+}} f(x) =\infty$, $\lim_{x\rightarrow 2^{-}} f(x) =-\infty$;
$\boldsymbol{4.}$
$\lim_{x\rightarrow 0^{+}} f(x) =0$, $\lim_{x\rightarrow 0^{-}} f(x) =\frac{1}{2}$;
$\boxed{144.}$
$\boldsymbol{1.}$
$\lim_{x\rightarrow \pi/2^{+}} tg x =-\infty \ne +\infty $ $=\lim_{x\rightarrow \pi/2^{-}} tg x$, preto $\lim_{x\rightarrow \pi/2} tg x$ neexistuje;
$\boldsymbol{2.}$
$\lim_{x\rightarrow 0}x \sgn x = 0$ (možno dokazovať pomocou jednostranných limít alebo využiť fakt, že $sgn$ je ohraničená funkcia a $\lim_{x\rightarrow 0}x  = 0$;
$\boldsymbol{3.}$ neexistuje ( $ \frac{\sin x}{x^{2}} = \frac{\sin x}{x} . \frac{1}{x}$);
$\boxed{146.}$
$\boldsymbol{2.}$
za daných predpokladov $\lim_{x\rightarrow 0} f(x)$ existuje práve vtedy, keď $\lim_{x\rightarrow 0^{+}} f(x)=0$;
$\boxed{149.}$
Nech sú dané funkcie $f,g $, nech $b \in \mathbb{R^{*}} $ je hromadný bod množiny $D(f) \cap D(g) $, nech $\lim_{x\rightarrow b} f(x)=a$, $\lim_{x\rightarrow b} g(x)=A$. Potom: $\boldsymbol{1.}$ ak $a \in (0,1),\, A= +\infty  $, tak 
$\lim_{x\rightarrow b} f(x)^{g(x)}=0$;
$\boldsymbol{2.}$ ak $a \in (0,1),\, A= -\infty  $, tak $\lim_{x\rightarrow b} f(x)^{g(x)}=+\infty$;
$\boldsymbol{3.}$ ak $a \in (1,+\infty)\cup \lbrace +\infty \rbrace,\, A= +\infty  $, tak $\lim_{x\rightarrow b} f(x)^{g(x)}=+\infty$; 
$\boldsymbol{4.}$ ak $a \in (1,+\infty)\cup \lbrace +\infty \rbrace,\, A= -\infty  $, tak $\lim_{x\rightarrow b} f(x)^{g(x)}=0$ (pri dôkazoch sa stačí obmedziť na také prstencové okolie bodu $b$, v ktorom je $f$ kladná (také okolie existuje, pretože $\lim_{x\rightarrow b} f(x)$ je kladné číslo alebo bod $+ \infty$)); 
$\boxed{150.}$
$\boldsymbol{1.}$  $\sqrt{\frac{2}{3}} $;
$\boldsymbol{2.}$  $\frac{1}{4} $;
$\boldsymbol{3.}$  $+\infty $;
$\boldsymbol{4.}$  $\frac{1}{2} $;
$\boldsymbol{5.}$  $0 $;
$\boldsymbol{6.}$  $0 $;
$\boldsymbol{7.}$  $+ \infty $;
$\boldsymbol{8.}$ neexistuje (limita sprava (zľava) je $+\infty, (0) $);
$\boxed{151.}$
$\boldsymbol{1.}$  $e^{3} $;
$\boldsymbol{2.}$  $e^{-2} $;
$\boldsymbol{3.}$  $e^{3} $;
$\boldsymbol{4.}$  $e^{-1} $;
$\boldsymbol{5.}$  $e^{ctg a} $;
$\boldsymbol{6.}$  $1 $;
$\boldsymbol{7.}$  $e $ (možno použiť substitúciu $\frac{1}{x}=t $, potom $t\rightarrow 0^{+} $ pre $x \rightarrow +\infty $);
$\boldsymbol{8.}$  $e^{-x^{2}/2} $;
$\boxed{152.}$
$\boldsymbol{1.}$  $1 $; (= $\lim_{x\rightarrow 0} ln[(1+x)^{1/x}  ] $; použije sa veta z pr. 124);
$\boldsymbol{2.}$  $1 $;
$\boldsymbol{3.}$  $\frac{1}{a} $; 
$\boldsymbol{5.}$  $0 $;
$\boldsymbol{6.}$  $n $ (využite, že $\lim_{x\rightarrow 0} \frac {ln [1+(x+ \sqrt{1-x^{2}}-1)]}{x+\sqrt{1-x^{2}}-1} =1$, analogický vzťah použite pre $ln (nx+ \sqrt{1-n^{2}x^{2}}) $);   
$\boxed{153.}$
$\boldsymbol{1.}$  $1 $ (možno použiť substitúciu  $e^{x}-1=t  $);
$\boldsymbol{2.}$  $1 $;
$\boldsymbol{3.}$  $4. ln \frac{2}{e} $ (v čitateli možno pripočítať a odpočítať číslo $4$, potom previesť na súčet dvoch limít);
$\boldsymbol{5.}$  $\frac{25}{4} $;
$\boldsymbol{6.}$  $1 $ (v čitateli možno pripočítať a odpočítať číslo $1$, potom zlomok rozšíriť výrazom $\frac{1}{x}$);
$\boldsymbol{7.}$  $ln x $;
$\boldsymbol{8.}$  $e^{2} $;
$\boxed{154.}$
$\boldsymbol{1.}$  $\frac{1}{3} $;
$\boldsymbol{3.}$  $\frac{1}{2} $;
$\boxed{153.}$
$\boldsymbol{1.}$  $a_{n+1}=(1-\frac{1}{2^{n+1}}) a_{n} < a_{n} \, (n \in \mathbb{N}) $, teda $ \lbrace a_{n} \rbrace_{n=1}^\infty $ je klesaajúca postupnosť; $ \forall n \in \mathbb{N}: \, a_{n} 	>0 $, teda $ \lbrace a_{n} \rbrace_{n=1}^\infty $  je zdola ohraničená; preto existuje konečná  $\lim_{n\rightarrow \infty} a_{n}  $(= $\inf \lbrace a_{n} ; n \in \mathbb{N}\rbrace $);
$\boldsymbol{2.}$ $ \lbrace a_{n} \rbrace_{n=1}^\infty $ je rastúca; z nerovnosti  $n!\geq 2^{n-1}  $ ($n \in \mathbb{N}  $) vyplýva  $a_{n}\leq 1+1+\frac{1}{2}+...+\frac{1}{2^{n-1}} = 2+1-\frac{1}{2^{n-1}} < 3$;
$\boldsymbol{3.}$ $\lbrace a_{n}\rbrace_{n=1}^{\infty}  $  je rastúca $a_{n+1}-a_{n}>0  $) a zhora ohraničená ( $a_{n} < 1,\, n \in \mathbb{N} $);
$\boldsymbol{4.}$ $\lbrace a_{n}\rbrace_{n=1}^{\infty}  $  je zhora ohraničená ( $a_{n} < \frac{1}{5}+...+ \frac{1}{5^{n}} = \frac{1}{4}(1- \frac{1}{5^{n}} < \frac{1}{4}$) a rastúca;
$\boldsymbol{5.}$ pre $n>10 $ je $a_{n}< a_{n-1} $, $\lbrace a_{n}\rbrace_{n=10}^{\infty} $ je klesajúca a zdola ohraničená  ($ \forall n \in \mathbb{N}: \, a_{n} 	>0 $);
$\boxed{156.}$
$\boldsymbol{2.}$ ak si najprv pomôžeme grafmi funkcií $y=\frac{1}{2}(x^{2}+1) $  a   $y=x $ nakreslenými do jedného obrázka (podobne ako obr. 2), zistíme, že postupnosť $\lbrace a_{n}\rbrace_{n=10}^{\infty} $ je rastúca a zhora ohraničená (napr. číslom  $ 1 $); pretože však i najlepšie nakreslený obrázok nieje dôkazom, treba obidve uvedené tvrdenia ešte dokázať; hľadaná limita vyhovuje rovnici  $ 2a=a^{2}+1 $, preto $\lim_{n\rightarrow \infty} a_{n} =0$ (druhý z koreňov rovnice $x= \frac{x}{2+x} $, t.j. číslo $-1$ nemôže byť hľadanou limitou, pretože platí: $\lim_{k\rightarrow \infty} a_{2k} = \lim_{k\rightarrow \infty} a_{2k-1}= \frac{1}{2}(1+\sqrt{5})$, preto $\lim_{n\rightarrow \infty} a_{n} = \frac{1}{2}(1+\sqrt{5})$;
$\boldsymbol{5.}$ pre $x=k \pi $  ($k \in \mathbb{Z}$) je postupnosť  $\lbrace a_{n}\rbrace_{n=1}^{\infty}$ konštantná, pre  $ x \in \bigcup\limits_{k \in \mathbb{Z}}2k \pi, (2k+1)\pi) $ klesajúca a zdola ohraničená, pre  $ x \in \bigcup\limits_{k \in \mathbb{Z}}(2k-1) \pi, 2k\pi) $ rastúca  a zhora ohraničená (pri dôkazoch monotónnosti využite platnosť nerovností  $0 < \sin x < x  $ pre  $x \in (0,\pi)  $, $x < \sin x < 0  $ pre  $x \in (-\pi,0)  $), $\lim_{n\rightarrow \infty}a_{n}=0$;
$\boxed{157.}$  $\frac{a_{n+1}}{a_{n}}= \frac{n+1}{n}. \frac{1}{(1+\frac{1}{n(n+2)})^{n+2}}<1$ (na odhad menovateľa v druhom súčiniteli použite nerovnosť z pr. 10.1), teda $\lbrace a_{n}\rbrace_{n=1}^{\infty }$ je klesajúca postupnosť (fakt, že $\lbrace (1+\frac{1}{n})^{n}\rbrace_{n=1}^{\infty }$ je rastúca postupnosť - ktorý pokladáme za známy z prednášok - možno dokázať podobne); $e=\lim_{n\rightarrow \infty} a_{n} = \inf \lbrace a_{n}; n\in \mathbb{N}\rbrace$;
$\boxed{159.}$ $\boldsymbol{1.}$ nie je (platí len implikácia "$\Rightarrow $ "; protipríkladom, dokumentujúcim neplatnosť implikácie "$\Leftarrow $ ", je napr. Dirichletova funkcia $\chi (x) $);
$\boldsymbol{2.}$ je;
$\boxed{160.}$ (H označuje množinu všetkých hromadných hodnôt postupnosti $\lbrace a_{n}\rbrace_{n=1}$; $a$ resp. $b$ označujú $\lim_{n\rightarrow \infty} \inf a_{n}$, resp. $\lim_{n\rightarrow \infty} \sup a_{n}$; v pr. 160.1-6 použijeme túto únahu: ak $\lbrace a_{k_{1}}\rbrace_{n=1}^{\infty}$,..., $\lbrace a_{k_{1}}\rbrace_{n=1}^{\infty}$ sú podpostupnosti $\lbrace a_{n}\rbrace_{n=1}^{\infty}$ také, že: a/ každá z nich má limitu; b/ množiny  $N_{1}:=\lbrace k_{1} (n); \, n \in \mathbb{N} \rbrace ,...,N_{1}:=\lbrace k_{1} (n); \, n \in \mathbb{N} \rbrace  $ sú po dvoch disjunktné a $N_{1}\cup ... \cup N_{1} = N $ (t.j. ak sa podarí postupnosť $\lbrace a_{n}\rbrace_{n=1}^{\infty}$ "rozdeliť" na konečný počet podpostupností, z ktorých každá má limitu), tak H pozostáva z limít postupností $\lbrace a_{k_{1}}\rbrace_{n=1}^{\infty}$,..., $\lbrace a_{k_{1}}\rbrace_{n=1}^{\infty}$;
$\boldsymbol{1.}$  $\lim_{k\rightarrow \infty} a_{2k}=-2$, $\lim_{k\rightarrow \infty}a_{2k-1}=2$, $ H= \lbrace -2,2\rbrace $, $a=-2, \, b=2 $;
$\boldsymbol{2.}$   $ H= \lbrace -4,0,2,6\rbrace $, $a=-4, \, b=6 $ ( $\lim_{k\rightarrow \infty} a_{4k} = 2$, $\lim_{k\rightarrow \infty} a_{4k+1} = 6$, $\lim_{k\rightarrow \infty} a_{4k+2} = -4$, $\lim_{k\rightarrow \infty} a_{4k+3} = 0$,  
$\boldsymbol{3.}$   $ H= \lbrace -1,-\frac{1}{2},0,\frac{1}{2},1\rbrace $, $a=-1, \, b=1 $;
$\boldsymbol{4.}$   $ H= \lbrace -\frac{1}{\sqrt{2}}-e,\frac{1}{\sqrt{2}}-e,e-1,e,e+1\rbrace $, $a=-\frac{1}{\sqrt{2}}-e, \, b=e+1 $;
$\boldsymbol{5.}$   $ H= \lbrace 0,\infty \rbrace $, $a=0, \, b=+\infty $;
$\boldsymbol{6.}$   $ H= \lbrace 1,2\rbrace $, $a=1, \, b=2 $;
 $\boldsymbol{7.}$ , $\boldsymbol{8.}$ $a=0,\, b=1$; o čísle $0$, resp. $1$ možno totiž dokázať, že vyhovuje obidvom podmienkam z vety 18 (v pr. 160.7,8 možno dokázať, že $ H= \langle0,1\rangle $);
$\boxed{161.}$ $\boldsymbol{4.}$  napr. $\lbrace 1,2,1,2,3,1,2,3,4,1,2,3,4,5,... \rbrace $;
$\boxed{163.}$ (skôr než začnete dokazovať, by ste si mali uvedomiť, že platí  $a,b \in \mathbb{R}, \, a<b$) sporom: keby $N_{\varepsilon}^{*} $ bola nekonečná pre niektoré $\varepsilon^{*} \in (0, \frac{b-a}{2} $), existovala by hromadná hodnota $c$ postupnosti $\lbrace a _{n} \rbrace_{n=1^{\infty}}  $, $c \in \langle a+\varepsilon^{*},b-\varepsilon^{*} \rangle $ (teda $c\ne a, \, c\ne b $);
$\boxed{164.}$ stačí dokázať: ak $a,b \in \mathbb{R}, \, a\leq b $, pričom platí $\forall \varepsilon >0: \, b< a+\varepsilon $, tak   $a=b $ (dôkaz možno vykonať sporom), tvrdenie pr. 164 je potom dôsledkom tohoto a vety 19	
$\boxed{165.}$ $\boldsymbol{1.}$ v $O(a) $ ( $O(a)$ je ľubovolné okolie bodu $a$) leží aspoň jeden prvok $z \in A $; existuje také okolie $O(z) $ prvku $z$, že $O(z) \subset  O(a) $ v $O(z) $ leží (pretože $z \in A $) nekonečne veľa členov postupnosti $\lbrace a_{n}\rbrace_{n=1}^{\infty} $, teda pre každé  $O(a) $ je množina $\lbrace n \in \mathbb{N}; \, a_{n}\in O(a)  \rbrace $ nekonečná;
$\boldsymbol{2.}$ nie; $A$ neobsahuje prvok $0$, čo je v spore s pr. 165.1;
$\boxed{166.}$ $\boldsymbol{1a/}$ postupnosť $\lbrace a _{n(k)}+b_{n(k)} \rbrace_{k=1}^{\infty}  $ vybraná z postupnosti  $\lbrace a _{n} + b_{n} \rbrace_{n=1}^{\infty}$ konverguje práve vtedy, keď konverguje postupnosť $\lbrace b _{n(k)} \rbrace_{k=1}^{\infty}$; preto množinu  $ H $ hromadných hodnôt postupnosti $\lbrace a _{n} + b_{n} \rbrace_{n=1}^{\infty}$ možno písať v tvare $H = a+h;\, h \in H_{1}$, kde  $a:=\lim_{n\rightarrow \infty} a_{n}$, $H_{1}$ je množina hromadných honôt postupnosti  $\lbrace  b_{n} \rbrace_{n=1}^{\infty}$; 
$\boldsymbol{1b/}$ dôkaz prvej nerovnosti možno založiť na tejto úvahe: ak pre postupnosť  $\lbrace  c_{n} \rbrace_{n=1}^{\infty}$ a číslo $c\in \mathbb{R}$ platí: pre každé $\varepsilon > 0 $ je množina  $\lbrace n\in \mathbb{N}; \, c_{n}\leq c- \varepsilon \rbrace$ konečná, tak  $ c \leq \lim_{n\rightarrow \infty} c_{n}$ (porovnaj s vetou 18); stačí teda ukázať, že číslo $ \lim_{n\rightarrow \infty} a_{n + \lim_{n\rightarrow \infty} b_{n}}$ vyhovuje tejto podmienke; posledná nerovnosť sa dokazuje analogicky;
$\boldsymbol{2.}$ napr. $\lbrace a_{n} \rbrace_{n=1}^{\infty} = \lbrace 0,1,2,3,0,1,2,3,...\rbrace $, $\lbrace a_{n} \rbrace_{n=1}^{\infty} = \lbrace 1,2,0,1,1,2,0,1,1,2,0,1...\rbrace $;
$\boxed{168.}$ nech je dané $O(a)$; ak $b \in (O(a)-\lbrace a \rbrace )\cap A^{)} $, tak existuje také $O(b)$, že $O(b) \subset O(a)-\lbrace a \rbrace$; v $O(b)$ leží aspoň jeden prvok z $A$;
$\boxed{169.}$ neexistuje, vyplýva to z pr. 168;
$\boxed{170.}$ doporučujeme dokazovať nepriamo, pokusy o priamy dôkaz radšej konzultujte s privolaným odborníkom;
$\boxed{171.}$ neplatnosť obrátenej implikácie dokumentuje postupnosť $a_{n}=(-1)^{n}$;
$\boxed{172.}$ (využite pr. 171); $\boldsymbol{1.}$ $0$;
$\boldsymbol{2.}$ $0$;
$\boxed{173.}$ $\boldsymbol{1.}$  ak $N= \max \lbrace m \in \mathbb{N}; \, a_{m} \leq K\rbrace$; ( $K > 1$ je dané), tak pre všetky $n \in \mathbb{N}, \, n>N $ platí $a_{n} \leq K $;
$\boldsymbol{2.}$ vyplýva to z pr. 173.1 a vety o limite zloženej funkcie;
$\boxed{174.}$ $\boldsymbol{1.}$ nie; $\boldsymbol{2.}$ áno;
$\boxed{175.}$ $\boldsymbol{1.}$ ak $\lim_{n \to \infty} \vert a_n \vert =0 $  tak aj $\lim_{n \to \infty}  a_n =0 $;
$\boldsymbol{2.}$ nepriamo, keby napr. množina $N^{+}$ bola konečná, vyplývala by z existencie $\lim_{n \to \infty} \vert a_n \vert =b $ existencia $\lim_{n \to \infty}  a_n =-b $;
$\boldsymbol{3.}$ nepriamo, keby $\mathbb{N}-(N^{+}\cup N^{-}) $ bola nekonečná, platilo by $\lim_{n \to \infty} \vert a_n \vert =0 $;
$\boxed{177.}$ napr. $f(x)=x $ pre $x \in \mathbb{Q} $,  $f(x)=-x $ pre $x \notin \mathbb{Q} $;
$\boxed{178.}$ sporom, ak $\sin n $ konverguje, tak $\lim_{n \to \infty} (\sin (n+2)-\sin n) =0 $, odtiaľ $\lim_{n \to \infty} \cos (n+1) =0 $; ak zo súčtového vzorca pre $\cos (n+1)$ vyjadríme $\sin n$, dostaneme $\lim_{n \to \infty} \sin n = \lim_{n \to \infty} \cos n =0 $, čo je v spore zo vzťahom $\sin^{2} n + \cos ^{2} n =1$;
neexistencia nevlastnej limity vyplýva z ohraničenosti postupnosti $\lbrace \sin n \rbrace _{n=1} ^{\infty}$;
$\boxed{179.}$ $\boldsymbol{1.}$ $\frac{1}{2} m.n.(n-m)$ (skontrolujte si, či Vami použitý postup je správny, keď $m=1$ alebo $n=1$);
$\boldsymbol{2.}$ $\frac{n.(n-1)}{2} a^{n-2}$ (možno použiť subst. $x-a=t$ ; je použitý postup správny pre $n=1$?);
$\boldsymbol{3.}$ $1 $;
$\boldsymbol{4.}$ $\frac{m-n}{2} $ (skontrolujte si správnosť použitého postupu, ak  $m=1$ alebo $n=1$);
$\boldsymbol{5.}$ $x^{2}+ax+a^{2}/3 $;
$\boldsymbol{6.}$ $n^{-n(n+1)/2}$ (skontrolujte si správnosť použitého postupu pre párne $n$, vtedy totiž exponent v menovateli nie je celočíselný);
$\boldsymbol{7.}$ $+\infty$;
$\boldsymbol{8.}$ $ \frac{2}{3}$( $\lim_{n \to \infty} \frac{2\frac{n(n+1)}{2}-n}{3\frac{n(n+1)}{2}-2n} $); 
$\boldsymbol{9.}$ neexistuje ($a_{2k}=-\frac{1}{2}$, $a_{2k-1}=1-\frac{k-1}{2k-1}$, $k\in \mathbb{N}$, kde $a_{n}=\frac{1-2+3-4+...+(-1)^{n-1}.n}{n}$);
$\boldsymbol{10.}$ $ \frac{3}{2}$;
$\boxed{180.}$ ak $\lim_{n \to \infty} a_{n}=a \in \mathbb{R} $, tak $\lbrace a_{n} \rbrace_{n=1} ^{\infty}$ je ohraničená, existuje teda $ B:= \sup_{n\in \mathbb{N}}a_{n} $, $A:= \inf_{n\in \mathbb{N}}a_{n} $, pritom $A \leq a\leq B $; predpokladajme $ A < a $, potom $ \exists N \in \mathbb{N} $  $ \forall n>N: \, a_{n}>A $ (stačí zvoliť $\varepsilon= \frac{a-A}{2} $ a použiť definíciu limity), potom najmenšie z čísel $a_{1},...,a_{N} $ musí byť rovné $A $; podrobné vypracovanie ako aj ostatné prípady prenechávame čitateľovi;
$\boxed{181.}$ R má limity v bodoch $+\infty$, $-\infty$; ak $\lim_{x \to \infty} R(x)=+\infty$,   $\lim_{x \to -\infty} R(x)=-\infty$ (ostatné prípady urobte sami), tak  $\exists a_{1} \in \mathbb{R} \, \forall x < a_{1}$, $x\in D(R):\, R(x)>1$, pritom $a_{1}< a_{2} $; ak má platiť  $ R(\mathbb{Z})=\mathbb{Q} $, musí byť $R( \mathbb{Z} \cap \langle a_{1}, a_{2}\rangle$) = $ \mathbb{Q} \cap \langle-1,1 \rangle$, čo je nemožné (množina vľavo je konečná, množina vpravo nekonečná);
$\boxed{182.}$ $\boldsymbol{1.}$ $ \frac{\alpha}{m} -\frac{\beta}{n}$ (stačí nájsť "vzorec" pre $ \lim_{x \to 0}\frac{\sqrt[k]{1+\vartheta x}-1}{x}$, $k\in \mathbb{Z}- \lbrace 0 \rbrace $);
$\boldsymbol{2.}$ $ \frac{\alpha}{m} -\frac{\beta}{n}$ ( v čitateli stačí pripočítať a odpočítať výraz  $\sqrt[m]{1+\alpha_{x}}$);
$\boldsymbol{3.}$ $1 $;
$\boldsymbol{4.}$ $-\frac{1}{2} $; 
$\boldsymbol{5.}$ $-\frac{\pi}{6} $ (je vhodné použiť modifikáciu vety o limite zloženej funkcie z pr. 124);
$\boldsymbol{6.}$ $5$;
$\boldsymbol{7.}$ $-\frac{1}{4} $;
$\boldsymbol{8.}$ $2 $ (stačí pripočítať a odpočítať $x$);
$\boldsymbol{9.}$ $2^{n}$ ($= \lim_{n \to \infty}((\frac{x-\sqrt{x^{2}-1}}{x})^{n} + (\frac{x+\sqrt{x^{2}-1}}{x})^{n}) $;
$\boxed{184.}$ $\boldsymbol{1.}$ $-\frac{9}{2^{7}} $; 
$\boldsymbol{2.}$ $\sin a $; 
$\boldsymbol{3.}$ $\sqrt{3} $ (pozri pr. 128.4); 
$\boldsymbol{4.}$ $\frac{2}{3\pi} $ (využite, že   $\lim_{u \to 0}\frac{\arctan u}{u}=1$; 
$\boldsymbol{5.}$ $+\infty $;
$\boldsymbol{6.}$ $3 $;
$\boldsymbol{7.}$ $-24 $;
$\boldsymbol{8.}$ $\sqrt{2} $ (ak použijeme subs. $\arccos (1-x)=t$, musíme si uvedomiť, že pre $x\rightarrow 0$ platí $t\rightarrow 0^{+}$, stačí teda hľadať $\lim_{t \to 0^{+}}\frac{t}{\sqrt{1-\cos t}}$;  $\lim_{t \to 0}\frac{t}{\sqrt{1-\cos t}}$ totiž neexistuje);
$\boldsymbol{9.}$  $+\infty $
$\boldsymbol{10.}$  $\frac{3}{2} $;
$\boldsymbol{11.}$  $\frac{5}{8}\sqrt{2}\pi ^{2}$;
$\boldsymbol{12.}$  $-\infty $;
$\boldsymbol{13.}$  $0 $;
$\boldsymbol{14.}$  $0 $;
$\boldsymbol{15.}$  $0 $;
$\boxed{185.}$ $\boldsymbol{2.}$  $0 $ (  $\lim_{n \to \infty}\sqrt[n]{a_{n}}= \frac{1}{\sqrt{2}} $, pre vhodne zvolené  $\varepsilon > 0 $ preto platí $\exists n_{0} \in \mathbb{N}, \, n>n_{0} $: $\sqrt[n]{a_{n}}<\frac{1}{\sqrt{2}}+\varepsilon < 1$); 
$\boxed{186.}$ $\boldsymbol{1,2.}$ možno využiť pr. 133.1;
$\boldsymbol{3.}$ možno použiť nerovnosť $n!>(\frac{n}{3})^{n} $ iná možnosť: pre $i=1,....,n $ platí $i.(n-i+1)\geq n $, odtiaľ $(n!)^{2}\geq n^{n} $;
$\boxed{187.}$ $\boldsymbol{1.}$  $1 $ (použite výsledok pr. 135);
$\boldsymbol{2.}$  $\max \lbrace a,b \rbrace $;
$\boldsymbol{3.}$  $3 $;
$\boldsymbol{4.}$  $11 $;
$\boldsymbol{5.}$  $0 $ (možno použiť pr. 133 a fakt, že $\lim_{n \to \infty}(1+ \frac{1}{n})^{n}=e $);
$\boldsymbol{6.}$  $+\infty $;
$\boldsymbol{7.}$  $+\infty $;
$\boldsymbol{8.}$  $\frac{1}{3} $ (využite pr. 186.1);
$\boldsymbol{9.}$  $1$;
$\boldsymbol{10.}$  $-\infty $;
$\boxed{188.}$ správny je postup v bode a/;
$\boxed{189.}$ Nech je daná funkcia $g$ a kladná funkcia $f$, nech $b$ je hromadný bod množiny $D(f)\cap D(g)$, nech $\lim_{x \to b} f(x)=0$. Potom: a/ ak $\lim_{x \to b} g(x)=+\infty$, tak $\lim_{x \to b} f(x)^{g(x)}=0$; b/ ak $\lim_{x \to b} g(x)=-\infty$, tak $\lim_{x \to b} f(x)^{g(x)}=+\infty$;
$\boxed{190.}$ $\boldsymbol{1.}$  $0$;
$\boldsymbol{2.}$  $e^{2}$;
$\boldsymbol{3.}$  $+\infty$;
$\boldsymbol{4.}$  $e^{-1/2}$;
$\boldsymbol{5.}$  $0$;
$\boldsymbol{6.}$  $0$;
$\boldsymbol{7.}$  $e^{(\beta^{2}-\alpha^{2}})/2$; (prípad $\alpha=\beta$ treba robiť samostatne;
$\boldsymbol{8.}$ neexistuje;
$\boldsymbol{9.}$  $0$ pre $a_{1}>a_{2}$, $e^({b_{1}-b_{2})/a_{1}}$ pre $a_{1}=a_{2}$ (prípad $a_{1}=a_{2}$, $b_{1}=b_{2}$ treba robiť samostatne);
$\boldsymbol{10.}$  $e^{-1}$;
$\boldsymbol{11.}$  $e^{-(a+b)}$ (prípady $a=0,\, b=0$ sa robia zvlášť);
$\boxed{191.}$ $\boldsymbol{1.}$  $- \ln 2$ (pre $x>0 \, x\ne 1$ je $\log_{x}2= \frac{1}{\log_{2}x}$;
$\boldsymbol{2.}$  $\ln 8$;
$\boldsymbol{3.}$  $\frac{1}{8}$;
$\boldsymbol{4.}$  $\alpha a^{\alpha}-1$ pre $\alpha\ne 0$, $0$ pre $\alpha=0$;
$\boldsymbol{5.}$  $a^{a}(1+\ln a)$;
$\boldsymbol{6.}$  $\sqrt{a.b}$ (prípad $a=b=1$ treba robiť samostatne);
$\boldsymbol{7.}$  $\ln x$ (= $\lim_{n \to \infty}\sqrt[n+1]{x}n^{2}(x^{1/n-1/(n+1)}-1)$; prípad $x=1$ treba urobiť samostatne);
$\boldsymbol{8.}$  $\frac{1}{\ln a - \ln b}$;
$\boldsymbol{9.}$  $\frac{1}{sqrt{a.b}}$ (prípad $a=b=1$ treba robiť samostatne);
$\boldsymbol{10.}$  $-2$;
$\boldsymbol{11.}$  $\frac{\alpha^{2}}{\beta ^{2}}$;
$\boldsymbol{12.}$  $-2$;
$\boxed{192.}$ $\boldsymbol{1a/}$  použite nerovnosti $\frac{[x]^{\alpha}}{a^{[x]+1}}$ $\leq \frac{x^{\alpha}}{a^{x}}\leq$ $\frac{([x]+1)^{\alpha}}{a^{[x]}}$ $x\geq0, \, a>1, \, \alpha >0$) a pr. 186.1;
$\boldsymbol{1b/}$  možno použiť substitúciu  $\log_{a}x=t$;
$\boldsymbol{2a/}$   $0$;
$\boldsymbol{2b/}$   $0$; 
$\boldsymbol{2c/}$   $1$;  
$\boxed{194.}$ (je zrejmé, že $g$ musí mať nevlastnú limitu); napr. $g(x)=1$ pre ($x<N, \, g(x)=f(x)$ pre $x \in \langle N+2k, N+2k+1 )$, $g(x)=2f(x)$ pre $x \in \langle N+2k+1, N+2k+2 )$ $k=1,2,...;N \in \mathbb{N}$ je číslo, pre ktoré platí $\forall x\geq N :\, f(x)>0$);
$\boxed{194.}$  $\boldsymbol{1.}$ $\lbrace a_{n} \rbrace _{n=1} ^{\infty}$ je klesajúca zdola ohraničená;
$\boldsymbol{2.}$ $\lbrace a_{n} \rbrace _{n=1} ^{\infty}$ je rastúca  a zhora ohraničená (použite nerovnosť  $\frac{1}{n^{2}}\leq \frac{1}{n-1}- \frac{1}{n}$  ($n>1$) alebo  $\frac{1}{(2^{k})^{2}}+\frac{1}{(2^{k}+1)^{2}}+...+\frac{1}{(2^{k+1}-1)^{2}}\leq \frac{1}{2^{k}}$;
$\boldsymbol{3.}$ $\lbrace a_{n} \rbrace _{n=1} ^{\infty}$ je rastúca  a zhora ohraničená (prísť na  odhad $a_{n}\leq (2+ \frac{1}{2}+ \frac{1}{4}+...+ \frac{1}{2^{n-2}} $) pre $n\geq 3 $ dá asi trocha námahy; iná možnosť: využiť nerovnosť $\ln (1+ \frac{1}{n})< \frac{1}{n} $ ( $ n \in \mathbb{N} $), ktorú dostaneme logaritmovaním nerovnosti z pr. 157, potom $(1+\frac{1}{2^{k}})=e^{ln(1+1/2^{k}})< e^{1/2^{k}}  $ );
$\boxed{196.}$   $\lbrace a_{n} \rbrace _{n=1} ^{\infty}$ je rastúca  a zhora ohraničená (ak postupne pre $k=n,n-1,...,1 $ použijeme nerovnosť $ k< 2^{2^{k}}$, dostaneme  $b_{n}< 2\sqrt{1+\sqrt{1+...+\sqrt{1+\sqrt{2}}}}   $  $\rightarrow n-1 $ odmocnín);
$\boxed{197.}$  $\boldsymbol{1.}$ ak zobrazíme na číselnej osi niekoľko prvých členov, zistíme, že by malo platiť $a_{2k}=1-(\frac{1}{4} +\frac{1}{4^{2}}+...+\frac{1}{4^{k-1}})$,  $ k>1  $,  $a_{2k+1}=\frac{1}{2}(1+\frac{1}{4} +\frac{1}{4^{2}}+...+\frac{1}{4^{k-1}}) $,  $ k\geq 1  $, čo možno dokázať indukciou; potom  $ \lim_{k \to \infty}a_{2k}  $ $= \lim_{k \to \infty}a_{2k+1}  = \frac{2}{3}$ $= \lim_{n \to \infty}a_n  $;
$\boldsymbol{2.}$ $\lim_{n \to \infty}a_n =a$, ak $a=b$; $ \lim_{n \to \infty}a_n =\frac{a}{3}+ \frac{2b}{3} $, ak $a \ne b $ (ak definujeme postupnosť $\lbrace t_n \rbrace_{n=1}^{\infty}$ vzťahom $a+(b-a)t_{n}=a_{n}$, dostaneme postupnosť z pr. 197.1);
$\boxed{198.}$  $\boldsymbol{1,2.}$ platí; obidva dôkazy sú analogické dôkazu vety o Heineho definícii limity;
$\boxed{200.}$ stačí dokázať, že čísla na pravých stranách rovností vyhovujú obidvom podmienkam z vety 18;
$\boxed{201.}$  $\boldsymbol{1.}$ pozri napr. pr.160.7,8;
$\boldsymbol{2.}$ nie (pozri pr.165.1);
$\boxed{202.}$ sporom; označme $a=\lim_{n \to \infty}a_n$, $b=\lim_{n \to \infty}a_n$, ak $c \in (a,b)$, ale $c \notin H$, tak existuje $\eta > 0 $ tak, že $(c-\eta ,c+\eta) \cap H=\varnothing $ (pozri pr. 165.1, pritom $a\notin (c-\eta ,c+\eta)$, $b\notin (c-\eta ,c+\eta)$; potom $\lbrace n \in \mathbb{N};\, a_{n }\in (c-\eta /2 ,c+\eta /2)) \rbrace$ je konečná, preto $\exists N_{1} \in \mathbb{N}\, \forall n\in \mathbb{N}\,  n > N_{1}: \, (a_{n} > c+ \eta /2 \, \lor \, a_{n} < c- \eta /2 )$; z podmienky  $\lim_{n \to \infty} (a_{n+1} - a_{n})=0$ vyplýva $\exists N_{2} \in \mathbb{N}\,  \forall n \in \mathbb{N},\, n >N_{2}: \quad \vert a_{n+1} - a_{n} \vert < \eta $; nech $N= \max \lbrace N_{1}, N_{2} \rbrace $, potom platí ( $\forall n \in \mathbb{N},\, n >N: \quad a_{n} < c-\eta /2 $) $\lor $ ( $\forall n \in \mathbb{N},\, n >N: \quad a_{n} > c+ \eta /2 $); to je ale spor s faktom, že $a$  aj   $b$ sú hromadné hodnoty postupnosti $\lbrace a_{n} \rbrace _{n=1} ^{\infty} $;
$\boxed{204.}$  $\boldsymbol{1.}$  všetky postupnosti, ktorých hromadnou hodnotou je číslo $0 $;
$\boldsymbol{2.}$  všetky postupnosti z pr. 204.1 a všetky postupnosti $\lbrace a_{n} \rbrace _{n=1} ^{\infty} $, pre ktoré platí $\exists  n \in \mathbb{N}, n\geq 2: \, a_{n} =0$;
$\boxed{204.}$  $\boldsymbol{1.}$ vyplýva to z vety 19 a z tvrdenia: ak $\lbrace c_{n} \rbrace _{n=1} ^{\infty} $ je postupnosť kladných čísel a $\limsup_{n \to \infty}c_n=L \in (0,\infty) $, tak $\liminf_{n \to \infty} \frac{1}{c_n}=\frac{1}{L} $;
$\boldsymbol{2.}$ pre všetky, ktoré vyhovujú podmienke $0 <\liminf_{n \to \infty}a_n < +\infty$;
$\boxed{206.}$  $\boldsymbol{1.}$ v prípade  $\lim_{n \to \infty}a_n =0 $ uvedená rovnosť iste platí, pretože vtedy  $\lim_{n \to \infty}a_n . b_{n} =0 $; v prípade  $\lim_{n \to \infty}a_n \ne 0 $ sa postupuje analogicky ako v pr. 166;
$\boxed{207.}$ sporom; nech teda ($*$) $\exists N \in \mathbb{N} \, \forall n \in \mathbb{N}, n>N $ $\exists k \in \mathbb{N} \, k<n:\, a_{k} \leq a_{n} $; nech $m= \min \lbrace a_{n}; n=1,...,N \rbrace$ (zrejme $m>0 $), potom $a_{N+1}\geq m $  (podľa ($*$) niektoré z čísel $a_{1}, ... , a_{N}$  je menšie alebo rovné číslu $a_{N+1}$) atď, teda $\forall n\in \mathbb{N}: \, a_{n}\geq m $, čo je spor s predpokladom $\liminf_{n \to \infty}a_n =0 $;










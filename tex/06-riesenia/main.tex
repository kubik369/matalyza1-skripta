\chapter{Riešenia}%\label{chapter:gramatiky}

\epigraph{
  \enquote{Jíž pohled na ty pítvorné vzorce,\\
  hemžící se odmocninami a trigonometrií,\\
  je s to vyhnat veškerou nezpednost z těla.}
}{\textit{K. Poláček: Hostinec U kamenného stolu}}

\section{Úvod}
$\boxed{1.}$
$\boldsymbol{1.}\ \exists x\in \mathbb{R}:x^2<x;\\
\boldsymbol{2.}\ \forall x\in \mathbb{R}\ \exists y\in \mathbb{N}:y=x;\\
\boldsymbol{3.}\ \exists x,y\in \mathbb{R}:x>y \wedge	x^2 \leq y^2;\\
\boldsymbol{4.}\ \exists x,y\in \mathbb{R}\ \forall y\in \mathbb{N}: x\geq z \vee y\geq z; \\
\boldsymbol{5.}\ $symbolický zápis uvedeného výroku je 

$\forall a\in \mathbb{R}\setminus \{0\}\ \forall b\in \mathbb{R}\setminus \{0\}:((\ln |a|>\ln |b|) \Rightarrow (a>b)),$ 

negácia je $\exists a,b\in \mathbb{R}\setminus \{0\}: (\ln|a|>\ln|b|\vee a\leq b);\\
\boldsymbol{6.}\ \exists a\in \mathbb{R^+}:(|\ln(a)|<|\ln(b)|\wedge a\geq b);\\
\boldsymbol{7.}\ \exists A\subset \mathbb{R},A\neq \emptyset \exists B\subset \mathbb{R},B\neq \emptyset:$

$((\forall x\in A\ \forall y\in B\ \exists c>0:|x-y|>c)\wedge A\cap B\neq \emptyset),$ 

teda slovne: existujú neprázdne množiny $A,B\subset \mathbb{R}$ také, že platí 

$A\cap B\neq \emptyset$ a súčasne $\forall x\in A\ \forall y\in B\ \exists c>0:|x-y|>c;\\
\boldsymbol{8.}\ \exists A\subset \mathbb{R},A\neq \emptyset\ \exists B\subset \mathbb{R},B\neq \emptyset\ \exists c\subset \mathbb{{R_{0}}^+},C\neq \emptyset:$

$((\forall x\in A\ \exists y\in B\ \forall x\in C:x+c<y)\wedge(\exists x\in A\ \exists y\in B: x\geq y));\\
\boldsymbol{9.}\ \exists A\subset \mathbb{N},A\neq \emptyset:$

$((\forall n\in A:(2\nmid n\vee 3\nmid n))\wedge (\exists n\in A: 2|n)\wedge (\exists n\in A:3|n));\\
\boxed{2.}
\boldsymbol{1.}\ $platí tvrdenie $\forall\ r,s\in\mathbb{Z}:((2|r\wedge 2|s)\Rightarrow 2|r+s),$ ale neplatí tvrdenie s opačnou implikáciou, t.j. $\forall r,s\in\mathbb{Z}:((2|r+s)\Rightarrow (2|r \wedge 2|s))$, treba preto doplniť ``stačí'' (ale nemožno doplniť ``je nutné'' ani ``je nutné a stačí'');\\
$\boldsymbol{2.}\ $``je nutné'';
$\boldsymbol{3.}\ $``stačí'';
$\boldsymbol{4.}\ $``stačí'';\\
$\boldsymbol{5.}\ $platí výrok $\forall x\in\mathbb{R}\setminus \{0\}:(\frac{1}{x}<1\iff (x>1 \vee x<0))$ (a teda platia aj výroky $\forall x\in\mathbb{R}\setminus \{0\}:(\frac{1}{x}<1\Rightarrow (x>1 \vee x<0))$ a 

$\forall x\in \mathbb{R}\setminus \{0\}:((x>` \vee x<0)\Rightarrow \frac{1}{x}<1)$), preto doplnením každého z uvedených výrazov vznikne pravdivý výrok;


\section{Množiny}
$\boxed{3.}$
nepriamo z negácie (t.j. z výroku $\exists a\in\mathbb{Q},b\in\mathbb{R}\setminus \mathbb{Q}:a+b\in\mathbb{Q}$) vyplýva (pretože rozdiel racionálnych čísel $a+b$, $a$ je racionálne číslo) $\exists b\in \mathbb{R}:b\in \mathbb{R}\setminus \mathbb{Q}\wedge b\in \mathbb{Q}$, čo je nepravdivý výrok;\\
$\boxed{4.}$
nepriamo; využite rovnosť $\sqrt{ab}=\frac{1}{2}((\sqrt{a})+\sqrt{b})^2-a-b$;\\
$\boxed{5.}$
$\boldsymbol{1.},\ \boldsymbol{2.}$ možno postupovať ako v známom dôkaze iracionálnosti čísla $\sqrt{2}$; $\boldsymbol{3.},\ \boldsymbol{4.}$ využite príklad č.$4$; $\boldsymbol{5.}$ nepriamo ($\sqrt{3}=\frac{6a+3}{4},$ kde $a=\frac{4\sqrt{3}-3}{6}$);\\
$\boxed{6.}$
pozor, odvodenie pravdivého výroku z danej nerovnosti nie je ešte jej dôkazom;
$\boldsymbol{4.}$ do nerovnosti $x^2+y^2\geq 2xy$ dosaďte za $(x,y)$ postupne $(a,b),(a,c),(b,c)$ a získané nerovnosti sčítajte;
$\boldsymbol{6.}$ vyplýva z nerovnosti $(\sqrt{x^2+1}-1)^2 \geq 0$;
$\boxed{7.}$ 
ak $m\ (M)$ je najmenší (najväčší) zlomok, tak $mb_1\leq a_1\leq Mb_1,...,mb_n\leq a_n\leq Mb_n$; súčet týchto nerovností vydeľte $(b_1+...+b_n)$;
$\boxed{8.}$
$\boldsymbol{3.}$ matematickou indukciou, v jej druhom kroku treba využiť nerovnosť $\frac{\sqrt{n}+1}{\sqrt{n+1}}$;
$\boldsymbol{4.}$ v druhom kroku indukcie stačí pripočítaním a odpočítaním čísla $\frac{1}{n+1}$ upraviť ľavú stranu nerovnosti tak, aby sme mohli využiť indukčný predpoklad $\frac{1}{n+2}+...+\frac{1}{2n+2}=(\frac{1}{n+1}+...+\frac{1}{2n})+(\frac{1}{2n+1}+\frac{1}{2n+2}-\frac{1}{n+1})$, pritom druhá zo zátvoriek je kladné číslo;\\
$\boxed{9.}$
$\boldsymbol{2.}$ pre $n=1$ má dokazovaná nerovnosť tvar $|x_1|\leq |x_1|$, čo zrejme platí; dôkaz pre $n=2$: podľa pr. $9.1$ je $x_1\leq |x_1|,-x_1\leq |x_1|,x_2\leq |x_2|,-x_2\leq |x_2|$; odtiaľ vyplýva $x_1+x_2 \leq |x_1|+|x_2|,-x_1-x_2\leq |x_1|+|x_2|$; preto $\max \{x_1+x_2,-x_1-x_2\}\leq |x_1|+|x_2|$; t.j. (podľa pr. $9.1$) $|x_1+x_2|\leq |x_1|+|x_2|$; pre $n>2$ možno postupovať analogicky alebo použiť matematickú indukciu: $|x_1+...+x_{n+1}|=|(x_1+...+x_n)+x_{n+1}|\leq |x_1+...+x_n|+|x_{n+1}|$ (iná možnosť dôkazu pre $n=2:|x_1+x_2|^2=|x_1|^2+2x_1x_2+|x_2|^2\leq (|x_1|+|x_2|)^2$);$3.$ do nerovnosti $|x+y|\leq |x|+|y|$ dosaďte za $(x,y)$ postupne $(a-b,b)$, $(b-a,a)$ a použite pr. $9.1$;\\
$\boxed{11.}$
$\boldsymbol{4.}$ podľa návodu stačí dokázať nerovnosť $2\leq (\frac{(n+2)}{n+1})^{n+1}=(1+\frac{1}{n+1})^{n+1}$, tá vyplýva z binomickej vety (alebo z pr. $10.1$); $\boldsymbol{5.}$ nerovnosť $((2n+2)!=)$ $(n+2)!$ $[(n+3)(n+4)...(2n+2)]\geq (n+2)!$ $[(n+2)^n]$ možno dokázať priamo: $k$-ty člen v hranatej zátvorke vľavo je väčší ako $k$-ty člen v hranatej zátvorke vpravo, $k=1,...,n$;
$\boldsymbol{6.}$ nerovnosť možno dokázať priamo použitím binomickej vety, iná možnosť: podľa návodu stačí dokázať nerovnosť $(\frac{(n+1)^2}{n*(n+2)})^{n+1}\geq 1$, na jej dôkaz možno použiť pr. $10.1$;\\
$\boxed{12.}$ 
$\boldsymbol{1.}$ $x_1(1-x_2)>1-x_2$; $\boldsymbol{2.}$ matematickou indukciou; ak neplatí $x_1=x_2=...=x_{n+1}=1$ (vtedy totiž dokazovaná nerovnosť zrejme platí), je aspoň jedno z čísel $x_1,x_2,...,x_{n+1}$ väčšie a aspoň jedno menšie než $1$; preznačme ich tak, aby $x_n>1,x_{n+1}<1$; ak na $n$ čísel $x+1,...,x_{n-1},x_n*x_{n+1}$ použijeme indukčný predpoklad (ktorý má podobu $\forall a_1,...,a_n\in \mathbb{R^+}:a_1*a_2*...*a_n=1\Rightarrow a_1+...+a_n\geq n$), dostaneme $x_1+...+x_{n-1}+x_n*x_{n+1}\geq n$; podľa pr. $12.1$ je potom $x_1+...+x_{n-1}+x_m+x_{n+1}>x_1+...+x_{n-1}+x_n*x_{n+1}+1\geq n+1$;
$3.$ čísla $\frac{x_1}{x_2},...,\frac{x_n}{x_1}$, resp. $\frac{x_1}{\sqrt[n]{x_1*...*x_n}},...,\frac{x_n}{\sqrt[n]{x_1*...*x_n}}$ vyhovujú predpokladom tvrdenia z pr. $12.2$; 
$\boxed{13.}$
$\boldsymbol{1.}$ do nerovnosti $\frac{1}{k^2}<\frac{1}{k(k-1)}=\frac{1}{k-1}-\frac{1}{k}$ $(k\geq 2)$ dosaďte postupne $k=2,...,k=n$ a sčítajte; $\boldsymbol{2.}$ analogicky využite nerovnosť $\frac{1}{n!}<\frac{1}{2^{n-1}}$ $(n\geq 3)$ (možno ju dokázať matematickou indukciou);
$\boxed{14.}$
$\boldsymbol{1.}$ zdola ohraničená $(\forall m\in A:\sqrt{2}\leq m)$; platí $\forall K\in\mathbb{R}\ \exists m\in A:m>K$ (pre dané $K$ stačí položiť $a=1,b=n^4$, kde $n\in\mathbb{N}\setminus \{1\}$ je číslo väčšie než $K$; existenciu takého $n$ zaručuje Archimedov princíp - pozri $[18,$ str. $31]$), teda $A$ nie je ohraničená zhora; $\boldsymbol{2.}$ ohraničená $(\forall b\in B: 0<b\leq \frac{1}{2})$; $\boldsymbol{3.}$ ohraničená (pre každé $x\in\mathbb{R}$, teda aj pre $x=n!$, platí $|\sin x|\leq 1$); $\boldsymbol{4.}$ ohraničená (pre všetky $x\in (2,3)$, teda aj pre všetky $x\in\mathbb{Q}\cap (2,3)$ paltí $\frac{\sqrt{2}}{\sqrt[3]{3}+\sqrt[4]{3}}\leq \frac{\sqrt{x}}{\sqrt[3]{x}+\sqrt[4]{x}}\leq \frac{\sqrt{3}}{\sqrt[3]{2}+\sqrt[4]{2}}$); $\boldsymbol{5.}$ platí $\mathbb{Z}\subset E$ $(x^2-z^2$, kde $z\in\mathbb{Z}$, je polynóm druhého stupňa s racionálnymi koeficientami, ktorého koreňom je číslo $z$), preto $E$ nie je zhora ani zdola ohraničená;
$\boxed{15.}$
nepriamo; z výroku $ \exists a\in A \quad \exists \beta > 0  \quad \forall b \in A : \quad \mid a - b \mid < \beta \quad $ vyplýva na základe nerovnosti $ \mid b \mid \leq \mid a \mid + \mid b - a \mid  $ tvrdenie $ \exists a\in A \quad \exists \beta > 0  \quad \forall b \in A : \quad \mid b \mid < \mid a \mid + \beta$ , čo znamená, že $ A $ je ohraničená množina; 
$\boxed{16.}$ 
sporom; ak platí $ ( \star ) $ a $ A $ je ohraničená, tak z toho a nerovnosti $ \mid x \mid \leq  \mid y \mid + \mid x - y \mid $ vyplýva, že aj $ B $ je ohraničená;
$\boxed{17.}$
$\boldsymbol{2.}$
platí (pri dôkaze vlastnosti (ii) vyhovuje príslušnej nerovnosti pre $\varepsilon > 1$ každé reálne číslo $ x_{\varepsilon}  $, pre $ 0 < \varepsilon \leq  1 $ každé číslo  $ x_{\varepsilon}  $  s vlastnosťou $ \mid x_{\varepsilon} \mid > \sqrt{ \frac{1 - \varepsilon }{2 \varepsilon}} $;
$\boldsymbol{3.}$ $ 2x^{2}+8x+1=2.(x+2)^{2} -7 $, odtiaľ vidno, že $ -2 $ nie je dolným ohraničením danej množiny  ( teda nevyhovuje podmienke (i), ale nevyhovuje podmienke (ii) z definície  suprema (existuje totiž menšie horné ohraničenie ako číslo 12), uvedená rovnosť teda neplatí;
$\boxed{18.}$
$\boldsymbol{1.}$ $ sup \, A = 3 $, $ inf \, A = 2 $;
$\boldsymbol{2.}$ $ inf \, B = 0 $, $ sup \, B = 1,\overline{1} =\frac{10}{9} $;
$\boldsymbol{3.}$ $ C = \left\lbrace  1, -1 \right\rbrace  $, teda $ inf \, C = -1 $, $ sup \, C = 1 $;
$\boxed{19.}$
označme $ \beta := sup \, B $, potom pre všetky $ x \in B $ a teda aj pre všetky $ x \in A $, platí $ x \leq \beta $; teda $ \beta $ je horné ohraničenie množiny $ A $; pretože $ sup \, A $ je najmenšie z horných ohraničení množiny $ A $, platí $ sup \, A \leq \beta $, tvrdenie pre infíma znie: Nech $ A \subset B  \subset \R  $, pričom $ A $ je neprázdna a $ B $ zdola ohraničená množina, potom $ A $ je zdola ohraničená a platí $ inf \, A \geq inf  \, B $. 
$\boxed{21.}$ 
označme $ \alpha := sup \, A $,  $ \beta := sup \, B $; pretože $ x \leq \alpha $ pre $ x \in A $, $ y \leq \beta $ pre $ y \in B $, je $ x+y \leq \alpha + \beta $ pre každé $ x \in A $ a každé $ y \in B $ ( tým je dokázaná vlastnosť (i)), ak $ x_{\varepsilon } > \alpha - \frac{\varepsilon}{2} $,  $ y_{\varepsilon } > \beta - \frac{\varepsilon}{2} $   ( $ \alpha, \, \beta $ majú vlastnosť (ii), preto také $ x_{\varepsilon } \in A $, $ y_{ \varepsilon }$  $ \in B $ pre dané ${\varepsilon } > 0 $ existujú), tak $ x_{\varepsilon } + y_{\varepsilon } > \alpha + \beta - \varepsilon$, pričom $ x_{\varepsilon } + y_{\varepsilon } \in A + B $ (tým je dokázaná vlastnosť (ii)), pre infíma platí  $ inf \, A + inf \,B = inf \, ( A + B ) $; 
$\boxed{22.}$ 
$\boldsymbol{1.}$
$ y = \sin x , u=y^{3} $;
$\boldsymbol{2.}$ 
$ y = x^{3}, u= \sin y $
$\boldsymbol{3.}$ 
$ y = \tan x, u= y^{2}, v = 5^{u} $;
$\boldsymbol{4.}$ 
$ y = b^{x}, u= \sqrt{y}, v= \sin u, w = v^{2}, z = \log_{3} w $;
$\boldsymbol{5.}$ 
$ y = \sin x , u= 2^{y}, v= \cos u, w = \sqrt{v}$;
$\boxed{23.}$ 
$\boldsymbol{1.}$
$ (- \infty , - \sqrt{3}\rangle \cup \langle 0 , \sqrt{3}\rangle $;
$\boldsymbol{2.}$
$ (- \infty , -2 ) \cup ( 2 , \infty) $;
$\boldsymbol{3.}$
$\bigcup\limits_{n=0}^\infty \langle 4 n^{2} \pi ^{2}, (2n+1)^{2} \pi ^{2}\rangle $;
$\boldsymbol{4.}$
$\bigcup\limits_{k \in Z} (e^{-x/2}.e^{2kx},e^{x/2}.e^{2kx}) $;
$\boldsymbol{5.}$
$ \lbrace (2k + 1) \pi  ; k \in Z  \rbrace \cup \bigcup\limits_{n  \in Z} (\langle 2n  \pi , \frac{\pi}{3} + 2n \pi\rangle \cup \langle \frac{4 \pi}{3}+2n \pi , \frac{3 \pi}{2} + 2n \pi \rangle) $;
$\boldsymbol{6.}$
$ \langle - \sqrt{ \frac{ \pi}{2}},\sqrt{ \frac{ \pi}{2} \rangle} \cup \bigcup\limits_{n=1}^\infty  
\langle  \sqrt{- \frac{ \pi}{2}+ 2n \pi} ,  \sqrt{\frac{ \pi}{2}+ 2n \pi}\rangle \cup \bigcup\limits_{n=1}^\infty  
 \langle -\sqrt{ \frac{ \pi}{2}+ 2n \pi} , -\sqrt{-\frac{ \pi}{2}+ 2n \pi}\rangle $;
 $\boldsymbol{7.}$
 $ (- \frac{1}{3}, \infty) $;
 $\boldsymbol{8.}$
 pre $ f(x) =3. \sin^{2}x - 4, \, g(x)= \ln x $ je množina $ D_{3} $ z definície zloženej funkcie prázdna, preto týmto predpisom nie je definovaná žiadna funkcia; 
 $\boldsymbol{9.}$
 $ \langle -3, -1) \cup (2, 3\rangle $;
 $\boldsymbol{10.}$
 $ \langle 1, 4 \rangle $;
 $\boldsymbol{11.}$ 
 $ (0, \frac{1}{4}) $;
 $\boldsymbol{12.}$
 $ \mathbb{Z} $;
 $\boldsymbol{13.}$
 $ \bigcup\limits_{k \in Z} ((- \frac{\pi}{2}+ 2k \pi, 2k \pi) \cup ( 2k \pi, \frac{\pi}{2}+ 2k \pi)) $;
 $\boldsymbol{14.}$
 $ (- \infty, 1) \cup \langle 2, \infty) $;
 $\boldsymbol{15.}$
 $ (2,3) $;
 $\boldsymbol{16.}$
 $ \bigcup\limits_{n \in Z} (( \frac{\pi}{4}+ n \pi,  \frac{3}{4} \pi + n \pi) $ ( pre $n$ musí platiť alebo $ \sin x + \cos x = 0 $ alebo $ 0 < (\sin x + \cos x)(\sin x - \cos x)= - \cos 2x $; využili sme, že podiel $ \frac{a}{b} $ je kladný práve vtedy, keď $ a.b > 0 $ ); 
 $\boldsymbol{17.}$
 $ (-6,\frac{-5 \pi}{3}\rangle \cup \langle \frac{- \pi}{3}, \frac{1}{6 }) $; 
 $\boxed{24.}$ 
 $\boldsymbol{1.}$
$ f(f(x)) = x^{4} $, $ f(g(x)) = 4^{x} $, $ g(f(x)) = 2^{x^{2}} $, $ g(g(x)) = 2^{2^{x}} $;\\
\begin{align*} 
\boldsymbol{2.}\ f(g(x))= g(f(x))= \left\{ \begin{array}{cc} 
                1 & \hspace{5mm} ak\ x > 0 \\
                -1 & \hspace{5mm} ak\ x < 0 , \\
                \end{array} \right.
\end{align*}
$ g(g(x))= x \, x \ne 0 $; \\
$\boldsymbol{3.}$
$ f(f(x)) = f(x) $, $ f(g(x)) = 0 $, $ g(f(x)) = g(x) $, $ g(g(x)) = 0 $;
\begin{align*} 
\boldsymbol{4.}\ f(f(x))= \left\{ \begin{array}{cc} 
                x^{4} & \hspace{5mm} ak\ x \in \langle0,1\rangle \\
                9x & \hspace{5mm} ak\ x \notin \langle0,1\rangle , \\
                \end{array} \right.
\end{align*}, 
\begin{align*} 
 f(f(x))= \left\{ \begin{array}{cc} 
                4x^{2} & \hspace{5mm} ak\ x \in \langle0, \frac{1}{2}\rangle \\
                6x & \hspace{5mm} ak\ x \in ( \frac{1}{2},1\rangle \\
                12x-6 & \hspace{5mm} ak\ x \notin \langle0,1\rangle , \\
                \end{array} \right.
\end{align*}
\begin{align*} 
g(f(x))= \left\{ \begin{array}{cc} 
                2x^{2} & \hspace{5mm} ak\ x \in \langle0,1\rangle \\
                12x-2 & \hspace{5mm} ak\ x \notin \langle0,1\rangle , \\
                \end{array} \right.
\end{align*}, 
\begin{align*} 
 f(f(x))= \left\{ \begin{array}{cc} 
                4x & \hspace{5mm} ak\ x \in \langle0, \frac{1}{2}\rangle \\
                8x-2 & \hspace{5mm} ak\ x \in ( \frac{1}{2},1\rangle \\
                16x-10 & \hspace{5mm} ak\ x \notin \langle0,1\rangle , \\
                \end{array} \right.
\end{align*}
\begin{align*} 
\boldsymbol{5.}\ f(f(x))= f(x), f(g(x))= \left\{ \begin{array}{cc} 
                0 & \hspace{5mm} ak\ x^{2} \notin \mathbb{Q} \\
                1 & \hspace{5mm} ak\ x^{2} \in \mathbb{Q}-\lbrace0\rbrace \\
                \end{array} \right.
\end{align*}
$ g(f(x))=1, \, x \in \mathbb{Q}, \, g(g(x))=x^{4}, \, x \ne 0 $;
\begin{align*} 
  f(f(x))= \left\{ \begin{array}{cc} 
                4x & \hspace{5mm} ak\ x \in \langle0, \frac{1}{2}\rangle \\
                8x-2 & \hspace{5mm} ak\ x \in ( \frac{1}{2},1\rangle \\
                16x-10 & \hspace{5mm} ak\ x \notin \langle0,1\rangle , \\
                \end{array} \right.
\end{align*}
\begin{align*} 
\boldsymbol{6.}\ f(f(x))= 0, \, f(g(x))= \left\{ \begin{array}{cc} 
                1 & \hspace{5mm} ak\ x \in \langle -2, -\sqrt{3} ) \cup (-1,1) \cup ( \sqrt{3, 2}\rangle \\
                0 & \hspace{5mm} ak\ x \in ( - \infty, -2) \cup \langle - \sqrt{3},2\rangle \cup \langle1, \sqrt{3}\rangle \cup (2, \infty), \\
                \end{array} \right.
\end{align*}, 
\begin{align*} 
g(f(x))= \left\{ \begin{array}{cc} 
                -2 & \hspace{5mm} ak\ \vert x \vert\leq 1  \\
                -1 & \hspace{5mm} ak\ \vert x \vert > 1 , \\
                \end{array} \right.
\end{align*}, 
\begin{align*} 
g(g(x))= \left\{ \begin{array}{cc} 
                x^{4}-4x^{2}+2 & \hspace{5mm} ak\ \vert x \vert\leq 2  \\
                -1 & \hspace{5mm} ak\ \vert x \vert > 2 , \\
                \end{array} \right.
\end{align*} 
$\boxed{25.}$ 
$\boldsymbol{1.}$
$ f_{n}(x)= \frac{x}{\sqrt{1+nx^{2}}} $;
$\boldsymbol{2.}$
$ f_{n}(x)= a. \frac{b^{n}-1}{b-1}+b^{n}x \quad pre \quad b \ne 1$ (použite vzťah $ 1+b+.....+b^{n-1}=\frac{b^{n}-1}{b-1} $ ),  $ f_{n}(x)= na+x \, \quad pre \quad b=1 $;
$\boldsymbol{3.}$
$ f_{n}(x)= \frac{x}{a. \frac{b^{n}-1}{b-1}x+b^{n}}, \quad D(f_{n}) = \mathbb{R}- \lbrace - \frac{b^{k}}{a. \frac{b^{n}-1}{b-1}}\, , k=1,.....,n\rbrace $ ( všetky uvedené predpisy sa dokážu matematickou indukciou );
$\boxed{26.}$ 
$\boldsymbol{1.}$
nerovnajú sa $ D(f) = (-\infty, 0\rangle \cup \langle 1, \infty) \, \ne \, \langle 1, \infty) = D(g), \quad A=\langle 1, \infty) $;
$\boldsymbol{2.}$
nerovnajú sa $ A = (2, \infty) $;
$\boldsymbol{3.}$
nerovnajú sa $ A = \mathbb{R}- \lbrace \frac{k \pi}{2}; \, k \in \mathbb{Z}\rbrace $;
$\boldsymbol{4.}$
rovnajú sa $ D(f)=D(g)= \mathbb{R} $, zlomok $ \vert \frac{\sqrt{x^{2}+1}-x}{\sqrt{x^{2}+1}+x}\vert  $ treba rozšíriť výrazom $ \vert\sqrt{x^{2}+1}+x\vert $);
$\boldsymbol{5.}$
rovnajú sa ($ f(x) = \vert x+2 \vert - \vert x-4 \vert $),
$\boxed{27.}$ 
$\boldsymbol{7.}$
$ y=-1+ \frac{2}{1-x} $;
$\boldsymbol{12.}$
$ y= \frac{2}{3}( \frac{1}{8})^{x}+2 $;
$\boxed{29.}$ 
$\boldsymbol{1.}$
$ g(x)=f(2x_{0}-x) $;
$\boldsymbol{2.}$
$ g(x)=2y_{0}-f(x) $;
$\boxed{31.}$ 
$\boldsymbol{2.}$
$ y=  \sqrt{2}\, \sin (\frac{\pi}{4}-x)=-\sqrt{2}\, \sin (x-\frac{\pi}{4}) $;
$\boldsymbol{2.}$
$ y=10 \, \sin (x+ \alpha) $, kde $ \alpha $ je určená podmienkami $ \alpha \in \langle 0, \frac{\pi}{2}\rangle, \,  \sin \alpha = 0,6 $;
$\boxed{32.}$ 
$ g(x)=f(x-a)-2a^{3} $;
$\boxed{35.}$ 
$\boldsymbol{1.}$
pretože $ \vert \sin \frac{1}{x}  \vert \leq 1 $ pre každé $ x \in (0,1) $, je $ f $ ohraničená na 
$ (0,1) $;
$\boldsymbol{2.}$
pre všetky $ x \geq 2$ platí $ 0  < \frac{1}{4+x^{2}  < \frac{1}{8}} $, teda $ f $ je ohraničená na $ \langle2, \infty) $;
$\boldsymbol{3.}$
pre každé $ x \in (1,2 \rangle $ je $ \frac{x+5}{x-1} >0 $, tj. $ f $ je zdola ohraničená na $ (1,2 \rangle  $; $ f $ nie je zhora ohraničená na $ (1,2 \rangle  $  ( nerovnica $ \frac{x+5}{x-1} >K $ má v intervale $ (1,2 \rangle  $ riešenie pre každé $ K \in \mathbb{R} $;
$\boxed{36.}$
$\exists x \in B: f(x) > f(x_{0})$;
$\boxed{38.}$
ak $ \alpha:= sup_{x \in \mathbb{R}} \, f(x), \quad \beta :=sup_{x \in \mathbb{R}} \, g(x) $, tak pre všetky $ x \in \mathbb{R} $ je $ f(x) \leq \alpha $, $ g(x) \leq \beta $, odtiaľ $ f(x)+g(x) \leq \alpha + \beta $; teda $ \alpha + \beta $ je horné ohraničenie množiny $ (f+g)(\mathbb{R}) $; pretože $ \gamma := sup_{x \in \mathbb{R}}  (f(x)+g(x))$ je jej najmenším horným ohraničením, musí platiť  $ \gamma\leq\alpha+\beta $ (ak si uvedomíme, že tento dôkaz je založený na rovnakých myšlienkach ako dôkazy z pr. 19 a 21, môžeme pomocou nich naše riešenie zapísať aj takto: Označme $ A := (f+g)(\mathbb{R}), \, B:= \, f(\mathbb{R})+g(\mathbb{R}) $, potom $ A \subset B $. Podľa pr. 19 $ sup \, A \leq sup \, B $ a podľa pr. 21 $ sup \, B  = sup \, f( \mathbb{R}) + g( \mathbb{R}) $ teda $ sup_{x \in \mathbb{R}} (f(x)+g(x)) = sup \,A \leq  sup \,B =  sup_{x \in \mathbb{R}}f (x) +  sup_{\pi \in \mathbb{R}} g(x).) $;
$\boxed{40.}$
$\boldsymbol{1.}$
$ f(x)= \sqrt{2}\sin (x+\frac{x}{4}) $ (pozri pr. 31), teda $ f $ je rastúca na každej z  množín$ \langle -\frac{3 \pi}{4}+2k \pi, \frac{\pi}{4}+2k \pi \rangle ,\, k \in \mathbb{Z} $ ale nie je rastúca na zjednotení týchto množín), $ f $ je klesajúca na každej z  množín $ \langle \frac{ \pi}{4}+2k \pi, \frac{5 \pi}{4}+2k \pi \rangle ,\, k \in \mathbb{Z} $  ale nie je klesajúca na zjednotení týchto množín);
$\boldsymbol{2.}$
$ f(x) =( \cos^{2} x+ \sin^{2} x) ^{2} -2 \sin^{2} \cos^{2} x = 1- \frac{1}{2} \sin^{2} 2x = 1- \frac{1}{2}( \frac{1- \cos 4x}{2}) $; $ f $ rastie na každej z množín $ \langle (2k-1) \pi, 2k \pi\rangle , \, k\in \mathbb{Z} $, klesá  na každej z množín $ \langle 2k \pi, (2k+1) \pi \rangle   \, k\in \mathbb{Z} $;
$\boxed{41.}$
dokazovanie sa zjednoduší, ak si uvedomíme, že postupvosť $ \lbrace x_{n}\rbrace_{n=1}^\infty $  je rastúca práve vtedy, keď platí $ \forall n \in \mathbb{N}: \, x_{n}< x_{n+1} $; analogické tvrdenia platia pre všetky monotónne postupnosti;
$\boxed{42.}$
$\boldsymbol{1.}$
$ \exists x,y \in M: \, x<y \, \land \, f(x)\leq f(y) $;
$\boldsymbol{2.}$
$( \exists x,y \in M: \, x<y \, \land \, f(x)<f(y)) \,\land \, ( \exists x,y \in M: \, x<y \, \land \, f(x)>f(y))  $ ( funkcia nie je monotónna práve vtedy, keď nie je nerastúca a nie je neklesajúca - dokážte!);
$\boxed{45.}$
$\boldsymbol{1.}$
rastúca;
$\boldsymbol{2.}$
klesajúca;
$\boldsymbol{3.}$
klesajúca;
$\boldsymbol{4.}$
rastúca;
$\boxed{46.}$
$\boldsymbol{1.}$
mnemotechnická pomôcka: priraďte rastúcej funkcii číslo 1, klesajúcej číslo -1, potom číslo priradené superpozícii rýdzomonotónnych funkcií je súčinom čísel priradených k jednotlivým zložkám; 
$\boldsymbol{2a/}$
$ x^{2}-6x+10=(x-3)^{2}+1  $ klesá na $ (-\infty,3 \rangle $, rastie na $ \langle3, \infty) $, $log u $ rastie na   $(0, \infty) $, teda podľa príkladu 46.1 $ f $ klesá na $ (-\infty,3 \rangle $, rastie na $ \langle3, \infty) $;
$\boldsymbol{2b/}$
$ f $ rastie na $ (-\infty,4 \rangle $, klesá na  $ \langle4, \infty) $;
$\boxed{48.}$
$\boldsymbol{2.}$
sporom, keby $ a \notin \mathbb{Q}, a>0 $ bola perióda funkcie $ \chi $, muselo by platiť $ 0=\chi(1-a)= \chi((1-a)+a)=1 $;
$\boxed{49.}$
$ g(x+ \frac{T}{a})=f(ax+b+T)=f(ax+b)=g(x), \, x \in \mathbb{R} $ tada $ \frac{T}{a} $ je perióda; keby nebola najmenšia, tak zo vzťahu $f(x)=g(\frac{x}{a}-\frac{b}{a}) $  vyplýva, že $ f $ má niektorú periódu menšiu ako $ T $, čo je spor;
$\boxed{51.}$
$\boldsymbol{1.}$
$ f(x+2T)=-f(x+T)=f(x), \, x\in \mathbb{R} $,  teda $ 2T $ je perióda;
$\boldsymbol{2.}$
$ f(x+2T)=f(x) $;
$\boldsymbol{3.}$
$ f(x+2T)=f(x) $;
$\boldsymbol{4.}$
$ f(x+3T)=f(x) $;
$\boxed{52.}$
ak pre začiatok pomôžeme svojej predstavivosti obrázkom, zistíme, že $ f $ musí mať periódu $ 2(b-a) $; dôkaz: podľa pr. 29.1 pre všetky $ x \in \mathbb{R} $  platí $ f(x)=f(2a-x) $ a $ f(x)=f(2b-x) $, ak na vyjadrenie $ f(2a-x) $ použijeme druhú z týchto rovností, dostaneme $ f(x)=f(2b-(2a-x))=f(x+2(b-a)), \, x \in \mathbb{R} $
$\boxed{53.}$
$\boldsymbol{1.}$
$ f^{-1}(x)= \frac{1-x}{1+x}; \, x \in \mathbb{R}-\lbrace-1\rbrace $;
$\boldsymbol{2.}$
$ f $ nie je prostá (rovnica $ f(y)=x $ má dve rôzne riešenia pre každé $ x \in (- \frac{1}{\sqrt{8}}, \frac{1}{\sqrt{8}}) - \lbrace 0 \rbrace $;
$\boldsymbol{4.}$
$ f^{-1}(x)=\frac{x}{1+\sqrt{1-x^{2}}}, \, x \in \langle-1,1\rangle $, 
ak $ x \in \mathbb{R} $ je dané a má platnosť $ f(y)=x $, tak $ y $ je riešenie rovnice $ ay^{2}-2y+x=0; \, pre x=0 $ je 
$ y=0 $, pre $ x \in \langle -1,1 \rangle - \lbrace 0 \rbrace $ sú jej riešeniami $ y_{1}= \frac{1+\sqrt{1-x^{2}}}{x} $ a  $ y_{2}= \frac{1-\sqrt{1-x^{2}}}{x} $, pre $ \vert x\vert > 1$ riešenie neexistuje; podmienka $ \vert y_{2} \vert \leq 1 $ je splnená pre všetky 
$ x \in \langle -1,1 \rangle - \lbrace 0 \rbrace $, $ \vert y_{1} \vert \leq 1  $ len pre 
$ x=1,-1 $ ale  pre  $ x=1,-1 $ je $ y _{1} = y_{2} $;
teda $ f^{-1}(x)=\frac{1-\sqrt{1-x^{2}}}{x} $ pre $ x \in \langle -1,1 \rangle - \lbrace 0 \rbrace $, $ f^{-1}(0)=0 $; to možno napísať v spoločnom tvare $ f^{-1}(x) = \frac{x}{1+ \sqrt{1-x^{2}}}, \, \vert x \vert \leq 1 $;
$\boldsymbol{5.}$
$ f^{-1}(x) = 1+\frac{1}{log_{3}x -1},x \ne 3 \,  $;
$\boldsymbol{6.}$
$ f^{-1}(x)=1- \sqrt{1+log_{2}x}, \, x\geq \frac{1}{2} $;
$\boldsymbol{7.}$
$ f^{-1}(x)= \frac{1}{2} log \frac{x}{2-x}, \, x \in (0,2) $;
$\boldsymbol{8.}$
$ ^{-1} (x)= \frac{1}{2}ln (x^{2}-2x-2), \, x>1+ \sqrt{3} $ (rovnica $ x=1+ \sqrt{3+e^{2y}} $ má riešenie len pre $ x>1+ \sqrt{3}) $;
$\boldsymbol{9.}$
$ f^{-1} (x)=10^{1/x}$, $x\ne 0  $ (platí $ log_{x}10 = \frac{1}{log x} $; 
$\boldsymbol{10.}$
$ f^{-1} (x)=2 + e^{2 log_{2}\frac{x}{2}}$, $x> 0  $;
$\boldsymbol{11.}$
$ f^{-1} (x)=x + \sqrt{x^{2}+1}$, $x \in \mathbb{R}  $;
$\boldsymbol{12.}$
$ f^{-1} (x)=x - \sqrt{x^{2}+1}$, $x \in \mathbb{R}  $;
$\boldsymbol{13.}$
$ f^{-1} (x)= \frac{1}{2}(x+ \frac{1}{x})$, $x \in \langle-1,0) \cup \langle1, \infty)  $; ( rovnicu $ x = y + \sqrt{y^{2}-1} $ vynásobte výrazom $ y -  \sqrt{y^{2}-1} $, potom možno vyjadriť $ y= \frac{1}{2}(x+ \frac{1}{x}) $, pri skúške správnosti zistíme, že vyhovujú len tie riešenia, pre ktoré je $ x - \frac{1}{x}\geq 0  $;
$\boldsymbol{14.}$
$ f^{-1} (x)= \frac{1}{2}( 2^{x}- 2^{-x})$, $x \in \mathbb{R}  $;
$\boldsymbol{15.}$
\begin{align*} 
f^{-1}(x)= \left\{ \begin{array}{cc} 
                x & \hspace{5mm} ak\ x < 0 \\
                 \frac{x}{2} & \hspace{5mm} ak\ x \geq 0  \\
                \end{array} \right.
\end{align*}

$\boldsymbol{16.}$
\begin{align*}
f^{-1}(x)= \left\{ \begin{array}{cc} 
                x & \hspace{5mm} ak\ x  \in  \mathbb{Q} \\
                 1-x & \hspace{5mm} ak\ x \notin  \mathbb{Q}  \\
                \end{array} \right.
\end{align*}
$\boldsymbol{17.}$
\begin{align*}
f^{-1}(x)= \left\{ \begin{array}{cc} 
                x +1& \hspace{5mm} ak\ x < 0 \\
                  \sqrt{x}& \hspace{5mm} ak\ x \in  \langle1,16\rangle  \\
                  log _{2} x& \hspace{5mm} ak\ x> 16 
                \end{array} \right.
\end{align*}
$\boxed{54.}$ 
treba ukázať, že platí rovnosť $ f^{-1}=f $, (graf funkcie $ f $ je symetrický podľa osi $ y = x $ s grafom funkcie  $ f^{-1} $) ;
$\boxed{55.}$ 
$\boldsymbol{2.}$
napr. funkcia z pr. 53.16 ;
$\boxed{56.}$
neplatí ( vhodný protipríklad už musíte nájsť sami );
$\boxed{54.}$
$\boldsymbol{1.}$
$ x/3 $;
$\boldsymbol{2.}$
$  \frac{1}{\sqrt{2}} $;
$\boldsymbol{3.}$
$  \frac{x}{3} $;
$\boldsymbol{4.}$
$  \frac{x}{6} $;
$\boldsymbol{5.}$
$  \sqrt{3} $;
$\boxed{58.}$
$\boldsymbol{1a/}$
$ \arcsin x+2 \pi $;
$\boldsymbol{1b/}$
$  \pi - \arcsin x $;
$\boxed{59.}$
$\boldsymbol{1.}$
$ f^{-1}(x)= \arcsin \sqrt[3]{x} $, $ \vert x \vert \leq1 $;
$\boldsymbol{2.}$
$ f^{-1}(x)= \pi - \arcsin \sqrt[3]{x} $, $ \vert x \vert \leq1 $ ( ak je $ x $ dané a má platiť $ f(y)=x $, tak postupne dostávame $ x = \sin^{3} y $, $  \sqrt[3]{x } = \sin y $, $ \arcsin \sqrt[3]{x} = \arcsin (\sin y) = \pi -y $);
$\boldsymbol{3.}$
$ f^{-1}(x)= 2( \pi + \arcsin \sqrt{x}) $, $  x \in \langle0,1\rangle $  (postupne $ x = \sin ^{2} \frac{y}{2} $, $ \sqrt{x}= \vert \sin \frac{y}{2}\vert = - \sin \frac{y}{2} $, $ \arcsin \sqrt{x}= \arcsin (- \sin \frac{y}{2})= - \arcsin ( \sin \frac{y}{2}) = - ( \pi - \frac{y}{2}))$;
$\boldsymbol{4.}$
$ f^{-1}(x)= \arctan x- \pi $ pre $ x\geq 0 $, $ f^{-1} (x)= \arctan x$ pre $ x<0 $;
$\boldsymbol{5.}$
$f^{-1}(x)= \cot x $, $0<\vert x \vert < \frac{ \pi}{2}$;
$\boldsymbol{6.}$
$ f^{-1}(x)= \sqrt[3]{1- \sin ^{2} \frac{\pi}{4}} $, $ x \in \langle0, 2 \pi \rangle $;
$\boldsymbol{7.}$
$ f^{-1}(x)= 1+ \cos  \frac{x-3}{4} $, $ x \in \langle 3, 3+4 \pi \rangle $;
$\boldsymbol{8.}$
$ f^{-1}(x)= \tan (log _{2} \frac{\pi}{8} $, $ x \in ( 2^{3- \frac{\pi}{2}}, 2^{3+ \frac{\pi}{2}}) $;
$\boldsymbol{9.}$
$ f^{-1} (x) = \frac{1+ \arcsin x}{3}$, $ \vert x \vert<1  $ ($ D (f^{-1}) $ sa v týchto príkladoch nájde ak $ f(D(f)) $);
$\boxed{60.}$
$\boldsymbol{1.}$
$ \sqrt{1-x^{2}} $;
$\boldsymbol{2.}$
$ \sqrt{1-x^{2}} $ (treba použiť vzorec $ \sin 2 \alpha = 2 \sin \alpha \cos \alpha $);
$\boldsymbol{3.}$
$ \frac{1}{1+x^{2}} $ (pre $ u \in (-\frac{\pi}{2},\frac{\pi}{2} ) $ platí pre $ \cos ^{2 }u = \frac{1}{1+\tan^{2}u} $);
$\boldsymbol{4.}$
$ \frac{1}{1+x^{2}} $;
$\boldsymbol{5.}$
$ \frac{x}{\sqrt{1+x^{2}}} $;
$\boldsymbol{6.}$
$ \frac{3x-x^{3}}{1-3x^{2}} $, $ x \ne \frac{1}{\sqrt{3}} $;
$\boxed{61.}$
$\boldsymbol{2.}$
$ x \in \langle-1,0\rangle $ (funkčné hodnoty obidvoch strán ležia v $ \langle - \frac{\pi}{2},\frac{\pi}{2}\rangle  $, možno využiť injektívnosť funkcie $ \sin/\langle - \frac{\pi}{2},\frac{\pi}{2}\rangle ) $;
$\boldsymbol{3.}$
$ x\geq 0 $ pre $ x<0 $ je $ \arccos \frac{1-x^{2}}{1+x^{2}}  \in (0,\pi\rangle $, $ 2 \arctan x \in (-\pi,0) $; pre $ x\geq 0 $ možno využiť injektívnosť funkcie $ \cos/\langle0,\pi) $ a vzťah $ \cos^{2} \alpha = \frac{1}{1+ \tan^{2}\alpha} $, $ \alpha \ne k \pi, \, k \in \mathbb{Z} $;
$\boldsymbol{4.}$
$ x \in \mathbb{R} $;
$\boldsymbol{5.}$
$ x > 1 $;
$\boxed{62.}$
$\boldsymbol{1.}$
čísla $ \arccos x, \, \frac{\pi}{2}-\arcsin x $ ležia v $ \langle0, \pi\rangle $, stačí využiť injektívnosť funkcie $ \cos/\langle0,\pi) $;
$\boldsymbol{1.}$
pre $ x>0 $ použite injektívnosť funkcie $ \cos/\langle0,\pi) $, pre $ x<0 $  injektívnosť funkcie $ \cos/( -\pi,0\rangle $;
$\boxed{64.}$
$\boldsymbol{1.}$
$ sh \, x\,\, ch\, y + sh \,y\,\, ch\, x $;
$\boldsymbol{2.}$
$ ch \, x\,\, ch\, y + sh \,y\,\, sh\, x $;
$\boldsymbol{3.}$
$ \sgn x. \sqrt{\frac{ch\, x -1}{2}} $ (stačí sčítať rovnosti 63.1 a 63.2, kam dosadíme $ \frac{x}{2} $ namiesto $ x $);
$\boldsymbol{4.}$
$ 2 \, ch \, \frac{x+y}{2}. ch\, \frac{x-y}{2}$ použite rovnosti $ x= \frac{x+y}{2}+ \frac{x-y}{2} $,  $ y= \frac{x+y}{2}- \frac{x-y}{2} $ a  vzorec pre $ ch \, ( \alpha\pm \beta) $);
$\boldsymbol{5.}$
$ 2 \, sh \, \frac{x+y}{2}. ch\, \frac{x-y}{2}$;
$\boxed{65.}$
platí
$\boxed{66.}$
neplatí; platí teda negácia uvedeného tvrdenia (na overenie jej platnosti stačí položiť $ A = \mathbb{R} $, $ B= \lbrace0\rbrace $);
$\boxed{68.}$
podľa (*) je každé $ x\in A $ dolné ohraničenie množiny $ B $, každé $ y \in B $ horné ohraničenie  množiny $ A $, preto $ A $ je zhora , $ B $ zdola ohraničená a platí $ \forall x \in A: \, x\leq \inf B $; teda $ \inf B $ je horné ohraničenie $ A $, odtiaľ $ \sup A\leq \inf B $; ďalej postupujeme sporom: ak $ \sup A > \inf B $, položme $ \varepsilon_{1}:= \sup A - \inf B $  zrejme , $ \varepsilon_{1}>0 $, z toho odvodíme spor s (**);
$\boxed{69.}$ 
ak $ k\geq 0 $, tak $ \sup kA = k \sup A $, $ \inf kA = k \inf A $; ak $ k\geq 0 $, tak $ \sup kA = k \inf A $, $ \inf kA = k \sup A $; (tieto tvrdenia treba ovšem dokázať, pritom dôkaz pre $ k<0 $ možno vykonať pomocou tvrdenia z pr. 20 a už dokázaných vzťahov pre $ k\geq 0 $, pretože $ kA = -((-k).A)) $;
$\boxed{71.}$
platí (treba dokázať, že číslo $ \sup _{x\in I} f(x) - \inf _{x\in I} f(x)  $ má obidve vlastnosti, ktoré charakterizujú suprémum množiny $ \lbrace \vert f(x)-f(y)\vert ;x,y \in I\rbrace $; iná možnosť: dokázať rovnosť $ \sup \lbrace \vert f(x)-f(y)\vert ;x,y \in I\rbrace =  \lbrace f(x)-f(y) ;x,y \in I\rbrace $, potom využiť rovnosť $ \sup (A-B) = \sup A - \inf B $);
$\boxed{72.}$
$\boldsymbol{2.}$
tento vzťah možno odvodiť z 72.1 rovnako, ako sa odvodzuje nerovnosť $ \vert \vert a \vert - \vert b \vert\vert \leq \vert a-b \vert $ z nerovnosti $ \vert a+b \vert \leq \vert a \vert + \vert b \vert $;
$\boxed{73.}$
napr. $ f(x) = 0 $ pre $ x \notin \mathbb{Q} $, $ f(x) = m $, ak $ x=\frac{n}{m} $, kde $ m \in \mathbb{N} $ a $ n \in \mathbb{Z} $ sú nesúdeliteľné; resp. $ g(x)=0 $ pre $ x \notin \mathbb{Q} $, $ g(x)=(-1)^{m}.m $, ak $ x= \frac{n}{m} $;
$\boxed{76.}$
ak $ T $ je perióda, tak $ 1= \sin 0 + \cos a.0 = \sin T + \cos aT = \sin (-T) + \cos a.(-T) = -\sin T + \cos aT $; odtiaľ $ \sin T = 0 $, tj. $ T $  je v tvare $ k \pi\ \,(k \in \mathbb{N})$; $ \cos aT = 1$, tj. $ aT = 2n \pi \, (n \in \mathbb{Z}) $, odtiaľ $ a=\frac{2n}{k} \in \mathbb{Q} $;
$\boxed{77.}$
neexistuje ; stačí dokázať, že funkcia, ktorej periódou je každé kladné iracionálne číslo, je konštantná (porovnaj pr. 77 a pr. 48);
$\boxed{79.}$
pozri pr. 29 a 52;
$\boxed{80.}$
$\boldsymbol{1.}$
$ (-1,1) $;
$\boldsymbol{2.}$
$ \bigcup\limits_{k \in \mathbb{Z}} (2k \pi , (2k+1)\pi) $;
$\boldsymbol{3.}$
$ (1,e) $;
$\boxed{81.}$
$\boldsymbol{1.}$
$ f(t) = t^{2} -5t+6 $ (ak je  dané $ t \in \mathbb{R} $ a chceme nájsť $ f(t) $, treba voliť $ x $ tak, aby platilo $ x+1=t $);
$\boldsymbol{2.}$
$ f(t)=t^{2}-2 $ pre $ \vert t \vert \geq 2 $, $ f(t) $ je ľubovoľne určené pre $ \vert t \vert < 2 $ (pri úpravách využite rovnosť  $ \frac{t+ \sqrt{t^{2}-4}}{2}= \frac{2}{t - \sqrt{t^{2}-4}}$);
$\boldsymbol{3.}$
$ f(t)=\frac{1}{t}+ \sqrt{1+ \frac{1}{t^{2}}}$,  pre  $t\ne 0$, $ f(0) $ je určené ľubovoľne;
$\boldsymbol{4.}$
$ f(t)=\frac{t^{2}}{(1-t)^{2}} $ pre $ t \ne 1 $, $ f(1) $ je určené ľubovoľne;
$\boxed{82.}$
nie;
$\boxed{83.}$
$\boldsymbol{1.}$
áno, napr. $ f(x) = \chi (x) $, $ g(x) = \chi (x)+1 $;
$\boldsymbol{2.}$
áno, pre $ g(\mathbb{R}) $ musí platiť: ak $ I \subset \mathbb{R} $ je interval, tak $ I\not\subset g(\mathbb{R} ) $; zoraďme spočitateľnú množinu $ \mathbb{Q} $ (definíciu spočitateľnosti pozri v Dodatku 1; k spočitateľnosti  množiny $ \mathbb{Q} $ pozri pr. 94) do prostej postupnosti $ \lbrace a _{n}\rbrace_{n=1}^\infty  $, potom funkcie definované predpismi $ g(x)=x $ pre $ x \notin \mathbb{Q} $, $ g(x)=n $ pre $ x=a_{n} $; $ f(x)=0 $ pre $ x \in \mathbb{N} \cup (\mathbb{R}-\mathbb{Q}) $, $ f(x)=1 $ pre $x \in \mathbb{Q} - \mathbb{N} $ vyhovujú zadaniu;
 $\boxed{84.}$
$\boldsymbol{1.}$
$ f(x)=3(\cos x - \frac{\sqrt{3}}{2})^{2}+ \frac{7}{4} $, funkcia $ g(u)=3(u - \frac{\sqrt{3}}{2})^{2}+ \frac{7}{4} $ má na $ \langle-1,1\rangle $ minimum v bode $u =\frac{\sqrt{3}}{2} $ maximum v bode $ u=-1 $; teda $ \frac{7}{4}= \min_{x\in \mathbb{R}} f(x) $ sa nadobúda pre $ x=\pm \frac{\pi}{6}+2k \pi \, k\in \mathbb{Z} $, $ 7+3\sqrt{3}= \max_{x\in \mathbb{R}} f(x) $ sa nadobúda pre $ x=(2k-1)\pi \, (k\in \mathbb{Z}) $; 
$\boldsymbol{2.}$
$ 1=3^{0}=\min_{x\in \mathbb{R}} f(x) $ pre $ x=0 $, $ \max_{x\in \mathbb{R}} f(x) $ neexistuje $ g(u)=3^{0} $ je rastúca a zhora neohraničená na $ \langle0,\infty) $);
$\boldsymbol{3.}$
$ \sqrt{a^{2}+b^{2}}=\max_{x\in \mathbb{R}}f(x) $ pre $ x=\alpha + \frac{\pi}{2}+2k \pi \, (k\in \mathbb{Z}) $, $ -\sqrt{a^{2}+b^{2}}=\min_{x\in \mathbb{R}}f(x) $ pre $ x=\alpha - \frac{\pi}{2}+2k \pi \, (k\in \mathbb{Z}) $, kde $ \alpha $ je určená podmienkami $ \sin \alpha = - \frac{a}{\sqrt{a^{2}+b^{2}}} $, $ \cos \alpha =  \frac{b}{\sqrt{a^{2}+b^{2}}} $ (pozri pr. 31);
$\boldsymbol{4.}$
$ f(x)=\frac{1}{\sqrt{2}} \sin (2x+\frac{\pi}{4})+\frac{1}{2} $; $ \max_{x\in \mathbb{R}}f(x)= \frac{\sqrt{2}+1}{2} $, $ \min_{x\in \mathbb{R}}f(x)=\frac{1-\sqrt{2}}{2} $;
$\boxed{85.}$
(pozri pr. 46) f rastie na $ (-\infty,-1\rangle $, klesá na $ \langle-1,0\rangle $, rastie na $ \langle0,1\rangle $, klesá na $ \langle1,\infty) $;
$\boxed{87.}$
$\boldsymbol{1.}$
pre $ x>0 $, $ y>0 $ využite injektívnosť $ cos/(0,\pi) $, pre $ x<0 $, $ y<0 $ injektívnosť $ cos/(-\pi,0) $, pre $ xy\leq 0 $ injektívnosť $ sin/(-\frac{\pi}{2},\frac{\pi}{2}) $, pritom tiež zistíte, že $ \varepsilon =0 $ pre $ xy<1 $, $ \varepsilon =1 $ pre $x>0 \, \land \, y > 0 \, \land \, xy>1 $, $ \varepsilon =-1 $ pre $x<0 \, \land \, y < 0 \, \land \, xy>1 $;
$\boxed{88.}$
$ f^{-1}(x)=\frac{1-2\pi+\arcsin(y-1)/2}{1+2\pi+\arcsin(y-1)/2} $, $ x \in \langle-1,3\rangle $;
$\boxed{89.}$
platí;



\section{Dodatok 1}
$\boxed{90.}$
$ \mathbb{Z} $ možno zoradiť do postupnosti $ 0,1,-1,2,-2,3,-3,..... $;
$\boxed{91.}$
zoraďme $ B $ do prostej postupnosti $ b_{1}, b_{2}..... $; "vyškrtaním" prnkov nepatriacich do $ A $ dostaneme prostú postupnosť, ktorá obsahuje všetky prvky z $ A $; pri konkrétnom zápise tejto "voľnej" úvahy využite princíp konštrukcie postupnosti matematickou indukciou: a/ určí sa prvý člen (nech $ k_{1}:= min \lbrace n \in \mathbb{N} ;  b_{n} \in A \rbrace $ , potom $ a_{1}=b_{k_{1}} $); b/ udá sa spôsob, akým určiť člen $ a_{n+1} $, ak sú už určené členy $ a_{1},...,a_{n} $ (ak $ a_{n}=b_{k_{n}} $, potom $ b_{n+1}:= \min \lbrace n \in \mathbb{N}; n>k_{n} \, \land b_{n} \in A \rbrace $, potom $ a_{n+1}:=b_{k_{n+1}} $);
$\boxed{92.}$
pre $ A\cap B=\varnothing $: všimnite si, ako sme "zoradili" množiny $ \lbrace1,2,3,...\rbrace $ a $\lbrace 0,-1,-2,...\rbrace $ v pr. 90; pre $ A \cap B \ne \varnothing $ použite ten istý postup na $ A $ a $ A-B $ (pozor  $ A-B $ môže byť aj nekonečná);
$\boxed{93.}$
ak $ A_{i}\cap B_{i}=\varnothing $ pre $ i \ne j $, $ i,j \in \mathbb{N} $, zoraďte $ A $ podľa schémy

\newpage


vo všeobecnom prípade využite rovnosť $ A=A_{1} \cup (A_{2}-A_{1}) \cup (A_{3}- (A_{1}\cup A_{2})... $;
$\boxed{94.}$
využite pr. 92 a 93 $ ( \mathbb{Q}-\mathbb{Q^{-} \cup \mathbb{Q^{+}_{O}}} $, kde $\mathbb{Q^{+}_{O}} =\bigcup\limits_{n \in \mathbb{R}} Q_{n} $, $  Q_{n}=\lbrace \frac{m}{n}; \, m\in \lbrace 0 \rbrace \cup \mathbb{N}\rbrace )$; 
$\boxed{95.}$
nepriamo, ak $ E-\langle-c,c\rangle $ je spočitateľná pre každé $ c $, tak $ E= \bigcup\limits_{n \in \mathbb{R}} (\langle-n,n\rangle \cap E)$  je spočitateľná;
$\boxed{96.}$
nepriamo, ak $ E_{1} $ je nespočitateľná, tak existuje $ n \in \mathbb{N} $ tak, že $ E_{1} \cap ( \frac{1}{n}, \infty) $ je nespočitateľná, teda nekonečná; potom $ S $ nie je zhora ohraničená;
$\boxed{97.}$
áno; $ A:= \lbrace x-y;  x,y \in E \rbrace $ je spočitateľná; ak $ a \in \mathbb{R}-A $   ($ \mathbb{R}-A $ je neprázdna množina, lebo $ \mathbb{R} $ je nespočitateľná), tak $ E+a \cap E = \varnothing $;
$\boxed{99.}$
z každého intervalu z $ \varphi $ vyberme jedno racionálne číslo (zrejme čísla vybrané z rôznych intervalov sú navzájom rôzne), takto vytvorená množina čísel je podmnožinou množiny $ \mathbb{Q} $, preto je spočitateľná;

\section{Limita funkcie}
$\boxed{100.}$
$\boldsymbol{1.}$
$ A=\lbrace 1/k \pi; \, k \in \mathbb{Z}- \lbrace 0 \rbrace \rbrace $, treba dokázať $ \forall O_{\varepsilon} (0) $ $ \exists x \in A; x \ne 0 \land x \in O_{\varepsilon} (0) $, čo možno zapísať v tvare $ \forall \varepsilon > 0 $ $ \exists k \in \mathbb{Z}- \lbrace0\rbrace : \vert 1/k \pi \vert < \varepsilon $, existencia riešenia poslednej nerovnosti vyplýva z Archimedovho princípu;
$\boldsymbol{3.}$
využite, že do $ A $ patria čísla $ 0,1;0,11;0,111...... $;
$\boxed{101.}$
$\boldsymbol{1.}$
$ A^{)} \langle0,1\rangle $ (nestačí len dokázať, že každý prvok z $ \langle0,1\rangle $ je hromadným bodom množiny $ A $, treba aj dokázať, že žiadny prvok z $ \mathbb{R}^{\chi} - \langle0,1\rangle $ už nepatrí do $ A^{)} $);
$\boldsymbol{2.}$
$ B^{)}= \lbrace0,\infty \rbrace $;
$\boldsymbol{3.}$
$ C ^{)}=\langle-2,+ \infty) \cup \lbrace+\infty\rbrace $;
$\boldsymbol{4.}$
$ D- \mathbb{Q} \cap (0,1) $, $ D^{)} = \langle0,1\rangle$ (pri dôkaze poslednej rovnosti využite fakt, že (nedegenerovaný) interval obsahuje aspoň jedno racionálne číslo);
$\boldsymbol{5.}$
$ E= \mathbb{Q}-\lbrace1\rbrace $, $ E^{)}=\mathbb{R}^{\chi} $;
$\boldsymbol{6.}$
$ F^{)}=\langle1,2\rangle $;
$\boxed{102.}$
$\boldsymbol{1.}$
$ \exists O(+ \infty):M \cap (+ \infty)=\varnothing $, to možno zapísať aj v tvare $ \exists x \in \mathbb{R} \quad \forall x \in M; x\leq K $;
$\boldsymbol{2.}$
$ \forall a \in \mathbb{R}^{\chi} \exists O(a); \quad (O(a)-\lbrace a \rbrace ) \cap M= \varnothing \ $ to možno zapísať aj v tvare $ \forall a \in \mathbb{R}^{\chi} \quad  \exists O(a) \quad \forall x \in M: \quad x=a \lor x \notin O(a)  $;
$\boxed{103.}$
$\boldsymbol{1.}$
ľubovoľná konečná neprázdna množina má túto vlastnosť;
$\boldsymbol{2.}$
napr. $ A= \mathbb{N} \cup \lbrace \frac{n+1}{n}; n\in \mathbb{N} \rbrace$;
$\boldsymbol{3.}$
napr. $ A= \mathbb{Z} $;
$\boldsymbol{4.}$
na $ A $ stačí zvoliť ľubovoľný interval;
$\boldsymbol{5.}$
napr. $ A= \lbrace m+ \frac{1}{n}; \, n \in \mathbb{N} \rbrace $, potom $ A^{)}= \mathbb{N} \cup \lbrace \infty \rbrace $;
$\boxed{104.}$
pri dôkaze využite druhú vlastnosť suprema;
$\boxed{105.}$
$\boldsymbol{1.}$
nerovnosti $ \vert \frac{3n^{2}+1}{5n^{2}-1} - \frac{3}{5} \vert < \varepsilon $ (pre dané $ \varepsilon> 0 $) vyhovujú všetky tie $ n \in \mathbb{N} $, pre ktoré platí $ n > \sqrt{\frac{5\varepsilon +8}{25 \varepsilon}} $; stačí teda položiť napr. $ n_{0}= [  \sqrt{\frac{5\varepsilon +8}{25 \varepsilon}}]   $ pre $ \varepsilon \in (0, \frac{2}{5}\rangle $ a $ n_{0}= 1 $ pre $ \varepsilon > \frac{2}{5} $ (pre $ \varepsilon > \frac{2}{5} $ je totiž $ [  \sqrt{\frac{5\varepsilon +8}{25 \varepsilon}}] =0 $ keby  sme v definícii limity postupnosti namiesto $n \in \mathbb{N}  $ požadovali $ n \in \mathbb{R} $ stačilo by pre dané $ \varepsilon > 0 $ položiť $ n_{0} \geq   \frac{5\varepsilon +8}{25 \varepsilon }$); a/ pre všetky $  n \in \mathbb{N} $; b/ pre všetky $ n\geq 8 $; c/ pre všetky $ n\geq 80 $;
$\boldsymbol{3.}$
nerovnosti $ \frac{n^{2}}{n+8}>K $ vyhovujú pre $ K > 0 $ všetky čísla $ n \in \mathbb{N} $ pre $ K \geq 0 $ všetky tie $ n \in \mathbb{N} $, pre ktoré platí $ n > \frac{1}{2} (K+ \sqrt{K^{2}+32K)} $ stačí teda položiť napr. $ n_{0}=1 $ pre $ K < 0 $, $ n_{0}= [ \frac{1}{2} (K+ \sqrt{K^{2}+32K)}] +1 $ pre $ K\geq 0 $ pripočítaním čísla 1 zaručíme, že aj v prípade $ [ \frac{1}{2} (K+ \sqrt{K^{2}+32K)}]=0 $ bude platiť $ n_{0} \in \mathbb{N}; \frac{n^{2}}{n+8}> 10^{3} $ platí pre všetky $ n > 1007 $;
$\boldsymbol{4.}$
nerovnosti $ \frac{5}{n}-n < K $ vyhovujú všetky tie $ n \in \mathbb{N} $, pre ktoré $ n > \frac{1}{2} ( \sqrt{K^{2}+20} -K)  $, stačí teda pre dané $ K \in \mathbb{R} $ položiť $ n_{0} = [ \frac{1}{2} ( \sqrt{K^{2}+20}-K )]+1 $; iné riešenie : pretože  $ \frac{5}{n}-n \leq 5-n $
pre $ n \in \mathbb{N} $ je každé riešenie nerovnice $ \frac{5}{n}-n < K $, teda nerovnici $ \frac{5}{n}-n < K $ iste vyhovujú všetky tie $ n $, pre ktoré $ a> 5-K $ (podobnými únahami možno zjednodušiť aj riešenia pr. 105.1-3);
$\boldsymbol{5.}$
ak $ q=0 $ , stačí položiť $ n_{0}=1 $ pre každé $ \varepsilon  > 0 $, ak $ q\neq 0 $, $ \vert q \vert < 1 $, sú riešeniami nerovnice $ \vert q^{n} \vert < \varepsilon $ všetky tie $ n \in \mathbb{N} $, pre ktoré platí $ n  > \frac{\ln \varepsilon}{\ln \vert q \vert} $, stačí teda položiť $ n_{0}=1 $ pre $ \varepsilon\geq 1 $, $ n_{0}= [  \frac{\ln \varepsilon}{\ln \vert q \vert} ]+1 $ pre $ \varepsilon \in (0,1) $;
$\boxed{106.}$
$\boldsymbol{1.}$
$ \lim_{n \to \infty}a_n=1$; nerovnosti $ \vert a_{n} -1 \vert < \varepsilon $ vyhovujú všetky tie párne $ n \in \mathbb{N} $, pre ktoré $ n > \frac{1}{\varepsilon} $ a všetky tie nepárne $ n \in \mathbb{N} $, pre ktoré $ n > \frac{1}{\sqrt{\varepsilon}} $, pre dané $ \varepsilon > 0 $ stačí teda položiť napr. $ n_{0}= [ \max \lbrace \frac{1}{\varepsilon} , \frac{1}{\sqrt{\varepsilon}} \rbrace ]+1 $;
$\boldsymbol{2.}$
$ \lim_{n \to \infty}a_n=0$;
$\boxed{108.}$
$\boldsymbol{1.}$
neexistujú, nerovnica $\vert a_{n}  -b \vert < \varepsilon   $ nemá totiž riešenie pre $ \varepsilon < 0 $;
$\boldsymbol{2.}$
len konštantná postupnosť $ b,b,b,b,...(b \in \mathbb{R}) $;
$\boldsymbol{3.}$
na overenie pravdivosti tvrdenia  $ \exists  \, \varepsilon > 0 \quad \exists \, n_{0 } \in \mathbb{N} \quad \forall n \in \mathbb{N} , \, n> n_{0}: \vert 2-7 \vert < \varepsilon $ stačí položiť $ \varepsilon = 6 $;
$\boldsymbol{4.}$
len konštantná postupnosť $ b,b,b,b,...(b \in \mathbb{R}) $;
$\boxed{109.}$
áno; z tvrdenia $ \exists N^{*}\in \mathbb{N} \quad \forall \, \varepsilon > 0 $ $ \forall n \in \mathbb{N} , \, n> N^{*} $: $ \vert a_{n} -b \vert < \varepsilon $ vyplýva totiž tvrdenie $ \forall \varepsilon > 0 $, $ \exists n_{0 }\in \mathbb{N} $, $ \forall n \in \mathbb{N}, \, n>n_{0} $: $ \vert a_{n} -b \vert < \varepsilon $; pre každé $ \varepsilon > 0  $ stačí položiť $ n_{0} = N^{*} $ ( z tvrdenia   $ \exists N^{*}\in \mathbb{N} \quad \forall \, \varepsilon > 0 $ $ \forall n \in \mathbb{N} , \, n> N^{*} $: $ \vert a_{n} -b \vert < \varepsilon $ vyplýva aj to, že pre  $ n> N^{*}$  je $ a_{n}=b $);
$\boxed{110.}$
$\boldsymbol{1.}$
všetky ohraničené postupnosti; 
$\boldsymbol{2.}$
len postupnosť   $ 0,0,0,... $;
$\boldsymbol{3.}$
všetky  ohraničené  postupnosti (každú z uvedených odpovedí treba samozrejme podrobne zdôvodniť);
$\boxed{111.}$
$\boldsymbol{2.}$
$\forall K \in \mathbb{R} \quad \exists \delta > 0  \,  \forall x \in D(f), \, x \ne a: \, \vert x-a \vert < \delta  $  $\Rightarrow   \, f(x)  > K  $; 
$\boldsymbol{3.}$
$\forall K \in \mathbb{R} \quad \exists \delta  >0  \, \forall x \in D(f), \, x \ne a: \, \vert x-a \vert < \delta  $  $\Rightarrow   \, f(x)  < K  $;
$\boldsymbol{4.}$
$ \forall \varepsilon >0  $  $\exists L \in \mathbb{R} \quad \forall x \in D(f): \, x>L $ $\Rightarrow \, \vert f(x) - b \vert  < \varepsilon   $;
$\boldsymbol{5.}$
$ \forall K \in \mathbb{R} \quad \exists L \in \mathbb{R} \quad \forall x \in D(f): \, x>L \, \Rightarrow \, f(x)>K $;
$\boldsymbol{6.}$
$ \forall K \in \mathbb{R} \quad \exists L \in \mathbb{R} \quad \forall x \in D(f): \, x>L \, \Rightarrow \, f(x)<K $;
$\boldsymbol{7.}$
$ \forall \varepsilon >0  $  $\exists L \in \mathbb{R} \quad \forall x \in D(f), \, x<L : \, \vert f(x) - b \vert  < \varepsilon   $;
$\boldsymbol{8.}$
$ \forall K \in \mathbb{R} \quad \exists L \in \mathbb{R} \quad \forall x \in D(f), \, x<L \, : \, f(x)>K $;
$\boldsymbol{9.}$
$ \forall K \in \mathbb{R} \quad \exists L \in \mathbb{R} \quad \forall x \in D(f), \, x<L \, : \, f(x)<K $;
$\boxed{112.}$
$\boldsymbol{2.}$
ak $ \vert x-1 \vert<\delta, \, x\ne 1 $, tak  $\frac{1}{(1-x)^{2}}> \frac{1}{ \delta ^{2}}$; pre dané $ K \in \mathbb{R} $ stačí položiť napr.  $ \delta = \sqrt{1+ \vert K \vert} $ (definíciu limity používame v takej podobe, ako je v pr. 111.2);
$\boldsymbol{4.}$
ak $x\geq 0 $ a $\vert x-8 \vert < \delta $, tak $\vert \sqrt[3]{x}-2\vert = \frac{\vert x-8 \vert}{\vert \sqrt[3]{x^{2}}+2 \sqrt[3]{x}+4 \vert}< \frac{\delta}{4}  $ (pre $x\geq 0  $ je $ \sqrt[3]{x^{2}}+2 \sqrt[3]{x}+4 \geq 4 $, pre dané $\varepsilon > 0  $ stačí položiť  $\delta = \min \lbrace 4 \varepsilon, 8 \rbrace   $ (podmienka  $\delta \leq 8 $ zaručuje, že pre všetky  $x \in O_{\delta}(8) $ platí nerovnosť  $  x > 0$, ktorú sme potrebovali pri odhadovaní výrazu $ \vert \sqrt[3]{x}-2 \vert $);
$\boldsymbol{5.}$
ak $ x<0 $ a $\vert x+2\vert< \delta $ , tak  $ \vert x^{2}-4 \vert = \vert x-2 \vert \vert x+2 \vert  <2 \delta $; pre dané $\varepsilon > 0$ stačí položiť  $\delta= \min \lbrace \frac{\varepsilon}{2},2 \rbrace $;
$\boxed{113.}$
$\boldsymbol{1.}$
$\exists \varepsilon  > 0 \quad \forall \delta > 0 \quad \exists x \in D(f): \, x \ne 0 \land \vert x \vert < \delta \land \, \vert f(x)-4 \vert > \varepsilon $;
$\boldsymbol{2.}$
$\forall a \in \mathbb {R^{*}} \quad \exists O(a) \quad \forall O_{\delta} (O) \quad \exists x \in D(f), \, x\in O_{ \delta}(O), \, x \ne 0: \, f(x)\notin O(a) $;
$\boxed{114.}$
z nerovnosti $ \vert \vert f(x) \vert - \vert b \vert \vert \leq \vert f(x) - b \vert$ vyplýva: ak pre všetky $ x \in O(a)$ platí $ \vert f(x) - b \vert < \varepsilon $ (pre dané $\varepsilon > 0  $ existenciu takého $O(a) $ zaručuje predpoklad $\lim_{x \to a}f(x)=b$), tak pre všetky $ x \in O(a) $ platí $ \vert \vert f(x) \vert - \vert b \vert \vert < \varepsilon  $; opačná implikácia neplatí (napr. $f(x)= \sgn (x-a) $; neexistenciu limity v bode $a$ môžeme dokázať rovnako ako v pr. 115.2);
$\boxed{115.}$
$\boldsymbol{1.}$
ak pre všetky $ z\in O(a )$ platí $ \vert f(z) - b \vert < \varepsilon $ a ak $ x, y \in O(a ) $, tak $ \vert f(x) - f(y) \vert \leq \vert f(x) - b \vert + \vert b - f(y) \vert < 2\varepsilon $;
$\boxed{116.}$
$\boldsymbol{1.}$
$ \frac{1}{2}$;
$\boldsymbol{2.}$
$0$;
$\boldsymbol{3.}$
$5^{-5}$ (uvedenú limitu  možno napr. zapísať ako súčin piatich limít, z ktorých každá je rovná $ \frac{1}{5}$);
$\boldsymbol{4.}$
$\frac{2}{9}$ (najprv dať na spoločného menovateľa;
$\boldsymbol{5.}$
$x+ \frac{a}{2}$ (limitovaný výraz má tvar $ \frac{1}{n} \quad [(n-1)x+ \frac{a}{n} (1+2+...+(n-1))]= \frac{n-1}{n}x+\frac{a}{n^{2}}. \frac{(n-1)n}{2}$);
$\boldsymbol{6.}$
$\frac{1}{3}$ (zlomok rozšíriť $3^{-n}$ a využiť pr. 105.5);
$\boxed{117.}$
$\boldsymbol{2.}$
$\frac{2}{3}$;	
$\boldsymbol{3.}$
$1$ (stačí dosadiť, daná funkcia je elementárna a definovaná v bode 0);
$\boldsymbol{4.}$
$1$;
$\boldsymbol{5.}$
$0$;
$\boldsymbol{6.}$
$(\frac{3}{2})^{10}$;
$\boldsymbol{7.}$
$\frac{m}{n}$;
$\boldsymbol{8.}$
$-\frac{1}{2}$ (spoločný menovateľ je $ x(x-1)(x-2))$;
$\boldsymbol{9.}$
$\frac{n(n+1)}{2}$ (delenie výrazom $ x^{-1} $ si uľahčíme, ak čitateľ napíšeme v tvare $ (x-1)+(x^{2}-1)+...(x^{n}-1))$;
$\boldsymbol{10.}$
$\frac{49}{24}(x^{100}-2x+1=(x^{100}-1)-2(x-1)$; $x^{50}-2x+1=(x^{50}-1)-2(x-1)$; tento prepis uľahčí delenie výrazom $x-1$);
$\boxed{118.}$
$\boldsymbol{1.,2.}$
napr. $f(x)=\chi (x), \, g(x)=1-\chi (x) $ (tým sme vlastne ukázali, že implikácie  vo vetách o limite súčtu a limite súčinu nemožno obrátiť);
$\boxed{119.}$
áno; napr. $ s_{n}=(-1)^{n}\frac{1}{n}$ možno dokázať, že pre každú postupnosť  $\lbrace a_{n}\rbrace_{n=1}^\infty $ vyhovujúcu pr. 119 platí $\lim_{n \to \infty}a_n =0$);
$\boxed{120.}$
$\boldsymbol{2.}$
$\frac{4}{3}$;
$\boldsymbol{3.}$
$\frac{1}{\sqrt{2a}}$ (= $\lim_{x \to a}( \frac{\sqrt{x-a}}{\sqrt{x+a}(x+a)}+\frac{1}{\sqrt{x+a}})$) ;
$\boldsymbol{4.}$
$-\frac{1}{16}$;
$\boldsymbol{5.}$
$\frac{2}{27}$ na úpravu čitateľa sa použije vzťah $A^{3}-B^{3}=(A-B)(A^{2}+AB+B^{2}) $, v menovateli stačí vyňať pred zátvorku premennú $ x $);
$\boldsymbol{6.}$
$\frac{1}{n!}$;
$\boxed{121.}$
$\boldsymbol{2.}$
$\frac{112}{27}$;
$\boldsymbol{3.}$
$\frac{7}{36}$;
$\boxed{122.}$
$\boldsymbol{1.}$
$\frac{5}{3}$ (subst. $t=\sqrt[15]{x} $);
$\boldsymbol{3.}$
$\frac{n}{m}$ (subst. $t=\sqrt[m.n]{x} $);
$\boldsymbol{4.}$
$\frac{1}{2}$;
$\boxed{122.}$
$\boldsymbol{1.}$
1 (zlomok rozšíriť výrazom $\frac{1}{\sqrt{x}}$; uvedomte si, že pri výpočte limity v čitateli aj v menovateli takto získaného zlomku sa používa veta o limite zloženej funkcie);
$\boldsymbol{2.}$
$\frac{1}{\sqrt{2}}$ (zlomok rozšíriť výrazom $\frac{1}{\sqrt{x}}$;
$\boldsymbol{3.}$
-2 (pre $x>0$ je $\vert x \vert=x $);
$\boldsymbol{4.}$
$\frac{a+b}{2}$ (rozšíriť výrazom $\sqrt{(x+a)(x+b)}+x $);
$\boldsymbol{5.}$
$\frac{2}{3}$;
$\boldsymbol{6.}$
$\frac{a_{1}+...+a_{n}}{n}$;
$\boldsymbol{7.}$
$-\frac{1}{4}$ (rozšíriť výrazom $ x+\sqrt{x^{2}+2x}+2\sqrt{x^{2}+x}$, potom napísať v tvare súčinu tak, aby jeden zo súčiniteľov bol $ \frac{2x}{x+\sqrt{x^{2}+2x}+2\sqrt{x^{2}+x}}$, jeho limitu už vieme vypočítať; z druhého súčiniteľa vyňať pred zátvorku $ x $ potom rozšíriť výrazom $\sqrt{x^{2}+2x}+x+1 $);
$\boldsymbol{8.}$
$\frac{1}{3}$;
$\boxed{125.}$
áno, podkladom pre konštrukciu funkcií $f,g $ môže byť rozbor dôkazov vety o limite zloženej funkcie a tvrdenia z pr. 124; $g $ nesmie spĺňať podmienku (*) z uvedenej vety, pre $f$ nesmie platiť $\lim_{x \to 2}f(x) = f(2)$);
$\boxed{126.}$
$\boldsymbol{1.}$ 5;
$\boldsymbol{2.}$
$\sin 1$ (stačí dosadiť);
$\boldsymbol{3.}$
$\frac{m}{n}$ (rozšíriť výrazom $\frac{1}{x}$);
$\boldsymbol{4.}$ 2 (rozšírte výrazom $x^{3}+2x$ a využite, že  $\lim_{x \to 0}\frac{\sin (x^{3}+2x)}{x^{3}+2x} = 1$ podľa vety o limite zloženej funkcie);
$\boldsymbol{5.}$
$\frac{1}{3}$;
$\boldsymbol{6.}$ $x$ (subst. $t= \frac{x}{2^{n}} $);
$\boxed{127.}$
$\boldsymbol{1.}$
$\frac{1}{2} $ (rozšíriť výrazom $1+ \cos x $ alebo použiť vzorec $1-\cos x =2 \sin ^{2}\frac{x}{2}$);
$\boldsymbol{2.}$
$\frac{1}{2} $;
$\boldsymbol{3.}$ 2 (stačí rozšíriť  výrazom $\frac{1}{x} $);
$\boldsymbol{4.}$
$\frac{1}{2} $;
$\boldsymbol{5.}$ 4 (v čitateli stačí odpočítať a pripočítať číslo 1 a použiť výsledok pr. 127.1);
$\boldsymbol{6.}$
$\frac{1}{p} $ (rozšíriť výrazom $\frac{1}{x} $; využiť, že $\lim_{x \to 0}\frac{1-\cos x}{x} = \lim_{x \to 0}\frac{1-\cos x}{x^{2}}x=0$, pozri pr. 127.1);
$\boxed{128.}$
$\boldsymbol{1.}$ $\cos a $ (použite vzorec $\sin x- \sin a = 2 \sin \frac{x-a}{2} \cos \frac{x+a}{2} $);
$\boldsymbol{2.}$
$-\frac{1}{\sin ^{2}a} $;
$\boldsymbol{3.}$
$- \cos a$;
$\boldsymbol{4.}$
$\frac{3}{2} \sin 2a $ (v čitateli napr. odpočítať a pripočítať výraz $\sin ( a+x) \sin a$);
$\boxed{129.}$
$\boldsymbol{1.}$ 
$\frac{1}{2} $;
$\boldsymbol{2.}$ -3 (subst. $ t= \sin x$);
$\boldsymbol{3.}$
$\frac{3}{4} $;
$\boldsymbol{5.}$
$\frac{2}{\pi} $;
$\boldsymbol{6.}$
1 (subst. $t = \arcsin x $;
$\boxed{130.}$
$\boldsymbol{1.}$ 
$\frac{1}{4} $;
$\boldsymbol{2.}$ 
$\frac{4}{3} $;
$\boldsymbol{3.}$ 
$-\frac{1}{12} $;
$\boldsymbol{4.}$ 
$\sqrt{2} $ (limitovaná funkcia je definovaná len pre $ x>0$, na jej definičnom obore teda platí $x= \sqrt{x^{2}} $);
$\boldsymbol{5.}$ 
$-\frac{1}{20} $;
$\boxed{131.}$
$\boldsymbol{1.}$ 
0 ($\lim_{x \to \infty}  \frac{1}{x}=0$, $\sin  $ je ohroničená funkcia);
$\boldsymbol{2.}$ 
$\frac{1}{2} $ (rozšíriť výrazom $\frac{1}{x^{2}} $);
$\boldsymbol{3.}$ 0;
$\boldsymbol{4.}$ 3;
$\boldsymbol{5.}$
$\frac{1}{2}$;
$\boxed{132.}$
$\boldsymbol{2.}$ 
$\frac{n}{2^{n}}= \frac{n}{(3/2)^{n}}.(3/4)^{n}\leq (3/4)^{n}$;
$\boldsymbol{3.}$
$( \frac{n}{\sqrt[3]{25}^{n}})^{3/2}\leq (\frac{n}{2^{n}})^{3/2}$ (rovnosti z pr. 132 možno dokázať aj na základe pr. 133.1;
$\boxed{133.}$
$\boldsymbol{1.}$ 
 $ 0 \leq a_{n} \leq \frac{a_{1}}{q}.q^{n}  $;
$\boldsymbol{2.}$ neplatí, stačí zvoliť  $ a_{n}= \frac{n+1}{2n}$;
$\boldsymbol{3a/.}$ $0$ (použiť pr. $133.1$ a nerovnosť $(1+ \frac{1}{n})^{n}\geq 2) $;$\boldsymbol{3b/.}$ $0$ (treba si uvedomiť, že tvrdenie z príkladu $133.1$ zostane v platnosti, aj keď $\frac{a_{n+1}}{a_{n}}\leq q$ bude platiť len pre všetky $n\geq K$, kde $K\in \mathbb{N}$ je dané);
$\boxed{134.}$
$\boldsymbol{1.}$ $0$ $((\frac{7}{n})^{n}\leq (\frac{7}{8})^{n}$ pre $n\geq 8$);$\boldsymbol{2.}$ $0$;
$\boxed{135.}$
$\boldsymbol{2.}$
pre $\omega(n):=\sqrt[n]{n}-1$ platí $0\leq \omega(n) \leq \sqrt{\frac{2}{n}}$;
$\boxed{136.}$
$\boldsymbol{1.}$  $0$ (rozšíriť výrazom $\frac{1}{n!}$);
$\boldsymbol{2.}$  $1$ (pre $n\geq 19 $ je $\sqrt[n]{4}\leq \sqrt[n]{\frac{5n+1}{n+5}} \leq \sqrt[n]{6} $; alebo všeobecnejšie: $\lim_{n\rightarrow\infty} \frac{5n+1}{n+5} =5$, preto pre dané $\varepsilon \in (0,5) $ existuje $O (\infty )$, v ktorom platí $5- \varepsilon \leq \frac{5n+1}{n+5} \leq 5+ \varepsilon $, a teda aj $\sqrt[n]{5- \varepsilon} \leq  \sqrt[n]{\frac{5n+1}{n+5}}\leq \sqrt[n]{5+ \varepsilon} $)
$\boldsymbol{3.}$  
$3$ $ (3^{n}-2^{n} = (3-2).(3^{n-1}+3^{n-2}.2+...+2^{n-1}) 3^{n-1}$; iná možnosť: $ \sqrt[n]{3^{n}-2^{n}}= 3 \sqrt[n]{1-(\frac{2}{3})^{n}}$, pritom $\lim_{n\rightarrow\infty} (1- ( \frac{2}{3})^{n})=1$, preto pre dané $\varepsilon \in (0,1) $  existuje $O( \infty )$, v ktorom platí $1- \varepsilon \leq 1- (\frac{2}{3})^{n} \leq 1+ \varepsilon $, a teda aj $ \sqrt[n]{1- \varepsilon } \leq \sqrt[n ]{1- (\frac{2}{3})^{n}} \leq \sqrt[n]{1+ \varepsilon }$; ďalšou možnosťou výpočtu limít z pr. 136.2-4. je použitie postupov uvedených v odseku 2.5);
$\boldsymbol{4.}$  $1 $  ($\frac{1}{2^{n}}\leq \frac{1}{2n} $, iná možnosť:  $\sqrt[n]{\frac{1}{n}-\frac{1}{2^{n}}}= \sqrt[n]{\frac{1}{n}}\sqrt[n]{1-\frac{n}{2^{n}}} $, ďalej využite analogicky ako v pr. 136.2,3 fakt, že $\lim_{n\rightarrow\infty} (1-\frac{n}{2^{n}})=1$);
$\boxed{137.}$
$\boldsymbol{1.}$  $-\infty$  ($\lim_{x\rightarrow-\infty} \sqrt{(x+a)(x+b)}=+\infty$, $\lim_{x\rightarrow-\infty} -x = + \infty$, preto $\lim_{x\rightarrow-\infty}( \sqrt{(x+a)(x+b)}-x)=+\infty$;
$\boldsymbol{2.}$  $-\infty$;
$\boldsymbol{3.}$  $+\infty$;
$\boldsymbol{4.}$  $+\infty$ (rozšíriť výrazom $\frac{1}{x^{2}} $; $x^{2}-\frac{5}{x}\rightarrow \, +\infty $, $1/(1-\frac{3}{x}+\frac{1}{x^{2}}) \rightarrow \, 1 $ pre  $x\rightarrow -\infty $);
$\boldsymbol{5.}$  $+\infty$;
$\boldsymbol{6.}$  $+\infty$;
$\boldsymbol{7.}$  $+\infty$ ($2+ \sin x \geq 1 $ pre $x \in \mathbb{R} $, preto $(2+ \sin x).x\geq x $ pre $x\geq0 $);
$\boxed{138.}$
$\boldsymbol{1.}$  $+\infty$;
$\boldsymbol{2.}$  $+\infty$ ; ($\frac{\sin x}{x}\rightarrow 1$,  $\frac{1}{x^{2}}\rightarrow +\infty$ pre $x\rightarrow 0 $);
$\boldsymbol{3.}$  $+\infty$ (rozšíriť výrazom $\frac{1}{x^{2}} $  alebo $\frac{1}{x^{4}} $);
$\boldsymbol{4.}$  $+\infty$;
$\boldsymbol{5.}$  $+\infty$;
$\boldsymbol{6.}$  $+\infty$;
$\boxed{139.}$  $\boldsymbol{3.}$ napr.  $f(x)= \frac{1}{(x-1)^{2}}+ \chi (x) $, $g(x)= -\frac{1}{(x-1)^{2}} $;
$\boxed{140.}$
$\boldsymbol{4.}$ napr. $a_{n}=n, \, b_{n}=(-1)^{n}. \frac{1}{n} $ ( príklady 139 a 140 ukazujú, že možno sformulovať všeobecné tvrdenia, ktoré by umožňovali výpočet limít neurčitých výrazov $ +\infty$ $ -\infty $, resp. $0.(+\infty) $);
$\boxed{141.}$ 
$\lim_{x\rightarrow\infty} R(x) =$ $ f(n)=\left\{\begin{matrix} +\infty, & \mbox{ak }n>m, \frac{a_{0}}{b_{0}}>0 \\ -\infty, & \mbox{ak }n>m, \frac{a_{0}}{b_{0}}<0 \\ \frac{a_{0}}{b_{0}} & \mbox {ak }n=m, \\ 0 & \mbox {ak }n<m\end{matrix}\right.$
najprv dokážte - napr. matematickou indukciou - že $\lim_{x\rightarrow-\infty}(c_{0}x^{s}+c_{1}x^{s-1}+...+c_{s})=$ 
$\left\{\begin{matrix} +\infty, & \mbox{ak } c_{0}>0 \\ - \infty, & \mbox{ak} c_{0}<0\end{matrix}\right. $);
$\boxed{142.}$
$\boldsymbol{1.}$
$\forall \varepsilon >0 \quad \exists \delta >0 \quad \forall x \in D(f): \, 0< x-a < \delta \, \Rightarrow \vert f(x) - b \vert < \varepsilon$;
$\boldsymbol{2.}$
$\forall \varepsilon >0 \quad \exists \delta >0 \quad \forall x \in D(f): \, -\delta< x-a < 0 \, \Rightarrow \vert f(x) - b \vert < \varepsilon$ (pritom predpokladáme, že $a$ je hromadný bod množiny $ D(f) \cap (a, \infty )$ resp. množiny $D(f) \cap (-\infty, a)$);
$\boxed{143.}$
$\boldsymbol{1.}$
$\lim_{x\rightarrow 1^{+}} f(x) =2$, $\lim_{x\rightarrow 1^{-}} f(x) =-2$;
$\boldsymbol{2.}$
$\lim_{x\rightarrow 0^{+}} f(x) =\sqrt{2}$, $\lim_{x\rightarrow 0^{-}} f(x) =-\sqrt{2}$ ($1-\cos 2x = 2\sin ^{2}x $);
$\boldsymbol{3.}$
$\lim_{x\rightarrow 2^{+}} f(x) =\infty$, $\lim_{x\rightarrow 2^{-}} f(x) =-\infty$;
$\boldsymbol{4.}$
$\lim_{x\rightarrow 0^{+}} f(x) =0$, $\lim_{x\rightarrow 0^{-}} f(x) =\frac{1}{2}$;
$\boxed{144.}$
$\boldsymbol{1.}$
$\lim_{x\rightarrow \pi/2^{+}} tg x =-\infty \ne +\infty $ $=\lim_{x\rightarrow \pi/2^{-}} tg x$, preto $\lim_{x\rightarrow \pi/2} tg x$ neexistuje;
$\boldsymbol{2.}$
$\lim_{x\rightarrow 0}x \sgn x = 0$ (možno dokazovať pomocou jednostranných limít alebo využiť fakt, že $sgn$ je ohraničená funkcia a $\lim_{x\rightarrow 0}x  = 0$;
$\boldsymbol{3.}$ neexistuje ( $ \frac{\sin x}{x^{2}} = \frac{\sin x}{x} . \frac{1}{x}$);
$\boxed{146.}$
$\boldsymbol{2.}$
za daných predpokladov $\lim_{x\rightarrow 0} f(x)$ existuje práve vtedy, keď $\lim_{x\rightarrow 0^{+}} f(x)=0$;
$\boxed{149.}$
Nech sú dané funkcie $f,g $, nech $b \in \mathbb{R^{*}} $ je hromadný bod množiny $D(f) \cap D(g) $, nech $\lim_{x\rightarrow b} f(x)=a$, $\lim_{x\rightarrow b} g(x)=A$. Potom: $\boldsymbol{1.}$ ak $a \in (0,1),\, A= +\infty  $, tak 
$\lim_{x\rightarrow b} f(x)^{g(x)}=0$;
$\boldsymbol{2.}$ ak $a \in (0,1),\, A= -\infty  $, tak $\lim_{x\rightarrow b} f(x)^{g(x)}=+\infty$;
$\boldsymbol{3.}$ ak $a \in (1,+\infty)\cup \lbrace +\infty \rbrace,\, A= +\infty  $, tak $\lim_{x\rightarrow b} f(x)^{g(x)}=+\infty$; 
$\boldsymbol{4.}$ ak $a \in (1,+\infty)\cup \lbrace +\infty \rbrace,\, A= -\infty  $, tak $\lim_{x\rightarrow b} f(x)^{g(x)}=0$ (pri dôkazoch sa stačí obmedziť na také prstencové okolie bodu $b$, v ktorom je $f$ kladná (také okolie existuje, pretože $\lim_{x\rightarrow b} f(x)$ je kladné číslo alebo bod $+ \infty$)); 
$\boxed{150.}$
$\boldsymbol{1.}$  $\sqrt{\frac{2}{3}} $;
$\boldsymbol{2.}$  $\frac{1}{4} $;
$\boldsymbol{3.}$  $+\infty $;
$\boldsymbol{4.}$  $\frac{1}{2} $;
$\boldsymbol{5.}$  $0 $;
$\boldsymbol{6.}$  $0 $;
$\boldsymbol{7.}$  $+ \infty $;
$\boldsymbol{8.}$ neexistuje (limita sprava (zľava) je $+\infty, (0) $);
$\boxed{151.}$
$\boldsymbol{1.}$  $e^{3} $;
$\boldsymbol{2.}$  $e^{-2} $;
$\boldsymbol{3.}$  $e^{3} $;
$\boldsymbol{4.}$  $e^{-1} $;
$\boldsymbol{5.}$  $e^{ctg a} $;
$\boldsymbol{6.}$  $1 $;
$\boldsymbol{7.}$  $e $ (možno použiť substitúciu $\frac{1}{x}=t $, potom $t\rightarrow 0^{+} $ pre $x \rightarrow +\infty $);
$\boldsymbol{8.}$  $e^{-x^{2}/2} $;
$\boxed{152.}$
$\boldsymbol{1.}$  $1 $; (= $\lim_{x\rightarrow 0} ln[(1+x)^{1/x}  ] $; použije sa veta z pr. 124);
$\boldsymbol{2.}$  $1 $;
$\boldsymbol{3.}$  $\frac{1}{a} $; 
$\boldsymbol{5.}$  $0 $;
$\boldsymbol{6.}$  $n $ (využite, že $\lim_{x\rightarrow 0} \frac {ln [1+(x+ \sqrt{1-x^{2}}-1)]}{x+\sqrt{1-x^{2}}-1} =1$, analogický vzťah použite pre $ln (nx+ \sqrt{1-n^{2}x^{2}}) $);   
$\boxed{153.}$
$\boldsymbol{1.}$  $1 $ (možno použiť substitúciu  $e^{x}-1=t  $);
$\boldsymbol{2.}$  $1 $;
$\boldsymbol{3.}$  $4. ln \frac{2}{e} $ (v čitateli možno pripočítať a odpočítať číslo $4$, potom previesť na súčet dvoch limít);
$\boldsymbol{5.}$  $\frac{25}{4} $;
$\boldsymbol{6.}$  $1 $ (v čitateli možno pripočítať a odpočítať číslo $1$, potom zlomok rozšíriť výrazom $\frac{1}{x}$);
$\boldsymbol{7.}$  $ln x $;
$\boldsymbol{8.}$  $e^{2} $;
$\boxed{154.}$
$\boldsymbol{1.}$  $\frac{1}{3} $;
$\boldsymbol{3.}$  $\frac{1}{2} $;
$\boxed{153.}$
$\boldsymbol{1.}$  $a_{n+1}=(1-\frac{1}{2^{n+1}}) a_{n} < a_{n} \, (n \in \mathbb{N}) $, teda $ \lbrace a_{n} \rbrace_{n=1}^\infty $ je klesaajúca postupnosť; $ \forall n \in \mathbb{N}: \, a_{n} 	>0 $, teda $ \lbrace a_{n} \rbrace_{n=1}^\infty $  je zdola ohraničená; preto existuje konečná  $\lim_{n\rightarrow \infty} a_{n}  $(= $\inf \lbrace a_{n} ; n \in \mathbb{N}\rbrace $);
$\boldsymbol{2.}$ $ \lbrace a_{n} \rbrace_{n=1}^\infty $ je rastúca; z nerovnosti  $n!\geq 2^{n-1}  $ ($n \in \mathbb{N}  $) vyplýva  $a_{n}\leq 1+1+\frac{1}{2}+...+\frac{1}{2^{n-1}} = 2+1-\frac{1}{2^{n-1}} < 3$;
$\boldsymbol{3.}$ $\lbrace a_{n}\rbrace_{n=1}^{\infty}  $  je rastúca $a_{n+1}-a_{n}>0  $) a zhora ohraničená ( $a_{n} < 1,\, n \in \mathbb{N} $);
$\boldsymbol{4.}$ $\lbrace a_{n}\rbrace_{n=1}^{\infty}  $  je zhora ohraničená ( $a_{n} < \frac{1}{5}+...+ \frac{1}{5^{n}} = \frac{1}{4}(1- \frac{1}{5^{n}} < \frac{1}{4}$) a rastúca;
$\boldsymbol{5.}$ pre $n>10 $ je $a_{n}< a_{n-1} $, $\lbrace a_{n}\rbrace_{n=10}^{\infty} $ je klesajúca a zdola ohraničená  ($ \forall n \in \mathbb{N}: \, a_{n} 	>0 $);
$\boxed{156.}$
$\boldsymbol{2.}$ ak si najprv pomôžeme grafmi funkcií $y=\frac{1}{2}(x^{2}+1) $  a   $y=x $ nakreslenými do jedného obrázka (podobne ako obr. 2), zistíme, že postupnosť $\lbrace a_{n}\rbrace_{n=10}^{\infty} $ je rastúca a zhora ohraničená (napr. číslom  $ 1 $); pretože však i najlepšie nakreslený obrázok nieje dôkazom, treba obidve uvedené tvrdenia ešte dokázať; hľadaná limita vyhovuje rovnici  $ 2a=a^{2}+1 $, preto $\lim_{n\rightarrow \infty} a_{n} =0$ (druhý z koreňov rovnice $x= \frac{x}{2+x} $, t.j. číslo $-1$ nemôže byť hľadanou limitou, pretože platí: $\lim_{k\rightarrow \infty} a_{2k} = \lim_{k\rightarrow \infty} a_{2k-1}= \frac{1}{2}(1+\sqrt{5})$, preto $\lim_{n\rightarrow \infty} a_{n} = \frac{1}{2}(1+\sqrt{5})$;
$\boldsymbol{5.}$ pre $x=k \pi $  ($k \in \mathbb{Z}$) je postupnosť  $\lbrace a_{n}\rbrace_{n=1}^{\infty}$ konštantná, pre  $ x \in \bigcup\limits_{k \in \mathbb{Z}}2k \pi, (2k+1)\pi) $ klesajúca a zdola ohraničená, pre  $ x \in \bigcup\limits_{k \in \mathbb{Z}}(2k-1) \pi, 2k\pi) $ rastúca  a zhora ohraničená (pri dôkazoch monotónnosti využite platnosť nerovností  $0 < \sin x < x  $ pre  $x \in (0,\pi)  $, $x < \sin x < 0  $ pre  $x \in (-\pi,0)  $), $\lim_{n\rightarrow \infty}a_{n}=0$;
$\boxed{157.}$  $\frac{a_{n+1}}{a_{n}}= \frac{n+1}{n}. \frac{1}{(1+\frac{1}{n(n+2)})^{n+2}}<1$ (na odhad menovateľa v druhom súčiniteli použite nerovnosť z pr. 10.1), teda $\lbrace a_{n}\rbrace_{n=1}^{\infty }$ je klesajúca postupnosť (fakt, že $\lbrace (1+\frac{1}{n})^{n}\rbrace_{n=1}^{\infty }$ je rastúca postupnosť - ktorý pokladáme za známy z prednášok - možno dokázať podobne); $e=\lim_{n\rightarrow \infty} a_{n} = \inf \lbrace a_{n}; n\in \mathbb{N}\rbrace$;
$\boxed{159.}$ $\boldsymbol{1.}$ nie je (platí len implikácia "$\Rightarrow $ "; protipríkladom, dokumentujúcim neplatnosť implikácie "$\Leftarrow $ ", je napr. Dirichletova funkcia $\chi (x) $);
$\boldsymbol{2.}$ je;
$\boxed{160.}$ (H označuje množinu všetkých hromadných hodnôt postupnosti $\lbrace a_{n}\rbrace_{n=1}$; $a$ resp. $b$ označujú $\lim_{n\rightarrow \infty} \inf a_{n}$, resp. $\lim_{n\rightarrow \infty} \sup a_{n}$; v pr. 160.1-6 použijeme túto únahu: ak $\lbrace a_{k_{1}}\rbrace_{n=1}^{\infty}$,..., $\lbrace a_{k_{1}}\rbrace_{n=1}^{\infty}$ sú podpostupnosti $\lbrace a_{n}\rbrace_{n=1}^{\infty}$ také, že: a/ každá z nich má limitu; b/ množiny  $N_{1}:=\lbrace k_{1} (n); \, n \in \mathbb{N} \rbrace ,...,N_{1}:=\lbrace k_{1} (n); \, n \in \mathbb{N} \rbrace  $ sú po dvoch disjunktné a $N_{1}\cup ... \cup N_{1} = N $ (t.j. ak sa podarí postupnosť $\lbrace a_{n}\rbrace_{n=1}^{\infty}$ "rozdeliť" na konečný počet podpostupností, z ktorých každá má limitu), tak H pozostáva z limít postupností $\lbrace a_{k_{1}}\rbrace_{n=1}^{\infty}$,..., $\lbrace a_{k_{1}}\rbrace_{n=1}^{\infty}$;
$\boldsymbol{1.}$  $\lim_{k\rightarrow \infty} a_{2k}=-2$, $\lim_{k\rightarrow \infty}a_{2k-1}=2$, $ H= \lbrace -2,2\rbrace $, $a=-2, \, b=2 $;
$\boldsymbol{2.}$   $ H= \lbrace -4,0,2,6\rbrace $, $a=-4, \, b=6 $ ( $\lim_{k\rightarrow \infty} a_{4k} = 2$, $\lim_{k\rightarrow \infty} a_{4k+1} = 6$, $\lim_{k\rightarrow \infty} a_{4k+2} = -4$, $\lim_{k\rightarrow \infty} a_{4k+3} = 0$,  
$\boldsymbol{3.}$   $ H= \lbrace -1,-\frac{1}{2},0,\frac{1}{2},1\rbrace $, $a=-1, \, b=1 $;
$\boldsymbol{4.}$   $ H= \lbrace -\frac{1}{\sqrt{2}}-e,\frac{1}{\sqrt{2}}-e,e-1,e,e+1\rbrace $, $a=-\frac{1}{\sqrt{2}}-e, \, b=e+1 $;
$\boldsymbol{5.}$   $ H= \lbrace 0,\infty \rbrace $, $a=0, \, b=+\infty $;
$\boldsymbol{6.}$   $ H= \lbrace 1,2\rbrace $, $a=1, \, b=2 $;
 $\boldsymbol{7.}$ , $\boldsymbol{8.}$ $a=0,\, b=1$; o čísle $0$, resp. $1$ možno totiž dokázať, že vyhovuje obidvom podmienkam z vety 18 (v pr. 160.7,8 možno dokázať, že $ H= \langle0,1\rangle $);
$\boxed{161.}$ $\boldsymbol{4.}$  napr. $\lbrace 1,2,1,2,3,1,2,3,4,1,2,3,4,5,... \rbrace $;
$\boxed{163.}$ (skôr než začnete dokazovať, by ste si mali uvedomiť, že platí  $a,b \in \mathbb{R}, \, a<b$) sporom: keby $N_{\varepsilon}^{*} $ bola nekonečná pre niektoré $\varepsilon^{*} \in (0, \frac{b-a}{2} $), existovala by hromadná hodnota $c$ postupnosti $\lbrace a _{n} \rbrace_{n=1^{\infty}}  $, $c \in \langle a+\varepsilon^{*},b-\varepsilon^{*} \rangle $ (teda $c\ne a, \, c\ne b $);
$\boxed{164.}$ stačí dokázať: ak $a,b \in \mathbb{R}, \, a\leq b $, pričom platí $\forall \varepsilon >0: \, b< a+\varepsilon $, tak   $a=b $ (dôkaz možno vykonať sporom), tvrdenie pr. 164 je potom dôsledkom tohoto a vety 19	
$\boxed{165.}$ $\boldsymbol{1.}$ v $O(a) $ ( $O(a)$ je ľubovolné okolie bodu $a$) leží aspoň jeden prvok $z \in A $; existuje také okolie $O(z) $ prvku $z$, že $O(z) \subset  O(a) $ v $O(z) $ leží (pretože $z \in A $) nekonečne veľa členov postupnosti $\lbrace a_{n}\rbrace_{n=1}^{\infty} $, teda pre každé  $O(a) $ je množina $\lbrace n \in \mathbb{N}; \, a_{n}\in O(a)  \rbrace $ nekonečná;
$\boldsymbol{2.}$ nie; $A$ neobsahuje prvok $0$, čo je v spore s pr. 165.1;
$\boxed{166.}$ $\boldsymbol{1a/}$ postupnosť $\lbrace a _{n(k)}+b_{n(k)} \rbrace_{k=1}^{\infty}  $ vybraná z postupnosti  $\lbrace a _{n} + b_{n} \rbrace_{n=1}^{\infty}$ konverguje práve vtedy, keď konverguje postupnosť $\lbrace b _{n(k)} \rbrace_{k=1}^{\infty}$; preto množinu  $ H $ hromadných hodnôt postupnosti $\lbrace a _{n} + b_{n} \rbrace_{n=1}^{\infty}$ možno písať v tvare $H = a+h;\, h \in H_{1}$, kde  $a:=\lim_{n\rightarrow \infty} a_{n}$, $H_{1}$ je množina hromadných honôt postupnosti  $\lbrace  b_{n} \rbrace_{n=1}^{\infty}$; 
$\boldsymbol{1b/}$ dôkaz prvej nerovnosti možno založiť na tejto úvahe: ak pre postupnosť  $\lbrace  c_{n} \rbrace_{n=1}^{\infty}$ a číslo $c\in \mathbb{R}$ platí: pre každé $\varepsilon > 0 $ je množina  $\lbrace n\in \mathbb{N}; \, c_{n}\leq c- \varepsilon \rbrace$ konečná, tak  $ c \leq \lim_{n\rightarrow \infty} c_{n}$ (porovnaj s vetou 18); stačí teda ukázať, že číslo $ \lim_{n\rightarrow \infty} a_{n + \lim_{n\rightarrow \infty} b_{n}}$ vyhovuje tejto podmienke; posledná nerovnosť sa dokazuje analogicky;
$\boldsymbol{2.}$ napr. $\lbrace a_{n} \rbrace_{n=1}^{\infty} = \lbrace 0,1,2,3,0,1,2,3,...\rbrace $, $\lbrace a_{n} \rbrace_{n=1}^{\infty} = \lbrace 1,2,0,1,1,2,0,1,1,2,0,1...\rbrace $;
$\boxed{168.}$ nech je dané $O(a)$; ak $b \in (O(a)-\lbrace a \rbrace )\cap A^{)} $, tak existuje také $O(b)$, že $O(b) \subset O(a)-\lbrace a \rbrace$; v $O(b)$ leží aspoň jeden prvok z $A$;
$\boxed{169.}$ neexistuje, vyplýva to z pr. 168;
$\boxed{170.}$ doporučujeme dokazovať nepriamo, pokusy o priamy dôkaz radšej konzultujte s privolaným odborníkom;
$\boxed{171.}$ neplatnosť obrátenej implikácie dokumentuje postupnosť $a_{n}=(-1)^{n}$;
$\boxed{172.}$ (využite pr. 171); $\boldsymbol{1.}$ $0$;
$\boldsymbol{2.}$ $0$;
$\boxed{173.}$ $\boldsymbol{1.}$  ak $N= \max \lbrace m \in \mathbb{N}; \, a_{m} \leq K\rbrace$; ( $K > 1$ je dané), tak pre všetky $n \in \mathbb{N}, \, n>N $ platí $a_{n} \leq K $;
$\boldsymbol{2.}$ vyplýva to z pr. 173.1 a vety o limite zloženej funkcie;
$\boxed{174.}$ $\boldsymbol{1.}$ nie; $\boldsymbol{2.}$ áno;
$\boxed{175.}$ $\boldsymbol{1.}$ ak $\lim_{n \to \infty} \vert a_n \vert =0 $  tak aj $\lim_{n \to \infty}  a_n =0 $;
$\boldsymbol{2.}$ nepriamo, keby napr. množina $N^{+}$ bola konečná, vyplývala by z existencie $\lim_{n \to \infty} \vert a_n \vert =b $ existencia $\lim_{n \to \infty}  a_n =-b $;
$\boldsymbol{3.}$ nepriamo, keby $\mathbb{N}-(N^{+}\cup N^{-}) $ bola nekonečná, platilo by $\lim_{n \to \infty} \vert a_n \vert =0 $;
$\boxed{177.}$ napr. $f(x)=x $ pre $x \in \mathbb{Q} $,  $f(x)=-x $ pre $x \notin \mathbb{Q} $;
$\boxed{178.}$ sporom, ak $\sin n $ konverguje, tak $\lim_{n \to \infty} (\sin (n+2)-\sin n) =0 $, odtiaľ $\lim_{n \to \infty} \cos (n+1) =0 $; ak zo súčtového vzorca pre $\cos (n+1)$ vyjadríme $\sin n$, dostaneme $\lim_{n \to \infty} \sin n = \lim_{n \to \infty} \cos n =0 $, čo je v spore zo vzťahom $\sin^{2} n + \cos ^{2} n =1$;
neexistencia nevlastnej limity vyplýva z ohraničenosti postupnosti $\lbrace \sin n \rbrace _{n=1} ^{\infty}$;
$\boxed{179.}$ $\boldsymbol{1.}$ $\frac{1}{2} m.n.(n-m)$ (skontrolujte si, či Vami použitý postup je správny, keď $m=1$ alebo $n=1$);
$\boldsymbol{2.}$ $\frac{n.(n-1)}{2} a^{n-2}$ (možno použiť subst. $x-a=t$ ; je použitý postup správny pre $n=1$?);
$\boldsymbol{3.}$ $1 $;
$\boldsymbol{4.}$ $\frac{m-n}{2} $ (skontrolujte si správnosť použitého postupu, ak  $m=1$ alebo $n=1$);
$\boldsymbol{5.}$ $x^{2}+ax+a^{2}/3 $;
$\boldsymbol{6.}$ $n^{-n(n+1)/2}$ (skontrolujte si správnosť použitého postupu pre párne $n$, vtedy totiž exponent v menovateli nie je celočíselný);
$\boldsymbol{7.}$ $+\infty$;
$\boldsymbol{8.}$ $ \frac{2}{3}$( $\lim_{n \to \infty} \frac{2\frac{n(n+1)}{2}-n}{3\frac{n(n+1)}{2}-2n} $); 
$\boldsymbol{9.}$ neexistuje ($a_{2k}=-\frac{1}{2}$, $a_{2k-1}=1-\frac{k-1}{2k-1}$, $k\in \mathbb{N}$, kde $a_{n}=\frac{1-2+3-4+...+(-1)^{n-1}.n}{n}$);
$\boldsymbol{10.}$ $ \frac{3}{2}$;
$\boxed{180.}$ ak $\lim_{n \to \infty} a_{n}=a \in \mathbb{R} $, tak $\lbrace a_{n} \rbrace_{n=1} ^{\infty}$ je ohraničená, existuje teda $ B:= \sup_{n\in \mathbb{N}}a_{n} $, $A:= \inf_{n\in \mathbb{N}}a_{n} $, pritom $A \leq a\leq B $; predpokladajme $ A < a $, potom $ \exists N \in \mathbb{N} $  $ \forall n>N: \, a_{n}>A $ (stačí zvoliť $\varepsilon= \frac{a-A}{2} $ a použiť definíciu limity), potom najmenšie z čísel $a_{1},...,a_{N} $ musí byť rovné $A $; podrobné vypracovanie ako aj ostatné prípady prenechávame čitateľovi;
$\boxed{181.}$ R má limity v bodoch $+\infty$, $-\infty$; ak $\lim_{x \to \infty} R(x)=+\infty$,   $\lim_{x \to -\infty} R(x)=-\infty$ (ostatné prípady urobte sami), tak  $\exists a_{1} \in \mathbb{R} \, \forall x < a_{1}$, $x\in D(R):\, R(x)>1$, pritom $a_{1}< a_{2} $; ak má platiť  $ R(\mathbb{Z})=\mathbb{Q} $, musí byť $R( \mathbb{Z} \cap \langle a_{1}, a_{2}\rangle$) = $ \mathbb{Q} \cap \langle-1,1 \rangle$, čo je nemožné (množina vľavo je konečná, množina vpravo nekonečná);
$\boxed{182.}$ $\boldsymbol{1.}$ $ \frac{\alpha}{m} -\frac{\beta}{n}$ (stačí nájsť "vzorec" pre $ \lim_{x \to 0}\frac{\sqrt[k]{1+\vartheta x}-1}{x}$, $k\in \mathbb{Z}- \lbrace 0 \rbrace $);
$\boldsymbol{2.}$ $ \frac{\alpha}{m} -\frac{\beta}{n}$ ( v čitateli stačí pripočítať a odpočítať výraz  $\sqrt[m]{1+\alpha_{x}}$);
$\boldsymbol{3.}$ $1 $;
$\boldsymbol{4.}$ $-\frac{1}{2} $; 
$\boldsymbol{5.}$ $-\frac{\pi}{6} $ (je vhodné použiť modifikáciu vety o limite zloženej funkcie z pr. 124);
$\boldsymbol{6.}$ $5$;
$\boldsymbol{7.}$ $-\frac{1}{4} $;
$\boldsymbol{8.}$ $2 $ (stačí pripočítať a odpočítať $x$);
$\boldsymbol{9.}$ $2^{n}$ ($= \lim_{n \to \infty}((\frac{x-\sqrt{x^{2}-1}}{x})^{n} + (\frac{x+\sqrt{x^{2}-1}}{x})^{n}) $;
$\boxed{184.}$ $\boldsymbol{1.}$ $-\frac{9}{2^{7}} $; 
$\boldsymbol{2.}$ $\sin a $; 
$\boldsymbol{3.}$ $\sqrt{3} $ (pozri pr. 128.4); 
$\boldsymbol{4.}$ $\frac{2}{3\pi} $ (využite, že   $\lim_{u \to 0}\frac{\arctan u}{u}=1$; 
$\boldsymbol{5.}$ $+\infty $;
$\boldsymbol{6.}$ $3 $;
$\boldsymbol{7.}$ $-24 $;
$\boldsymbol{8.}$ $\sqrt{2} $ (ak použijeme subs. $\arccos (1-x)=t$, musíme si uvedomiť, že pre $x\rightarrow 0$ platí $t\rightarrow 0^{+}$, stačí teda hľadať $\lim_{t \to 0^{+}}\frac{t}{\sqrt{1-\cos t}}$;  $\lim_{t \to 0}\frac{t}{\sqrt{1-\cos t}}$ totiž neexistuje);
$\boldsymbol{9.}$  $+\infty $
$\boldsymbol{10.}$  $\frac{3}{2} $;
$\boldsymbol{11.}$  $\frac{5}{8}\sqrt{2}\pi ^{2}$;
$\boldsymbol{12.}$  $-\infty $;
$\boldsymbol{13.}$  $0 $;
$\boldsymbol{14.}$  $0 $;
$\boldsymbol{15.}$  $0 $;
$\boxed{185.}$ $\boldsymbol{2.}$  $0 $ (  $\lim_{n \to \infty}\sqrt[n]{a_{n}}= \frac{1}{\sqrt{2}} $, pre vhodne zvolené  $\varepsilon > 0 $ preto platí $\exists n_{0} \in \mathbb{N}, \, n>n_{0} $: $\sqrt[n]{a_{n}}<\frac{1}{\sqrt{2}}+\varepsilon < 1$); 
$\boxed{186.}$ $\boldsymbol{1,2.}$ možno využiť pr. 133.1;
$\boldsymbol{3.}$ možno použiť nerovnosť $n!>(\frac{n}{3})^{n} $ iná možnosť: pre $i=1,....,n $ platí $i.(n-i+1)\geq n $, odtiaľ $(n!)^{2}\geq n^{n} $;
$\boxed{187.}$ $\boldsymbol{1.}$  $1 $ (použite výsledok pr. 135);
$\boldsymbol{2.}$  $\max \lbrace a,b \rbrace $;
$\boldsymbol{3.}$  $3 $;
$\boldsymbol{4.}$  $11 $;
$\boldsymbol{5.}$  $0 $ (možno použiť pr. 133 a fakt, že $\lim_{n \to \infty}(1+ \frac{1}{n})^{n}=e $);
$\boldsymbol{6.}$  $+\infty $;
$\boldsymbol{7.}$  $+\infty $;
$\boldsymbol{8.}$  $\frac{1}{3} $ (využite pr. 186.1);
$\boldsymbol{9.}$  $1$;
$\boldsymbol{10.}$  $-\infty $;
$\boxed{188.}$ správny je postup v bode a/;
$\boxed{189.}$ Nech je daná funkcia $g$ a kladná funkcia $f$, nech $b$ je hromadný bod množiny $D(f)\cap D(g)$, nech $\lim_{x \to b} f(x)=0$. Potom: a/ ak $\lim_{x \to b} g(x)=+\infty$, tak $\lim_{x \to b} f(x)^{g(x)}=0$; b/ ak $\lim_{x \to b} g(x)=-\infty$, tak $\lim_{x \to b} f(x)^{g(x)}=+\infty$;
$\boxed{190.}$ $\boldsymbol{1.}$  $0$;
$\boldsymbol{2.}$  $e^{2}$;
$\boldsymbol{3.}$  $+\infty$;
$\boldsymbol{4.}$  $e^{-1/2}$;
$\boldsymbol{5.}$  $0$;
$\boldsymbol{6.}$  $0$;
$\boldsymbol{7.}$  $e^{(\beta^{2}-\alpha^{2}})/2$; (prípad $\alpha=\beta$ treba robiť samostatne;
$\boldsymbol{8.}$ neexistuje;
$\boldsymbol{9.}$  $0$ pre $a_{1}>a_{2}$, $e^({b_{1}-b_{2})/a_{1}}$ pre $a_{1}=a_{2}$ (prípad $a_{1}=a_{2}$, $b_{1}=b_{2}$ treba robiť samostatne);
$\boldsymbol{10.}$  $e^{-1}$;
$\boldsymbol{11.}$  $e^{-(a+b)}$ (prípady $a=0,\, b=0$ sa robia zvlášť);
$\boxed{191.}$ $\boldsymbol{1.}$  $- \ln 2$ (pre $x>0 \, x\ne 1$ je $\log_{x}2= \frac{1}{\log_{2}x}$;
$\boldsymbol{2.}$  $\ln 8$;
$\boldsymbol{3.}$  $\frac{1}{8}$;
$\boldsymbol{4.}$  $\alpha a^{\alpha}-1$ pre $\alpha\ne 0$, $0$ pre $\alpha=0$;
$\boldsymbol{5.}$  $a^{a}(1+\ln a)$;
$\boldsymbol{6.}$  $\sqrt{a.b}$ (prípad $a=b=1$ treba robiť samostatne);
$\boldsymbol{7.}$  $\ln x$ (= $\lim_{n \to \infty}\sqrt[n+1]{x}n^{2}(x^{1/n-1/(n+1)}-1)$; prípad $x=1$ treba urobiť samostatne);
$\boldsymbol{8.}$  $\frac{1}{\ln a - \ln b}$;
$\boldsymbol{9.}$  $\frac{1}{sqrt{a.b}}$ (prípad $a=b=1$ treba robiť samostatne);
$\boldsymbol{10.}$  $-2$;
$\boldsymbol{11.}$  $\frac{\alpha^{2}}{\beta ^{2}}$;
$\boldsymbol{12.}$  $-2$;
$\boxed{192.}$ $\boldsymbol{1a/}$  použite nerovnosti $\frac{[x]^{\alpha}}{a^{[x]+1}}$ $\leq \frac{x^{\alpha}}{a^{x}}\leq$ $\frac{([x]+1)^{\alpha}}{a^{[x]}}$ $x\geq0, \, a>1, \, \alpha >0$) a pr. 186.1;
$\boldsymbol{1b/}$  možno použiť substitúciu  $\log_{a}x=t$;
$\boldsymbol{2a/}$   $0$;
$\boldsymbol{2b/}$   $0$; 
$\boldsymbol{2c/}$   $1$;  
$\boxed{194.}$ (je zrejmé, že $g$ musí mať nevlastnú limitu); napr. $g(x)=1$ pre ($x<N, \, g(x)=f(x)$ pre $x \in \langle N+2k, N+2k+1 )$, $g(x)=2f(x)$ pre $x \in \langle N+2k+1, N+2k+2 )$ $k=1,2,...;N \in \mathbb{N}$ je číslo, pre ktoré platí $\forall x\geq N :\, f(x)>0$);
$\boxed{194.}$  $\boldsymbol{1.}$ $\lbrace a_{n} \rbrace _{n=1} ^{\infty}$ je klesajúca zdola ohraničená;
$\boldsymbol{2.}$ $\lbrace a_{n} \rbrace _{n=1} ^{\infty}$ je rastúca  a zhora ohraničená (použite nerovnosť  $\frac{1}{n^{2}}\leq \frac{1}{n-1}- \frac{1}{n}$  ($n>1$) alebo  $\frac{1}{(2^{k})^{2}}+\frac{1}{(2^{k}+1)^{2}}+...+\frac{1}{(2^{k+1}-1)^{2}}\leq \frac{1}{2^{k}}$;
$\boldsymbol{3.}$ $\lbrace a_{n} \rbrace _{n=1} ^{\infty}$ je rastúca  a zhora ohraničená (prísť na  odhad $a_{n}\leq (2+ \frac{1}{2}+ \frac{1}{4}+...+ \frac{1}{2^{n-2}} $) pre $n\geq 3 $ dá asi trocha námahy; iná možnosť: využiť nerovnosť $\ln (1+ \frac{1}{n})< \frac{1}{n} $ ( $ n \in \mathbb{N} $), ktorú dostaneme logaritmovaním nerovnosti z pr. 157, potom $(1+\frac{1}{2^{k}})=e^{ln(1+1/2^{k}})< e^{1/2^{k}}  $ );
$\boxed{196.}$   $\lbrace a_{n} \rbrace _{n=1} ^{\infty}$ je rastúca  a zhora ohraničená (ak postupne pre $k=n,n-1,...,1 $ použijeme nerovnosť $ k< 2^{2^{k}}$, dostaneme  $b_{n}< 2\sqrt{1+\sqrt{1+...+\sqrt{1+\sqrt{2}}}}   $  $\rightarrow n-1 $ odmocnín);
$\boxed{197.}$  $\boldsymbol{1.}$ ak zobrazíme na číselnej osi niekoľko prvých členov, zistíme, že by malo platiť $a_{2k}=1-(\frac{1}{4} +\frac{1}{4^{2}}+...+\frac{1}{4^{k-1}})$,  $ k>1  $,  $a_{2k+1}=\frac{1}{2}(1+\frac{1}{4} +\frac{1}{4^{2}}+...+\frac{1}{4^{k-1}}) $,  $ k\geq 1  $, čo možno dokázať indukciou; potom  $ \lim_{k \to \infty}a_{2k}  $ $= \lim_{k \to \infty}a_{2k+1}  = \frac{2}{3}$ $= \lim_{n \to \infty}a_n  $;
$\boldsymbol{2.}$ $\lim_{n \to \infty}a_n =a$, ak $a=b$; $ \lim_{n \to \infty}a_n =\frac{a}{3}+ \frac{2b}{3} $, ak $a \ne b $ (ak definujeme postupnosť $\lbrace t_n \rbrace_{n=1}^{\infty}$ vzťahom $a+(b-a)t_{n}=a_{n}$, dostaneme postupnosť z pr. 197.1);
$\boxed{198.}$  $\boldsymbol{1,2.}$ platí; obidva dôkazy sú analogické dôkazu vety o Heineho definícii limity;
$\boxed{200.}$ stačí dokázať, že čísla na pravých stranách rovností vyhovujú obidvom podmienkam z vety 18;
$\boxed{201.}$  $\boldsymbol{1.}$ pozri napr. pr.160.7,8;
$\boldsymbol{2.}$ nie (pozri pr.165.1);
$\boxed{202.}$ sporom; označme $a=\lim_{n \to \infty}a_n$, $b=\lim_{n \to \infty}a_n$, ak $c \in (a,b)$, ale $c \notin H$, tak existuje $\eta > 0 $ tak, že $(c-\eta ,c+\eta) \cap H=\varnothing $ (pozri pr. 165.1, pritom $a\notin (c-\eta ,c+\eta)$, $b\notin (c-\eta ,c+\eta)$; potom $\lbrace n \in \mathbb{N};\, a_{n }\in (c-\eta /2 ,c+\eta /2)) \rbrace$ je konečná, preto $\exists N_{1} \in \mathbb{N}\, \forall n\in \mathbb{N}\,  n > N_{1}: \, (a_{n} > c+ \eta /2 \, \lor \, a_{n} < c- \eta /2 )$; z podmienky  $\lim_{n \to \infty} (a_{n+1} - a_{n})=0$ vyplýva $\exists N_{2} \in \mathbb{N}\,  \forall n \in \mathbb{N},\, n >N_{2}: \quad \vert a_{n+1} - a_{n} \vert < \eta $; nech $N= \max \lbrace N_{1}, N_{2} \rbrace $, potom platí ( $\forall n \in \mathbb{N},\, n >N: \quad a_{n} < c-\eta /2 $) $\lor $ ( $\forall n \in \mathbb{N},\, n >N: \quad a_{n} > c+ \eta /2 $); to je ale spor s faktom, že $a$  aj   $b$ sú hromadné hodnoty postupnosti $\lbrace a_{n} \rbrace _{n=1} ^{\infty} $;
$\boxed{204.}$  $\boldsymbol{1.}$  všetky postupnosti, ktorých hromadnou hodnotou je číslo $0 $;
$\boldsymbol{2.}$  všetky postupnosti z pr. 204.1 a všetky postupnosti $\lbrace a_{n} \rbrace _{n=1} ^{\infty} $, pre ktoré platí $\exists  n \in \mathbb{N}, n\geq 2: \, a_{n} =0$;
$\boxed{204.}$  $\boldsymbol{1.}$ vyplýva to z vety 19 a z tvrdenia: ak $\lbrace c_{n} \rbrace _{n=1} ^{\infty} $ je postupnosť kladných čísel a $\limsup_{n \to \infty}c_n=L \in (0,\infty) $, tak $\liminf_{n \to \infty} \frac{1}{c_n}=\frac{1}{L} $;
$\boldsymbol{2.}$ pre všetky, ktoré vyhovujú podmienke $0 <\liminf_{n \to \infty}a_n < +\infty$;
$\boxed{206.}$  $\boldsymbol{1.}$ v prípade  $\lim_{n \to \infty}a_n =0 $ uvedená rovnosť iste platí, pretože vtedy  $\lim_{n \to \infty}a_n . b_{n} =0 $; v prípade  $\lim_{n \to \infty}a_n \ne 0 $ sa postupuje analogicky ako v pr. 166;
$\boxed{207.}$ sporom; nech teda ($*$) $\exists N \in \mathbb{N} \, \forall n \in \mathbb{N}, n>N $ $\exists k \in \mathbb{N} \, k<n:\, a_{k} \leq a_{n} $; nech $m= \min \lbrace a_{n}; n=1,...,N \rbrace$ (zrejme $m>0 $), potom $a_{N+1}\geq m $  (podľa ($*$) niektoré z čísel $a_{1}, ... , a_{N}$  je menšie alebo rovné číslu $a_{N+1}$) atď, teda $\forall n\in \mathbb{N}: \, a_{n}\geq m $, čo je spor s predpokladom $\liminf_{n \to \infty}a_n =0 $;











\section{Dodatok 2}
$\boxed{208.}$  $\boldsymbol{1.}$ nie je otvorená (2 a 3 nie sú vnútorné body), nie je uzavretá (neobsahuje svoje hromadné body 0, 1);
$\boldsymbol{2.}$ otvorená , nie je uzavretá (neobsahuje napr. svoj hromadný bod 0);
$\boldsymbol{3.}$ nie je otvorená (bod 0),  nie je uzavretá (napr.  bod 1);
$\boldsymbol{4.}$ uzavretá, nie je otvorená;
$\boldsymbol{5.}$  nie je otvorená, nie je uzavretá;
$\boldsymbol{6.}$  $\chi (R)=\lbrace 0,1\rbrace $,  nie je otvorená, je uzavretá;
$\boldsymbol{7.}$ nie je otvorená,  nie je uzavretá (neobsahuje svoj hromadný bod 0);
$\boldsymbol{8.}$ $f(R)=\lbrace \frac{1}{n}; n \in \mathbb{N} \rbrace  \cup \lbrace 0 \rbrace$; je  uzavretá, nie je otvorená;
$\boldsymbol{9.}$  otvorená , nie je uzavretá;
$\boxed{209.}$  $\boldsymbol{2.}$ stačí dokázať, že doplnky sú otvorené; využite de Morganove vzorce $\mathbb{R}-(A \cup B)= (\mathbb{R} - A) \cap ( \mathbb{R} - B) $, $\mathbb{R}-(A \cap B)= (\mathbb{R} - A) \cup ( \mathbb{R} - B) $ a pr. 209.1;
$\boldsymbol{3.}$ $ A-B=A \cap (\mathbb{R} - B)$, ďalej využite pr. 209.1,2;
$\boldsymbol{4.}$ nech $x \in \bigcup\limits_{\alpha \in I }A_{\alpha} $, potom  $x \in A_{\alpha_{0}}  $ pre  niektoré $\alpha_{0}\in I $; $ A_{\alpha_{0}} $ je otvorená množina, teda pre niektoré okolie $O(x)$ bodu $x$ platí $O(x) \subset A_{\alpha_{0}}$; pretože $A_{\alpha_{0}}\subset \bigcup\limits_{\alpha \in I}A_{\alpha} $, platí aj $O(x) \subset \bigcup\limits_{\alpha \in I}A_{\alpha}$; teda každý prvok množiny $\bigcup\limits_{\alpha \in I}A_{\alpha}$ je jej vnútorným bodom;
$\boxed{210.}$ $A+B=\bigcup\limits_{b \in B} ( \lbrace b \rbrace + A)$, každá z množín $\lbrace b \rbrace +A $ je otvorená, zjednotenie ľubovoľného systému otvorených množín je otvorená množina (pozri pr. 209.4);
$\boxed{211.}$ $\boldsymbol{1.}$ $(a,b)=\bigcup\limits_{n \in \mathbb{N}} \langle a+ \frac{b-a}{n+2}, b-\frac{b-a}{n+2} \rangle $;
$\boldsymbol{2.}$ nepriamo, ak $\langle a,b \rangle =\bigcup\limits_{t \in I}A_{t} $, kde $ A_{t} $ je otvorený interval pre každé $t \in I $, tak $\exists t_{0} \in I: \, b \in A_{t_{0}} \subset \langle a,b \rangle$, teda $b$ je vnútorný bod množiny $\langle a,b \rangle$, čo neplatí;
$\boxed{212.}$ $C =\bigcup\limits_{b \in B}C_{b} $, kde $C_{b}= \lbrace \vert x-b \vert; x \in A \rbrace $; pre $b \notin A $  je množina $ C_{b}$ otvorená (ak $ \varepsilon  $-okolie bodu  $x$ patrí do $A$, tak $ \varepsilon  $-okolie bodu $\vert x-b \vert $ patrí do $ C_{b}$); pre $b \in A $ možno $ C_{b}$ písať v tvare $\lbrace 0 \rbrace \cup \lbrace \vert x-b \vert ; x \in A- \lbrace b \rbrace \rbrace $, pritom $A-\lbrace b\rbrace  $ je otvorená podľa pr. 209.3 a $b \notin A-\lbrace b \rbrace $, preto - v predchádzajúcej úvahe stačí nahradiť množinu $A$   množinou $  A- \lbrace b \rbrace $  - je $\lbrace \vert x-b \vert; x \in A - \lbrace b \rbrace  \rbrace$ otvorená množina;
dokončenie dôkazu ponechávame čitateľovi;
$\boxed{213.}$ sporom, ak $\varnothing \ne A \ne \mathbb{R}$ je súčasne uzavretá a otvorená, tak $ A $ aj $ \mathbb{R} -A $ sú otvorené a neprázdne; nech $ a \in A $, $b \in \mathbb{R} - A$, predpokladajme $ a<b $ (pre $ a>b $ je postup analogický); nech $ c:= \sup \lbrace x \in (a,b); \, x \in A \rbrace $, potom $ c \ne a $  (lebo  $ A $  je otvorená), $ c \ne b$  (lebo  $\mathbb{R} - A $  je otvorená); pritom  $ c  $ nemôže ležať v $ A $ ($ c $ by muselo patriť do $ A $  spolu s niektorým svojim okolím, teda existovali by prvky patriace do $ A $  a ležiace v intervale $ (c,b) $, čo odporuje vlastnostiam $ c $  ako suprema) a nemôže patriť ani do $\mathbb{R} - A $, čo je spor; 
$\boxed{214.}$ využite, že množina $A \subset \mathbb{R}$ je kompaktná práve vtedy, keď je uzavretá a ohraničená (tvrdenie pr. 209.2 možno matematickou indukciou rozšíriť na konečný počet množín);
$\boxed{215.}$ ak $N:= \max \lbrace n \in \mathbb{N};\, \frac{1}{n} \geq \epsilon \rbrace $, tak konečné podpokrytie je $\lbrace (-\varepsilon , \varepsilon),  ( 1- \varepsilon , 1+ \varepsilon),..., (\frac{1-\varepsilon}{2^{N}},\frac{1+\varepsilon}{2^{N}}) \rbrace$;
$\boxed{216.}$ nie, každý y uvedených intervalov obsahuje práve jeden prvok nekonečnej množiny $E$;
$\boxed{217.}$ napr. $\lbrace ( \frac{1}{n},2);\, n \in \mathbb{N} \rbrace $;
$\boxed{218.}$ $\boldsymbol{1.}$ z množiny $A_{n}$ vyberme prvok $a_{n}$ ( potom $\lbrace a_{n} \rbrace _{n=k}  ^{\infty}$ je postupnosť prvkov z  $A_{k}$), z $\lbrace a_{n} \rbrace _{n=1}  ^{\infty}$ možno vybrať konvergentnú postupnosť, jej limita leží v každej z množín  $A_{n}$, $ n \in \mathbb{N}$, teda leží aj v $ A $;
$\boldsymbol{2.}$ $\bigcap\limits_{n \in \mathbb{N}} A_{n} = \bigcap\limits_{n \in \mathbb{N}} B_{n}$, kde $B_{n}:= A_{1}\cap A_{2}\cap ... \cap A_{n}$;  $\lbrace B_{n} \rbrace _{n=1}  ^{\infty}$ vyhovuje podmienkam z pr. 218.1;
$\boxed{219.}$ $\boldsymbol{1.}$ áno (využite, že každú otvorenú množinu možno písať v tvare zjedotenia otvorených intervalov);
$\boldsymbol{2.}$ nie (túto vlastnosť majú len konečné množiny; využite, že $\lbrace \lbrace a \rbrace;a \in A \rbrace$  je pokrytie množiny $A \ne \varnothing $  uzavretými množinami);
$\boldsymbol{3.}$ nie (uvažujte napr. množinu $\langle 0,1\rangle $ a jej pokrytie $\lbrace \langle -\frac{1}{4}, \frac{1}{4} \rangle$, $\langle \frac{3}{4}, \frac{5}{4} \rangle \rbrace \cup $ $\lbrace \langle \frac{1}{4} +\frac{1}{n} , \frac{3}{4} - \frac{1}{n} \rangle$; $n\geq 3 \rbrace $; platí ale, že každá množina s uvedenou vlastnosťou je kompaktná, využite pr. 219.1);
$\boldsymbol{4.}$ nie (uvažujte napr. množinu $\langle 0,1\rangle$ a jej  pokrytie $\lbrace \langle 1, 2 ) \rbrace \cup \lbrace\langle 0, \frac{n-1}{n} ); \, n \geq 2 \rbrace $; platí ale,  že každá množina s uvedenou vlastnosťou je kompaktná);
$\boxed{220.}$ využite ekvivalenciu $a/ \iff c/$ z vety 21; ak $c_{n} \in A+B$, $n \in \mathbb{N} $, tak existujú $a_{n} \in A$, $b_{n} \in B$ ($n \in \mathbb{N} $) tak, že $\lbrace c_{n} \rbrace _{n=1}  ^{\infty}$ $=\lbrace a _{n}+b_{n} \rbrace _{n=1}  ^{\infty}$; z $\lbrace a_{n} \rbrace _{n=1}  ^{\infty}$ možno vybrať konvergentnú $\lbrace a_{n_{k}} \rbrace _{k=1}  ^{\infty}$, z $\lbrace b_{n} \rbrace _{n=1}  ^{\infty}$ možno vybrať konvergentnú $\lbrace b_{n_{k}} \rbrace _{k=1}  ^{\infty}$, potom  $\lbrace c_{n_{k_{l}}} \rbrace _{l=1}  ^{\infty}$ je konvergentná a jej limita leží v množine  $A+B $; podobne pre $A.B$;
 





\section{Spojitosť funkcie}
$\boxed{221.}$ $\exists \varepsilon > 0 \, \forall \delta > 0 \, \exists  x_{\delta} \in D(f): \vert x_{\delta} -a \vert < \delta \land \vert f(x_{\delta}) - f(a) \vert \geq \varepsilon $;
$\boxed{222.}$ $\boldsymbol{1.}$ $0$ je hromadný bod množiny $D(f)$, $\lim_{x \to 0} f(x) = 1 = f(0)$, teda $f$ je spojitá v bode $0$;
$\boldsymbol{2.}$  nie je spojitá v bode $0$ ($0$  je hromadný bod $D(f)$  a   $\lim_{x \to 0} f(x) $ neexistuje);
$\boldsymbol{3.}$ nie je spojitá v bode $-1$;
$\boldsymbol{4.}$ nie je spojitá v bode $0$;
$\boldsymbol{5.}$ je spojitá v bode $0$;
$\boxed{223.}$ $\boldsymbol{1.}$ $f(0)$ treba položiť rovné $\frac{3}{2} $ ($= \lim_{x \to 0} f(x)$);
$\boldsymbol{2.}$ $f(0):=\sqrt{e}$;
$\boldsymbol{3.}$ $f(0):=0$;
$\boldsymbol{4.}$ $f(1):=1$;
$\boldsymbol{5.}$ $f(-3):=0$;
$\boxed{224.}$ $\boldsymbol{1.}$ spojitá (vyplýva to z viet 1 a 3);
$\boldsymbol{3.}$ spojitá  v každom bode množiny $\mathbb{R}- \lbrace -1,0,1 \rbrace $, nespojitá v bodoch $-1,0,1 $;
$\boldsymbol{4.}$ spojitá len  v bodoch množiny $(\mathbb{R}- \mathbb{Z}) \cup \lbrace 0 \rbrace $;
$\boldsymbol{5.}$ $f(x)=1$ pre $x \in \langle0,1)$, $f(x):=\frac{1}{2}$  pre $x =1$, $f(x):=0$  pre $x > 1$; $f$ je spojitá práve v bodoch množiny $\langle 0, +\infty) -\lbrace 1 \rbrace $;
$\boldsymbol{6.}$ $f(x)=1$ pre $ \vert x \vert \leq 1$, $f(x)=x^{2}$  pre $ \vert x \vert >1$; $f$ je spojitá;
$\boldsymbol{7.}$ (množina $D(f)= \mathbb{Z}$ pozostáva len z izolovaných bodov);
$\boxed{227.}$ nemožno; napr. 1. ak $x_{0}$ je izolovaný bod množiny $D(f)$, je funkcia $g\circ f $ spojitá v bode $x_{0}$ ($x_{0}$ je potom totiž aj izolovaný bod množiny  $D(g\circ f)$); alebo 2. $f(x)= const$, $x \in  \mathbb{R}$  ; $ g $ je ľubovoľná funkcia nespojitá v bode $ f(a)$;
$\boxed{228.}$ pri dôkaze spojitosti funkcie $\vert f \vert$ v bode $a$ využite nerovnosť $\vert \vert r \vert - \vert s \vert \vert \leq \vert r-s \vert$ ($r,s \in \mathbb{R}$) a spojitosť funkcie  $f$  v bode $a$; ďalej platí $\max \lbrace f,g \rbrace =\frac{f+g+\vert f-g\vert}{2}$, analogicky možno vyjadriť $\min \lbrace f,g \rbrace $;
$\boxed{229.}$ funkcia $\chi $ nie je spojitá v žiadnom bode  množiny $\mathbb{R}$;
$\boxed{230.}$ nech $a \in \mathbb{R} - \mathbb{Q} $, nech $\varepsilon  >0$ je dané; nech $\overline{n} \in \mathbb{N} $ je také, že $\frac{1}{\overline{n}}< \varepsilon $; pre každé $n \in \lbrace1,...,\overline{n} \rbrace $ existuje $\delta_{n}>0 $ tak, že v $O_{\delta_{n}}(a)$ neležia body s funkčnými hodnotami  $\frac{1}{n} $; pre dané  $\varepsilon $ potom stačí položiť  $\delta = \min \lbrace \delta_{1},...,\delta_{\overline{n}} \rbrace$; dôkaz druhého tvrdenia z pr. 230 prenechávame čitateľovi;
$\boxed{232.}$ $\boldsymbol{1.}$ $-1$ je bod neodstrániteľnej nespojitosti, bod nespojitosti 2. druhu;
$\boldsymbol{2.}$ $1$ a $-2$ sú body neodstrániteľnej nespojitosti, body nespojitosti 2. druhu;
$\boldsymbol{3.}$ $0$ je bod odstrániteľnej nespojitosti;
$\boldsymbol{4.}$ $1$ a $0$ sú body odstrániteľnej nespojitosti, $-1$ je bod neodstrániteľnej nespojitosti, bod nespojitosti 2. druhu;
$\boldsymbol{5.}$ $2$ a $-2$ sú body odstrániteľnej nespojitosti;
$\boldsymbol{6.}$ $0$ je bod neodstrániteľnej nespojitosti, bod nespojitosti 2. druhu;
$\boldsymbol{7.}$ $0$ je bod odstrániteľnej nespojitost;
$\boldsymbol{8.}$ $0$ je bod odstrániteľnej nespojitost, $1$ je bod neodstrániteľnej nespojitosti, bod nespojitosti 2. druhu;
$\boldsymbol{9.}$ $-1$ je bod neodstrániteľnej nespojitost, bod nespojitosti 1. druhu;
$\boldsymbol{10.}$ každý prvok množiny $\lbrace \frac{1}{n}; \, n \in \mathbb{Z} -\lbrace 0 \rbrace \rbrace$ je bod neodstrániteľnej nespojitosti, bod nespojitosti 1. druhu;  $0$ je bod odstrániteľnej nespojitost  $\lim_{x \to 0} f(x) =1$; platí totiž $f \langle ( \frac{1}{n+1}, \frac{1}{n} \rangle ) $ pre  $n \in \mathbb{N}$,  $f (( \frac{1}{n}, -\frac{1}{n+1} \rangle ) = \langle 1, \frac{n+1}{n})$ pre  $n \in \mathbb{N}$;
$\boldsymbol{11.}$ $f(x) = x$, $D(f) = \mathbb{R} - \lbrace 0,1 \rbrace$; $0$ a $1$ sú body odstrániteľnej nespojitosti;
$\boldsymbol{12.}$ každý prvok  $a \in\mathbb{R}- \mathbb{Z} $ je bod neodstrániteľnej nespojitosti, bod nespojitosti 2. druhu;
$\boldsymbol{13.} $ platí $g(\varphi (x)) = g(x)$;  $0$ a  $2$ sú body odstrániteľnej nespojitost (všimnite si napriek tomu, že ;$\varphi $ je spojitá len v bode $1$, je $g \circ \varphi $ spojitá fubkcia);
$\boxed{234.}$ $\boldsymbol{2.}$ pre spojitú funkciu  $f(x) = x- \cos x$ platí  $f(0)<0$, $f(\frac{\pi}{2})>0$;
$\boldsymbol{3.}$ $\lim_{x \to \infty} P(x) $ a $\lim_{x \to -\infty} P(x) $ sú nevlastné a opačných znamienok, preto musia existovať $a,b \in \mathbb{R}$ tak, že $P(a)> 0$, $P(b)< 0 $ ;
$\boxed{235.}$ využite nerovnosti $\min \lbrace f(x_{1}),..., f(x_{n}) \rbrace \leq \frac{1}{n}(f(x_{1})+...+f(x_{n}))\leq \max \lbrace f(x_{1}),..., f(x_{n}) \rbrace$;
$\boxed{236.}$ ak $f(0)=0$ alebo   $f(1)=1$, je dôkaz skončený; ak  $f(0)\ne 0$, $f(1)\ne 1$, tak  pre  spojitú  funkciu  $g(x) = f(x)-x $ platí  $fg(0)>0$, $fg(1)< 0$;
$\boxed{238.}$ stačí dokázať tvrdenie: každá nekonštantná funkcia s uvedenou vlastnosťou je spojitá;
$\boxed{240.}$ $f$ nadobúda na $\langle a,b \rangle$ minimum, ktoré (pretože je funkčnou hodnotou), je kladné;
$\boxed{241.}$ ak $a,b \in \mathbb{R}$, tak funkcia $g$ určená predpisom $g(a) = \lim_{x \to a^{+} }f(x), \, g(b) = \lim_{x \to b^{-}} f(x), \, g(x)=f(x)$ pre $x \in (a,b)$ je spojitá, a teda ohraničená na kompakte $\langle a,b \rangle$; $f$ je potom ohraničená ako zúženie ohraničenej funkcie; vo všeobecnosti (tj. ak $a, b \in \mathbb{R^{*}}$) možno dokazovať nasledovne: $f$ je ohraničená na niektorom okolí $O(a)$ (to vyplýva z existencie vlastnej  $\lim_{x \to a} f(x)$)  a na niektorom okolí $O(b)$; množina $A:=(a,b)-(O(a) \cup O(b)) $ môže byť prázdna, jednoprvková (v týchto prípadoch je dôkaz skončený) alebo kompaktný interval, na tom je $f$ ohraničená;
$\boxed{242.}$ ak  $\lim_{x \to \infty }P(x) = \lim_{x \to -\infty }P(x)= +\infty $, tak existuje $a>0$ tak, že $\forall x \in \mathbb{R}, \, \vert x \vert > a: \, f(x) > f(0) $; na kompakte $\langle -a,a \rangle$ nadobúda $f$ minimum, ktoré je globálnym minimom na $ \mathbb{R} $;
$\boxed{245.}$ dôkaz je analogický dôkazu veta 5;
$\boxed{246.}$ napr. $f(x)=e^{x} \sin \frac{1}{x}$, $ x \in (0,1 \rangle$; 
$\boxed{247.}$ ($\forall x \in (a,b) \quad \forall\varepsilon  >0 \quad   \exists \delta >0 \,  \forall y \in (a,b): \, \vert x-y \vert < \delta \, \Rightarrow \, \vert f(x)-f(y) \vert < \varepsilon$)  $\land$ ( $\exists \varepsilon _{0} >0 \quad \forall \delta  >0 \, \exists x_{\delta}, y_{\delta}  \in (a,b): \vert  x_{\delta}- y_{\delta}\vert <  \delta $  $\land$  $\vert f(x_{\delta})-f(y_{\delta}) \vert \geq \varepsilon_{0} $);
$\boxed{248.}$  $\boldsymbol{1.}$ áno;
$\boldsymbol{3.}$  nie;
$\boldsymbol{4.}$ áno;
$\boldsymbol{6.}$ áno (Ťažkosti spôsobuje bod $0$, keby sme boli na intervale $\langle \eta , +\infty)$, kde $\eta >0$ , mohli by sme použiť odhad $ \vert \sqrt{x} - \sqrt{y} \vert =\vert \frac{x-y}{\sqrt{x} + \sqrt{y}}\vert \leq \frac{\vert x-y \vert}{2 \sqrt{\eta}} $. Možmo uvažovať nasledovne: nech je da né $\varepsilon  >0 $; ak $0 <  x <  \varepsilon ^{2} $, $0 <  y <  \varepsilon ^{2} $, tak  $ \vert \sqrt{x} - \sqrt{y} \vert < \varepsilon$. Zostáva sa ešte zaoberať prípadom $x\geq \frac{\varepsilon^{2}}{2} $, $y\geq \frac{\varepsilon^{2}}{2} $; potom $ \vert \sqrt{x} - \sqrt{y} \vert \leq \frac{\vert x-y \vert}{2 \sqrt{\eta}}$; teda pre dané $\varepsilon$ a $x,y \geq \frac{\varepsilon^{2}}{2}$ stačí položiť $\delta <  \sqrt{2}\varepsilon^{2} $, tj. napr. $\delta =\frac{\varepsilon^{2}}{2}$. Celkovo sme dokázali: ak  $ \vert x-y  \vert < \frac{\varepsilon^{2}}{2}$, tak $ \vert \sqrt{x} - \sqrt{y} \vert < \varepsilon$.);
 $\boldsymbol{7.}$ áno  (využite vzorec $ \sin x - \sin y = 2 \sin \frac{x-y}{2} \cos \frac{x+y}{2}$  a nerovnosť $\sin \alpha < \vert \alpha\vert$);
$\boxed{249.}$ "$\Rightarrow$ ": k číslu $\frac{\varepsilon}{2}$	nájdime $\delta$ z definície rovnomernej spojitosti, nech pre  body $a=x_{0 }< x_{1}<...< x_{n}=b$ platí $\vert x_{i} -x_{i-1}\vert < \delta$ ($i=1,...,n $), potom graf funkcie $\varphi $ je lomená čiara s vrcholmi  $(x_{0 }, f(x_{0})),(x_{1 }, f(x_{1})), ..., (x_{n }, f(x_{n}))$; pri dôkaze implikácie "$\Leftarrow$" využite, že každá spojitá po častiach lineárna funkcia definovaná na ohraničenom intervale $I$ je rovnomerne spojitá (to možno dokázať priamo z definície), $\vert f(x)-f(y) \vert \leq  \vert f(x)-\varphi(x) \vert + \vert \varphi(x)-\varphi(y) \vert + \vert \varphi(y)-f(y) \vert$;
$\boxed{250.}$ návodom je dôkaz implikácie "$\Rightarrow$ " v pr. 249;
$\boxed{251.}$ $\boldsymbol{1.}$ áno ($f$ je spojitá, a teda podľa vety 6 aj rovnomerne spojitá na kompakte $\langle -1,1 \rangle $);
$\boldsymbol{3.}$ áno (zvoľme $a>0$ pevné; rovnomernú spojitosť na $\langle a,+ \infty) $ možno dokázať z definície, rovnomerná spojitosť na $\langle 0,a \rangle $ vyplýva z vety 6; ak je funkcia rovnomerne spojitá na $\langle 0,a \rangle $ a na  $\langle a,+ \infty) $, tak je rovnomerne spojitá na ich zjednotení);
$\boldsymbol{4.}$ áno ( $f$ je spojitá na kompakte $\langle 0,\pi \rangle$);
$\boldsymbol{5.}$ nie (rodeľme každý z intervalov $\langle \frac{\pi}{2}+k \pi,\frac {3 \pi}{2}+ k\pi \rangle $, $k=0,1,2,...,$ na $n$ rovnakých častí bodmi $x_{0}^{(k)}:= \frac{\pi}{2}+k\pi$,  $x_{1}^{(k)}:= \frac{\pi}{2}+k\pi + \frac{\pi}{n}$,...,$x_{n-1}^{(k)}:= \frac{\pi}{2}+k\pi + \frac{(n-1)\pi}{n}$, $x_{n}^{(k)}:= \frac{3\pi}{2}+k\pi $. Pretože $2\pi({1+k})=$ $\vert f(x_{0}^{(k)})-f(x_{n}^{(k)})  \vert\leq $ $\vert f(x_{0}^{(k)})-f(x_{1}^{(k)})  \vert+...+\vert f(x_{n-1}^{(k)})-f(x_{n}^{(k)})  \vert $, musí byť aspoň jedno z čísel na pravej strane tejto  nerovnosti väčšie ako $\frac{2\pi(1+k)}{n} $. Teda pre ľubovoľné $n \in \mathbb{N}$ existuje v každom z intervalov $\langle \frac{\pi}{2}+k\pi,\frac{3\pi}{2}+k\pi\rangle $ dvojica ($x_{k},y_{k} $) taká, že $\vert x_{k}-y_{k}\vert \leq  \frac{\pi}{n}$ a $\vert f(x_{k})-f(y_{k})\vert > \frac{2\pi(1+k)}{n}$, pritom $\frac{2\pi(1+k)}{n} \rightarrow +\infty$ pre $k \rightarrow +\infty$);
$\boldsymbol{6.}$ áno ($\lim_{x \to 1}f(x)=\lim_{x \to -1}f(x)=0 $,ďalej pozri riešenie pr. 251.2);
$\boldsymbol{7.}$ áno (Ak $x,y\geq 1 $, $0< x-y < \eta  $, tak  $\frac{x}{y}=\frac{x-y+y}{y} < 1+\eta $.
Pretože $x>y\geq 1 $ je nerovnosť  $\vert\ln x - \ln y \vert < \varepsilon $ ekvivalentná s nerovnosťou $\frac{x}{y}< e^{\varepsilon}=1+(e^{\varepsilon}-1) $, stačí pre dané $\varepsilon>0$ položiť $\delta= \varepsilon-1 $.);
$\boxed{252.}$ ak $a:=\lim_{x \to \infty }f(x)$, tak k danému  $\varepsilon >0$ existuje $O(+\infty) $ tak, že $f(O(+\infty)) \subset (a-\frac{\varepsilon}{2}, a+\frac{\varepsilon}{2}) $; pre $ x,y \in O(+\infty)$ potom platí $\vert  f(x)-f(y) \vert < \varepsilon$; existuje kompaktný interval $I$ obsahujúci množinu $ A:= \langle 0,+\infty)-O(+\infty) $ (pozn.: v prípade $A=\varnothing$ netreba už vlastne postupovať ďalej, pre dané $\varepsilon>0$ stačí za $\delta$
 zvoliť ľubovoľné kladné číslo; úvahu s intervalom   $I$  možno ale použiť aj na tento prípad, vyhneme sa tým predlžovaniu dôkazu spôsobenému rozlišovaním jednotlivých možmostí) na  $I$ je  $f$ rovnomerne spojitá, a preto k číslu  $\varepsilon$ možno nájsť príslušné  $\delta$ z definície rovnomernej spojitosti (na  $I$; zvyšok úvah, ako nájsť  $\delta$ vyhovujúce definícii rovnomernej spojitosti na celom intervale  $\langle 0, +\infty)$, prenecháme čitateľovi; 
$\boxed{254.}$ $\boldsymbol{1.}$ sporom; nech $f$ je rovnomerne spojitá na $(a,b)$, zvoľme $\varepsilon >0$ pevné, napr.   $ \varepsilon=1$ a nájdime k nemu príslušné $\delta$ z definície rovnomernej spojitosti; potom pre ľubovoľné $x \in (b-\delta, b) $ je $\vert x-(b-\delta) \vert < \delta$, preto $\vert f(x)-f(b-\delta)) \vert < \varepsilon$; funkcia $f$ je teda ohraničená na $(b-\delta, b) $, čo je spor s tým, že $\lim_{x \to b^{-} }f(x)=+\infty$;
$\boldsymbol{2.}$ ak neexistuje  $\lim_{x \to b^{-} }f(x)$, tak existujú postupnosti $\lbrace a_{n} \rbrace _{n=1}^{\infty} $, $\lbrace b_{n} \rbrace _{n=1}^{\infty} $ ( $a_{n}, b_{n} \in (a,b))$ tak, že $\lim_{n \to \infty }a_{n}=\lim_{n \to \infty }b_{n}=b$, ($A:=)$  $\lim_{n \to \infty }f(a_{n}) \ne \lim_{n \to \infty }f(b_{n})$ ($B:=)$ (pozri pr. 199); ak $A, B \in \mathbb{R}, \, A > B $ (ostatné prípady sú analogické), tak v ľubovoľne malom okolí bodu  $b$ existujú body s funkčnými hodnotami väčšími ako  $ A-\frac{A-B}{3} $ aj body s funkčnými hodnotami menšími ako  $ B+\frac{A-B}{3} $; (iný možný dôkaz tvrdení z pr.254: ak  $f $ je rovnomerne spojitá na  $(a,b) $, tak v bode  $b $ je splnené Cauchyho-Bolzanovo kritérium konvergencie);
$\boxed{255.}$ inšpiráciou pre dôkaz  $" \Leftarrow "$ je pr. 251.2, pri dôkaze $" \Rightarrow "$ využite pr. 254;
$\boxed{256.}$ $\boldsymbol{1.}$ využite pr. 255 a 241;
$\boldsymbol{2.}$ pozri pr. 248.3;
$\boxed{257.}$ $\boldsymbol{1.}$ každý prvok množiny $\lbrace (-1)^{k}\sqrt{n}; \, k \in \lbrace -1,1 \rbrace, n \in \mathbb{N}\rbrace $ je bod neodstrániteľnej nespojitosti, bod  nespojitosti 1. druhu; $0$ je bod neodstrániteľnej nespojitosti, bod  nespojitosti 2. druhu;
$\boldsymbol{2.}$ každý prvok množiny $\lbrace \frac{(-1)^{k}}{n}; \, k \in \lbrace -1,1 \rbrace, n \in \mathbb{N}\rbrace $ je bod neodstrániteľnej nespojitosti, bod  nespojitosti 1. druhu; $0$ je bod neodstrániteľnej nespojitosti, bod  nespojitosti 2. druhu;
$\boldsymbol{3.}$  $0$ je bod neodstrániteľnej nespojitosti, bod  nespojitosti 2. druhu;
$\boldsymbol{4.}$ každý prvok množiny $\lbrace \frac{k \pi}{2}; \, k \in  \mathbb{Z}- \lbrace 0 \rbrace\rbrace $ je bod neodstrániteľnej nespojitosti, bod  nespojitosti 1. druhu; 
$\boldsymbol{5.}$ $f$ je spojitá funkcia;
$\boldsymbol{6.}$ $f$ je spojitá funkcia; 
$\boxed{258.}$ (pozri pr. 192) $\boldsymbol{1.}$ $0$;
$\boldsymbol{2.}$ $1$;
$\boldsymbol{3.}$ $0$; 
$\boxed{259.}$ (možno využiť pr. 228) ($F(x)= \min \lbrace c, \max \lbrace f(x), -c \rbrace \rbrace $); 
$\boxed{260.}$ obrátená implikácia neplatí (položme napr. $ a_{n}=f(q_{n})$, kde $f(x)=0$ pre $x \in \mathbb{Q}-\mathbb{N}, \, f(x)=1 $ pre $x \in \mathbb{N}$ ;
$\boxed{261.}$ $f$ je spojitá v každom bode množiny $A= \lbrace (\mathbb{R}-\mathbb{Q} \cap (0+\infty))$ a nespojitá v bodoch množiny $\mathbb{R} -A$;
$\boxed{262.}$  $f$ je spojitá;
$\boxed{263.}$ $\boldsymbol{1.}$ nevyplýva (vhodný protipríklad už nájdite sami); uvedená podmienka je ekvivalentná s podmienkou:  $f$ je ohraničená v každom okolí bodu  $x_{0}$;
$\boldsymbol{2.}$ nevyplýva (uvažujte napr. $f(x) = x$ pre $x<0$,   $f(x) = x+1$ pre $x\geq 0$); uvedená podmienka je ekvivalentná s podmienkou:  $f$ má spojitú inverznú funkciu;
$\boxed{264.}$ $\boldsymbol{1.}$ nie (možno využiť pr. 232.13);
$\boldsymbol{2.}$ áno (pri dôkaze možno využiť pr.239);
$\boxed{266.}$ napr. $f(x)=x$ pre $x \in \mathbb{Q}- \lbrace 0,1 \rbrace $, $f(x)=-x$ pre $x \in \mathbb{R}- \mathbb{Q} $, $f(0)=1$, $f(1)=0$;
$\boxed{267.}$ Sporom. Zvoľme $x_{0}$, platí $1_{0}:=\lim_{x \to x_{0} }f(x) \ne f(x_{0})$, preto $\exists \varepsilon_{0} > 0 $  $ \exists\delta_{0 } > 0: f(x_{0}) \notin \langle 1_{0}-\varepsilon_{0}, 1_{0}+\varepsilon_{0} \rangle \, (=:B_{0}) \, \land f(\langle x_{0}+\delta_{0}/3, x_{0}+\delta_{0}\rangle ) \subset B_{0}$. Pre bod $x_{1} :=x_{0}+2\delta_{0}/3$ platí $l_{1}:= \lim_{x \to x_{1} }f(x) \ne f(x_{1}) $, preto $\exists 0< \varepsilon_{1} < \varepsilon_{0}/2 $  $ \exists 0<\delta_{1 } < \delta_{0}/3: f(x_{1}) \notin \langle 1_{1}-\varepsilon_{1}, 1_{1}+\varepsilon_{1} \rangle \, (=:B_{0}) \, \land f(\langle x_{1}+\delta_{1}/3, x_{1}+\delta_{1}\rangle ) \subset \langle l_{1}-\varepsilon_{1},l_{1}+\varepsilon_{1} \rangle$; teraz položme  $x_{2} :=x_{1}+2\delta_{1}/3$ atď. Tak dostaneme postupnosť prvkov $\lbrace x_{n} \rbrace _{n=0}^{\infty}$, postupnosť uzavretých intervalov  $\lbrace B_{n} \rbrace _{n=0}^{\infty} = \lbrace \langle l_{n}-\varepsilon_{n} , l_{n}+\varepsilon_{n}\rangle \rbrace _{n=0}^{\infty}$ a postupnosť uzavretých intervalov  $\lbrace A_{n} \rbrace _{n=0}^{\infty}= \lbrace \langle x_{n}+ \delta_{n}/3, x_{n}+\delta_{n} \rangle \rbrace _{n=0}^{\infty}$ s vlastnosťami: $A_{0}\supset A_{1} \supset  A_{2} \supset A_{3} \supset ..., B_{0}\supset B_{1} \supset B_{2} \supset...,$  $\delta_{n+1}< \delta_{n}/3 $; $\varepsilon_{n+1}< \varepsilon_{n}/2$,  $x_{n+1}\in A_{n}- A_{n+1} $, $f(x_{n+1})\in B_{n} $, $f(A_{n})\subset B_{n} $. Označme $a,b$ tie prvky, pre ktoré platí $\lbrace a\rbrace = \bigcap\limits_{n=0}^\infty A_{n} $, $\lbrace b\rbrace = \bigcap\limits_{n=0}^\infty B_{n} $; potom platí  $ a= \lim_{n \to \infty}x_n , \,  b= \lim_{n \to \infty}f(x_n)$, preto $\lim_{n \to \infty}f(x=b)$. Súčasne ale (pretože $a \in A_{n} $ pre každé $n=0,1,...) $ je $f(a) \in B_{n} (n=0,1,...)$, preto $f(a)=b $, čo je spor.
$\boxed{268.}$ $\boldsymbol{1.}$ každá z množín $ A_{n}:=\lbrace x \in \mathbb{R}; \, f(x)>\frac{1}{n}\rbrace $,  $ B_{n}:=\lbrace x \in \mathbb{R}; \, f(x)<\frac{1}{n}\rbrace $ ($ n \in \mathbb{N}$ je konečná a $A = \bigcup\limits_{n=1}^\infty (A_{n} \cup  B_{n} $;
$\boldsymbol{2.}$ množina  $ M=\lbrace x \in \mathbb{R};f(x)=0 \rbrace  $ je konečná, funkcia  $g(x) = \frac{1}{f(x)}, x \in \mathbb{R}-M $, vhodne dodefinovaná v bodoch množiny $M$, vyhovuje predpokladom pr. 268.1;
$\boxed{271.}$ existujú postupnosti $\lbrace a_{n}\rbrace_{n=1}^{\infty} $, $\lbrace b_{n}\rbrace_{n=1}^{\infty}$ tak, že $  \lim_{n \to \infty}a_n =\lim_{n \to \infty}b_n = +\infty$, $(a:=\lim_{n \to \infty}f(a_n)<\lim_{n \to \infty}f(b_n) = +\infty (=:b$, $a,b \in \mathbb{R}$ (pozri pr. 199); potom stačí položiť napr. $A=\frac{a+b}{2} $;
$\boxed{273.}$ sporom; ak $g(x):=f(x+ \frac{T}{2})-f(x) \ne 0$ pre každé $x\in \mathbb{R}$, tak $g$ nemení znamienko na $ \mathbb{R}$, potom ale (ak napr. $g(x) >0 $ na  $\mathbb{R} $ je  $f(x+T) - f(x)= (f(x+T)-f(x+T/2))+(f(x+T/2)-f(x)) >0$;
$\boxed{274.}$ $\boldsymbol{2.}$ napr. $f(x)=0$ pre  $x\in \mathbb{R}-(\mathbb{N} \cup \lbrace \frac{1}{n}; n\in \mathbb{N} \rbrace$), $f(n)=\frac{1}{n} \in \mathbb{R}$;
$\boldsymbol{3.}$ napr. $f(x)=x$ pre  $x\in \langle 0,1\rangle $, $f(x)=x-1$ pre  $x\in (2,3\rangle $;
$\boxed{276.}$ $\boldsymbol{1.}$ pri dôkaze rovnomernej spojitosti funkcie $f.g$ využite, že $f$ aj $g$ sú na $(a,b)$ ohraničené (pozri pr. 256.1);
$\boxed{277.}$ tvrdenie je nepravdivé;
$\boxed{278.}$ $\boldsymbol{1,2.}$ každá funkcia rovnomerne spojitá na $\mathbb{R}$;
$\boldsymbol{3.}$ každá konštantná funkcia;


\section{Diferenciálny počet funkcií jednej premennej}
$\boxed{279.}$
$\boldsymbol{1.}\ 0,2;\boldsymbol{2.}\ 0;\boldsymbol{3.}\ 2;\boldsymbol{4.}\ 1;\boldsymbol{5.}\ 2;$
$\boxed{280.}$
$\boldsymbol{1.}\ f'(1)=+\infty;\boldsymbol{2.}\ \lim_{x\rightarrow 0+}\frac{\sgn x}{x}=\lim_{x\rightarrow 0+}\frac{1}{x}=\infty=\lim_{x\rightarrow 0-}-\frac{1}{x}=\lim_{x\rightarrow 0-}\frac{\sgn x}{x};$
$\boxed{281.}$
$\boldsymbol{1.}\ f'(x)=3x^2+2,D(f')=\mathbb{R};\boldsymbol{2.}\ f'(x)=\frac{4}{3}\sqrt[3]{x},D(f')=\mathbb{R};\boldsymbol{3.}\ f'(x)=-\frac{2x}{(1+x^2)^2},D(f')=\mathbb{R};\boldsymbol{4.}\ f'(x)=2^{x+1} \ln 2,D(f')=\mathbb{R};\boldsymbol{5.}\ f'(x)=\frac{1}{x},D(f')=(0,\infty);\boldsymbol{6.}\ f'(x)=-\frac{1}{{\sin}^2 n},D(f')=\mathbb{R}\setminus \{k\pi; k\in\mathbb{Z}\};\boldsymbol{7.}\ f'(x)=\frac{2}{3(x+1)^{\frac{1}{3}}},D(f')=\mathbb{R}\setminus {\ -1}\,$ pre $x=-1$ platí $\lim_{h\rightarrow 0+}\frac{f(-1+h)-f(-1)}{h}=\lim_{h\rightarrow 0+}\frac{1}{\sqrt[3]{h}}=\infty \neq -\infty=\lim_{h\rightarrow 0-}\frac{1}{\sqrt[3]{h}}=\lim_{h\rightarrow 0-}\frac{f(-1+h)-f(-1)}{h}$, teda $f'(-1)$ neexistuje; $\boldsymbol{8.}\ f'(x)=1,D(f')=\mathbb{R}\setminus \{0\}$; pre $x=0$ platí $f'(0)=\infty$ (pozri pr. $280.2$), preto $0\notin D(f')$;
$\boxed{282.}$
$\boldsymbol{1.}\ f'_+(\frac{\pi}{2})=1,f'_-(\frac{\pi}{2})=-1;\boldsymbol{2.}\ f'_+(1)=-\pi,f'_-(1)=0;\boldsymbol{3.}\ f'_+(2)=1,f'_-(2)=-1;\boldsymbol{4.}\ f'_+(3)=1,f'_-(3)=-1;$
$\boxed{283.}$
$\boldsymbol{1.}\ f'_+(0)=1=f'_-(0)$, teda $f'(0)=1;\boldsymbol{2.}\ f'_-(-1)=-2,f'_+(-1)=\lim_{x\rightarrow -1_+}\frac{-2x-0}{x+1}$ ($f(-1)$ sa vypočíta dosadením čísla $-1$ do výrazu $x^2-1$)$=\lim_{x\rightarrow -1_+}-\frac{2x}{x+1}=\infty;f'_+(-1)\notin f'_-(-1)$, preto $f'(-1)$ neexistuje; $\boldsymbol{3.}\ f'_+(0)=0=f'_-(0)$, teda $f'(0)=0;\boldsymbol{4.}\ f'_+(\pi)=f'_-(\pi)=0,$ teda $f'(\pi)=0$;
$\boxed{284.}$
$\boldsymbol{1.}\ \lim_{h\rightarrow 0}\frac{f(h)-f(-h)}{2h}=\lim_{h\rightarrow 0}(\frac{1}{2}(\frac{f(h)-f(0)}{h}+\frac{f(0)-f(-h)}{h})),$ pri výpočte limity druhého sčítanca použite substitúciu $t=-h$; $\boldsymbol{2.}\ $ nevyplýva, stačí uvažovať $f(x)=|x|$; 
$\boxed{286.}$
$\boldsymbol{1.}\ y'=1-2x;\boldsymbol{2.}\ y'=x^{\frac{2}{3}}+\sqrt{5}x^{-\sqrt{5}-1}$; $\boldsymbol{3.}\ y'=\frac{2(1-2x)}{(1-x-x^2)^2};\boldsymbol{4.}\ y'=\frac{6-\sqrt[3]{x^2}}{6\sqrt{x}(2+\sqrt[3]{x^2})^2};\boldsymbol{5.}\ y'=30(3x-7)^9;\boldsymbol{6.}\ y'=\frac{1+2x^2}{\sqrt{1+x^2}};\boldsymbol{7.}\ y'=-(p+q+(p-q)x)*\frac{(1-x)^{p-1}}{(1+x)^{q+1}};\boldsymbol{8.}\ y'=-\frac{1}{(1+x^2)^{\frac{3}{2}}};\boldsymbol{9.}\ y'=\frac{7\sqrt[5]{2}}{65\sqrt[5]{x^4}\sqrt[13]{(9+7\sqrt[5]{2x})^12}};\boldsymbol{10.}\ y'=\frac{4\sqrt{x+\sqrt{x}}*\sqrt{x}+2\sqrt{x}+1}{8\sqrt{x}*\sqrt{x+\sqrt{x}}*\sqrt{x+\sqrt{x+\sqrt{x}}}};$
$\boxed{287.}$
$f'(a)=\lim_{x\rightarrow a}\frac{(x-a)\varphi (x)}{x-a}=\varphi (a);$
$\boxed{288.}$
$\boldsymbol{1.}\ f'_+(a)=\varphi (a)\neq -\varphi (a)=f'_-(a);\boldsymbol{2.}\ f'(a)=\lim_{x\rightarrow a}(\varphi (x) \sgn (x-a))*|x-a|^{\varepsilon}=0$ (funkcia $\varphi (x) sgn (x-a)$ je ohranižená, $\lim_{x\rightarrow a}|x-a|^{\varepsilon}=0$);
$\boxed{289.}$
$\boldsymbol{1.}\ y'=\cos x - x\sin x;\boldsymbol{2.}\ y'=\frac{x^2}{(\cos x + x\sin x)^2};\boldsymbol{3.}\ y'=\frac{\sin x \cos x - 2x}{2\sqrt{x}\sin ^2 x},x>0,x\neq \frac{k\pi}{2},k\in \mathbb{N}$ (Doteraz sme pri zápisoch využívali dohodu, podľa ktorej definičný obor funkcie - pokiaľ nie je výslovne určený - je množina všetkých tých $x\in \mathbb{R}$, pre ktorý má daný predpis 'zmysel'. Funkcia $f(x)=\frac{\sqrt{x}}{\tan x}$ je definovaná na $D(f)=\{x\in \mathbb{R};x>0 \wedge x\neq \frac{k\pi}{2},k\in \mathbb{N}\}$ a podľa vety o derivácii podielu má v každom jej bode konečnú deriváciu, preto $D(f')=D(f)$; množina $D(f)$ sa v tomto prípade nezhoduje s množinou $\{x\in \mathbb{R}; x>0 \wedge x\neq k\pi, k\in \mathbb{N} \}$ všetkých tých reálnych čísel, pre ktoré má výraz $f'(x)=\frac{(\sin x \cos x - 2x)}{(2\sqrt{x}\sin ^2 x)}$ 'zmysel');$\boldsymbol{4.}\ y'=n\sin^{n-1}x \cos (n+1)x; \boldsymbol{5.}\ y'=-\sin 2x*\cos (\cos 2x);\boldsymbol{6.}\ y'=2\sin x * \frac{\cos x \sin x^2 - x\sin x \cos x^2}{\sin^2 x^2};\boldsymbol{7.}\ y'=-\cos 2x,x\neq \frac{k\pi}{2}, x\neq -\frac{\pi}{4}+k\pi,k\in \mathbb{Z}$ (pozri poznámku k pr. $289.3$); $\boldsymbol{8.}\ y'=\frac{x^4-1}{x^3\cos^2 (x^2+x^{-2})\sqrt{1+\tan^2(x^2+x^{-2})}}$
$\boxed{290.}$ 
$\boldsymbol{1.}$ $\ y'= \frac{1}{2}(\sqrt{2^{x}}\ln 2-\sqrt{5^{-x}} \ln 5)$;
$\boldsymbol{2.}$ $\ y'=-2x e^{-x^{2}}$;
$\boldsymbol{3.}$ $\ y'= e^{x}x^{2}$;
$\boldsymbol{4.}$ $\ y'= 2^{1+ \sin x^{2}}.x \cos x^{2}\ln 2$; 
$\boldsymbol{5.}$ $\ y'= -e^{\sqrt{\frac{1-x}{1+x}}}(1+x)^{-3/2}(1-x)^{-1/2}$;
$\boldsymbol{6.}$ $\ y'= e^{ax}\sqrt{a^{2}+b^{2}}\sin bx$;
$\boldsymbol{7.}$ $\ y'=(\frac{a}{b})^{x}(\frac{b}{x})^{a}(\frac{x}{a})^{b}(\ln a-\ln b-\frac{a}{x}+\frac{b}{x})$;
$\boldsymbol{8.}$ $\ y'= x^{2}e^{-x}\sin x$;
$\boxed{291.}$ 
$\boldsymbol{1.}$ možno (v opačnom prípade by existovala vlastná alebo nevlastná $\ g'(a) $, pretože  $\ g=(f+g)-f)$;
$\boldsymbol{2.}$ nemožno (ak zvolíme $f(x)=\sqrt[3]{x}$,  $g_{1}(x)=\vert x \vert$, $g_{2}(x)=\sgn x .\sqrt[3]{x}$, tak existuje nevlastná $(f+g_{1})'(0)$, a neexistuje $(f+g_{2})'(0)$; za uvedených predpokladov ale nemôže existovať vlastná $(f+g)'(a)$;
$\boldsymbol{3.}$ nemožno (ak zvolíme $f_{1}(x)=g_{1}(x)=\vert x \vert$, $f_{2}(x)=x+\vert x \vert$, $g_{2}(x)=x-\vert x \vert$, $f_{3}(x)=(1+ \sgn x)\sqrt[3]{x}$, $g_{3}(x)=(1- \sgn x)\sqrt[3]{x}$, tak $(f_{1}+g_{1})'(0)$ neexistuje, $(f_{2}+g_{2})'(0)$ je vlastná a $(f_{3}+g_{3})'(0)$ je nevlastná);
$\boxed{292.}$ dôkaz prebieha analogicky ako dôkaz vety o derivácii zloženej funkcie pre prípad vlastvých derivácií;
$\boxed{293.}$ 
$\boldsymbol{1.}$ $\ y'= \frac{1}{\ln 3 . \sin x}, \, x \in \bigcup\limits_{k \in \mathbb{Z}}(2k\pi, (2k+1)\pi)$ (pozri poznámku k pr. 289.3);
$\boldsymbol{2.}$ $\ y'= \frac{x}{x^{4}-1}$, $\vert x\vert >1 $;
$\boldsymbol{3.}$ $\ y'= \frac{1}{\sqrt{x^{2}+1}}$;
$\boldsymbol{4.}$ $\ y'= \frac{1}{(1+x^{2})(1+x)^{2}}$, $ x >-1 $;
$\boldsymbol{5.}$ $\ y'= \ln (x+\sqrt{1+x^{2}})$;
$\boldsymbol{6.}$ $\ y'= \frac{(2x+1) e^{\sqrt{\log _{2}(x^{2}+x+1)}}}{\ln 4(x^{2}+x+1)\sqrt{\log_{2}(x^{2}+x+1)}}$;
$\boldsymbol{7.}$ $\ y'= \frac{2\sin a}{(1-x^{2})(1-x^{2}\cos ^{2}a)}$,$\vert x\vert <1 $ ;
$\boldsymbol{8.}$ $\ y'= 2\sin \ln x $;
$\boldsymbol{9.}$ $\ y'= \frac{x^{2}+\sgn x}{x\sqrt{x^{2}+1}}$;
$\boldsymbol{10.}$ $\ y'= \frac{1}{x}(\frac{\ln x. \log _{3}x}{\ln 2}+\log _{2}x.\log _{3}x+\frac{\log _{2}x.\ln x}{\ln 3})$;
$\boxed{294.}$ 
$\boldsymbol{1.}$ $\ y'=\frac{x+2}{2\sqrt{x}(x+1)} $;
$\boldsymbol{2.}$ $\ y'=\arcsin x \frac{x}{\sqrt{(1-x^{2})}} $;
$\boldsymbol{3.}$ $\ y'=\frac{\pi}{2\sqrt{(1-x^{2})}\arcsin ^{2}x} $; využite pr. 62.1);
$\boldsymbol{4.}$ $\ y'=\frac{1}{2\sqrt{x}(1-x^{2})} $;
$\boldsymbol{5.}$ $\ y'=\arcsin \sqrt {\frac{x}{1+x}}, \, x\geq 0 $ (tabuľkovým derivovaním možno nájsť hodnoty $\ y'$ len pre $x>0$ pretože v bode $0$ má funkcia $\sqrt{x}$ nevlastnú deriváciu; derivácia funkcie $x \arcsin \sqrt{\frac{x}{1+x}} + \arctan \sqrt{x}-\sqrt{x}$ v bode $0$ sa vypočíta na základe definície derivácie;
$\boldsymbol{6.}$ $\ y'=\frac{1}{2x\sqrt{x-1}\arccos \frac{1}{\sqrt{x}}} $;
$\boldsymbol{7.}$ $\ y'=\frac{3^{\arctan (2x+\pi)}.\ln 9}{1+(2x+\pi)^{2}} $;
$\boldsymbol{8.}$ $\ y'=\frac{\sin x. \sgn (\cos x-\cos a)}{1-\cos a. \cos x} $, $\cos x\ne \cos a $ (využite rovnosť $(\cos x- \cos a )^{2}=(1-\cos x. \cos a )^{2}-(\sin x. \sin a )^{2}$ ;
$\boldsymbol{9.}$ $\ y'=(\arcsin x)^{2} $; (tabuľkovým derivovaním možno $\ y'$ nájsť len pre $ \vert x\vert <1$, pretože $arcsin  $ má v bodoch $1,-1 $ nevlastnú deriváciu; platnosť uvedeného vzťahu pre body $1,-1 $ by sa nemusela overovať z definície; neskôr - pozri pr. 384 - uvedieme efektívnejší spôsob nájdenia derivácie v bodoch $1,-1 $);
$\boldsymbol{10.}$ $\ y'=\frac{\cos x}{\sqrt{3-2\sin ^{2}x}} $;
$\boxed{295.}$ 
$\boldsymbol{2.}$ $\ y'=\frac{24-x-5x^{2}}{3\sqrt{x-1}\sqrt[3]{(x+2)^{5}}\sqrt{(x+3)^{5}}} $; (uvedený postup možno použiť pre $x>1$; derivácia v bode $1$  je nevlastná;
$\boldsymbol{3.}$ $\ y'=\frac{ctg 3x}{1-\sin 3x}.\sqrt[3]{\frac{\sin 3x}{1-\sin 3x}} $, $x\in (0,\frac{\pi}{6} )$;
$\boldsymbol{5.}$ $\ y'=\frac{1}{2}.(1-\ln x) \sqrt[x]{x} $;
$\boldsymbol{6.}$ $\ y'=x(2\ln x+1)x^{x^{2}} $;
$\boldsymbol{7.}$ $\ y'=(\cos x)^{\sin x}(\cos x.\ln\cos x-\tan x.\sin x)^{\cos x}(\sin x.\ln \sin x-\cos x. ctg x)$, $x\in (0,\frac{\pi}{2} ) $ ;
$\boxed{296.}$ 
$\boldsymbol{1.}$ napr. $f(x)= g(x)=\vert x-a\vert $;
$\boldsymbol{2.}$ napr. $f(x)=x-a, \, g(x)=\vert x-a\vert $;
$\boxed{297.}$ dôkaz sa robí analogicky ako v prípade vlastnej $\ f'(a)$;
$\boxed{299.}$ 
$\boldsymbol{1.}$ $\ y'=\frac{\sqrt{b^{2}-a^{2}}}{a+b \cos x},\,x\in (0,\frac{\pi}{2} )  $;
$\boldsymbol{2.}$ $\ y'=-2\cos x.\arctan(\sin x) $;
$\boldsymbol{3.}$ $\ y'=-\frac{\arccos x}{x^{2}},\,0<\vert x\vert<1 )  $;
$\boldsymbol{4.}$ $\ y'=\ln^{2}(x+\sqrt{1+x^{2})}$;
$\boldsymbol{5.}$ $\ y'=\frac{a^{2}+b^{2}}{(x^{2}+b^{2})(x+a)}$, $x > -a$;
$\boldsymbol{6.}$ $\ y'=\frac{1}{x^{4}+1}$, $ \vert x \vert\ne 1$;
$\boldsymbol{7.}$ $\ y'=-e^{-x}arcctg\, e^{x}$;
$\boldsymbol{8.}$ $\ y'=\frac{\cos x.\sin ^{2}x}{\sqrt{\cos ^{2}x-2\sin x}}$, $ \sin x \leq\sqrt{2}- 1$ (bolo možné použiť substitúciu $t=1+\sin x $);
$\boldsymbol{9.}$ $\ y'=\frac{m}{\sqrt{1-x^{2}} }.2\cos (m\arcsin x)e^{m \arcsin x},\,\vert x\vert<1 )$;
$\boldsymbol{10.}$ $\ y'=\frac{x}{x^{2}-2x\cos a+1}$;
$\boldsymbol{11.}$ $\ y'=(5x-1)(x-1)(x+1)^{2}$ pre $x > -1$, $\ y'=-(5x-1)(x-1)(x+1)^{2}$ pre $x < -1$, $y_{+}'(-1)=y_{-}'(-1)=0$, teda $y'(-1)=0$; to možno naraz zapísať v tvare $\ y'=(5x-1)(x-1)(x+1)^{2}\sgn (x+1)$;
$\boldsymbol{12.}$ $\ y'=\pi n \sin 2\pi x$ pre ; $x \in (n,n+1),\, y_{+}'(n)=y_{-}'(n)=0$, teda $y'(n)=0$ ($n\in \mathbb{Z}$); to možno naraz zapísať  $y'=\pi[x] \sin 2\pi x$;
$\boldsymbol{13.}$ $\ y'= \left\{\begin{matrix} -1, & \mbox{ak }x <1 \\ 2x-3, & \mbox{ak }x \in \langle 1,2\rangle \\ 1, & \mbox {ak  }x >2\end{matrix}\right. $;
$\boldsymbol{14.}$ $\ y'= \left\{\begin{matrix} \frac{1}{1+x^{2}}, & \mbox{ak } -1<x\leq 1 \\ \frac{1}{2}, & 
         \mbox{ak   } \vert x \vert > 1\end{matrix}\right.$; $y_{+}'(-1)=\frac{1}{2}$, $y_{-}'(-1)=+\infty$, teda $y'(-1)$ neexistuje;
$\boldsymbol{15.}$ $\ y'=th^{3}x$, pritom sa použije vzťah $\frac{1}{ch^{2}x}=1-th^{2}x$;  
$\boldsymbol{16.}$ $\ y'=(\ln x)^{x}.(\ln(\ln x)+\frac{1}{\ln x}-\frac{2\ln x}{x}/(x^{\ln x})$, $ x >1$;
$\boldsymbol{17.}$ $\ y'=(\frac{\sin 2x . \arctan x}{\sqrt{1-\sin ^{4}x}\arcsin (\sin ^{2}x}+\frac{\ln \arcsin (\sin ^{2}x)}{1+x^{2}}).(\arcsin (\sin ^{2}x))^{\arctan x}$;
$\boldsymbol{18.}$ $\ y'=x^{x^{x}}.x^{x}(\ln ^{2}x+\ln x+\frac{1}{x})$;
$\boldsymbol{19.}$  použitím vzťahu $\log _{x }e=\frac{1}{\ln x}$   možno daný výraz upraviť na $y=e^{2}-2xe^{2}+ex^{2}$  $x>0, \, x\ne 1)$; teda $\ y'=ex(x-e)$,  $x>0, \, x\ne 1)$;
$\boxed{300.}$ $a=2x_{0}$, $b=-x_{0}^{2}$;
$\boxed{301.}$ dôkaz sa robí matematickou indukciou, pričom sa použije veta o derivácii súčinu; $f'(0)=10001$;
$\boxed{302.}$ $\boldsymbol{1.}$ $\ y'=\frac{ff'+gg'}{\sqrt{f^{2}+g^{2}}}$;
$\boldsymbol{2.}$ $\ y'=\frac{f'g+fg'}{f^{2}+g^{2}}$;
$\boldsymbol{3.}$ $\ y'=\frac{fg'.\ln f+gf'.\ln g}{f.g.\ln^{2}f}$ (treba použiť vzťah  $\log _{f}g=\frac{\ln g}{\ln f}$);
$\boldsymbol{4.}$ $\ y'=2x f'(x^{2})$;
$\boldsymbol{5.}$ $\ y'=\sin 2x f'(\sin ^{2}x)-g'(\cos ^{2}x))$;
$\boldsymbol{6.}$ $\ y'=e^{f(x)}(e^{x}.f'(e^{x})+f(e^{x}).f'(x))$;
$\boxed{303.}$ $\boldsymbol{1.}$ $P_{n}(x)=(x+x^{2}+...+x^{n})'=(\frac{x^{n+1}-x}{x-1})'=\frac{nx^{n+1}-(n+1)x^{n}+1}{(x-1)^{2}} $ pre $x \ne 1$,  $P_{n}(1)=\frac{n(n+1)}{2} $;
$\boldsymbol{2.}$ $Q_{n}(x)=(\frac{x^{2n+1}-x}{x^{2}-1})'=\frac{(2n-1)x^{2n+2}-(2n+1)x^{2n}+x^{2}+1}{(x^{2}-1)^{2}}$; 
$\boldsymbol{3.}$ $R_{n}(x)=(xP_{n}(x))'=\frac{n^{2}x^{n+2}-(2n^{2}+2n-1)x^{n+1}+(n+1)^{2}x^{n}-x-1}{(x-1)^{3}}$; 
$\boldsymbol{4.}$ $T_{n}(x)=(\sum_{k=1}^n \sin kx)'=(\frac{\cos \frac{x}{2}- \cos \frac{(2n+1)\pi}{2}}{2\sin \frac{x}{2} })'=\frac{-1+(n+1)\cos nx -n \cos (n+1)x}{4 \sin ^{2}\frac{x}{2}}$;
$\boxed{304.}$ $\boldsymbol{1.}$ $\alpha >0 $;
$\boldsymbol{2.}$ $\alpha >1 $;
$\boldsymbol{3.}$ $\alpha >2 $ (pre $x \ne 0$ možno  $f'(x)$ nájsť tabuľkovým derivovaním, $f'(0)$ sa musí hľadať priamo z definície; treba si uvedomiť, že $\lim_{x \to 0}\cos \frac{1}{x} $ neexistuje, $\lim_{x \to 0}x^{\beta} \sin \frac{1}{x}=0 $ pre $\beta >0 $; pre $\beta \leq 0 $ táto limita neexistuje);
$\boxed{305.}$ nie (funkcia $\sqrt[3]{x^{2}}$ nemá vlastnú ani nevlastnú deriváciu v bode $0$; jednostranné derivácie sú nevlastné a opačných znamienok); $y'(0)=0$ (vypočíta sa z definície);
$\boxed{306.}$ $\boldsymbol{1.}$ napr. $f(x)=(x-1)^{2}(x-2)^{2} $ pre $x \in \mathbb{Q}, \, f(x)=0 $ pre $x \in \mathbb{R} -\mathbb{Q}$;
$\boldsymbol{2.}$ napr. funkcia určená podmienkami $f(x)=\sqrt[3]{\frac{x}{4}}$ pre $x \in \langle -\frac{1}{2},\frac{1}{2}\rangle$, $f(x+1)=f(x)+1 $ ($x \in \mathbb{R}$);
$\boldsymbol{3.}$ napr. $f(x)=ax+g(x)$, kde $g$ je určená podmienkami $g(x)=0$ pre $x \leq 0, \, g(x)=x^{2}(x-1)^{2}\sin \frac{1}{x(x-1)} $  pre $x \in (0,1), \,g(x+1)=g(x) $ pre $x \geq 0 $;
$\boxed{308.}$ napr. $f(x)=(1+\sgn (x-1)). \frac{\arccos(\cos \pi(x-1)}{\pi}$, tj. $f(x)=0 $ pre $x<1 $, $f(x)=x-k $ pre $x\in \langle k, k+1), \, k \in \mathbb{N} $, $k$ nepárne, $f(x)=k+1-x$ pre $x\in \langle k, k+1), \, k \in \mathbb{N} $, $k$ párne;
$\boxed{309.}$ $\boldsymbol{1.}$ napr. $f(x)=-\sqrt[3]{x-a}$;
$\boldsymbol{2.}$ napr. $f(x)=\sgn (x-a)$;
$\boldsymbol{3.}$ napr. $f(x)=\sgn (x-a).(\sin \frac{1}{x-a}-2)$ pre  $x \ne a $, $f(a)=0$;
$\boxed{310.}$ $\boldsymbol{1.}$ $\beta /b=f(\sqrt{2}), \, (f^{-1})'(\frac{6}{5})=\frac{1}{2}$;
$\boldsymbol{2.}$ $b=f(0), \, (f^{-1})'(-\frac{1}{2})=\frac{1}{2}$;
$\boldsymbol{3.}$ $b=f(0), \, (f^{-1})'(1)=5$; 
$\boldsymbol{4.}$ $b=f(\sqrt{2}), \, (f^{-1})'(0)= - \frac{1}{4\sqrt{2}}$;
$\boldsymbol{5.}$ $b=f(\frac{1}{\sqrt{2}}), \, (f^{-1})'(\frac{3}{4})= \frac{1}{\sqrt{2}}$;
$\boxed{311.}$ pretože $f^{-1}(f(x))=x $ v niektorom okolí bodu $a$, je $f^{-1}(f(a)).f'(a)=1 $; treba predpokladať existenciu vlasnej  $(f^{-1})'(f(a)) $ a vlasnej  $f'(a)\ne 0 $;
$\boxed{312.}$ $\boldsymbol{1.}$ $f'(x)= (1-x^{2})^{-1/2}$;
$\boldsymbol{2.}$ $f'(x)= -(1-x^{2})^{-1/2}$;
$\boldsymbol{3.}$ $f'(x)= \frac{1}{1+x^{2}}$;
$\boldsymbol{4.}$ $f'(x)= -\frac{1}{1+x^{2}}$;
$\boldsymbol{5.}$ $f'(x)= \frac{1}{x}, \, x>0$;
$\boxed{313.}$ dôkaz je analogický dôkazu vety o derivácii inverznej funkcie;
$\boxed{314.}$ $\boldsymbol{1\alpha/}$ $y= \sqrt[3]{4}(x+1)$; $\boldsymbol{1\beta/}$ $y= 3$;
$\boldsymbol{2.}$ $y= x+1$;
$\boldsymbol{3.}$ $x=1$;
$\boldsymbol{4.}$ $A=(0,1), y=1$;
$\boldsymbol{5.}$ $A=(1,\frac{1}{2}), x+2y-2=0$;
$\boxed{315.}$ $\boldsymbol{1.}$ $12x-4y-13=0$;
$\boldsymbol{2.}$ $4x-4y+3=0$;
$\boxed{316.}$ $\boldsymbol{1.}$ $b^{2}-4ac=0$;
$\boldsymbol{2.}$ $a=\frac{1}{2e}$;
$\boxed{317.}$ $\boldsymbol{1.}$ $d(\frac{\ln x}{\sqrt{x}})(a)=\frac{2- \ln a}{2a\sqrt{a}}\mathrm{d}x (a)$; (teda  $d(\frac{\ln x}{\sqrt{x}})(a)$ je funkcia daná predpisom $\frac{2- \ln a}{2a\sqrt{a}}(x-a)$);,
$\boldsymbol{2.}$ $\frac{a}{\sqrt{A^{2}+a^{2}}}\mathrm{d}x (a)$;
$\boldsymbol{3.}$ $\frac{2a}{a^{2}-1}\mathrm{d}x (a)$ ($\vert a \vert <1) $;
$\boldsymbol{4.}$ $\frac{1}{\cos ^{3}a}\mathrm{d}x (a)$;
$\boxed{318.}$ $\boldsymbol{1.}$ $dy(a)=(u(a).v(a).w'(a)+u(a).v'(a).w(a)+u'(a).v(a).w(a) )\mathrm{d}x (a)$, čo sa stručne zapisuje v tvare $dy=u.v.dw+u.dv.w+du.v.w$;
$\boldsymbol{2.}$ $dy=\frac{v^{2}.du-2u.v.dv}{v^{4}}$;
$\boldsymbol{3.}$ $dy=\frac{w.du-u.dw}{u^{2}+w^{2}}$;
$\boldsymbol{4.}$ $dy=\frac{u.du-v.dv}{u^{2}+v^{2}}$;
$\boxed{319.}$ $\boldsymbol{1.}$ $\sqrt{a^{2}+x^{2}}\approx a+\frac{x}{2a}$ pre $x$ blízke $0$;
$\boldsymbol{2.}$ $\ln{x}+\sqrt{1+x^{2}} \approx x$ pre $x$ blízke $0$;
$\boldsymbol{3.}$ $\arctan (1+x^{2}) \approx \frac{\pi}{4}+\frac{1}{2} x$ pre $x$ blízke $0$;
$\boldsymbol{4.}$ $\sqrt[n]{a^{n}+x}\approx a+\frac{x}{na^{n-1}}$ pre $x$ blízke $0$;
$\boxed{320.}$ "$\Rightarrow $": stačí položiť $\varphi (x)=A+\frac{\omega(x)}{x-a}$ pre $x \ne a$, $\varphi(a)=0$; "$\Leftarrow $": $\varphi(x)=0$ možno písať v tvare $A+(\varphi(x)-A)$, kde $A=\lim_{x \to a}\varphi(x)$; potom $\varphi(x)=(\varphi(x)-A)(x-a)$;
$\boxed{321.}$ $\boldsymbol{1.}$  $y''=\frac{2x^{3}+3x}{(1+x^{2})^{3/2}} $;
$\boldsymbol{2.}$  $y''=3x(1-x^{2})^{-5/2} $;   
$\boldsymbol{3.}$  $y''=2(2x^{2}-1)e^{-x^{2}} $;   
$\boldsymbol{4.}$  $y''=2\arctan x+\frac{2x}{1+x^{2}} $;
$\boldsymbol{5.}$  $y''=\frac{1}{x}, \, x>0 $;
c
$\boxed{322.}$ $\boldsymbol{2.}$  $y^{V}=2^{2}.3^{3}.5! $; $y^{VI}=0$;
$\boldsymbol{3.}$  $y^{(10)}=-(3.5.7.....17)2^{-10}.x^{-19/2}$;
$\boldsymbol{4.}$  $y=-1+\frac{2}{1-x}, \, y^{(22)}=2.221.(1-x)^{-23} $;
$\boldsymbol{6.}$  $y=\frac{1}{2}(\sin 6x-\sin 2x), \, y^{(15)}=\frac{1}{2}(2^{15}\cos 2x -6^{15}\cos 6x) $;
$\boldsymbol{7.}$  $y=\frac{1}{2}\sin 2x(\cos 2x-\cos 4x)= \frac{1}{4}\sin 4x -\frac{1}{4}(\sin 6x-\sin 2x) , \, y^{(10)}=\frac{1}{4}(-4^{10}\sin 4x +6^{10}\sin 6x-2^{10}\sin 2x) $;
$\boxed{323.}$ $\boldsymbol{1.}$  $((1-x)^{-1/2})^{(100)}=\frac{3.5.....199}{2^{100}}(1-x)^{-201/2} $, $((1-x)^{-1/2})^{(99)}=\frac{3.5.....197}{2^{99}}(1-x)^{-199/2} $, $y^{(100)}=\frac{3.5.....197}{2^{100}}\sqrt{(1-x)^{201}}(399-x) $;
$\boldsymbol{2.}$  $y^{V}=\frac{-6}{x^{4}}, x>0$;
$\boldsymbol{3.}$  $y^{(50)}=2^{49}(1225-2x^{2})\sin 2x + 50.2^{50}x \cos 2x$;
$\boldsymbol{4.}$  $y'''=-\frac{9\sin 3x}{\sqrt[7]{(1-3x)^{3}}}(27x^{2}-18x-1)-\frac{\cos 3x}{\sqrt[3]{(1-3x)^{10}}}(243x^{2}-162x-1)$;
$\boldsymbol{5.}$  $y^{(101)}=3^{99}(10100+18x-9x^{2})\sin 3x + 202.3^{100}(x-1). \cos 3x$;
$\boldsymbol{6.}$ Leibnizovým vzorcom sa vypočíta  $15$. derivácia funkcie; $y'=2x+\sin 2x - 2 \sin x-2x \cos x$,   $y^{(16)}=-2^{15}\cos 2x+32 \cos x - 2x \sin x$; 
$\boldsymbol{7.}$  $y=\frac{1}{2}x(\sin 3x -\sin x)$, $y^{(100)}=\frac{1}{2}(x(3^{100}\sin 3x \sin x)+100\cos x - 100.3^{99}\cos 3x)$;
$\boxed{325.}$ $\boldsymbol{1.}$ $y^{n}=\frac{(-1)^{n-1}(n-1)!a^{n}}{(ax+b)^{n}}, ax+b>0$;
$\boldsymbol{2.}$ $y^{(n)}=n![\frac{(-1)^{n}}{x^{n+1}}+\frac{1}{(1-x)^{n+1}}](y=\frac{1}{x}+\frac{1}{1-x}) $;
$\boldsymbol{3.}$ $y^{(n)}=\frac{(-1)^{n}n!}{(x-2)^{n+1}}+\frac{n!}{(1-x)^{n+1}}(y=\frac{1}{x-2}+\frac{1}{1-x}) $;
$\boldsymbol{4.}$ $y^{(n)}=\frac{1.3.....(2n-1)}{(1-2x)^{2n+1}}$;
$\boldsymbol{5.}$  $\ y'=\sin 2x$;  $y^{(n)}=(\sin 2x)^{(n-1)}=2^{n-1}\sin (2x+\frac{(n-1)\pi}{2})$;
$\boldsymbol{6.}$  $\ y^{(n)}=\frac{1}{2}(8^{n}\sin (8x+\frac{n\pi}{2})-2^{n}\sin (2x+\frac{n\pi}{2}))$  $y=\frac{1}{2}(\sin 8x- \sin 2x))$;
$\boldsymbol{7.}$  $\ y^{(n)}=4^{n-1}\cos (4x+\frac{n\pi}{2})$ (pozri návod k pr. 40.2);
$\boldsymbol{8.}$  $\ y^{(n)}=\frac{1}{2} \sin ax.(\sin (a+b)x +\sin (a-b)x)=\frac{1}{4}(2\cos bx-\cos(2a+b)x -\cos (2a-b)x)$, $\ y^{(n)}=\frac{1}{4} (2b^{n} \cos(bx+\frac{n\pi}{2})-(2a+b)^{n}\cos ((2a+b)x+\frac {n\pi}{2})-(2a-b)^{n}\cos ((2a-b)x+\frac{n\pi}{2}))$;
$\boxed{326.}$ $\boldsymbol{1.}$ $y^{(n)}=2^{x-1}.\ln^{n}2 (x-1+\frac{n}{\ln 2}) $ (použil sa Leibnizov vzorec);
$\boldsymbol{2.}$ $((1+x)^{-1/3})^{(n)} =(-1)^{n}\frac {1.4.7.....(3n-2)}{3^{n}}(1+x)^{-(3n+1)/3}$,  $y^{(n)}=(-1)^{n+1}\frac{1.4.7.....(3n-5)}{3^{n}(1+x)^{n+1/3}}.(2x+3n)$ pre $n\geq 2$, $\ y'=\frac{3+2x}{\sqrt[3]{(1+x)^{4}}}$ (výsledok pre $\ y'$ sa musí uviesť samostatne, pretože vzťah pre $((1+x)^{-1/3})^{(n-1)}$, ktorý dosádzame do Leibnizovho vzorca, neplatí pre $n=1$);
$\boldsymbol{3.}$ $y^{n}=b^{n-2}.\sin (bx+\frac{n\pi}{2})(b^{2}x^{2}-n(n-1))+2nb^{n-1} x \sin (bx+\frac{n-1}{2}\pi )$ (pri úprave sa použil vzťah $\sin (A-\pi)=-\sin A$; pre $n\geq 2$ má Leibnizov vzorec tvar  $y^{(n)}=x^{2}(\sin bx)^{(n)}+ \begin{pmatrix}n \\1 \end{pmatrix} 2x.(\sin bx)^{(n-1)}+ \begin{pmatrix}n \\2 \end{pmatrix} 2.(\sin bx)^{(n-2)}$, $\ y'$ treba vypočítať samostatne, zhodou okolností možno výsledky písať v uvedenej spoločnej podobe);
$\boldsymbol{4.}$ $y^{(n)}=(-1)^{n}e^{-x}(x^{2}+(n-1)(n-2))$;
$\boldsymbol{5.}$ $(\log _{2}(1-3x))^{n}=-\frac{(n-1)!3^{n}}{(1-3x)^{n}\ln 2}$,  $y^{(n)}=-\frac{(n-2)!3^{n-1}}{(1-3x)^{n}\ln 2}(3x-n)$ pre $n>1$, $\ y'=\log _{2}(1-3x)-\frac{3x}{(1-3x)\ln 2}$ ($\ y'$ uvádzame samostatne z toho istého dôvodu ako v pr. 326.1); 
$\boldsymbol{6.}$ $y^{(n)}=(-3)^{n-2}e^{2-3x}(36x^{2}-12(9+2n)x+4n^{2}+32n+81)$ pre $n\geq 2$, $\ y'=e^{2-3x}(3-2x)(6x-7)$ ($\ y'$ treba vypočítať samostatne, pretože Leibnizov vzorec pre $n\geq 2$ má iný tvar ako pre $n=1$);
$\boxed{327.}$ $\boldsymbol{1.}$ $f(x)=\frac{2}{3}.\frac{1}{1-2x}+\frac{1}{3}.\frac{1}{1+x}$, $f^{(n)}(0)=\frac{n!}{3}(2^{n+1}+(-1)^{n})$;   
$\boldsymbol{2.}$ $f^{(n)}(x)=2^{n}x^{2}e^{2x}+2^{n}nxe^{2x}+n(n-1)2^{n-2}e^{2x}$ (použili sme Leibnizov vzorec pre $n\geq 2$, samostatne vypočítali $\ y'$ a presvedčili sa, že sa dá vyjadriť tým istým vzťahom), $f^{(n)}(0)=n(n-1)2^{n-2}$; 
$\boldsymbol{3.}$ $f^{(n)}(x)=\frac {(2n-x)}{2^{n}(1-x)^{n+1/2}}\prod_{k=1}^{n-1} (2k-1)$ pre $n\geq 2$, $ f'(x)=\frac{2-x}{2(1-x)^{3/2}}$; $f^{(n)}(0)=\frac{n}{2^{n-1}}\prod_{k=1}^{n-1} (2k-1)$ pre $n\geq 2$, $f^{(1)}(0)=1$;
$\boldsymbol{4.}$ $f(x)=\frac{1}{2}(\frac{1}{1-x)}-\frac{1}{1+x})$, $f^{(n)}(0)=\frac{1}{2}n!(1-(-1)^{n})$;$\boldsymbol{5.}$ platí $(1-x^{2}).f''(x)-x.f'(x)=0$, odtiaľ $f^{(n)}(0)=(n-2)^{2}(0)$ (odvodí sa Leibnizovým vzorcom za predpokladu $n\geq 4$, ale zhodou okolností platí tento rekurentný vzťah aj pre $n=3$; vzťah pre  $n=3$ dostaneme, ak prvú uvedenú identitu raz zderivujeme);  $f'(0)=1$, $f''(0)=0$, $f^{(2k+2)}(0)=0$, $f^{(2k+1)}(0)=1^{2}.3^{2}.....(2k-1)^{2}$  ($k=0,1,...$);
$\boldsymbol{7.}$ $f(x)=g(x).g(x)$, kde $g(x)=\frac{1}{2}(\frac{1}{1-x}+\frac{1}{1+x})$; $g^{(k)}(0)=0$    pre $k$ nepárne, $g^{(k)}(0)=k!$ pre $k$ párne; na súčin  $g.g$  použijeme Leibnizov vzorec;  $f^{(n)}(0)=0$   pre $n$ nepárne,  $f^{(2n)}(0)=\sum_{k=0}^m(2 k)!$;
$\boxed{329.}$  nepriamo, ak nie je prostá, tak vyhovuje predpokladom Rolleho vety na niektorom intervale;
$\boxed{330.}$ $n+1$ nulových bodov vytvorí $n$ intervalov, na každom z nich sú splnené predpoklady Rolleho vety; tak dostaneme $n$ nulových bodov funkcie $f'$, tie vytvoria $n-1$ intervalov, naktorých $f'$ vyhovuje predpokladom Rolleho vety atď.;
$\boxed{331.}$ Ak $x_{1}<x_{2}<...<x_{k}$ sú korene polynómu $P_{n}$, tak podľa návodu k pr. 330 existuje $k-1$ koreňov polynómu $P'_{n}$, tak, že $x_{1}<c_{1}<x_{2}<c_{2}...<c_{k-1}<x_{k}$. Ďalej: ak $x_{i}$ je m-násobný koreň $P_{n}$ ($m\geq 2$), tak $x_{i}$ je $m+1$-násobný koreň $P'_{n}$. Súčet násobností takto získaných koreňov je $n-1$ (podľa základnej vety algebry nemôže byť väčší než $n-1$; ak násobnosť koreňa $c_{i}$ ($i=1,...,k-1$) zdola odhadneme číslom $1$, je číslo $n-1$ aj dolným odhadom súčtu všetkých násobností.
$\boxed{332.}$ z návodu k pr.331 vyplýva: ak súčet násobností reálnych koreňov polynómu $P_{n}$ je  $k$ tak súčet násobností reálnych koreňov $P'_{n}$ je aspoň $k-1$ ($n\geq 2, k\geq 2 $); derivácia polynómu $x^{3}-3x^{2}+6x-1$ nemá reálne korene;
$\boxed{333.}$ treba dokázať, že existujú body $c,d \in \mathbb{R}$, $a<c<d<b$, v ktorých $f(c)=f(d)$; na intervale $\langle c,d\rangle$  potom $f$ vyhovuje predpokladom  Rolleho vety; 
$\boxed{334.}$ $f$ musí byť spojitá na $(a,b)$; keby bola naviac spojitá aj v bodoch $a,b$, vyhovovala by predpokladom  Rolleho vety, čo by bolo v spore z podmienkou $f'(x)\ne 0$ pre všetky  $x \in(a,b)$; zadaniu vyhovuje napr. funkcia  $f$ daná predpisom  $f(x)=x$ pre  $x \in(a,b\rangle $,  $f(x)=b$;
$\boxed{335.}$  nie je;  $y$ totiž nemá vlastnú ani nevlastnú deriváciu v bode $0$;
$\boxed{336.}$ $\boldsymbol{1.}$ zvoľme pevne $a\in I$; ak $x\in I$, tak podľa Lagrangeovej vety  $f(x)-f(a)=0.(x-a)$, tj. $f(x)=f(a)$;
$\boldsymbol{2.}$ z pr. 336.1 vyplýva, že $M$ nemôže byť otvorený interval; zadaniu vyhovuje napr. $M=\mathbb{R}-\lbrace 0\rbrace$, $f(x)=\sgn x/M$;
$\boxed{338.}$ postup je rovnaký ako v pr. 337, v pr. 338.3 treba robiť úvahy zvlášť pre $(-\infty, -1)$ a zvlášť pre $(1,\infty)$ (množina $(-\infty, -1)\cup (1,\infty)$ nie je interval);
$\boxed{339.}$ uvedieme dva návody: 1. zvoľme pevne $a \in I $; ak $x \in I $, tak podľa Lagrangeovej vety  $f(x)-f(a)=k.(x-a)$; 2. funkcia $g(x)=kx+b $ má požadovanú vlastnosť ($g'(x) * k$ na $I$), ak ju má aj funkcia $f$, je $(g-f)'=0$ na $I$, podľa pr. 336 potom $g-f$ $* const$ na $I$;
$\boxed{340.}$ sporom; ak $f'(x)=\sgn x$, tak podľa pr. 339 $f(x)=\vert x\vert +c$, ale  $\vert x\vert +c$ nemá deriváciu v bode $0$;
$\boxed{341.}$ $\boldsymbol{2.}$ treba využiť, že z nerovnosti $0<y<c<x$ vyplýva pre $p>1$ nerovnosť $y^{p-1}<c^{p-1}<x^{p-1}$;
$\boldsymbol{4.}$  $\ln \frac{a}{b}= \ln a-\ln b$; ak $0<b<c<a$, tak $\frac{1}{a}<\frac{1}{c}<\frac{1}{b}$;
$\boxed{342.}$ $\boldsymbol{1.}$ $f$ je na $(a,b)$ spojitá (má totiž v každom bode deriváciu), na každom intervale $\langle x,y \rangle \subset (a,b)$ vyhovuje predpokladom Lagrangeovej vety; ak $\vert f'(x)\vert \leq K$ pre všetky $x \in (a,b)$, tak $\vert f(x)-f(y)\vert \leq K \vert x-y\vert\quad (x,y \in (a,b)$), odtiaľ vyplýva naše tvrdenie na základe definície rovnomernej spojitosti;
$\boxed{343.}$ nepriamo;  ak $\vert f'(x)\vert \leq K$   $x \in (a,b)$, tak použitím Lagrangeovej vety $\vert f(x)-f(A)\vert \leq K \vert x-A\vert \leq K(b-a)$  ($A \in (a,b)$ je pevne zvolené), teda $ \vert f(x) \vert \leq \vert f(A) \vert + \vert f(x)-f(A)  \vert \leq  \vert f(A) \vert+K(b-a)$, tj. $f$ je uhraničená na $(a,b)$;
$\boxed{344.}$ zvoľme $a \in I$ pevné; pre $a \in I, \, x \ne a$ označme $z(x)$ to číslo $z$ ležiace medzi $x,a $ pre ktoré $F(x) - F(a) = f(z)(x-a)$ (ak je takých čísel viacej, vyberieme jedno z nich); dokážte implikáciu: ak $x\rightarrow a$, tak $z(x) \rightarrow a$; potom $F'(a)=\lim_{x \to a}\frac{F(x) - F(a)}{x-a}=\lim_{x \to a}f(z(x))=\lim_{z \to a}f(z)=f(a)$ ($f$ je na $I$ spojitá);
$\boxed{345.}$ nie; stačí uvážiť $f(x)=x^{3}$, interval $\langle -1,1 \rangle$, $c=0$;
$\boxed{346.}$ číslo $c_{1}$, pre ktoré $f(b)-f(a)=f'(c_{1})(b-a)$, nemusí byť totožné s číslom $c_{2}$, pre ktoré $g(b)-g(a)=g'(c_{2})(b-a)$; dokumentujeme to na príklade $f(x)=x^{3}$, $g(x)=x^{2}$, $a=0$, $b=1$; potom $f(1)-f(0)=f'(\frac{1}{\sqrt{3}})(1-0)$,  $g(1)-g(0)=g'(\frac{1}{2})(1-0)$, ale $\frac{f(1)-f(0)}{g(1)-g(0)}=\frac{f'(\frac{2}{3}}{g'(\frac{2}{3}})$;
$\boxed{347.}$ stačí použiť Cauchyho vetu pre funkcie $f(x)$ a $\frac{1}{x}$ na intervale $\langle 1,2 \rangle$ ( a samozrejme predtým preveriť splnenie predpokladov (i), (ii));
$\boxed{348.}$ ak rozšírime zlomok na ľavej strane rovnosti výrazom $\frac{1}{a.b}$, vidíme, že treba použiť Cauchyho vetu pre funkcie $\frac{f(x)}{x}$  a  $\frac{1}{x}$ na $\langle a,b \rangle$;
$\boxed{349.}$ $\boldsymbol{1.}$ klesajúca na $(-\infty,-1 \rangle$ a na $\langle 1,\infty) $,rastúca na $\langle -1,1\rangle $;
$\boldsymbol{2.}$ klesajúca na $(-\infty,-1 \rangle$ a na $\langle 1,\infty) $,rastúca na $\langle -1,1\rangle $;
$\boldsymbol{3.}$  na každom z intervalov $((2k-1)\pi,(2k+1)\pi)$ je $y'$ kladná, teda  $y$ rastie na každom z intervalov $\langle (2k-1)\pi,(2k+1)\pi\rangle, \, k\in \mathbb{Z}$, odtiaľ vyplýva, že $y$ rastie na $\mathbb{R}= \bigcup_{k\in \mathbb{Z}}\langle (2k-1)\pi,(2k+1)\pi\rangle $;
$\boldsymbol{4.}$ rastie na každom z intervalov $\langle 2k\pi, \frac {\pi}{3}+2k\pi \rangle$, $\langle \frac{2\pi}{3}+2k\pi,(2k+1)\pi \rangle$, $\langle \frac{7\pi}{6}+2k\pi,\frac{11\pi}{6}+2k\pi \rangle$, klesá na každom z intervalov $\langle \frac{\pi}{3}+2k\pi,\frac{2\pi}{3}+2k\pi \rangle$, $\langle (2k+1)\pi,\frac{7\pi}{6}+2k\pi \rangle$, $\langle \frac{11\pi}{6}+2k\pi,(2k+2)\pi \rangle$, $k \in \mathbb{Z}$;
$\boldsymbol{5.}$ rastie na každom z intervalov $\langle \frac{1}{2k+1}, \frac{1}{2k}\rangle$ ($k=\pm 1,\pm 2,\pm 3,...)$ a na $\langle 1,\infty$, klesá na $\langle \frac{1}{2k+2}, \frac{1}{2k+1}\rangle$ ($k=\pm 1,\pm 2,\pm 3,...)$ a na $(-\infty , -1\rangle$;
$\boldsymbol{6.}$ klesá na $(-\infty , 0\rangle$ a na $\langle 2\log _{2}e,\infty)$, rastie na $\langle 0, 2\log _{2}e\rangle$ ($\log _{2}e=\frac{1}{\ln 2} $);
$\boldsymbol{7.}$ klesá na $(-\infty , -1\rangle$ a na $(0,1\rangle$, rastie na $\langle -1, 0)$ a na $\langle 1,\infty)$;
$\boldsymbol{8.}$ rastie na $\langle e^{-7\pi/12+2k\pi},e^{13\pi/12+2k\pi}\rangle$,   klesá na $\langle e^{13\pi/12+2k\pi},e^{17\pi/12+2k\pi}\rangle$, $k\in \mathbb{Z}$ (všimnite si, že číslo $0$ neleží v žiadnom z týchto intervalov);
$\boxed{350.}$ pre dané $a\in \mathbb{R}$ je funkcia $F(x):= \frac{f(x)-f(a)}{x-a}$ kladná na $\mathbb{R}- \lbrace a \rbrace$, teda $\lim_{x \to a}F(x)\geq 0$;
$\boxed{351.}$ $a$ stačí zvoliť tak, aby všetky korene polynómu $P'_{n}$ ležali v $\langle -a,a \rangle$, potom $P'_{n}$ nemení znamienko na $(-\infty, -a)$ a na $(a, \infty)$;
$\boxed{352.}$ $\boldsymbol{1.}$ označme $f(x)=e^{x}$, $g(x)=1+x$; potom $f(0)=g(0)$, $f'(x)>g'(x)$ pre všetky $x>0$, $f'(x)<g'(x)$ pte všetky $x<0$; odtiaľ už vyplýva na základe vety 12 uvedená nerovnosť;
$\boldsymbol{2.}$ nerovnosť $\sin x< x$ iste platí pre $x>1$, petože vtedy $\sin x\leq 1<x$; ak $f(x):=x$, tak $f(0)=g(0)$, $f'(x)>g'(x)$ pre všetky $x \in (0,1 \rangle$, teda $\sin x<x$ platí aj pre $x \in (0,1 \rangle$ (nerovnosť $f'(x)>g'(x)$ neplatí pre všetky $x>0$ - pre  $x>0$ platí len nerovnosť $f'(x)\geq g'(x)$ - preto vetu 12 nemôžeme použiť na dôkaz nerovnosti $\sin x<x$ na celom intervale $(0, \infty)$); ak $f(x):= \sin x$, $g(x):=x-x^{3}/6 $,tak $f(0)=g(0)$, $f'(0)=g'(0)$ a  pre všetky $x>0 $ platí  $f''(x)>g''(x)$  (to vyplýva z predtým dokázanej nerovnosti $\sin x<x$ pre  $x>0 $);
$\boldsymbol{4.}$ ak $f(x):=\tan x$, $g(x):=x+\frac{x^{3}}{3}$, tak $f(0)=g(0)$, $f'(0)=g'(0)$, $f''(x)=2 \cos ^{-3}x.\sin x$, $g''(x)=2x$; pre  $x \in (0,\frac{\pi}{2})$ je $0<\cos x<1$, teda $\cos^{-2}x>1 $, súčasne  $\tan x>x $ (pozri pr. 352.3), teda $\tan x. \cos^{-2}x=\cos^{-3}x.\sin x>x$; odtiaľ vyplýva $f''(x)>g''(x)$ pre  $x \in (0,\frac{\pi}{2})$;
$\boldsymbol{7.}$ ak  $f(x):=1+2\ln x$, $g(x):=x^{2}$, tak $f(1)=g(1)$, $f'(x)<g'(x)$ pre všetky $x>1$, $f'(x)>g'(x)$ pre všetky $x\in(0,1)$;
$\boxed{353.}$ ak do uvedenej nerovnosti dosadíme $x=\frac{1}{n-1},...,x=\frac{1}{2n}$ a získané nerovnosti sčítame, dostaneme odhad $\ln (\frac{2n+1}{n})<\frac{1}{n}+...+\frac{1}{2n}<\ln (\frac{2n}{n-1}) $, $n>1$;
$\boxed{354.}$ nech $M=\lbrace a_{1},...,a_{m} \rbrace$, $a_{1}<a_{2}<...<a_{m}$; potom $f$ rastie na každom z intervalov  $(a,a_{1} \rangle$, $\langle a_{1},a_{2} \rangle$,..., $\langle a_{m},b)$; odtiaľ vyplýva, že $f$ rastie na $(a,b)$;
$\boxed{355.}$ funkcia $f-g$ je neklesajúca na $\langle a, \infty)$ a platí $f(a)-g(a)>0$; odtiaľ vyplýva, že $f(x)-g(x)\geq f(a)-g(a)>0$ pre všetky $x\geq a$;
$\boxed{356.}$ nie (príslušný protiklad už musíte nájsť sami);
$\boxed{357.}$ $\boldsymbol{1.}$ rýdzo konvexná na $(-\infty ,1\rangle$, rýdzo konkávna na $\langle 1, \infty)$;
$\boldsymbol{2.}$ rýdzo konvexná na $\langle 0, \infty)$, rýdzo konkávna na $(-\infty ,0\rangle$;
$\boldsymbol{3.}$ rýdzo konkávna na každom z intervalov $\langle 2k\pi, (2k+1)\pi \rangle$, rýdzo konvexná na  každom z intervalov $\langle (2k-1)\pi, 2k\pi \rangle$, $k\in \mathbb{Z}$;
$\boldsymbol{4.}$ rýdzo konvexná na $\langle -1,1 \rangle$, rýdzo konkávna na $(-\infty ,-1\rangle$ a na $\langle  1,\infty )$; 
$\boldsymbol{5.}$ rýdzo konvexná na $\langle e^{\pi/4-(2k+1)\pi},e^{\pi/4-2k\pi} \rangle$, rýdzo konkávna na $\langle e^{\pi/4-2k\pi},e^{\pi/4-(2k-1)\pi} \rangle$, $k\in \mathbb{Z}$;
$\boldsymbol{5.}$ rýdzo konkávna na $(-\infty ,0\rangle$ a na $\langle 0, \infty)$, ale nie na ich zjednotení;
$\boxed{358.}$ $\boldsymbol{1.}$ $p,q$ nepárne, $p>q>0$ (funkcia $x^{p/q}$ je pre $x<0$ definovaná len vtedy, keď $p$ je párne alebo $p$ aj $q$ sú nepárne; derivácia v bode $0$ je vlastná len pre $p\geq q$);
$\boldsymbol{2.}$ $9b^{2}-24ac>0$;
$\boldsymbol{3.}$ $\vert a \vert \leq 2$ (pre $\vert a \vert < 2$ je  $y''>0$ na $\mathbb{R}$, teda $y$ je rýdzo konvexná na $\mathbb{R}$; pre  $\vert a \vert =2$ má $y''$ nulovú hodnotu len v jednom bode, inde $y''>0$; dá sa dokázať, že aj v tomto prípade je  $y$ rýdzo konvexná (pozri pr. 462);
$\boxed{360.}$ $\boldsymbol{3.}$ stačí obidve strany vydeliť číslom $2$ a využiť konvexnosť funkcie $x \ln x$;
$\boldsymbol{4.}$ $y=\frac{2}{\pi}x$ je rovnica spojnice bodov $(0,0)$, $(\frac{\pi}{2},1)$ ležiacich na grafe funkcie $\sin $;
$\boldsymbol{6.}$ $y=x-1$ je rovnica spojnice bodov $(1,0)$, $(2,1)$ ležiacich na grafe konkávnej funkcie $y=\log _{2}x $;
$\boxed{361.}$ sporom; $f$ by musela byť rýdzo konkávna, a teda aj nekonštantná na $\mathbb{R}$; ak $a < b$, $f(a)\ne  f(b)$, tak sa spojnica bodov $(a,f(a))$, $(b,f(b))$ pretína s osou $Ox$, pritom na $\mathbb{R}-\langle a,b \rangle$ leží graf funcie $f$ pod touto spojnicou, to vedie k sporu s kladnosťou funkcie $f$ (každý z uvedených krokov treba samozrejme podrobne zdôvodniť);
$\boxed{362.}$ $\boldsymbol{1.}$ pretože z definície konvexnosti funkcie $f$  na $I$ vyplýva $\forall x,y,z\in I, \quad x<z<y: \, f(z) <f(x)+(f(y)-f(x)).\frac{z-x}{y-x}$ (v definícii stačí položiť  $z=px+qy$ a odtiaľ vyjadriť $q=\frac{z-x}{y-x}$, môžeme dokázať, že funkcia $F(x):=\frac{f(x)-f(a)}{x-a}$ je rastúca na $I\cap (a,+\infty)$ a na $I\cap (-\infty,a)$; pritom $f'(a)=\lim_{x \to a}F(x)$; odtiaľ vyplýva $\forall x\in I, \, x>a: \quad \frac{f(x)-f(a)}{x-a}>f'(a)$, $\forall x\in I, \, x<a: \quad \frac{f(x)-f(a)}{x-a}<f'(a)$; z toho po úprave  $\forall x\in I, \, x \ne a: \quad f(x)>f(a)+f'(a)(x-a)$ (úvahy pre krajné a pre vnútorné body intervalu $I$ treba robiť každú samostatne);
$\boxed{363.}$ $f'(0)=0$ (vypočíta sa priamo z definície); pre $x>0$ je  $f(x)>0$, pre  $x<0$  je $f(x)<0$; $f''(x)=12x+(6x-\frac{1}{x})\cos\frac{1}{x}+4\sin\frac{1}{x}$ je spojitá na $\mathbb{R}-\lbrace 0 \rbrace$ a  na každom z intervalov $(-\varepsilon ,0)$, $(0,\varepsilon)$, $\varepsilon>0$, nadobúda kladné aj záporné hodnoty (vyplýva to z faktu, že pre $x=\frac{1}{2k\pi}$ je $f''(x)=18x-\frac{1}{x}$, pre  $x=\frac{1}{(2k+1)\pi}$ je $f''(x)=6x+\frac{1}{x}$ a obidve tieto funkcie majú v bode $0$ jednostranné limity $+\infty$ a $-\infty$); preto každý z intervalov $(-\varepsilon ,0)$, $(0,\varepsilon)$, $\varepsilon>0$ obsahuje intervaly, na ktorých je $f$ rýdzo konvexná, aj intervaly na ktorých je  $f$ rýdzo konkávna;
$\boxed{364.}$ $\boldsymbol{1.}$ lok. minimum: $f(0)=0$, lok. maximum: $f(1)=f(-1)=1$;
$\boldsymbol{2.}$ lok. minimum: $f(-2)=-9$, $f(3)=-\frac{161}{4}$; lok. maximum: $f(0)=7$;
$\boldsymbol{3.}$ lok. minimum: $f(k\pi-\frac{\pi}{4})=-\frac{e^{(4k-1)\pi/4}}{\sqrt{2}}$, $k$ párne;  lok. maximum: $f(k\pi-\frac{\pi}{4})=\frac{e^{(4k-1)\pi/4}}{\sqrt{2}}$, $k$ nepárne $(\cos x+\sin x=\sqrt{2}\sin (x+\frac{\pi}{4}))$; 
$\boxed{365.}$ $\boldsymbol{1.}$ lok. minimum: $f(1)=f(3)=3$; lok. maximum: $f(2)=4$;
$\boldsymbol{2.}$ lok. minimum: $f(1)=0$, $f(-\frac{5+\sqrt{13}}{6})$; lok. maximum: $f(-\frac{\sqrt{13}-5}{6})$;
$\boldsymbol{3.}$ lok. minimum: $f(\frac{7}{5})=-\frac{1}{24}$; 
$\boldsymbol{4.}$ lok. minimum: $f(1)=0$; lok. maximum: $f(e^{2})=\frac{4}{e^{2}}$;
$\boldsymbol{5.}$ lok. minimum: $f(-1)=0$; lok. maximum: $f(9)=\frac{10^{10}}{9}$;
$\boxed{366.}$ $\boldsymbol{1.}$ nemá;
$\boldsymbol{2.}$ má v bode 0 lok. maximum $f(\frac{3}{2})=\frac{1}{4}$ (ak $f$ je spojitá, tak $\vert f \vert$ nadobúda lokálne extrémy v bodoch lokálnych extrémov funkcie $f$ a v nulových bodoch funkcie $f$);
$\boldsymbol{2.}$ lok. maximum: $f(-\frac{1}{\sqrt[4]{27}})=f(\frac{1}{\sqrt[4]{27}})=\frac{2}{3\sqrt{3}}$; lok. minimum: $f(0)=0$ (posledné tvrdenie nevyplýva z viet 16 a 17 - $f'(0)$ totiž neexistuje - a treba ho preto dokázať samostatne);
$\boldsymbol{3.}$ lok. maximum: $f(3)=3$ (bod 3 nie je vnútorný bod $D(f)$ uvedené tvrdenie teda nevyplýva z viet 15 - 17 a treba ho dokázať); 
$\boldsymbol{4.}$ lok. minimum: $f(\frac{3}{4})=-\frac{3}{4\sqrt[3]{4}}$ (pretože neexistuje vlastná $f'(1)$, preskúmame bod 1 zvlášť: buď priamo ukážeme, že $f$ nemá v bode $1$ lokálvy extrém, alebo dokážeme, že $f'(1)$ je nevlastná a použijeme vetu 15);
$\boxed{368.}$ $\boldsymbol{1.}$ $M=11$, $m=2$;
$\boldsymbol{2.}$ $M=100,01$; $m=2$;
$\boldsymbol{3.}$ $M=8$; $m$ neexistuje ($f$ je rastúca na $\langle 0,4 \rangle$; zrejme $0=\inf _{x\in (0,4\rangle}f(x)$);
$\boldsymbol{4.}$  $M=1$ ($f$ rastie na  $\langle 0,\frac{\pi}{4} \rangle$, klesá na $\langle \frac{\pi}{4} ,frac{\pi}{2})$), $m$ neexistuje ($\lim_{x \to \pi/2-}f(x)=-\infty$);
$\boldsymbol{5.}$ $M$ neexistuje,  $m=4$;
$\boldsymbol{6.}$ $M=\frac{\pi}{4}$; $m=0$;
$\boxed{369.}$ $\boldsymbol{3.}$ vyšetrite monotónnosť funkcie $\frac{x^{2}+1}{x^{2}+x+1}$ a nájdite jej limity v bodoch $+\infty $, $-\infty $; na základe toho si načrtnite jej graf; obrázok je návodom k vlastnému dôkazu;
$\boxed{370.}$ $\boldsymbol{1.}$ dokážte, že $P'$ musí pri prechode cez tento bod zmeniť znamienko; využite, že $P'$ ako polynóm nepárneho stupňa má v bodoch $+\infty$, $-\infty$ nevlastné limity opačných znamienok;
$\boxed{371.}$ nemôže; nepriamo: nech $a<b$ sú také stacionárne body; funkcia $f/\langle a,b \rangle$ nadobúda v niektorom bode $c$ kompaktu $\langle a,b \rangle$ svoje minimum; treba dokázať $c\in (a,b)$ a potom využiť vetu 15;
$\boxed{372.}$ $f'$ je spojitá na $\mathbb{R}-\lbrace 0 \rbrace$ a v každom ľavom, resp. pravom okolí bodu $O$ nadobúda kladné aj záporné hodnoty, ďalej pozri myšlienku návodu k pr. 363;
$\boxed{373.}$ $\boldsymbol{1.}$ výška  $v=\frac{2R}{\sqrt{3}}$, polomer podstavy $r=\frac{\sqrt{2} R}{\sqrt{3}}$ (existuje medzi valcami vpísanými do tejto gule valec s najmenším objemom?);
$\boldsymbol{2.}$  $v=\frac{2R\sqrt{5-\sqrt{5}}}{\sqrt{10}}$,  $r=\frac{R\sqrt{5+\sqrt{5}}}{\sqrt{10}}$;
$\boxed{374.}$ bod $(\frac{-1+\sqrt{33}}{8},\frac{1+3\sqrt{33}}{8})$;
$\boxed{375.}$ ak vzdialenosť bodov $(a,b)$ a  $(c,f(c))$ je maximálna (minimálna), tak  $f'(c)=\frac{c-a}{f(c)-b}$, to vyplýva z vety 15; pri dôkaze faktu, že dotyčnica v bode $(c,f(c))$	je kolmá na spojnicu bodov $(a,b)$ a $(c,f(c))$,  treba rozlíšíť prípady $a=c$,	$a\ne c$;
$\boxed{376.}$ vo vydialenosti $\sqrt{ab}$ ($cm$) ($\varphi =\arctan \frac{b}{x-\arctan \frac{a}{x}}$, $b>a$);
$\boxed{377.}$ platí $f(x)=f(x)-f(a)$, $g(x)=g(x)-g(a)$; potom zlomok rozšíriť výrazom $\frac{1}{x-a}$; pozor, vetu 18 nemožno použiť (nie je zaručené splnenie predpokladu 1);
$\boxed{378.}$ $\boldsymbol{2.}$ $2$;
$\boldsymbol{3.}$ $\frac{1}{3}$;
$\boldsymbol{4.}$ $0$; (l'Hospitalovo pravidlo použiť n-krát za sebou);
$\boldsymbol{5.}$ $0$ (najprv subst. $t=\frac{1}{x^{2}}$);
$\boldsymbol{6.}$ $-\frac{1}{2}$ (najprv zlomok rozšíriť členom x a využiť, že $\lim_{x \to 0}\frac{x}{\tan x}=1$; použitie l'Hospitalovho pravidla sa tak zjednoduší);
$\boldsymbol{7.}$ $4$ (napísať v tvare súčtu dvoch limít; využiť, že $\lim_{x \to 0}\frac{x^{3}\cos \pi/x}{x^{2}}=0$, l'Hospitalovo pravidlo použiť len na výpočet druhej limity);
$\boxed{379.}$ $\boldsymbol{1.}$ $0$;
$\boldsymbol{2.}$ $0$;
$\boldsymbol{3.}$ $2$;
$\boldsymbol{4.}$ $0$;
$\boxed{380.}$ $\boldsymbol{1.}$ $1$;
$\boldsymbol{2.}$ $e$;
$\boldsymbol{3.}$ $e^{1/6}$;
$\boldsymbol{4.}$ $e^{-1/2}$;
$\boldsymbol{5.}$ $1$ (po prvom použití l'Hospitalovo pravidla dostaneme výraz $\frac{x}{\ln \frac{1}{x}}$; ten - hoci v bode $0$ nie je typu $\frac{\infty}{\infty}$ ani  $\frac{0}{0}$ - vyhovuje všetkým predpokladom vety 18 (zvlášť si všimnite predpoklad 3), možno teda ešte raz použiť l'Hospitalovo pravidlo; $\lim_{x \to 0}(x/\ln (1/x))$ možno samozrejme nájsť aj bez toho, stačí použiť vetu 10 z kapitoly 2);
$\boxed{381.}$ $\boldsymbol{1.}$ $\frac{1}{2}$; (po úprave na spoločného menovateľa zlomok rozšírte členom $x$ a využite, že  $\lim_{x \to 0}\frac{e^{x}-1}{x}=1$);
$\boldsymbol{2.}$ $\frac{2}{3}$; (po úprave na spoločného menovateľa  využite, že  $\lim_{x \to 0}\frac{x^{2}}{\sin ^{2}x}=1$);
$\boldsymbol{3.}$ $\frac{1}{2}$ (najprv subst. $\frac{1}{x}=1$);
$\boxed{382.}$ $\boldsymbol{1,2.}$ nie, nie je splnená podmienka 4 z vety 18;
$\boldsymbol{3.}$ nie, nie je splnená podmienka 3 z vety 18 (výpočet jednotlivých limít: 1. zlomok rozšíriť členom $x$ a použiť pr. 131.1; 2. využiť, že $\lim_{x \to 0}x\sin \frac{1}{x}=0$; 3. dosadiť);
$\boxed{383.}$ správny je druhý postup; derivácia menovateľa je $\cos x.e^{\sin x}.(x+2\cos x+\cos x.\sin x$), tá má nulové hodnoty v každom okolí bodu  $+\infty$, nie je teda splnená podmienka 2 z vety 18, preto sme neboli oprávnení používať  l'Hospitalovo pravidlo;
$\boxed{384.}$ $\boldsymbol{1.}$ na výpočet $\lim_{x \to a}\frac{f(x)-f(a)}{x-a}$ použite l'Hospitalovo pravidlo;
$\boldsymbol{2a/}$  $0=\lim_{x \to 0}f'(x)=f'(0)$;
$\boldsymbol{2b/}$  $f'(1)=+\infty$, $f'(-1)=+\infty$ (použili sme unalógiu tvrdenia z pr. 384.1 pre jednostranné derivácie);
$\boldsymbol{2c/}$  $f'(0)=\frac{1}{\sqrt{2}}$ (nezabudnite najprv preveriť spojitosť funkcie $f$ v bode 0);
$\boxed{385.}$ $\boldsymbol{1.}$ $\frac{1}{\sqrt{2}}+\frac{1}{\sqrt{2}}(x+\frac{1}{2})-\frac{1}{2:\sqrt{2}}(x+\frac{1}{2})^{2}+\frac{1}{2:\sqrt{2}}(x+\frac{1}{2})^{3}-\frac{15}{4:\sqrt{2}}(x+\frac{1}{2})^{4}$;
$\boldsymbol{2.}$ $x+\frac{x^{3}}{3}$;
$\boldsymbol{3.}$ $\frac{1}{\sqrt{2}}+\sqrt{2}x-\sqrt{2}x^{2}-\frac{4\sqrt{2}}{3!}x^{3}+\frac{8\sqrt{2}}{4!}x^{4}+\frac{16\sqrt{2}}{5!}x^{5}$;
$\boldsymbol{4.}$ $\sum_{k=0}^N (-2)^{k}(x+1)^{k}$ ($f^{(k)}(-1)=(-1)^{k}k!2^{k}$;
$\boldsymbol{5.}$ Taylorovým polynómom stupňa i je funkcia $T(x)=\frac{1}{2}$; Taylorove polynómy stupňov $2m$ a $2m+1$ ($m \in \mathbb{R}$) sa zhodujú a majú tvar $\frac{1}{2}+\sum_{k=0}^N \frac{(-1)^{k+1}4^{2k-1}}{(2k)!}(x-\frac{\pi}{4})^{2k}$ (predpis funkcie $f$ možno upraviť pomocou vzorcov $\sin ^{2}\alpha =\frac{1-\cos 2\alpha}{2}$, $\cos ^{2} \alpha =\frac{1+\cos 2\alpha}{2}$);
$\boxed{386.}$ platí (*) $A_{0}+...+A_{n}(x-a)^{n}+o((x-a)^{n})=f(x)=f(a)+...+\frac{f^{(n)}}{n!}(x-a)^{n}+o((x-a)^{n})$ (druhá rovnosť vyplýva z vety 19, odtiaľ vyplýva rovnosť limity pravej a ľavej strany v bode $a$, preto $A_{0}=f(a)$, ak po odstránení $A_{0}$, resp. $f(a)$ z ľavej resp. pravej strany vydelíme obidve strany členom $(x-a)$ (a využijeme, že $\frac{o((x-a)^{n})}{x-a}=o((x-a)^{n-1})$), dostaneme $A_{1}+A_{2}(x-a)+...+A_{n}(x-a)^{n-1}+o((x-a)^{n-1})=f'(a)+...+\frac{f^{(n)}}{n!}(x-a)^{n-1}+o((x-a)^{n-1})$, odtiaľ opäť vyplýva rovnosť limít v bode a atď;
$\boxed{387.}$ $\boldsymbol{1.}$  $e^{\frac{x}{7}}=\sum_{k=0}^n \frac{x^{k}}{7^{k}.k!}+o(x^{n})$;
$\boldsymbol{3.}$ Maclavrinove polynómy stupňa 1, 2, ..., 5 sa zhodujú a majú tvar $T(x)=1$; Maclavrinove polynómy stupňa $n=6m, 6m+1,...,6m+5$ $(m \in \mathbb{N}$ sa zhodujú a majú tvar $T(x)=1+ \sum_{k=1}^m (-1)^{k}\frac{x^{6k}}{(2k!)}$, preto možno písať $\cos x^{3}=1+\sum_{k=1}^m (-1)^{k}.\frac{x^{6k}}{(2k!)}+o(x^{6m+5})$;
$\boldsymbol{5.}$ $\frac{1}{2+3x^{2}}=\frac{1}{2}. \frac{1}{1-(-\frac{3}{2}x^{2})}=\frac{1}{2}+\sum_{k=1}^m (-\frac{3}{2})^{k}x^{2k}+o(x^{2m+1})$,  Maclavrinov polynóm stupňa 1 má tvar $T(x)=\frac{1}{2}$ (využili sme vzorec V;  Maclavrinove polynómy stupňa $n=2m$ a $n=2m+1$ sa zhodujú);
$\boldsymbol{5.}$ $x^{3}\sin 3x= \sum_{k=1}^m (-1)^{k+1}\frac{3^{2k-1}x^{2k+2}}{(2k-1)!}+e(x^{2m+3})$ (teda Maclavrinov polynóm stupňa 1, 2, 3 majú rovnaký  tvar $T(x)=0$,  Maclavrinove polynómy stupňov  $n=2m+2, \, n=2m+3\quad (m\in \mathbb{N})$ sa zhodujú);
$\boxed{388.}$ $\boldsymbol{2.}$ $1+x-x^{2}+\frac{3}{2}x^{3} $ (najprv sme použili vzorec VI pre $\alpha=\frac{1}{3} $);
$\boldsymbol{3.}$ $-\frac{x^{3}}{8}$ (obidve strany rovnosti $\ln (1-\frac{x}{2})=-\frac{x}{2}+o(x)$ sme umocnili na tretiu);
$\boldsymbol{4.}$	$x+\frac{x^{2}}{2} -\frac{2x^{3}}{3}+\frac{x^{4}}{4} $ (položili sme  $ z=x^{2}+x $  a najprv použili vzorec IV; ak  $R(z)=o(z^{4})$, tak  $R(x^{2}+x)= o(z^{4}) $;
$\boldsymbol{5.}$ $\cos x-\ln (1+x)=(1-\frac{x^{2}}{2}+\frac{x^{4}}{24}+o(x^{4})). (x-\frac{x^{2}}{2}+\frac{x^{3}}{3}-\frac{x^{4}}{4}+\frac{x^{5}}{5}+ o(x^{5}))=x-\frac{x^{2}}{2}-\frac{x^{3}}{6}+\frac{3x^{5}}{40}+ o(x^{5}))$;
$\boxed{389.}$ $\boldsymbol{1.}$  $\sum_{k=0}^m\frac{e^{-2}}{k!}(x+1)^{2k} $ pre $n=2m$ aj pre $n=2m+1$;
$\boldsymbol{2.}$  $\sum_{k=0}^m\frac{(-1)^{k}e^{2}2^{k-2}}{k!}(k^{2}+3k+4)(x+1)^{k} $ $f(t+1)=e^{2}.(e^{-2t}-2te^{-2t}+t^{2}e^{-2t})=e^{2}[\sum_{k=0}^n\frac{(-1)^{k}2^{k}t^{k}}{k!} +o(t^{n})+\sum_{i=0}^{n-1}\frac{(-1)^{i+1}2^{i+1}t^{i+1}}{i!} +o(t^{n})+\sum_{j=0}^{n-2}\frac{(-1)^{j}2^{j}t^{j}}{j!} +o(t^{n})]$ $=e^{2}[\sum_{k=0}^n\frac{(-1)^{k}2^{k}t^{k}}{k!}+\sum_{k=1}^n\frac{(-1)^{k}2^{k}t^{k}}{(k-1)!}+\sum_{k=2}^n\frac{(-1)^{k}2^{k-2}t^{k}}{(k-2)!}+o(t^{n})]=$ (položili sme $i+1=k,j+2=k$)$=e^{2}[1-2t+\sum_{k=2}^n\frac{(-1)^{k}2^{k}t^{k}}{k!}-2t+\sum_{k=2}^n\frac{(-1)^{k}2^{k}t^{k}}{(k-1)!}+\sum_{k=2}^n\frac{(-1)^{k}2^{k}t^{k}}{(k-2)!}+o(t^{n})]$;
$\boldsymbol{3.}$   $\frac{x}{2}-\frac{x^{2}}{4}+\sum_{k=3}^n\frac{(-1)^{(k+1)}(k-1)x^{k}}{k(k-2)}$ pre  $n\geq 3$ Maclavrinov polynóm stupňa 1, resp. 2 je $\frac{x}{2}-\frac{x^{2}}{4}(\ln \sqrt{1+x}=\frac{1}{2}\ln (1+x))$;
$\boldsymbol{4.}$ $f(x)=1-x^{2}+\sum_{k=2}^m 2.(-1)^{k}x^{2k}+o(x^{2m+1})$;
$\boldsymbol{5.}$ $f(x)=-\frac{7}{6}+\sum_{k=1}^n (-\frac{5}{3})(\frac{10}{21})^{k}(x+\frac{1}{10})^{k}+o((x+\frac{1}{10})^{n})$  $(f(t-\frac{1}{10})=\frac{1}{2}-\frac{5}{3}.\frac{1}{1-\frac{10t}{21}})$;
$\boldsymbol{6.}$ $f(x)= \ln 6 - \sum_{k=1}^n \frac{1}{k}(\frac{1}{2^{k}}+\frac{1}{3^{k}})(x-1)^{k}+o(x-1)^{n})$ (pre $t<2$ platí $f(t+1)=\ln ((t-2)(t-3))=\ln (2-t)+\ln (3-t)=\ln 2 +\ln (1-\frac{t}{2}) +\ln 3+\ln (1-\frac{t}{3}))$;
$\boxed{390.}$ $\boldsymbol{1.}$ $\frac{1}{2} $  ($= \lim_{x \to 0}\frac{x^{3}/2+o(x^{3})}{x^{3}})$;
$\boldsymbol{3.}$ $\frac{1}{3} $;
$\boldsymbol{4.}$ $10 $ ($= \lim_{x \to 0}(-\frac{x^{3}/6+o(x^{3})}{x^{2}+o(x^{2})})$);
$\boldsymbol{5.}$ $\frac{4}{3} $;
$\boldsymbol{6.}$ $\frac{1}{2} $ ($\ln (\cos x +\frac{x^{2}}{2})= \ln (1+\cos x -1 +\frac{x^{2}}{2}))$, potom použiť vzorec IV; ak $R(z)=o(z^{2})$, tak $R( \cos x -1+\frac{x^{2}}{2})= o(x^{4}))$;
$\boxed{391.}$ $\boldsymbol{1.}$ $\frac{60!}{30!} $  ($f^{(60)}(0)$ možno vyjadriť pomocou koeficientu pri $x^{60}$ v Maclavrinovom polynóme stupňa $n\geq 60$ funkcie $f$);
$\boldsymbol{2.}$ $0$  ($\frac{1}{1+x+x^{2}}=\frac{1-x}{1-x^{3}}$ pre $x \ne 1$);
$\boxed{392.}$ $\boldsymbol{1.}$ $\bigtriangleup < \frac{e}{11!}$ (použili sme Lagrangeov tvar zvyšku);
$\boldsymbol{2.}$ $\bigtriangleup < \frac{\cos 0,5}{3840}$ (polynóm vpravo je Maclavrinovým polynómom stupňa 3 aj stupňa 4 funkcie $\sin$, preto sme odhadli zvyšky v obidvoch prípadoch a z odhadov vybrali ten menší);
$\boldsymbol{3.}$ $\bigtriangleup < 0,0015$;
$\boxed{393.}$ $\boldsymbol{2.}$ podľa vety 20 $e^{x}=1+x+\frac{e^{\upsilon (x)}}{2}x^{2}$, pritom $\upsilon (x)<x$;
$\boxed{394.}$ obr. 5, funkcia je nepárna;
$\boxed{395.}$ obr. 6;
$\boxed{396.}$ obr. 7;
$\boxed{397.}$ obr. 8;
$\boxed{Obr.5.}$  $y=3x-x^{3}$
$\boxed{Obr.6.}$  $y=x^{4}/(1+x)^{3}$
$\boxed{Obr.7.}$  $y=(x+1)^{3}/(x-1)^{2}$
$\boxed{Obr.8.}$  $y=(x^{2}+x-1)/(x^{2}-2x+1)$
$$*$$
$$*$$
$$*$$
$$*$$
(Pri popise obrázkov sme použili tieto označenia: n - bod, v ktorom sa funkčná hodnota rovná 0; m - bod, v ktorom funkcia nadobúda lokálne minimum; M - bod, v ktorom funkcia nadobúda lokálne maximum; i - inflexný bod; ak je os Ox (Oy) asymptotou, je k nej pripísaná jej rovnica y=0 (x=0), ostatné asymptoty sú vyznačené čiarkovane; s výnimkou obr. 17 a 23 sa jednotka dĺžky na Ox a na Oy zhodujú a jednotka dĺžky je označená len na Oy.)
$\boxed{398.}$ obr. 9; nepárna funkcia, $y_{-}'(-1)=y_{+}'(-1)=+\infty $; pozor: na intervale $\langle -\sqrt{\frac{3}{2}},0 \rangle$ $y$ nie je konvexná, hoci je konvexná na intervale $\langle -\sqrt{\frac{3}{2}},-1 \rangle$ a na intervale $\langle -1,0 \rangle$; podobná poznámka sa vzťahuje aj na interval $\langle 0,\sqrt{\frac{3}{2}} \rangle$;
$\boxed{399.}$ obr. 10; 
$\boxed{400.}$ obr. 11;	 nepárna fukcia;
$\boxed{401.}$ obr. 12;  párna funkcia, $y_{-}'(-1)=y_{-}'(1)=-\infty $, $y_{+}'(-1)=y_{+}'(1)=+\infty $; $y$ nie je konkávna na množine $\mathbb{R}$, hoci je konkávna na intervaloch $(-\infty ,-1\rangle , \langle -1,1 \rangle $ a $\langle 1, \infty)$;
$\boxed{402.}$ obr. 13;  $y_{-}'(0)=-\infty $, $y_{+}'(0)=+\infty $; $y$ nie je konkávna na intervale $\langle -2+\sqrt{3}, +\infty)$, hoci je konkávna na intervaloch $\langle -2+\sqrt{3},0 \rangle$ a $\langle 0,+\infty)$;
$\boxed{403.}$ obr. 14; funkcia s periódou $2\pi$;
$\boxed{404.}$ obr. 15; párna funkcia s periódou $2\pi$;
$\boxed{405.}$ obr. 16; $y_{-}'(0)=-\infty $, $y_{+}'(0)=+\infty $; $y$ nie je konkávna na intervale $\langle \frac{2-\sqrt{6}}{3},\frac{2+\sqrt{6}}{3} \rangle$;
$\boxed{Obr.9.}$  $y=x \sqrt{\vert}x^{2}-1 \vert)$
$\boxed{Obr.10.}$  $y=(x-2)/\sqrt{x^{2}+1}$
$\boxed{Obr.11.}$  $y=x/\sqrt[3]{x^{2}-1}$
$$*$$
$$*$$
$$*$$
$$*$$
$\boxed{406.}$ obr. 17;
$\boxed{407.}$ obr. 18;
$\boxed{408.}$ obr. 19; nepárna funkcia,  $y_{-}'(1)=y_{+}'(-1)=1 $,  $y_{+}'(1)=y_{-}'(-1)=-1 $; pozor $-1,1$ nie sú podľa nami používanej definície inflexnými bodmi (pri úprave výrazu pre $y'$ nezabudnite, že  $\sqrt{a^{2}}=\vert a \vert $, pri výpočte $y_{+}'(1),y_{+}'(-1)$ použite obdobu tvrdenia z pr. 384.1 pre jednostranné derivácie);
$\boxed{409.}$ obr. 20; $0 \notin D(y)$, existuje ale $\lim_{x \to 0-}y(x)=0$;
$\boxed{410.}$ obr. 21; 
$\boxed{411.}$ obr. 22; 0 je bod odstrániteľnej nespojitosti,  $\lim_{x \to 0}y(x)=1$;
$\boxed{413.}$ $(f.g)'(a)=f(a).g'(a)$;
$\boxed{414.}$ tvrdenie vyplýva zo vzťahu $\frac{f^{n}(x)-f^{n}(a)}{x-a}=\frac{f(x)-f(a)}{x-a}.(f^{n-1}(x)+f^{n-2}(x).f(a)+...+f^{n-1}(a))$;
$\boxed{417.}$ $\boldsymbol{1.}$ neplatí
$$*$$
$$*$$
$$*$$
$$*$$
$\boxed{Obr.12.}$  $y=(x+1)^{2/3}+(x-1)^{2/3}$
$\boxed{Obr.13.}$  $y=\sqrt[3]{x^{2}/(x+1)}$
$\boxed{Obr.14.}$  $y=\frac{1}{2}\sin 2x +\cos x$
($x+\sin x$ je neperiodická funkcia s periodickou deriváciou);
$\boldsymbol{2.}$ platí;
$\boxed{418.}$ $(f^{2})_{\pm}'(a)=2f(a).f_{\pm}'(a)$; nutná a postačujúca podmienka je $f(a)=0 $;
$\boxed{419.}$ $\boldsymbol{1.}$  $\sum_{j=1}^m$  ($\sum_{l=1}^n$ ($\varphi _{lj}'.\prod _{k=1,k\ne l}^n \varphi _{kj}))$;
$\boldsymbol{2.}$  $\sum_{j=1}^m$  ($\sum_{l=1}^n$ ($\varphi _{lj}'.\prod _{j=1}^m \sum_{j=1}^n \varphi _{kj}))$;
$\boxed{420.}$ $\boldsymbol{1.}$ $F'(x)=3x^{2}+3x+20$;
 $\boldsymbol{2.}$ $F'(x)=6x^{2}$;
$\boxed{421.}$ $\boldsymbol{1.}$ $f(x)=\vert x-a\vert$, $g(x)=x^{2}$;
$\boldsymbol{2.}$ $f(x)=( x-a)^{2}$, $g(x)=\vert x \vert$;	
$\boldsymbol{3.}$ $f(x)=2( x-a)+\vert x-a \vert $, $g(x)=2x - \vert x \vert$ (všimnite si, že jednostranné derivácie$ g\circ f$ možno v tomto prípade vypočítať pomocou $f_{+}'(a)$ a $g_{+}'(0)$, resp. $f_{-}'(a)$ a $g_{-}'(0))$;
$\boxed{422.}$ za daných predpokladov musí platiť
$$*$$
$$*$$
$$*$$
$$*$$
$\boxed{Obr.15.}$  $y=\cos x / \cos 2x$
$\boxed{Obr.16.}$  $y=x^{2/3}e^{-x}$
$f(a)=0$, $f'(a)\ne 0$, ďalej pozri pr. 412;
$\boxed{423.}$ 1. $f(x)=ax$ pre $x\ne \frac{1}{n}$, $x\ne n$ ($n \in \mathbb{N}$) (v tomto prípade využívame možnosť širšieho chápaania pojmu derivácia pre $f'(0))$, alebo 2. $f$ je nepárna funkcia definovaná pre $x\geq 0$ nasledovne: zoraďme spočitateľnú množinu ($\mathbb{Q}\cap \langle 0, \infty)) - \lbrace x \in \mathbb{R}:x=ap^{2}+ap, \, p\in \langle 0, \infty) \cap (\mathbb{Q}- \mathbb{N}) \rbrace $ (tá obsahuje aj všetky čísla tvaru $\frac{a}{n}$, kde $n \in \mathbb{N}$) do prostej postupnosti $\lbrace b_{n} \rbrace _{n=1} ^{\infty}$, potom položíme $f(x)=ax$ pre $x \in (0,\infty) - \mathbb{Q}$, $f(x)=ax^{2}+ax$ pre $x \in \langle 0,\infty )\cap (\mathbb{Q}- \mathbb{N})$, $f(n)=b_{n}$ $ n \in \mathbb{N}$ ( v tomto prípade nevyužívame možnosť širšieho chápania pojmu derivácia ani pre $f'(0)$ ani pre ($f^{-1})'(0))$;
$$*$$
$$*$$
$$*$$
$$*$$
$\boxed{Obr.17.}$  $y=\ln x / \sqrt{x}$
$\boxed{Obr.18.}$  $y=x \arctan x$
$\boxed{424.}$ $f''(0)=0$ (najprv treba odvodiť vzorec pre druhú deriváciu inverznej funkcie: ($f^{-1})''(x)=-\frac{f''(f^{-1}(x))}{(f')^{3}(f^{-1}(x))}$; stanovte predpoklady, za ktorých možno toto odvodenie vykonať);
$\boxed{425.}$ $\boldsymbol{1.}$ $y^{(n)}=\frac{3^{n}}{4}\cos (3x+\frac{n\pi}{2})+\frac{3}{4}\cos (x+ \frac{n\pi}{2}) $;
$\boldsymbol{2.}$ $y^{(n)}=2^{2n-3}\cos (4x+\frac{n\pi}{2}+2^{n-1}.\cos (2x+\frac{n\pi}{2})$;
$$*$$
$$*$$
$$*$$
$$*$$
$\boxed{Obr.19.}$  $y=\ln x / \arcsin (2x/(1+x^{2})$
$\boxed{Obr.20.}$  $y=(x+2)e^{1/x}$
$\boxed{Obr.21.}$  $y=\arccos((1-x)/(1-2x))$
$\boxed{Obr.22.}$  $y=x^{x}$
$\boldsymbol{3.}$ $y^{(n)}=72^{x}\ln ^{n-1} 72. ((2x-1)\ln 72+2n)$;
$\boldsymbol{4.}$ úpravu $\ln (x^{2}-3x+2)=\ln (x-2)+\ln(x-1)$ nemožno priamo použiť (porovnaj s pr. 26.2), treba si pomôcť nasledovne: pre všetky $x \in D((\ln f)'(x)=(\ln \vert f \vert )'(x)$; $y^{(n)}=(n-2)!(-1)^{n}.(\frac{x-2n}{(x-2)^{n}}+\frac{x-n}{(x-1)^{n}})$ pre $n\geq 2$; $y'=\ln (x^{2}-3x+2)+\frac{2x^{2}-3x}{x^{2}-3x+2}$;
$\boldsymbol{5.}$ $y^{(n)}=(\sqrt{2})^{n}e^{x} \sin (x+\frac{n\pi}{4})$ (treba použiť rovnosť $\sin x +\cos x=\sqrt{2} \sin (x+\frac{\pi}{4}) $ a indukciu);
$\boxed{426.}$ $f^{(n)}(a)=n\varphi ^{(n-1)}(a)$ (na výpočet $f^{(n)}(a)$ nemožno použiť vetu 7, existencia $\varphi ^{(n)}(a)$ nie je zaručená);
$\boxed{429.}$ pozri pr. 329;
$\boxed{430.}$ $a_{0}+...+a_{n}x^{n}=(a_{0}\frac{x}{1}+...+a_{n}\frac{x^{n+1}}{n+1})'$, polynóm v zátvorke vpravo má korene 0,  1;
$\boxed{431.}$ pozri pr. 332;
$\boxed{432.}$ pozri pr. 331, polynóm $(x^{2}-1)^{n}$ má n-násobné korene 1 a -1;
$\boxed{433.}$ sporom, použiť pr. 331;
$\boxed{435.}$ možno postupovať indukciou, pričom dôkaz pre dané n sa robí sporom; ak P je polynóm s n zmenami znamienok, tak $(x^{-k}P(x))'$ (kde $k\in \mathbb{N}$ je vhodne zvolené, pozri návod k pr. 434) má n-1 zmien znamienok, pritom $(x^{k+1}(x^{-k}P(x))'$ je polynóm, ktorý má rovnaké kladné korene ako $(x^{-k}P(x))'$ a tiež $n-1$ zmien znamienok;
$\boxed{437.}$ $f$ je polynóm stupňa $k$, kde $k\leq n-1$ (pozri návod k pr. 339);
$\boxed{438.}$ ak $f'(x_{0})=0$, tak $\frac{f(x)-f'(x_{0})}{f'(x_{0})} =\frac{f'(c(x))}{f'(x_{0})}(x-x_{0})$, pričom uvedený podiel derivácií je ohraničená funkcia;
$\boxed{441.}$ pozri pr. 343;
$\boxed{444.}$  možno použiť pr. 443 tak, ako sa používa Rolleho veta pri dôkaze Lagrangeovej vety o strednej hodnote;
$\boxed{445.}$ sporom	 keby to nebola pravda, tak by existoval interval $I\subset (a,b)$, na ktorom by funkcia $f$ mala nulovú deriváciu, čo je spor s jej injektívnosťou (treba si uvedomiť, že $f$ aj $f^{-1}$ sú monotónne spojité funkcie);
$\boxed{446.}$ treba použiť Cauchyho vetu pre funkcie $x^{2}.f(x),x^{2}$;
$\boxed{447.}$ (pozri tiež pr. 352)
$\boldsymbol{1.}$ ak $f(x):=\ln x$, $g(x):=\frac{x-1}{\sqrt{x}}$, tak $f(1)=g(1)$, $f'(x)< g'(x)$ pre $x>0, x\ne 1$, preto $f(x)< g(x)$ pre $x>1$,  $f(x)>g(x)$ pre $x\in (0,1)$;
$\boldsymbol{2.}$ najprv zlogaritmovať, potom položiť $x=\frac{1}{t}$;
$\boldsymbol{3.}$ môžeme predpokladať $x\leq y$; položme $t=\frac{x}{y}$, dostaneme ekvivalentnú nerovnosť $(1+t^{\alpha})^{1/\alpha}>(1+t^{\beta})^{1/\beta}$, $t\in (0,1 \rangle$, potom zlogaritmovať;
$\boxed{449.}$ $\boldsymbol{3.}$ funkcia $h:=f-g$ vyhovuje všetkým predpokladom tvrdenia z pr. 449.1;
$\boxed{450.}$ označme $O(c)$ to okolie bodu $c\in I$,v ktorom platí $\forall x,y \in O(c)$, $x<c<y:f(x)<f(c)<f(y)$; nech $a,b\in I$, $a<b$, potom $\langle a,b \rangle$ je kompatk, z otvoreného pokrytia $\lbrace O(c);c\in \langle a,b \rangle \rbrace $ vyberme konečné podpokrytie a z neho také konečné pokrytie $\lbrace O(c_{1}),...,O(c_{n})\rbrace$, že $O(c_{i})\cap O(c_{i+1})\ne \varnothing$ pre $i=1,...,n-1,$ $a<c_{1}<x_{1}<c_{2}<x_{2}<...<x_{n-1}<c_{n}<b$, pričom  $a\in O(c_{1})$, $b\in O(c_{n})$, $x_{i}\in O(c_{i})\cap O(c_{i+1})$ pre $i=1,...,n-1$; potom $f(a)<f(c_{1})<f(x_{1}<...<f(b))$; teda $f(a)<f(b))$, ak $a<b$;
$\boxed{451.}$ $\boldsymbol{1.}$ to, že $f$ je rastúca v bode 0, možno dokázať (aspoň) dvoma spôsobmi: a/ priamo z definície: pre $x\in (0,1)$ je $x+x^{2} \sin \frac{2}{x}\geq x-x^{2} = x(1-x)>0$, pre $x\in (-1,0)$ je $x+x^{2} \sin \frac{2}{x}< x-x^{2}<0$ využili sme nerovnosť $\vert \sin \frac{2}{x}\vert \leq 1$ pre $x\ne 0$); b/ na základe pr. 412, pretože $f'(0)=1$;
$\boldsymbol{2.}$ $f'$ je spojitá na $\mathbb{R}- \lbrace 0 \rbrace$ a v ľubovoľne malom okolí bodu 0 nadobúda kladné aj záporné hodnoty (stačí vypočítať $f'(\frac{2}{\pi/2+2k\pi})$ pre $k \in \mathbb{Z}$ a $f'(\frac{1}{k\pi})$ pre $k \in \mathbb{Z} - \lbrace 0 \rbrace $, preto $f$ nemôže byť rastúca na žiadnom okolí bodu 0 (porovnaj aj s pr.350);
$\boxed{452.}$ uvedieme dva návody: a/ ukázať, že $f$ je klesajúca v každom bode $x\in I$, ďalej porovnaj s pr.450; b/ ak $M$ je konečná, porovnaj s pr.354;  ak $M$ je nekonečná, tak má aspoň jeden hromadný bod, ten musí byť krajným bododm intervalu $I$ (označme krajné body intervalu $I$ ako a, b, $a<b$); ak je tým hromadným bodom len bod b, možno  $M$ zoradiť do rastúcej postupnosti  $\lbrace a_{n} \rbrace _{n=1}^{\infty}$, $\lim_{n \to \infty}a_n =b$, pričom $f$ je klesajúca na $\langle a_{n}, a_{n+1} \rangle$  $n\in \mathbb{N}$; úvahy pre prípad, že množina hromadných bodov množiny $M$ je $\lbrace a \rbrace$, resp. $\lbrace a,b \rbrace$ sú podobné;
$\boxed{453.}$ matematickou indukciou; $f(\lambda _{1}x_{1}+...+  \lambda _{n}x_{n}+  \lambda _{n+1}x_{n+1})=f((1-\lambda _{n+1})(\frac{\lambda  _{1}}{1-\lambda _{n+1}}x_{1}+...+\frac{\lambda _{n}}{1-\lambda _{n+1}}x_{n})+\lambda _{n+1}x_{n+1})$, pritom  $\lambda _{1}+...+\lambda _{n}=1-\lambda _{n+1}$;
$\boxed{454.}$ $\boldsymbol{1.}$ využiť Jensenovu nerovnosť pre funkciu $f(x)=x^{2}$ (zvoliť  $\lambda _{1}=...=\lambda _{n}=\frac{1}{n}$);
$\boldsymbol{2.}$ zlogaritmovať; funkcia - $\ln x$ je konvexná na $\mathbb{R}^{+}$;
$\boldsymbol{3.}$ vynásobiť obidve strany číslom $1/n^{r}$ a využiť Jensenovu nerovnosť pre funkciu $f(x)=x^{r}$, $r>1, x>0$;
$\boxed{456.}$ ak je polynóm P párnou funkciou, tak $P(x)=a_{0}+a_{2}x^{2}+...+a_{2m}x^{2m}$; súčet klaných násobkov konvexných funkcií je konvexná funkcia;
$\boxed{458.}$ stačí dokázať druhé z uvedených tvrdení; pre $x\in (c,b)$ je funkcia $F(x):=\frac{f(x)-f(c)}{x-c}$ neklesajúca (pozri návod k pr. 362.1) a zdola ohraničená (napr. číslom $=\frac{f(a)-f(c)}{a-c}$; pre $x\in (c,b)$ ležia body grafu funkcie $f$  nad alebo na spojnici bodov $(a,f(a)), (c,f(c))$, preto $\lim_{x \to c+}F(x)$  existuje a je konečná (z uvedeného tiež vyplýva, že $f_{-}'(c)\leq f_{+}'(c)$);
$\boxed{459.}$ stačí použiť pr. 362.2;
$\boxed{461.}$ inverzná k rastúcej konvexnej (konkávnej) je rastúca konvexná (konkávna), ku klesajúcej  konvexnej (konkávnej) je klesajúca konvexná (konkávna); pri  dôkaze použite definíciu konvexnosti prepísanú na tvar $\forall x,y,z\in I,\, x<z<y: f(z)\leq f(x) +\frac{f(y)-f(x)}{y-x}(z-x)$ (porovnaj s návodom k pr. 362.1) a inšpirujte sa obrázkom;
$\boxed{463.}$ $\boldsymbol{1.}$ lok. maximum $3\sqrt[3]{3}$ pre $x=-3$, lok. minimum $-\sqrt[3]{44}$ pre $x=2$ (stačí nájsť lokálne extrémy odmocnenca a využiť, že $f(z)=\sqrt[3]{z}$ je rastúca funkcia);
$\boldsymbol{2.}$  pre $n$ párne: lok. maximum $n^{n}e^{1-n}$ pre $x=n-1$, lok. minimum 0 pre  $x=-1$; pre $n$ nepárne: lok. maximum $n^{n}e^{1-n}$ pre $x=n-1$;
$\boldsymbol{3.}$ ak $n$ je  párne: lok. maximum  $1$ pre $x=0$; ak $n$ je  nepárne: nemá lok. extrémy;
$\boldsymbol{4.}$ lok. minimum 0 pre  $x=0$ (pri úprave výrazu pre $y'$ využite vzorce $\sin \alpha \pm \cos \alpha = \sqrt{2}\sin (\frac{\pi}{4}\pm \alpha )$; $y$ rastie pre $x\geq 0$, klesá pre $x\leq 0$);
$\boxed{464.}$ $\boldsymbol{2.}$  najprv vydeľte výrazom $(x+a)$, potom položte $z=\frac{x}{a}$  a vyšetrite priebeh funkcie $f(z)=\frac{\sqrt[n]{1+z^{n}}}{1+z}$ $z\geq 0$;
$\boxed{465.}$  $x=\frac{H}{2}$ (dostrek je $\sqrt{2gx}t$, pričom $t$ vypočítame z rovnice $\frac{gt^{2}}{2=H-x}$, tj. $t$ je čas, za ktorý teleso spadne voľným pádom z výšky $H-x$);
$\boxed{466.}$  $(a^{2/3}+b^{2/3})^{3/2}$;
$\boxed{467.}$  $\boldsymbol{1.}$ $-3$;
$\boldsymbol{2.}$ $\frac{1}{3}$;
$\boldsymbol{3.}$ $0$ (rozšírte členom x a využite, že $\lim_{x \to 0}\frac{\ln (1+x)}{x}=1)$;
$\boldsymbol{4.}$ $+\infty$ (rozšírte členom x, l'Hospitalovo pravidlo použite len na prvý člen v čitateli);
$\boldsymbol{5.}$ $0$ (v exponente vyňať $x^{n}$, l'Hospitalovo pravidlo použite len na druhý člen v zátvorke);
$\boldsymbol{6.}$ $0$ (upraviť na tvar $e^{f(x)}$, v $f(x)$ vyňať $\ln ^{2}x$ pred zátvorku, l'Hospitalovo pravidlo použite len na druhý člen v zátvorke);
$\boldsymbol{7.}$ $1$ pri úpravách využite, že $\lim_{x \to 0}x^{x}=1)$;
$\boldsymbol{8.}$ $-1$;
$\boldsymbol{9.}$ $+\infty$;
$\boldsymbol{10.}$ $-\frac{1}{6}$ (odpočítať a pripočítať $\sqrt{x^{2}+x+1}$; prvý rozdiel možno previesť na typ $\frac{0}{0}$ substitúciou $x=\frac{1}{t}$ a vyňatím $\frac{1}{t}$; druhý rozdiel možno úpraviť na tvar $\ln (1+\frac{x}{e^{x}}).\frac{\sqrt{x^{2}+x+1}}{x})$;
$\boldsymbol{11.}$ a (vyňať $(x+a)^{1/x}$ a upraviť na tvar a $(x+a)^{1/x}-x(\frac{x^{1/(x+a)}}{(x+a)^{1/x}}-1)$, prvý člen v zátvorke upraviť na tvar $e^{u}$ a využiť, že $\lim_{u \to 0}\frac{e^{u}-1}{u}=1)$;
$\boxed{468.}$ pri dôkaze neexistencie $\lim_{x \to +\infty}\frac{f(x)}{g(x)})$ využite, že v bodoch $k\pi - \frac{\pi}{4}$ má funkcia $\frac{f}{g}$ nevlastné jednostranné limity opačného znamienka, preto  $\frac{f}{g}$ nie je zhora ani zdola ohraničená v žiadnom okolí bodu  $+\infty$;
$\boxed{469.}$ $\boldsymbol{1.}$ $\sum_{k=0}^n (-1)^{k}(k+1)x^{3k+1}$ pre $n=3n+1, 3n+2, 3n+3$ ($n\in \mathbb{N} \cup \lbrace 0 \rbrace $);
$\boldsymbol{2.}$ $\sum_{k=1}^n c^{k-1}_{-1/2}.(-1)^{k-1}\frac{x^{2k}}{2^{2k-1}}+\sum_{k=0}^n c^{k}_{-1/2}.(-1)^{k}\frac{x^{2k+1}}{2^{2k}}$, kde $c_{\alpha}^{0}=1$, $c_{\alpha}^{k}=\frac{\alpha(\alpha -1).....(\alpha -k+1)}{k!}$, $k\in \mathbb{N}$ (pre $x \in (-2,2)$ platí $f(x)=x(2+x)(4-x^{2})^{-1/2}$);
$\boldsymbol{3.}$ $\sum_{k=0}^n \frac{e^{-27}3^{k}}{k!}(x+3)^{2k+1}$ pre $n=2n+1,2n+2$ ($n=0,1,...)$;
$\boldsymbol{4.}$ $ 1+ \sum_{k=1}^n \frac{(-1)^{k-1}3^{k}}{k.2^{k}\ln 2}(x-4)^{2k}$ pre $n=2n,2n+1$  ($n\in \mathbb{N}$), Taylorov polynóm stupňa 1 má tvar $T(x)=1$ (využite vzorec $\log _{a} b=\frac{\ln b}{\ln a}$);
$\boldsymbol{5.}$ $\sum_{k=1}^n c^{k-1}_{-1/3}.(-1)^{k-1}(x-1)^{2k}$, Taylorov polynóm stupňa 1 má tvar $T(x)=0$ (symbol $c^{i}_{\alpha}$ pozri v návode k pr.469.2);
$\boldsymbol{6.}$ $-\frac{\sin 2}{2}+\frac{\sin 2}{2}(x+1)+\sum_{k=0}^n \frac{(-1)^{k}2^{2k}}{(2k+1)!}(x+1)^{4k+3}+\sum_{k=0}^n \frac{(-1)^{k+1}2^{2k}}{(2k+1)!}(x+1)^{4k+2}$; ($\sin \alpha .\cos \beta =\frac{1}{2}(\sin (\alpha +\beta)+\sin (\alpha -\beta))$);
$\boxed{470.}$ hodnoty $f^{(k)}(0)$ potrebné na nájdenie Maclaurinovho polynómu funkcie $f$ možno vyjadriť z koeficientov Maclaurinovho polynómu funkcie $f'$;
$\boldsymbol{1.}$ $x+\sum_{k=1}^n \frac{(-1)^{k}.1.3.....(2k-1)}{(2k+1)2^{k}k!} x^{2k+1}$ pre $n=2n+1,2n+2$ ($n\in \mathbb{N}$), Maclaurinove polynómy stupňa 1 a 2 majú tvar $T(x)=x$;
$\boldsymbol{2.}$ $\sum_{k=0}^n \frac{(-1)^{k}x^{2k+1}}{2k+1}$ pre $n=2n+1,2n+2$ ($n=0,1,...)$;
$\boldsymbol{3.}$ $\frac{\pi}{2}-x- \sum_{k=1}^n \frac{1.3.....(2n-1)}{(2k+1)2^{k}k!}x^{2k+1}$ pre $n=2n+1,2n+2$ ($n\in \mathbb{N}$), Maclaurinove polynómy stupňa 1 resp. 2 má tvar $T(x)=\frac{\pi}{2}- x$;
$$*$$
$$*$$
$$*$$
$$*$$
$\boxed{Obr.23.}$  $y=e^{-2x}\sin ^{2} x$
$\boxed{Obr.24.}$  $y^{3}=6x^{2}-x^{3}$
$\boxed{Obr.25.}$  $y^{2}=x^{4}(x+1)$
$\boxed{471.}$  nech $e=\frac{p}{q}$ ($p\in \mathbb{Z}, q\in \mathbb{N}$); zvoľme $n\geq q$; platí $\frac{p}{q}=e=1+\frac{1}{1!}+...+\frac{1}{n!}+\frac{\vartheta}{n!}$, pričom $\vartheta \in(0,1)$; ak z tejto rovnosti vyjadríme $\vartheta$, zistíme, že $\vartheta \in \mathbb{Z}$, čo je spor;
$\boxed{472.}$ ak uvedenú rovnosť porovnáme s Taylorovým vzorcom $(n+1)$-vého stupňa so zvyškom v Peanovom tvare, dostaneme po úprave: $\vartheta (x).\frac{f^{(n)}(a+\vartheta (x)(x-a))-f^{(n)}(a)}{\vartheta (x)(x-a)}=\frac{1}{n+1}.f^{(n+1)}(a)+\frac{o(x-a)}{x-a}$;
$\boxed{473.}$ $\boldsymbol{1.}$ $\frac{19}{90}$;
$\boldsymbol{2.}$ $\frac{1}{2}$;
$\boldsymbol{3.}$ $e^{2/3}$ ($=\lim_{x \to 0}(1+\frac{2}{3}x^{3}+o(x^{3}))^{1/x^{3}}$);
$\boldsymbol{4.}$ $\frac{1}{2}$ (subst. $\frac{1}{x}=t$;
$\boldsymbol{5.}$ $\frac{1}{3}$ (pred zátvorku vyňať x, potom subst. $\frac{1}{x}=t$);
$\boldsymbol{6.}$ $-\frac{1}{4}$;
$\boxed{474.}$ hodnoty $f(0),f(1)$ zapíšte pomocou Taylorovho polynómu stupňa 1 so stredom v bode x a so zvyškom v Lagrangeovom tvare, potom jednu z týchto rovností odčítajte od druhej;
$\boxed{475.}$ $\boldsymbol{1.}$ pre $M_{2}=0$ dokážeme uvedené tvrdenie samostatne; postup pre $M_{2}\ne 0$: z rovnosti (*) $f(x+h)=f(x)+f'(x)h+f''(c)h^{2}/2$ vyjadríme $f'(x)$; odtiaľ dostaneme: pre každé $h>0$ platí $M_{1}\leq 2M_{0}/h+M_{2}h/2$; odtiaľ $M_{2}(h-M_{1}/M_{2})^{2}+4M_{0}-M_{1}^{2}M_{2}\geq 0$  pre $h>0$, potom položíme $h=M_{1}/M_{2}$;
$\boldsymbol{2.}$ postup je analogický ako v pr. 475.1, naviac využijeme ešte rovnosť $f(x-h)=f(x)-f'(x).h+f''(d)h^{2}/2$, ktorú odčítame od rovnosti (*);
$\boxed{476.}$ $\boldsymbol{1.}$ obr. 23; stačí zostrojiť graf na niektorom intervale dĺžky $\pi$ a potom využiť vzťah  $y(x+\pi)=e^{-2\pi}$ $y(x)\quad (x\in \mathbb{R}) $; 
$\boldsymbol{2.}$ obr. 24; $y_{-}'(0)=-\infty$, $y_{+}'(0)=+\infty$, $y'(6)=-\infty$; pozor: $y$ nie je konkávna na $(-\infty ,6)$; podľa nami používanej definície nie je bod 6 inflexným bodom;
$\boldsymbol{3.}$ obr. 25; dotyčnica v bode $(-1,0)$ má rovnicu $x=-1$;
$\boldsymbol{4.}$ obr. 26; (pri riešení nerovníc  $y'\ne 0, y''\ne 0$ 	nezabúdajte, že pred umocňovaním na druhú treba skontrolovať, či obidve strany nerovnice majú rovnaké znamienko, a že pred násobením číslom $\alpha $ treba zistiť znamienko tohto čísla);
$$*$$
$$*$$
$$*$$
$$*$$
$\boxed{Obr.26.}$  $y=1-x+\frac{x^{3}}{3+x}$
$\boldsymbol{5.}$ obr. 27; dotyčnica v bode $(1,0)$ má rovnicu $x=1$; ak $y=x\sqrt{1-x}/(1+x)$, tak $y''=(1+x)^{-3}(1-x)^{-3/2}(-x^{3}-6x^{2}+15x-12)/4$, pritom  $-x^{3}-6x^{2}+15x-12<0$ pre $x>-5$ (to yistíme, ak vyšetríme rast, klesanie a lokálne extrémy funkcie $-x^{3}-6x^{2}+15x-12$ na intervale $(-5,\infty$ );
$\boxed{477.}$ $\boldsymbol{1.}$ obr. 28;
$\boldsymbol{2.}$ obr. 29.$$*$$
$$*$$
$$*$$
$$*$$
$\boxed{Obr.27.}$  $y^{2}=x^{2}(1-x)(1+x)^{-2}$
$\boxed{Obr.28.}$
$\boxed{Obr.29.}$



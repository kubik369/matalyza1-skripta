$\boxed{279.}$
$\boldsymbol{1.}\ 0,2;\boldsymbol{2.}\ 0;\boldsymbol{3.}\ 2;\boldsymbol{4.}\ 1;\boldsymbol{5.}\ 2;$
$\boxed{280.}$
$\boldsymbol{1.}\ f'(1)=+\infty;\boldsymbol{2.}\ \lim_{x\rightarrow 0+}\frac{\sgn x}{x}=\lim_{x\rightarrow 0+}\frac{1}{x}=\infty=\lim_{x\rightarrow 0-}-\frac{1}{x}=\lim_{x\rightarrow 0-}\frac{\sgn x}{x};$
$\boxed{281.}$
$\boldsymbol{1.}\ f'(x)=3x^2+2,D(f')=\mathbb{R};\boldsymbol{2.}\ f'(x)=\frac{4}{3}\sqrt[3]{x},D(f')=\mathbb{R};\boldsymbol{3.}\ f'(x)=-\frac{2x}{(1+x^2)^2},D(f')=\mathbb{R};\boldsymbol{4.}\ f'(x)=2^{x+1} \ln 2,D(f')=\mathbb{R};\boldsymbol{5.}\ f'(x)=\frac{1}{x},D(f')=(0,\infty);\boldsymbol{6.}\ f'(x)=-\frac{1}{{\sin}^2 n},D(f')=\mathbb{R}\setminus \{k\pi; k\in\mathbb{Z}\};\boldsymbol{7.}\ f'(x)=\frac{2}{3(x+1)^{\frac{1}{3}}},D(f')=\mathbb{R}\setminus {\ -1}\,$ pre $x=-1$ platí $\lim_{h\rightarrow 0+}\frac{f(-1+h)-f(-1)}{h}=\lim_{h\rightarrow 0+}\frac{1}{\sqrt[3]{h}}=\infty \neq -\infty=\lim_{h\rightarrow 0-}\frac{1}{\sqrt[3]{h}}=\lim_{h\rightarrow 0-}\frac{f(-1+h)-f(-1)}{h}$, teda $f'(-1)$ neexistuje; $\boldsymbol{8.}\ f'(x)=1,D(f')=\mathbb{R}\setminus \{0\}$; pre $x=0$ platí $f'(0)=\infty$ (pozri pr. $280.2$), preto $0\notin D(f')$;
$\boxed{282.}$
$\boldsymbol{1.}\ f'_+(\frac{\pi}{2})=1,f'_-(\frac{\pi}{2})=-1;\boldsymbol{2.}\ f'_+(1)=-\pi,f'_-(1)=0;\boldsymbol{3.}\ f'_+(2)=1,f'_-(2)=-1;\boldsymbol{4.}\ f'_+(3)=1,f'_-(3)=-1;$
$\boxed{283.}$
$\boldsymbol{1.}\ f'_+(0)=1=f'_-(0)$, teda $f'(0)=1;\boldsymbol{2.}\ f'_-(-1)=-2,f'_+(-1)=\lim_{x\rightarrow -1_+}\frac{-2x-0}{x+1}$ ($f(-1)$ sa vypočíta dosadením čísla $-1$ do výrazu $x^2-1$)$=\lim_{x\rightarrow -1_+}-\frac{2x}{x+1}=\infty;f'_+(-1)\notin f'_-(-1)$, preto $f'(-1)$ neexistuje; $\boldsymbol{3.}\ f'_+(0)=0=f'_-(0)$, teda $f'(0)=0;\boldsymbol{4.}\ f'_+(\pi)=f'_-(\pi)=0,$ teda $f'(\pi)=0$;
$\boxed{284.}$
$\boldsymbol{1.}\ \lim_{h\rightarrow 0}\frac{f(h)-f(-h)}{2h}=\lim_{h\rightarrow 0}(\frac{1}{2}(\frac{f(h)-f(0)}{h}+\frac{f(0)-f(-h)}{h})),$ pri výpočte limity druhého sčítanca použite substitúciu $t=-h$; $\boldsymbol{2.}\ $ nevyplýva, stačí uvažovať $f(x)=|x|$; 
$\boxed{286.}$
$\boldsymbol{1.}\ y'=1-2x;\boldsymbol{2.}\ y'=x^{\frac{2}{3}}+\sqrt{5}x^{-\sqrt{5}-1}$; $\boldsymbol{3.}\ y'=\frac{2(1-2x)}{(1-x-x^2)^2};\boldsymbol{4.}\ y'=\frac{6-\sqrt[3]{x^2}}{6\sqrt{x}(2+\sqrt[3]{x^2})^2};\boldsymbol{5.}\ y'=30(3x-7)^9;\boldsymbol{6.}\ y'=\frac{1+2x^2}{\sqrt{1+x^2}};\boldsymbol{7.}\ y'=-(p+q+(p-q)x)*\frac{(1-x)^{p-1}}{(1+x)^{q+1}};\boldsymbol{8.}\ y'=-\frac{1}{(1+x^2)^{\frac{3}{2}}};\boldsymbol{9.}\ y'=\frac{7\sqrt[5]{2}}{65\sqrt[5]{x^4}\sqrt[13]{(9+7\sqrt[5]{2x})^12}};\boldsymbol{10.}\ y'=\frac{4\sqrt{x+\sqrt{x}}*\sqrt{x}+2\sqrt{x}+1}{8\sqrt{x}*\sqrt{x+\sqrt{x}}*\sqrt{x+\sqrt{x+\sqrt{x}}}};$
$\boxed{287.}$
$f'(a)=\lim_{x\rightarrow a}\frac{(x-a)\varphi (x)}{x-a}=\varphi (a);$
$\boxed{288.}$
$\boldsymbol{1.}\ f'_+(a)=\varphi (a)\neq -\varphi (a)=f'_-(a);\boldsymbol{2.}\ f'(a)=\lim_{x\rightarrow a}(\varphi (x) \sgn (x-a))*|x-a|^{\varepsilon}=0$ (funkcia $\varphi (x) sgn (x-a)$ je ohranižená, $\lim_{x\rightarrow a}|x-a|^{\varepsilon}=0$);
$\boxed{289.}$
$\boldsymbol{1.}\ y'=\cos x - x\sin x;\boldsymbol{2.}\ y'=\frac{x^2}{(\cos x + x\sin x)^2};\boldsymbol{3.}\ y'=\frac{\sin x \cos x - 2x}{2\sqrt{x}\sin ^2 x},x>0,x\neq \frac{k\pi}{2},k\in \mathbb{N}$ (Doteraz sme pri zápisoch využívali dohodu, podľa ktorej definičný obor funkcie - pokiaľ nie je výslovne určený - je množina všetkých tých $x\in \mathbb{R}$, pre ktorý má daný predpis 'zmysel'. Funkcia $f(x)=\frac{\sqrt{x}}{\tan x}$ je definovaná na $D(f)=\{x\in \mathbb{R};x>0 \wedge x\neq \frac{k\pi}{2},k\in \mathbb{N}\}$ a podľa vety o derivácii podielu má v každom jej bode konečnú deriváciu, preto $D(f')=D(f)$; množina $D(f)$ sa v tomto prípade nezhoduje s množinou $\{x\in \mathbb{R}; x>0 \wedge x\neq k\pi, k\in \mathbb{N} \}$ všetkých tých reálnych čísel, pre ktoré má výraz $f'(x)=\frac{(\sin x \cos x - 2x)}{(2\sqrt{x}\sin ^2 x)}$ 'zmysel');$\boldsymbol{4.}\ y'=n\sin^{n-1}x \cos (n+1)x; \boldsymbol{5.}\ y'=-\sin 2x*\cos (\cos 2x);\boldsymbol{6.}\ y'=2\sin x * \frac{\cos x \sin x^2 - x\sin x \cos x^2}{\sin^2 x^2};\boldsymbol{7.}\ y'=-\cos 2x,x\neq \frac{k\pi}{2}, x\neq -\frac{\pi}{4}+k\pi,k\in \mathbb{Z}$ (pozri poznámku k pr. $289.3$); $\boldsymbol{8.}\ y'=\frac{x^4-1}{x^3\cos^2 (x^2+x^{-2})\sqrt{1+\tan^2(x^2+x^{-2})}}$
$\boxed{290.}$ 
$\boldsymbol{1.}$ $\ y'= \frac{1}{2}(\sqrt{2^{x}}\ln 2-\sqrt{5^{-x}} \ln 5)$;
$\boldsymbol{2.}$ $\ y'=-2x e^{-x^{2}}$;
$\boldsymbol{3.}$ $\ y'= e^{x}x^{2}$;
$\boldsymbol{4.}$ $\ y'= 2^{1+ \sin x^{2}}.x \cos x^{2}\ln 2$; 
$\boldsymbol{5.}$ $\ y'= -e^{\sqrt{\frac{1-x}{1+x}}}(1+x)^{-3/2}(1-x)^{-1/2}$;
$\boldsymbol{6.}$ $\ y'= e^{ax}\sqrt{a^{2}+b^{2}}\sin bx$;
$\boldsymbol{7.}$ $\ y'=(\frac{a}{b})^{x}(\frac{b}{x})^{a}(\frac{x}{a})^{b}(\ln a-\ln b-\frac{a}{x}+\frac{b}{x})$;
$\boldsymbol{8.}$ $\ y'= x^{2}e^{-x}\sin x$;
$\boxed{291.}$ 
$\boldsymbol{1.}$ možno (v opačnom prípade by existovala vlastná alebo nevlastná $\ g'(a) $, pretože  $\ g=(f+g)-f)$;
$\boldsymbol{2.}$ nemožno (ak zvolíme $f(x)=\sqrt[3]{x}$,  $g_{1}(x)=\vert x \vert$, $g_{2}(x)=\sgn x .\sqrt[3]{x}$, tak existuje nevlastná $(f+g_{1})'(0)$, a neexistuje $(f+g_{2})'(0)$; za uvedených predpokladov ale nemôže existovať vlastná $(f+g)'(a)$;
$\boldsymbol{3.}$ nemožno (ak zvolíme $f_{1}(x)=g_{1}(x)=\vert x \vert$, $f_{2}(x)=x+\vert x \vert$, $g_{2}(x)=x-\vert x \vert$, $f_{3}(x)=(1+ \sgn x)\sqrt[3]{x}$, $g_{3}(x)=(1- \sgn x)\sqrt[3]{x}$, tak $(f_{1}+g_{1})'(0)$ neexistuje, $(f_{2}+g_{2})'(0)$ je vlastná a $(f_{3}+g_{3})'(0)$ je nevlastná);
$\boxed{292.}$ dôkaz prebieha analogicky ako dôkaz vety o derivácii zloženej funkcie pre prípad vlastvých derivácií;
$\boxed{293.}$ 
$\boldsymbol{1.}$ $\ y'= \frac{1}{\ln 3 . \sin x}, \, x \in \bigcup\limits_{k \in \mathbb{Z}}(2k\pi, (2k+1)\pi)$ (pozri poznámku k pr. 289.3);
$\boldsymbol{2.}$ $\ y'= \frac{x}{x^{4}-1}$, $\vert x\vert >1 $;
$\boldsymbol{3.}$ $\ y'= \frac{1}{\sqrt{x^{2}+1}}$;
$\boldsymbol{4.}$ $\ y'= \frac{1}{(1+x^{2})(1+x)^{2}}$, $ x >-1 $;
$\boldsymbol{5.}$ $\ y'= \ln (x+\sqrt{1+x^{2}})$;
$\boldsymbol{6.}$ $\ y'= \frac{(2x+1) e^{\sqrt{\log _{2}(x^{2}+x+1)}}}{\ln 4(x^{2}+x+1)\sqrt{\log_{2}(x^{2}+x+1)}}$;
$\boldsymbol{7.}$ $\ y'= \frac{2\sin a}{(1-x^{2})(1-x^{2}\cos ^{2}a)}$,$\vert x\vert <1 $ ;
$\boldsymbol{8.}$ $\ y'= 2\sin \ln x $;
$\boldsymbol{9.}$ $\ y'= \frac{x^{2}+\sgn x}{x\sqrt{x^{2}+1}}$;
$\boldsymbol{10.}$ $\ y'= \frac{1}{x}(\frac{\ln x. \log _{3}x}{\ln 2}+\log _{2}x.\log _{3}x+\frac{\log _{2}x.\ln x}{\ln 3})$;
$\boxed{294.}$ 
$\boldsymbol{1.}$ $\ y'=\frac{x+2}{2\sqrt{x}(x+1)} $;
$\boldsymbol{2.}$ $\ y'=\arcsin x \frac{x}{\sqrt{(1-x^{2})}} $;
$\boldsymbol{3.}$ $\ y'=\frac{\pi}{2\sqrt{(1-x^{2})}\arcsin ^{2}x} $; využite pr. 62.1);
$\boldsymbol{4.}$ $\ y'=\frac{1}{2\sqrt{x}(1-x^{2})} $;
$\boldsymbol{5.}$ $\ y'=\arcsin \sqrt {\frac{x}{1+x}}, \, x\geq 0 $ (tabuľkovým derivovaním možno nájsť hodnoty $\ y'$ len pre $x>0$ pretože v bode $0$ má funkcia $\sqrt{x}$ nevlastnú deriváciu; derivácia funkcie $x \arcsin \sqrt{\frac{x}{1+x}} + \arctan \sqrt{x}-\sqrt{x}$ v bode $0$ sa vypočíta na základe definície derivácie;
$\boldsymbol{6.}$ $\ y'=\frac{1}{2x\sqrt{x-1}\arccos \frac{1}{\sqrt{x}}} $;
$\boldsymbol{7.}$ $\ y'=\frac{3^{\arctan (2x+\pi)}.\ln 9}{1+(2x+\pi)^{2}} $;
$\boldsymbol{8.}$ $\ y'=\frac{\sin x. \sgn (\cos x-\cos a)}{1-\cos a. \cos x} $, $\cos x\ne \cos a $ (využite rovnosť $(\cos x- \cos a )^{2}=(1-\cos x. \cos a )^{2}-(\sin x. \sin a )^{2}$ ;
$\boldsymbol{9.}$ $\ y'=(\arcsin x)^{2} $; (tabuľkovým derivovaním možno $\ y'$ nájsť len pre $ \vert x\vert <1$, pretože $arcsin  $ má v bodoch $1,-1 $ nevlastnú deriváciu; platnosť uvedeného vzťahu pre body $1,-1 $ by sa nemusela overovať z definície; neskôr - pozri pr. 384 - uvedieme efektívnejší spôsob nájdenia derivácie v bodoch $1,-1 $);
$\boldsymbol{10.}$ $\ y'=\frac{\cos x}{\sqrt{3-2\sin ^{2}x}} $;
$\boxed{295.}$ 
$\boldsymbol{2.}$ $\ y'=\frac{24-x-5x^{2}}{3\sqrt{x-1}\sqrt[3]{(x+2)^{5}}\sqrt{(x+3)^{5}}} $; (uvedený postup možno použiť pre $x>1$; derivácia v bode $1$  je nevlastná;
$\boldsymbol{3.}$ $\ y'=\frac{ctg 3x}{1-\sin 3x}.\sqrt[3]{\frac{\sin 3x}{1-\sin 3x}} $, $x\in (0,\frac{\pi}{6} )$;
$\boldsymbol{5.}$ $\ y'=\frac{1}{2}.(1-\ln x) \sqrt[x]{x} $;
$\boldsymbol{6.}$ $\ y'=x(2\ln x+1)x^{x^{2}} $;
$\boldsymbol{7.}$ $\ y'=(\cos x)^{\sin x}(\cos x.\ln\cos x-\tan x.\sin x)^{\cos x}(\sin x.\ln \sin x-\cos x. ctg x)$, $x\in (0,\frac{\pi}{2} ) $ ;
$\boxed{296.}$ 
$\boldsymbol{1.}$ napr. $f(x)= g(x)=\vert x-a\vert $;
$\boldsymbol{2.}$ napr. $f(x)=x-a, \, g(x)=\vert x-a\vert $;
$\boxed{297.}$ dôkaz sa robí analogicky ako v prípade vlastnej $\ f'(a)$;
$\boxed{299.}$ 
$\boldsymbol{1.}$ $\ y'=\frac{\sqrt{b^{2}-a^{2}}}{a+b \cos x},\,x\in (0,\frac{\pi}{2} )  $;
$\boldsymbol{2.}$ $\ y'=-2\cos x.\arctan(\sin x) $;
$\boldsymbol{3.}$ $\ y'=-\frac{\arccos x}{x^{2}},\,0<\vert x\vert<1 )  $;
$\boldsymbol{4.}$ $\ y'=\ln^{2}(x+\sqrt{1+x^{2})}$;
$\boldsymbol{5.}$ $\ y'=\frac{a^{2}+b^{2}}{(x^{2}+b^{2})(x+a)}$, $x > -a$;
$\boldsymbol{6.}$ $\ y'=\frac{1}{x^{4}+1}$, $ \vert x \vert\ne 1$;
$\boldsymbol{7.}$ $\ y'=-e^{-x}arcctg\, e^{x}$;
$\boldsymbol{8.}$ $\ y'=\frac{\cos x.\sin ^{2}x}{\sqrt{\cos ^{2}x-2\sin x}}$, $ \sin x \leq\sqrt{2}- 1$ (bolo možné použiť substitúciu $t=1+\sin x $);
$\boldsymbol{9.}$ $\ y'=\frac{m}{\sqrt{1-x^{2}} }.2\cos (m\arcsin x)e^{m \arcsin x},\,\vert x\vert<1 )$;
$\boldsymbol{10.}$ $\ y'=\frac{x}{x^{2}-2x\cos a+1}$;
$\boldsymbol{11.}$ $\ y'=(5x-1)(x-1)(x+1)^{2}$ pre $x > -1$, $\ y'=-(5x-1)(x-1)(x+1)^{2}$ pre $x < -1$, $y_{+}'(-1)=y_{-}'(-1)=0$, teda $y'(-1)=0$; to možno naraz zapísať v tvare $\ y'=(5x-1)(x-1)(x+1)^{2}\sgn (x+1)$;
$\boldsymbol{12.}$ $\ y'=\pi n \sin 2\pi x$ pre ; $x \in (n,n+1),\, y_{+}'(n)=y_{-}'(n)=0$, teda $y'(n)=0$ ($n\in \mathbb{Z}$); to možno naraz zapísať  $y'=\pi[x] \sin 2\pi x$;
$\boldsymbol{13.}$ $\ y'= \left\{\begin{matrix} -1, & \mbox{ak }x <1 \\ 2x-3, & \mbox{ak }x \in \langle 1,2\rangle \\ 1, & \mbox {ak  }x >2\end{matrix}\right. $;
$\boldsymbol{14.}$ $\ y'= \left\{\begin{matrix} \frac{1}{1+x^{2}}, & \mbox{ak } -1<x\leq 1 \\ \frac{1}{2}, & 
         \mbox{ak   } \vert x \vert > 1\end{matrix}\right.$; $y_{+}'(-1)=\frac{1}{2}$, $y_{-}'(-1)=+\infty$, teda $y'(-1)$ neexistuje;
$\boldsymbol{15.}$ $\ y'=th^{3}x$, pritom sa použije vzťah $\frac{1}{ch^{2}x}=1-th^{2}x$;  
$\boldsymbol{16.}$ $\ y'=(\ln x)^{x}.(\ln(\ln x)+\frac{1}{\ln x}-\frac{2\ln x}{x}/(x^{\ln x})$, $ x >1$;
$\boldsymbol{17.}$ $\ y'=(\frac{\sin 2x . \arctan x}{\sqrt{1-\sin ^{4}x}\arcsin (\sin ^{2}x}+\frac{\ln \arcsin (\sin ^{2}x)}{1+x^{2}}).(\arcsin (\sin ^{2}x))^{\arctan x}$;
$\boldsymbol{18.}$ $\ y'=x^{x^{x}}.x^{x}(\ln ^{2}x+\ln x+\frac{1}{x})$;
$\boldsymbol{19.}$  použitím vzťahu $\log _{x }e=\frac{1}{\ln x}$   možno daný výraz upraviť na $y=e^{2}-2xe^{2}+ex^{2}$  $x>0, \, x\ne 1)$; teda $\ y'=ex(x-e)$,  $x>0, \, x\ne 1)$;
$\boxed{300.}$ $a=2x_{0}$, $b=-x_{0}^{2}$;
$\boxed{301.}$ dôkaz sa robí matematickou indukciou, pričom sa použije veta o derivácii súčinu; $f'(0)=10001$;
$\boxed{302.}$ $\boldsymbol{1.}$ $\ y'=\frac{ff'+gg'}{\sqrt{f^{2}+g^{2}}}$;
$\boldsymbol{2.}$ $\ y'=\frac{f'g+fg'}{f^{2}+g^{2}}$;
$\boldsymbol{3.}$ $\ y'=\frac{fg'.\ln f+gf'.\ln g}{f.g.\ln^{2}f}$ (treba použiť vzťah  $\log _{f}g=\frac{\ln g}{\ln f}$);
$\boldsymbol{4.}$ $\ y'=2x f'(x^{2})$;
$\boldsymbol{5.}$ $\ y'=\sin 2x f'(\sin ^{2}x)-g'(\cos ^{2}x))$;
$\boldsymbol{6.}$ $\ y'=e^{f(x)}(e^{x}.f'(e^{x})+f(e^{x}).f'(x))$;
$\boxed{303.}$ $\boldsymbol{1.}$ $P_{n}(x)=(x+x^{2}+...+x^{n})'=(\frac{x^{n+1}-x}{x-1})'=\frac{nx^{n+1}-(n+1)x^{n}+1}{(x-1)^{2}} $ pre $x \ne 1$,  $P_{n}(1)=\frac{n(n+1)}{2} $;
$\boldsymbol{2.}$ $Q_{n}(x)=(\frac{x^{2n+1}-x}{x^{2}-1})'=\frac{(2n-1)x^{2n+2}-(2n+1)x^{2n}+x^{2}+1}{(x^{2}-1)^{2}}$; 
$\boldsymbol{3.}$ $R_{n}(x)=(xP_{n}(x))'=\frac{n^{2}x^{n+2}-(2n^{2}+2n-1)x^{n+1}+(n+1)^{2}x^{n}-x-1}{(x-1)^{3}}$; 
$\boldsymbol{4.}$ $T_{n}(x)=(\sum_{k=1}^n \sin kx)'=(\frac{\cos \frac{x}{2}- \cos \frac{(2n+1)\pi}{2}}{2\sin \frac{x}{2} })'=\frac{-1+(n+1)\cos nx -n \cos (n+1)x}{4 \sin ^{2}\frac{x}{2}}$;
$\boxed{304.}$ $\boldsymbol{1.}$ $\alpha >0 $;
$\boldsymbol{2.}$ $\alpha >1 $;
$\boldsymbol{3.}$ $\alpha >2 $ (pre $x \ne 0$ možno  $f'(x)$ nájsť tabuľkovým derivovaním, $f'(0)$ sa musí hľadať priamo z definície; treba si uvedomiť, že $\lim_{x \to 0}\cos \frac{1}{x} $ neexistuje, $\lim_{x \to 0}x^{\beta} \sin \frac{1}{x}=0 $ pre $\beta >0 $; pre $\beta \leq 0 $ táto limita neexistuje);
$\boxed{305.}$ nie (funkcia $\sqrt[3]{x^{2}}$ nemá vlastnú ani nevlastnú deriváciu v bode $0$; jednostranné derivácie sú nevlastné a opačných znamienok); $y'(0)=0$ (vypočíta sa z definície);
$\boxed{306.}$ $\boldsymbol{1.}$ napr. $f(x)=(x-1)^{2}(x-2)^{2} $ pre $x \in \mathbb{Q}, \, f(x)=0 $ pre $x \in \mathbb{R} -\mathbb{Q}$;
$\boldsymbol{2.}$ napr. funkcia určená podmienkami $f(x)=\sqrt[3]{\frac{x}{4}}$ pre $x \in \langle -\frac{1}{2},\frac{1}{2}\rangle$, $f(x+1)=f(x)+1 $ ($x \in \mathbb{R}$);
$\boldsymbol{3.}$ napr. $f(x)=ax+g(x)$, kde $g$ je určená podmienkami $g(x)=0$ pre $x \leq 0, \, g(x)=x^{2}(x-1)^{2}\sin \frac{1}{x(x-1)} $  pre $x \in (0,1), \,g(x+1)=g(x) $ pre $x \geq 0 $;
$\boxed{308.}$ napr. $f(x)=(1+\sgn (x-1)). \frac{\arccos(\cos \pi(x-1)}{\pi}$, tj. $f(x)=0 $ pre $x<1 $, $f(x)=x-k $ pre $x\in \langle k, k+1), \, k \in \mathbb{N} $, $k$ nepárne, $f(x)=k+1-x$ pre $x\in \langle k, k+1), \, k \in \mathbb{N} $, $k$ párne;
$\boxed{309.}$ $\boldsymbol{1.}$ napr. $f(x)=-\sqrt[3]{x-a}$;
$\boldsymbol{2.}$ napr. $f(x)=\sgn (x-a)$;
$\boldsymbol{3.}$ napr. $f(x)=\sgn (x-a).(\sin \frac{1}{x-a}-2)$ pre  $x \ne a $, $f(a)=0$;
$\boxed{310.}$ $\boldsymbol{1.}$ $\beta /b=f(\sqrt{2}), \, (f^{-1})'(\frac{6}{5})=\frac{1}{2}$;
$\boldsymbol{2.}$ $b=f(0), \, (f^{-1})'(-\frac{1}{2})=\frac{1}{2}$;
$\boldsymbol{3.}$ $b=f(0), \, (f^{-1})'(1)=5$; 
$\boldsymbol{4.}$ $b=f(\sqrt{2}), \, (f^{-1})'(0)= - \frac{1}{4\sqrt{2}}$;
$\boldsymbol{5.}$ $b=f(\frac{1}{\sqrt{2}}), \, (f^{-1})'(\frac{3}{4})= \frac{1}{\sqrt{2}}$;
$\boxed{311.}$ pretože $f^{-1}(f(x))=x $ v niektorom okolí bodu $a$, je $f^{-1}(f(a)).f'(a)=1 $; treba predpokladať existenciu vlasnej  $(f^{-1})'(f(a)) $ a vlasnej  $f'(a)\ne 0 $;
$\boxed{312.}$ $\boldsymbol{1.}$ $f'(x)= (1-x^{2})^{-1/2}$;
$\boldsymbol{2.}$ $f'(x)= -(1-x^{2})^{-1/2}$;
$\boldsymbol{3.}$ $f'(x)= \frac{1}{1+x^{2}}$;
$\boldsymbol{4.}$ $f'(x)= -\frac{1}{1+x^{2}}$;
$\boldsymbol{5.}$ $f'(x)= \frac{1}{x}, \, x>0$;
$\boxed{313.}$ dôkaz je analogický dôkazu vety o derivácii inverznej funkcie;
$\boxed{314.}$ $\boldsymbol{1\alpha/}$ $y= \sqrt[3]{4}(x+1)$; $\boldsymbol{1\beta/}$ $y= 3$;
$\boldsymbol{2.}$ $y= x+1$;
$\boldsymbol{3.}$ $x=1$;
$\boldsymbol{4.}$ $A=(0,1), y=1$;
$\boldsymbol{5.}$ $A=(1,\frac{1}{2}), x+2y-2=0$;
$\boxed{315.}$ $\boldsymbol{1.}$ $12x-4y-13=0$;
$\boldsymbol{2.}$ $4x-4y+3=0$;
$\boxed{316.}$ $\boldsymbol{1.}$ $b^{2}-4ac=0$;
$\boldsymbol{2.}$ $a=\frac{1}{2e}$;
$\boxed{317.}$ $\boldsymbol{1.}$ $d(\frac{\ln x}{\sqrt{x}})(a)=\frac{2- \ln a}{2a\sqrt{a}}\mathrm{d}x (a)$; (teda  $d(\frac{\ln x}{\sqrt{x}})(a)$ je funkcia daná predpisom $\frac{2- \ln a}{2a\sqrt{a}}(x-a)$);,
$\boldsymbol{2.}$ $\frac{a}{\sqrt{A^{2}+a^{2}}}\mathrm{d}x (a)$;
$\boldsymbol{3.}$ $\frac{2a}{a^{2}-1}\mathrm{d}x (a)$ ($\vert a \vert <1) $;
$\boldsymbol{4.}$ $\frac{1}{\cos ^{3}a}\mathrm{d}x (a)$;
$\boxed{318.}$ $\boldsymbol{1.}$ $dy(a)=(u(a).v(a).w'(a)+u(a).v'(a).w(a)+u'(a).v(a).w(a) )\mathrm{d}x (a)$, čo sa stručne zapisuje v tvare $dy=u.v.dw+u.dv.w+du.v.w$;
$\boldsymbol{2.}$ $dy=\frac{v^{2}.du-2u.v.dv}{v^{4}}$;
$\boldsymbol{3.}$ $dy=\frac{w.du-u.dw}{u^{2}+w^{2}}$;
$\boldsymbol{4.}$ $dy=\frac{u.du-v.dv}{u^{2}+v^{2}}$;
$\boxed{319.}$ $\boldsymbol{1.}$ $\sqrt{a^{2}+x^{2}}\approx a+\frac{x}{2a}$ pre $x$ blízke $0$;
$\boldsymbol{2.}$ $\ln{x}+\sqrt{1+x^{2}} \approx x$ pre $x$ blízke $0$;
$\boldsymbol{3.}$ $\arctan (1+x^{2}) \approx \frac{\pi}{4}+\frac{1}{2} x$ pre $x$ blízke $0$;
$\boldsymbol{4.}$ $\sqrt[n]{a^{n}+x}\approx a+\frac{x}{na^{n-1}}$ pre $x$ blízke $0$;
$\boxed{320.}$ "$\Rightarrow $": stačí položiť $\varphi (x)=A+\frac{\omega(x)}{x-a}$ pre $x \ne a$, $\varphi(a)=0$; "$\Leftarrow $": $\varphi(x)=0$ možno písať v tvare $A+(\varphi(x)-A)$, kde $A=\lim_{x \to a}\varphi(x)$; potom $\varphi(x)=(\varphi(x)-A)(x-a)$;
$\boxed{321.}$ $\boldsymbol{1.}$  $y''=\frac{2x^{3}+3x}{(1+x^{2})^{3/2}} $;
$\boldsymbol{2.}$  $y''=3x(1-x^{2})^{-5/2} $;   
$\boldsymbol{3.}$  $y''=2(2x^{2}-1)e^{-x^{2}} $;   
$\boldsymbol{4.}$  $y''=2\arctan x+\frac{2x}{1+x^{2}} $;
$\boldsymbol{5.}$  $y''=\frac{1}{x}, \, x>0 $;
c
$\boxed{322.}$ $\boldsymbol{2.}$  $y^{V}=2^{2}.3^{3}.5! $; $y^{VI}=0$;
$\boldsymbol{3.}$  $y^{(10)}=-(3.5.7.....17)2^{-10}.x^{-19/2}$;
$\boldsymbol{4.}$  $y=-1+\frac{2}{1-x}, \, y^{(22)}=2.221.(1-x)^{-23} $;
$\boldsymbol{6.}$  $y=\frac{1}{2}(\sin 6x-\sin 2x), \, y^{(15)}=\frac{1}{2}(2^{15}\cos 2x -6^{15}\cos 6x) $;
$\boldsymbol{7.}$  $y=\frac{1}{2}\sin 2x(\cos 2x-\cos 4x)= \frac{1}{4}\sin 4x -\frac{1}{4}(\sin 6x-\sin 2x) , \, y^{(10)}=\frac{1}{4}(-4^{10}\sin 4x +6^{10}\sin 6x-2^{10}\sin 2x) $;
$\boxed{323.}$ $\boldsymbol{1.}$  $((1-x)^{-1/2})^{(100)}=\frac{3.5.....199}{2^{100}}(1-x)^{-201/2} $, $((1-x)^{-1/2})^{(99)}=\frac{3.5.....197}{2^{99}}(1-x)^{-199/2} $, $y^{(100)}=\frac{3.5.....197}{2^{100}}\sqrt{(1-x)^{201}}(399-x) $;
$\boldsymbol{2.}$  $y^{V}=\frac{-6}{x^{4}}, x>0$;
$\boldsymbol{3.}$  $y^{(50)}=2^{49}(1225-2x^{2})\sin 2x + 50.2^{50}x \cos 2x$;
$\boldsymbol{4.}$  $y'''=-\frac{9\sin 3x}{\sqrt[7]{(1-3x)^{3}}}(27x^{2}-18x-1)-\frac{\cos 3x}{\sqrt[3]{(1-3x)^{10}}}(243x^{2}-162x-1)$;
$\boldsymbol{5.}$  $y^{(101)}=3^{99}(10100+18x-9x^{2})\sin 3x + 202.3^{100}(x-1). \cos 3x$;
$\boldsymbol{6.}$ Leibnizovým vzorcom sa vypočíta  $15$. derivácia funkcie; $y'=2x+\sin 2x - 2 \sin x-2x \cos x$,   $y^{(16)}=-2^{15}\cos 2x+32 \cos x - 2x \sin x$; 
$\boldsymbol{7.}$  $y=\frac{1}{2}x(\sin 3x -\sin x)$, $y^{(100)}=\frac{1}{2}(x(3^{100}\sin 3x \sin x)+100\cos x - 100.3^{99}\cos 3x)$;
$\boxed{325.}$ $\boldsymbol{1.}$ $y^{n}=\frac{(-1)^{n-1}(n-1)!a^{n}}{(ax+b)^{n}}, ax+b>0$;
$\boldsymbol{2.}$ $y^{(n)}=n![\frac{(-1)^{n}}{x^{n+1}}+\frac{1}{(1-x)^{n+1}}](y=\frac{1}{x}+\frac{1}{1-x}) $;
$\boldsymbol{3.}$ $y^{(n)}=\frac{(-1)^{n}n!}{(x-2)^{n+1}}+\frac{n!}{(1-x)^{n+1}}(y=\frac{1}{x-2}+\frac{1}{1-x}) $;
$\boldsymbol{4.}$ $y^{(n)}=\frac{1.3.....(2n-1)}{(1-2x)^{2n+1}}$;
$\boldsymbol{5.}$  $\ y'=\sin 2x$;  $y^{(n)}=(\sin 2x)^{(n-1)}=2^{n-1}\sin (2x+\frac{(n-1)\pi}{2})$;
$\boldsymbol{6.}$  $\ y^{(n)}=\frac{1}{2}(8^{n}\sin (8x+\frac{n\pi}{2})-2^{n}\sin (2x+\frac{n\pi}{2}))$  $y=\frac{1}{2}(\sin 8x- \sin 2x))$;
$\boldsymbol{7.}$  $\ y^{(n)}=4^{n-1}\cos (4x+\frac{n\pi}{2})$ (pozri návod k pr. 40.2);
$\boldsymbol{8.}$  $\ y^{(n)}=\frac{1}{2} \sin ax.(\sin (a+b)x +\sin (a-b)x)=\frac{1}{4}(2\cos bx-\cos(2a+b)x -\cos (2a-b)x)$, $\ y^{(n)}=\frac{1}{4} (2b^{n} \cos(bx+\frac{n\pi}{2})-(2a+b)^{n}\cos ((2a+b)x+\frac {n\pi}{2})-(2a-b)^{n}\cos ((2a-b)x+\frac{n\pi}{2}))$;
$\boxed{326.}$ $\boldsymbol{1.}$ $y^{(n)}=2^{x-1}.\ln^{n}2 (x-1+\frac{n}{\ln 2}) $ (použil sa Leibnizov vzorec);
$\boldsymbol{2.}$ $((1+x)^{-1/3})^{(n)} =(-1)^{n}\frac {1.4.7.....(3n-2)}{3^{n}}(1+x)^{-(3n+1)/3}$,  $y^{(n)}=(-1)^{n+1}\frac{1.4.7.....(3n-5)}{3^{n}(1+x)^{n+1/3}}.(2x+3n)$ pre $n\geq 2$, $\ y'=\frac{3+2x}{\sqrt[3]{(1+x)^{4}}}$ (výsledok pre $\ y'$ sa musí uviesť samostatne, pretože vzťah pre $((1+x)^{-1/3})^{(n-1)}$, ktorý dosádzame do Leibnizovho vzorca, neplatí pre $n=1$);
$\boldsymbol{3.}$ $y^{n}=b^{n-2}.\sin (bx+\frac{n\pi}{2})(b^{2}x^{2}-n(n-1))+2nb^{n-1} x \sin (bx+\frac{n-1}{2}\pi )$ (pri úprave sa použil vzťah $\sin (A-\pi)=-\sin A$; pre $n\geq 2$ má Leibnizov vzorec tvar  $y^{(n)}=x^{2}(\sin bx)^{(n)}+ \begin{pmatrix}n \\1 \end{pmatrix} 2x.(\sin bx)^{(n-1)}+ \begin{pmatrix}n \\2 \end{pmatrix} 2.(\sin bx)^{(n-2)}$, $\ y'$ treba vypočítať samostatne, zhodou okolností možno výsledky písať v uvedenej spoločnej podobe);
$\boldsymbol{4.}$ $y^{(n)}=(-1)^{n}e^{-x}(x^{2}+(n-1)(n-2))$;
$\boldsymbol{5.}$ $(\log _{2}(1-3x))^{n}=-\frac{(n-1)!3^{n}}{(1-3x)^{n}\ln 2}$,  $y^{(n)}=-\frac{(n-2)!3^{n-1}}{(1-3x)^{n}\ln 2}(3x-n)$ pre $n>1$, $\ y'=\log _{2}(1-3x)-\frac{3x}{(1-3x)\ln 2}$ ($\ y'$ uvádzame samostatne z toho istého dôvodu ako v pr. 326.1); 
$\boldsymbol{6.}$ $y^{(n)}=(-3)^{n-2}e^{2-3x}(36x^{2}-12(9+2n)x+4n^{2}+32n+81)$ pre $n\geq 2$, $\ y'=e^{2-3x}(3-2x)(6x-7)$ ($\ y'$ treba vypočítať samostatne, pretože Leibnizov vzorec pre $n\geq 2$ má iný tvar ako pre $n=1$);
$\boxed{327.}$ $\boldsymbol{1.}$ $f(x)=\frac{2}{3}.\frac{1}{1-2x}+\frac{1}{3}.\frac{1}{1+x}$, $f^{(n)}(0)=\frac{n!}{3}(2^{n+1}+(-1)^{n})$;   
$\boldsymbol{2.}$ $f^{(n)}(x)=2^{n}x^{2}e^{2x}+2^{n}nxe^{2x}+n(n-1)2^{n-2}e^{2x}$ (použili sme Leibnizov vzorec pre $n\geq 2$, samostatne vypočítali $\ y'$ a presvedčili sa, že sa dá vyjadriť tým istým vzťahom), $f^{(n)}(0)=n(n-1)2^{n-2}$; 
$\boldsymbol{3.}$ $f^{(n)}(x)=\frac {(2n-x)}{2^{n}(1-x)^{n+1/2}}\prod_{k=1}^{n-1} (2k-1)$ pre $n\geq 2$, $ f'(x)=\frac{2-x}{2(1-x)^{3/2}}$; $f^{(n)}(0)=\frac{n}{2^{n-1}}\prod_{k=1}^{n-1} (2k-1)$ pre $n\geq 2$, $f^{(1)}(0)=1$;
$\boldsymbol{4.}$ $f(x)=\frac{1}{2}(\frac{1}{1-x)}-\frac{1}{1+x})$, $f^{(n)}(0)=\frac{1}{2}n!(1-(-1)^{n})$;$\boldsymbol{5.}$ platí $(1-x^{2}).f''(x)-x.f'(x)=0$, odtiaľ $f^{(n)}(0)=(n-2)^{2}(0)$ (odvodí sa Leibnizovým vzorcom za predpokladu $n\geq 4$, ale zhodou okolností platí tento rekurentný vzťah aj pre $n=3$; vzťah pre  $n=3$ dostaneme, ak prvú uvedenú identitu raz zderivujeme);  $f'(0)=1$, $f''(0)=0$, $f^{(2k+2)}(0)=0$, $f^{(2k+1)}(0)=1^{2}.3^{2}.....(2k-1)^{2}$  ($k=0,1,...$);
$\boldsymbol{7.}$ $f(x)=g(x).g(x)$, kde $g(x)=\frac{1}{2}(\frac{1}{1-x}+\frac{1}{1+x})$; $g^{(k)}(0)=0$    pre $k$ nepárne, $g^{(k)}(0)=k!$ pre $k$ párne; na súčin  $g.g$  použijeme Leibnizov vzorec;  $f^{(n)}(0)=0$   pre $n$ nepárne,  $f^{(2n)}(0)=\sum_{k=0}^m(2 k)!$;
$\boxed{329.}$  nepriamo, ak nie je prostá, tak vyhovuje predpokladom Rolleho vety na niektorom intervale;
$\boxed{330.}$ $n+1$ nulových bodov vytvorí $n$ intervalov, na každom z nich sú splnené predpoklady Rolleho vety; tak dostaneme $n$ nulových bodov funkcie $f'$, tie vytvoria $n-1$ intervalov, naktorých $f'$ vyhovuje predpokladom Rolleho vety atď.;
$\boxed{331.}$ Ak $x_{1}<x_{2}<...<x_{k}$ sú korene polynómu $P_{n}$, tak podľa návodu k pr. 330 existuje $k-1$ koreňov polynómu $P'_{n}$, tak, že $x_{1}<c_{1}<x_{2}<c_{2}...<c_{k-1}<x_{k}$. Ďalej: ak $x_{i}$ je m-násobný koreň $P_{n}$ ($m\geq 2$), tak $x_{i}$ je $m+1$-násobný koreň $P'_{n}$. Súčet násobností takto získaných koreňov je $n-1$ (podľa základnej vety algebry nemôže byť väčší než $n-1$; ak násobnosť koreňa $c_{i}$ ($i=1,...,k-1$) zdola odhadneme číslom $1$, je číslo $n-1$ aj dolným odhadom súčtu všetkých násobností.
$\boxed{332.}$ z návodu k pr.331 vyplýva: ak súčet násobností reálnych koreňov polynómu $P_{n}$ je  $k$ tak súčet násobností reálnych koreňov $P'_{n}$ je aspoň $k-1$ ($n\geq 2, k\geq 2 $); derivácia polynómu $x^{3}-3x^{2}+6x-1$ nemá reálne korene;
$\boxed{333.}$ treba dokázať, že existujú body $c,d \in \mathbb{R}$, $a<c<d<b$, v ktorých $f(c)=f(d)$; na intervale $\langle c,d\rangle$  potom $f$ vyhovuje predpokladom  Rolleho vety; 
$\boxed{334.}$ $f$ musí byť spojitá na $(a,b)$; keby bola naviac spojitá aj v bodoch $a,b$, vyhovovala by predpokladom  Rolleho vety, čo by bolo v spore z podmienkou $f'(x)\ne 0$ pre všetky  $x \in(a,b)$; zadaniu vyhovuje napr. funkcia  $f$ daná predpisom  $f(x)=x$ pre  $x \in(a,b\rangle $,  $f(x)=b$;
$\boxed{335.}$  nie je;  $y$ totiž nemá vlastnú ani nevlastnú deriváciu v bode $0$;
$\boxed{336.}$ $\boldsymbol{1.}$ zvoľme pevne $a\in I$; ak $x\in I$, tak podľa Lagrangeovej vety  $f(x)-f(a)=0.(x-a)$, tj. $f(x)=f(a)$;
$\boldsymbol{2.}$ z pr. 336.1 vyplýva, že $M$ nemôže byť otvorený interval; zadaniu vyhovuje napr. $M=\mathbb{R}-\lbrace 0\rbrace$, $f(x)=\sgn x/M$;
$\boxed{338.}$ postup je rovnaký ako v pr. 337, v pr. 338.3 treba robiť úvahy zvlášť pre $(-\infty, -1)$ a zvlášť pre $(1,\infty)$ (množina $(-\infty, -1)\cup (1,\infty)$ nie je interval);
$\boxed{339.}$ uvedieme dva návody: 1. zvoľme pevne $a \in I $; ak $x \in I $, tak podľa Lagrangeovej vety  $f(x)-f(a)=k.(x-a)$; 2. funkcia $g(x)=kx+b $ má požadovanú vlastnosť ($g'(x) * k$ na $I$), ak ju má aj funkcia $f$, je $(g-f)'=0$ na $I$, podľa pr. 336 potom $g-f$ $* const$ na $I$;
$\boxed{340.}$ sporom; ak $f'(x)=\sgn x$, tak podľa pr. 339 $f(x)=\vert x\vert +c$, ale  $\vert x\vert +c$ nemá deriváciu v bode $0$;
$\boxed{341.}$ $\boldsymbol{2.}$ treba využiť, že z nerovnosti $0<y<c<x$ vyplýva pre $p>1$ nerovnosť $y^{p-1}<c^{p-1}<x^{p-1}$;
$\boldsymbol{4.}$  $\ln \frac{a}{b}= \ln a-\ln b$; ak $0<b<c<a$, tak $\frac{1}{a}<\frac{1}{c}<\frac{1}{b}$;
$\boxed{342.}$ $\boldsymbol{1.}$ $f$ je na $(a,b)$ spojitá (má totiž v každom bode deriváciu), na každom intervale $\langle x,y \rangle \subset (a,b)$ vyhovuje predpokladom Lagrangeovej vety; ak $\vert f'(x)\vert \leq K$ pre všetky $x \in (a,b)$, tak $\vert f(x)-f(y)\vert \leq K \vert x-y\vert\quad (x,y \in (a,b)$), odtiaľ vyplýva naše tvrdenie na základe definície rovnomernej spojitosti;
$\boxed{343.}$ nepriamo;  ak $\vert f'(x)\vert \leq K$   $x \in (a,b)$, tak použitím Lagrangeovej vety $\vert f(x)-f(A)\vert \leq K \vert x-A\vert \leq K(b-a)$  ($A \in (a,b)$ je pevne zvolené), teda $ \vert f(x) \vert \leq \vert f(A) \vert + \vert f(x)-f(A)  \vert \leq  \vert f(A) \vert+K(b-a)$, tj. $f$ je uhraničená na $(a,b)$;
$\boxed{344.}$ zvoľme $a \in I$ pevné; pre $a \in I, \, x \ne a$ označme $z(x)$ to číslo $z$ ležiace medzi $x,a $ pre ktoré $F(x) - F(a) = f(z)(x-a)$ (ak je takých čísel viacej, vyberieme jedno z nich); dokážte implikáciu: ak $x\rightarrow a$, tak $z(x) \rightarrow a$; potom $F'(a)=\lim_{x \to a}\frac{F(x) - F(a)}{x-a}=\lim_{x \to a}f(z(x))=\lim_{z \to a}f(z)=f(a)$ ($f$ je na $I$ spojitá);
$\boxed{345.}$ nie; stačí uvážiť $f(x)=x^{3}$, interval $\langle -1,1 \rangle$, $c=0$;
$\boxed{346.}$ číslo $c_{1}$, pre ktoré $f(b)-f(a)=f'(c_{1})(b-a)$, nemusí byť totožné s číslom $c_{2}$, pre ktoré $g(b)-g(a)=g'(c_{2})(b-a)$; dokumentujeme to na príklade $f(x)=x^{3}$, $g(x)=x^{2}$, $a=0$, $b=1$; potom $f(1)-f(0)=f'(\frac{1}{\sqrt{3}})(1-0)$,  $g(1)-g(0)=g'(\frac{1}{2})(1-0)$, ale $\frac{f(1)-f(0)}{g(1)-g(0)}=\frac{f'(\frac{2}{3}}{g'(\frac{2}{3}})$;
$\boxed{347.}$ stačí použiť Cauchyho vetu pre funkcie $f(x)$ a $\frac{1}{x}$ na intervale $\langle 1,2 \rangle$ ( a samozrejme predtým preveriť splnenie predpokladov (i), (ii));
$\boxed{348.}$ ak rozšírime zlomok na ľavej strane rovnosti výrazom $\frac{1}{a.b}$, vidíme, že treba použiť Cauchyho vetu pre funkcie $\frac{f(x)}{x}$  a  $\frac{1}{x}$ na $\langle a,b \rangle$;
$\boxed{349.}$ $\boldsymbol{1.}$ klesajúca na $(-\infty,-1 \rangle$ a na $\langle 1,\infty) $,rastúca na $\langle -1,1\rangle $;
$\boldsymbol{2.}$ klesajúca na $(-\infty,-1 \rangle$ a na $\langle 1,\infty) $,rastúca na $\langle -1,1\rangle $;
$\boldsymbol{3.}$  na každom z intervalov $((2k-1)\pi,(2k+1)\pi)$ je $y'$ kladná, teda  $y$ rastie na každom z intervalov $\langle (2k-1)\pi,(2k+1)\pi\rangle, \, k\in \mathbb{Z}$, odtiaľ vyplýva, že $y$ rastie na $\mathbb{R}= \bigcup_{k\in \mathbb{Z}}\langle (2k-1)\pi,(2k+1)\pi\rangle $;
$\boldsymbol{4.}$ rastie na každom z intervalov $\langle 2k\pi, \frac {\pi}{3}+2k\pi \rangle$, $\langle \frac{2\pi}{3}+2k\pi,(2k+1)\pi \rangle$, $\langle \frac{7\pi}{6}+2k\pi,\frac{11\pi}{6}+2k\pi \rangle$, klesá na každom z intervalov $\langle \frac{\pi}{3}+2k\pi,\frac{2\pi}{3}+2k\pi \rangle$, $\langle (2k+1)\pi,\frac{7\pi}{6}+2k\pi \rangle$, $\langle \frac{11\pi}{6}+2k\pi,(2k+2)\pi \rangle$, $k \in \mathbb{Z}$;
$\boldsymbol{5.}$ rastie na každom z intervalov $\langle \frac{1}{2k+1}, \frac{1}{2k}\rangle$ ($k=\pm 1,\pm 2,\pm 3,...)$ a na $\langle 1,\infty$, klesá na $\langle \frac{1}{2k+2}, \frac{1}{2k+1}\rangle$ ($k=\pm 1,\pm 2,\pm 3,...)$ a na $(-\infty , -1\rangle$;
$\boldsymbol{6.}$ klesá na $(-\infty , 0\rangle$ a na $\langle 2\log _{2}e,\infty)$, rastie na $\langle 0, 2\log _{2}e\rangle$ ($\log _{2}e=\frac{1}{\ln 2} $);
$\boldsymbol{7.}$ klesá na $(-\infty , -1\rangle$ a na $(0,1\rangle$, rastie na $\langle -1, 0)$ a na $\langle 1,\infty)$;
$\boldsymbol{8.}$ rastie na $\langle e^{-7\pi/12+2k\pi},e^{13\pi/12+2k\pi}\rangle$,   klesá na $\langle e^{13\pi/12+2k\pi},e^{17\pi/12+2k\pi}\rangle$, $k\in \mathbb{Z}$ (všimnite si, že číslo $0$ neleží v žiadnom z týchto intervalov);
$\boxed{350.}$ pre dané $a\in \mathbb{R}$ je funkcia $F(x):= \frac{f(x)-f(a)}{x-a}$ kladná na $\mathbb{R}- \lbrace a \rbrace$, teda $\lim_{x \to a}F(x)\geq 0$;
$\boxed{351.}$ $a$ stačí zvoliť tak, aby všetky korene polynómu $P'_{n}$ ležali v $\langle -a,a \rangle$, potom $P'_{n}$ nemení znamienko na $(-\infty, -a)$ a na $(a, \infty)$;
$\boxed{352.}$ $\boldsymbol{1.}$ označme $f(x)=e^{x}$, $g(x)=1+x$; potom $f(0)=g(0)$, $f'(x)>g'(x)$ pre všetky $x>0$, $f'(x)<g'(x)$ pte všetky $x<0$; odtiaľ už vyplýva na základe vety 12 uvedená nerovnosť;
$\boldsymbol{2.}$ nerovnosť $\sin x< x$ iste platí pre $x>1$, petože vtedy $\sin x\leq 1<x$; ak $f(x):=x$, tak $f(0)=g(0)$, $f'(x)>g'(x)$ pre všetky $x \in (0,1 \rangle$, teda $\sin x<x$ platí aj pre $x \in (0,1 \rangle$ (nerovnosť $f'(x)>g'(x)$ neplatí pre všetky $x>0$ - pre  $x>0$ platí len nerovnosť $f'(x)\geq g'(x)$ - preto vetu 12 nemôžeme použiť na dôkaz nerovnosti $\sin x<x$ na celom intervale $(0, \infty)$); ak $f(x):= \sin x$, $g(x):=x-x^{3}/6 $,tak $f(0)=g(0)$, $f'(0)=g'(0)$ a  pre všetky $x>0 $ platí  $f''(x)>g''(x)$  (to vyplýva z predtým dokázanej nerovnosti $\sin x<x$ pre  $x>0 $);
$\boldsymbol{4.}$ ak $f(x):=\tan x$, $g(x):=x+\frac{x^{3}}{3}$, tak $f(0)=g(0)$, $f'(0)=g'(0)$, $f''(x)=2 \cos ^{-3}x.\sin x$, $g''(x)=2x$; pre  $x \in (0,\frac{\pi}{2})$ je $0<\cos x<1$, teda $\cos^{-2}x>1 $, súčasne  $\tan x>x $ (pozri pr. 352.3), teda $\tan x. \cos^{-2}x=\cos^{-3}x.\sin x>x$; odtiaľ vyplýva $f''(x)>g''(x)$ pre  $x \in (0,\frac{\pi}{2})$;
$\boldsymbol{7.}$ ak  $f(x):=1+2\ln x$, $g(x):=x^{2}$, tak $f(1)=g(1)$, $f'(x)<g'(x)$ pre všetky $x>1$, $f'(x)>g'(x)$ pre všetky $x\in(0,1)$;
$\boxed{353.}$ ak do uvedenej nerovnosti dosadíme $x=\frac{1}{n-1},...,x=\frac{1}{2n}$ a získané nerovnosti sčítame, dostaneme odhad $\ln (\frac{2n+1}{n})<\frac{1}{n}+...+\frac{1}{2n}<\ln (\frac{2n}{n-1}) $, $n>1$;
$\boxed{354.}$ nech $M=\lbrace a_{1},...,a_{m} \rbrace$, $a_{1}<a_{2}<...<a_{m}$; potom $f$ rastie na každom z intervalov  $(a,a_{1} \rangle$, $\langle a_{1},a_{2} \rangle$,..., $\langle a_{m},b)$; odtiaľ vyplýva, že $f$ rastie na $(a,b)$;
$\boxed{355.}$ funkcia $f-g$ je neklesajúca na $\langle a, \infty)$ a platí $f(a)-g(a)>0$; odtiaľ vyplýva, že $f(x)-g(x)\geq f(a)-g(a)>0$ pre všetky $x\geq a$;
$\boxed{356.}$ nie (príslušný protiklad už musíte nájsť sami);
$\boxed{357.}$ $\boldsymbol{1.}$ rýdzo konvexná na $(-\infty ,1\rangle$, rýdzo konkávna na $\langle 1, \infty)$;
$\boldsymbol{2.}$ rýdzo konvexná na $\langle 0, \infty)$, rýdzo konkávna na $(-\infty ,0\rangle$;
$\boldsymbol{3.}$ rýdzo konkávna na každom z intervalov $\langle 2k\pi, (2k+1)\pi \rangle$, rýdzo konvexná na  každom z intervalov $\langle (2k-1)\pi, 2k\pi \rangle$, $k\in \mathbb{Z}$;
$\boldsymbol{4.}$ rýdzo konvexná na $\langle -1,1 \rangle$, rýdzo konkávna na $(-\infty ,-1\rangle$ a na $\langle  1,\infty )$; 
$\boldsymbol{5.}$ rýdzo konvexná na $\langle e^{\pi/4-(2k+1)\pi},e^{\pi/4-2k\pi} \rangle$, rýdzo konkávna na $\langle e^{\pi/4-2k\pi},e^{\pi/4-(2k-1)\pi} \rangle$, $k\in \mathbb{Z}$;
$\boldsymbol{5.}$ rýdzo konkávna na $(-\infty ,0\rangle$ a na $\langle 0, \infty)$, ale nie na ich zjednotení;
$\boxed{358.}$ $\boldsymbol{1.}$ $p,q$ nepárne, $p>q>0$ (funkcia $x^{p/q}$ je pre $x<0$ definovaná len vtedy, keď $p$ je párne alebo $p$ aj $q$ sú nepárne; derivácia v bode $0$ je vlastná len pre $p\geq q$);
$\boldsymbol{2.}$ $9b^{2}-24ac>0$;
$\boldsymbol{3.}$ $\vert a \vert \leq 2$ (pre $\vert a \vert < 2$ je  $y''>0$ na $\mathbb{R}$, teda $y$ je rýdzo konvexná na $\mathbb{R}$; pre  $\vert a \vert =2$ má $y''$ nulovú hodnotu len v jednom bode, inde $y''>0$; dá sa dokázať, že aj v tomto prípade je  $y$ rýdzo konvexná (pozri pr. 462);
$\boxed{360.}$ $\boldsymbol{3.}$ stačí obidve strany vydeliť číslom $2$ a využiť konvexnosť funkcie $x \ln x$;
$\boldsymbol{4.}$ $y=\frac{2}{\pi}x$ je rovnica spojnice bodov $(0,0)$, $(\frac{\pi}{2},1)$ ležiacich na grafe funkcie $\sin $;
$\boldsymbol{6.}$ $y=x-1$ je rovnica spojnice bodov $(1,0)$, $(2,1)$ ležiacich na grafe konkávnej funkcie $y=\log _{2}x $;
$\boxed{361.}$ sporom; $f$ by musela byť rýdzo konkávna, a teda aj nekonštantná na $\mathbb{R}$; ak $a < b$, $f(a)\ne  f(b)$, tak sa spojnica bodov $(a,f(a))$, $(b,f(b))$ pretína s osou $Ox$, pritom na $\mathbb{R}-\langle a,b \rangle$ leží graf funcie $f$ pod touto spojnicou, to vedie k sporu s kladnosťou funkcie $f$ (každý z uvedených krokov treba samozrejme podrobne zdôvodniť);
$\boxed{362.}$ $\boldsymbol{1.}$ pretože z definície konvexnosti funkcie $f$  na $I$ vyplýva $\forall x,y,z\in I, \quad x<z<y: \, f(z) <f(x)+(f(y)-f(x)).\frac{z-x}{y-x}$ (v definícii stačí položiť  $z=px+qy$ a odtiaľ vyjadriť $q=\frac{z-x}{y-x}$, môžeme dokázať, že funkcia $F(x):=\frac{f(x)-f(a)}{x-a}$ je rastúca na $I\cap (a,+\infty)$ a na $I\cap (-\infty,a)$; pritom $f'(a)=\lim_{x \to a}F(x)$; odtiaľ vyplýva $\forall x\in I, \, x>a: \quad \frac{f(x)-f(a)}{x-a}>f'(a)$, $\forall x\in I, \, x<a: \quad \frac{f(x)-f(a)}{x-a}<f'(a)$; z toho po úprave  $\forall x\in I, \, x \ne a: \quad f(x)>f(a)+f'(a)(x-a)$ (úvahy pre krajné a pre vnútorné body intervalu $I$ treba robiť každú samostatne);
$\boxed{363.}$ $f'(0)=0$ (vypočíta sa priamo z definície); pre $x>0$ je  $f(x)>0$, pre  $x<0$  je $f(x)<0$; $f''(x)=12x+(6x-\frac{1}{x})\cos\frac{1}{x}+4\sin\frac{1}{x}$ je spojitá na $\mathbb{R}-\lbrace 0 \rbrace$ a  na každom z intervalov $(-\varepsilon ,0)$, $(0,\varepsilon)$, $\varepsilon>0$, nadobúda kladné aj záporné hodnoty (vyplýva to z faktu, že pre $x=\frac{1}{2k\pi}$ je $f''(x)=18x-\frac{1}{x}$, pre  $x=\frac{1}{(2k+1)\pi}$ je $f''(x)=6x+\frac{1}{x}$ a obidve tieto funkcie majú v bode $0$ jednostranné limity $+\infty$ a $-\infty$); preto každý z intervalov $(-\varepsilon ,0)$, $(0,\varepsilon)$, $\varepsilon>0$ obsahuje intervaly, na ktorých je $f$ rýdzo konvexná, aj intervaly na ktorých je  $f$ rýdzo konkávna;
$\boxed{364.}$ $\boldsymbol{1.}$ lok. minimum: $f(0)=0$, lok. maximum: $f(1)=f(-1)=1$;
$\boldsymbol{2.}$ lok. minimum: $f(-2)=-9$, $f(3)=-\frac{161}{4}$; lok. maximum: $f(0)=7$;
$\boldsymbol{3.}$ lok. minimum: $f(k\pi-\frac{\pi}{4})=-\frac{e^{(4k-1)\pi/4}}{\sqrt{2}}$, $k$ párne;  lok. maximum: $f(k\pi-\frac{\pi}{4})=\frac{e^{(4k-1)\pi/4}}{\sqrt{2}}$, $k$ nepárne $(\cos x+\sin x=\sqrt{2}\sin (x+\frac{\pi}{4}))$; 
$\boxed{365.}$ $\boldsymbol{1.}$ lok. minimum: $f(1)=f(3)=3$; lok. maximum: $f(2)=4$;
$\boldsymbol{2.}$ lok. minimum: $f(1)=0$, $f(-\frac{5+\sqrt{13}}{6})$; lok. maximum: $f(-\frac{\sqrt{13}-5}{6})$;
$\boldsymbol{3.}$ lok. minimum: $f(\frac{7}{5})=-\frac{1}{24}$; 
$\boldsymbol{4.}$ lok. minimum: $f(1)=0$; lok. maximum: $f(e^{2})=\frac{4}{e^{2}}$;
$\boldsymbol{5.}$ lok. minimum: $f(-1)=0$; lok. maximum: $f(9)=\frac{10^{10}}{9}$;
$\boxed{366.}$ $\boldsymbol{1.}$ nemá;
$\boldsymbol{2.}$ má v bode 0 lok. maximum $f(\frac{3}{2})=\frac{1}{4}$ (ak $f$ je spojitá, tak $\vert f \vert$ nadobúda lokálne extrémy v bodoch lokálnych extrémov funkcie $f$ a v nulových bodoch funkcie $f$);
$\boldsymbol{2.}$ lok. maximum: $f(-\frac{1}{\sqrt[4]{27}})=f(\frac{1}{\sqrt[4]{27}})=\frac{2}{3\sqrt{3}}$; lok. minimum: $f(0)=0$ (posledné tvrdenie nevyplýva z viet 16 a 17 - $f'(0)$ totiž neexistuje - a treba ho preto dokázať samostatne);
$\boldsymbol{3.}$ lok. maximum: $f(3)=3$ (bod 3 nie je vnútorný bod $D(f)$ uvedené tvrdenie teda nevyplýva z viet 15 - 17 a treba ho dokázať); 
$\boldsymbol{4.}$ lok. minimum: $f(\frac{3}{4})=-\frac{3}{4\sqrt[3]{4}}$ (pretože neexistuje vlastná $f'(1)$, preskúmame bod 1 zvlášť: buď priamo ukážeme, že $f$ nemá v bode $1$ lokálvy extrém, alebo dokážeme, že $f'(1)$ je nevlastná a použijeme vetu 15);
$\boxed{368.}$ $\boldsymbol{1.}$ $M=11$, $m=2$;
$\boldsymbol{2.}$ $M=100,01$; $m=2$;
$\boldsymbol{3.}$ $M=8$; $m$ neexistuje ($f$ je rastúca na $\langle 0,4 \rangle$; zrejme $0=\inf _{x\in (0,4\rangle}f(x)$);
$\boldsymbol{4.}$  $M=1$ ($f$ rastie na  $\langle 0,\frac{\pi}{4} \rangle$, klesá na $\langle \frac{\pi}{4} ,frac{\pi}{2})$), $m$ neexistuje ($\lim_{x \to \pi/2-}f(x)=-\infty$);
$\boldsymbol{5.}$ $M$ neexistuje,  $m=4$;
$\boldsymbol{6.}$ $M=\frac{\pi}{4}$; $m=0$;
$\boxed{369.}$ $\boldsymbol{3.}$ vyšetrite monotónnosť funkcie $\frac{x^{2}+1}{x^{2}+x+1}$ a nájdite jej limity v bodoch $+\infty $, $-\infty $; na základe toho si načrtnite jej graf; obrázok je návodom k vlastnému dôkazu;
$\boxed{370.}$ $\boldsymbol{1.}$ dokážte, že $P'$ musí pri prechode cez tento bod zmeniť znamienko; využite, že $P'$ ako polynóm nepárneho stupňa má v bodoch $+\infty$, $-\infty$ nevlastné limity opačných znamienok;
$\boxed{371.}$ nemôže; nepriamo: nech $a<b$ sú také stacionárne body; funkcia $f/\langle a,b \rangle$ nadobúda v niektorom bode $c$ kompaktu $\langle a,b \rangle$ svoje minimum; treba dokázať $c\in (a,b)$ a potom využiť vetu 15;
$\boxed{372.}$ $f'$ je spojitá na $\mathbb{R}-\lbrace 0 \rbrace$ a v každom ľavom, resp. pravom okolí bodu $O$ nadobúda kladné aj záporné hodnoty, ďalej pozri myšlienku návodu k pr. 363;
$\boxed{373.}$ $\boldsymbol{1.}$ výška  $v=\frac{2R}{\sqrt{3}}$, polomer podstavy $r=\frac{\sqrt{2} R}{\sqrt{3}}$ (existuje medzi valcami vpísanými do tejto gule valec s najmenším objemom?);
$\boldsymbol{2.}$  $v=\frac{2R\sqrt{5-\sqrt{5}}}{\sqrt{10}}$,  $r=\frac{R\sqrt{5+\sqrt{5}}}{\sqrt{10}}$;
$\boxed{374.}$ bod $(\frac{-1+\sqrt{33}}{8},\frac{1+3\sqrt{33}}{8})$;
$\boxed{375.}$ ak vzdialenosť bodov $(a,b)$ a  $(c,f(c))$ je maximálna (minimálna), tak  $f'(c)=\frac{c-a}{f(c)-b}$, to vyplýva z vety 15; pri dôkaze faktu, že dotyčnica v bode $(c,f(c))$	je kolmá na spojnicu bodov $(a,b)$ a $(c,f(c))$,  treba rozlíšíť prípady $a=c$,	$a\ne c$;
$\boxed{376.}$ vo vydialenosti $\sqrt{ab}$ ($cm$) ($\varphi =\arctan \frac{b}{x-\arctan \frac{a}{x}}$, $b>a$);
$\boxed{377.}$ platí $f(x)=f(x)-f(a)$, $g(x)=g(x)-g(a)$; potom zlomok rozšíriť výrazom $\frac{1}{x-a}$; pozor, vetu 18 nemožno použiť (nie je zaručené splnenie predpokladu 1);
$\boxed{378.}$ $\boldsymbol{2.}$ $2$;
$\boldsymbol{3.}$ $\frac{1}{3}$;
$\boldsymbol{4.}$ $0$; (l'Hospitalovo pravidlo použiť n-krát za sebou);
$\boldsymbol{5.}$ $0$ (najprv subst. $t=\frac{1}{x^{2}}$);
$\boldsymbol{6.}$ $-\frac{1}{2}$ (najprv zlomok rozšíriť členom x a využiť, že $\lim_{x \to 0}\frac{x}{\tan x}=1$; použitie l'Hospitalovho pravidla sa tak zjednoduší);
$\boldsymbol{7.}$ $4$ (napísať v tvare súčtu dvoch limít; využiť, že $\lim_{x \to 0}\frac{x^{3}\cos \pi/x}{x^{2}}=0$, l'Hospitalovo pravidlo použiť len na výpočet druhej limity);
$\boxed{379.}$ $\boldsymbol{1.}$ $0$;
$\boldsymbol{2.}$ $0$;
$\boldsymbol{3.}$ $2$;
$\boldsymbol{4.}$ $0$;
$\boxed{380.}$ $\boldsymbol{1.}$ $1$;
$\boldsymbol{2.}$ $e$;
$\boldsymbol{3.}$ $e^{1/6}$;
$\boldsymbol{4.}$ $e^{-1/2}$;
$\boldsymbol{5.}$ $1$ (po prvom použití l'Hospitalovo pravidla dostaneme výraz $\frac{x}{\ln \frac{1}{x}}$; ten - hoci v bode $0$ nie je typu $\frac{\infty}{\infty}$ ani  $\frac{0}{0}$ - vyhovuje všetkým predpokladom vety 18 (zvlášť si všimnite predpoklad 3), možno teda ešte raz použiť l'Hospitalovo pravidlo; $\lim_{x \to 0}(x/\ln (1/x))$ možno samozrejme nájsť aj bez toho, stačí použiť vetu 10 z kapitoly 2);
$\boxed{381.}$ $\boldsymbol{1.}$ $\frac{1}{2}$; (po úprave na spoločného menovateľa zlomok rozšírte členom $x$ a využite, že  $\lim_{x \to 0}\frac{e^{x}-1}{x}=1$);
$\boldsymbol{2.}$ $\frac{2}{3}$; (po úprave na spoločného menovateľa  využite, že  $\lim_{x \to 0}\frac{x^{2}}{\sin ^{2}x}=1$);
$\boldsymbol{3.}$ $\frac{1}{2}$ (najprv subst. $\frac{1}{x}=1$);
$\boxed{382.}$ $\boldsymbol{1,2.}$ nie, nie je splnená podmienka 4 z vety 18;
$\boldsymbol{3.}$ nie, nie je splnená podmienka 3 z vety 18 (výpočet jednotlivých limít: 1. zlomok rozšíriť členom $x$ a použiť pr. 131.1; 2. využiť, že $\lim_{x \to 0}x\sin \frac{1}{x}=0$; 3. dosadiť);
$\boxed{383.}$ správny je druhý postup; derivácia menovateľa je $\cos x.e^{\sin x}.(x+2\cos x+\cos x.\sin x$), tá má nulové hodnoty v každom okolí bodu  $+\infty$, nie je teda splnená podmienka 2 z vety 18, preto sme neboli oprávnení používať  l'Hospitalovo pravidlo;
$\boxed{384.}$ $\boldsymbol{1.}$ na výpočet $\lim_{x \to a}\frac{f(x)-f(a)}{x-a}$ použite l'Hospitalovo pravidlo;
$\boldsymbol{2a/}$  $0=\lim_{x \to 0}f'(x)=f'(0)$;
$\boldsymbol{2b/}$  $f'(1)=+\infty$, $f'(-1)=+\infty$ (použili sme unalógiu tvrdenia z pr. 384.1 pre jednostranné derivácie);
$\boldsymbol{2c/}$  $f'(0)=\frac{1}{\sqrt{2}}$ (nezabudnite najprv preveriť spojitosť funkcie $f$ v bode 0);
$\boxed{385.}$ $\boldsymbol{1.}$ $\frac{1}{\sqrt{2}}+\frac{1}{\sqrt{2}}(x+\frac{1}{2})-\frac{1}{2:\sqrt{2}}(x+\frac{1}{2})^{2}+\frac{1}{2:\sqrt{2}}(x+\frac{1}{2})^{3}-\frac{15}{4:\sqrt{2}}(x+\frac{1}{2})^{4}$;
$\boldsymbol{2.}$ $x+\frac{x^{3}}{3}$;
$\boldsymbol{3.}$ $\frac{1}{\sqrt{2}}+\sqrt{2}x-\sqrt{2}x^{2}-\frac{4\sqrt{2}}{3!}x^{3}+\frac{8\sqrt{2}}{4!}x^{4}+\frac{16\sqrt{2}}{5!}x^{5}$;
$\boldsymbol{4.}$ $\sum_{k=0}^N (-2)^{k}(x+1)^{k}$ ($f^{(k)}(-1)=(-1)^{k}k!2^{k}$;
$\boldsymbol{5.}$ Taylorovým polynómom stupňa i je funkcia $T(x)=\frac{1}{2}$; Taylorove polynómy stupňov $2m$ a $2m+1$ ($m \in \mathbb{R}$) sa zhodujú a majú tvar $\frac{1}{2}+\sum_{k=0}^N \frac{(-1)^{k+1}4^{2k-1}}{(2k)!}(x-\frac{\pi}{4})^{2k}$ (predpis funkcie $f$ možno upraviť pomocou vzorcov $\sin ^{2}\alpha =\frac{1-\cos 2\alpha}{2}$, $\cos ^{2} \alpha =\frac{1+\cos 2\alpha}{2}$);
$\boxed{386.}$ platí (*) $A_{0}+...+A_{n}(x-a)^{n}+o((x-a)^{n})=f(x)=f(a)+...+\frac{f^{(n)}}{n!}(x-a)^{n}+o((x-a)^{n})$ (druhá rovnosť vyplýva z vety 19, odtiaľ vyplýva rovnosť limity pravej a ľavej strany v bode $a$, preto $A_{0}=f(a)$, ak po odstránení $A_{0}$, resp. $f(a)$ z ľavej resp. pravej strany vydelíme obidve strany členom $(x-a)$ (a využijeme, že $\frac{o((x-a)^{n})}{x-a}=o((x-a)^{n-1})$), dostaneme $A_{1}+A_{2}(x-a)+...+A_{n}(x-a)^{n-1}+o((x-a)^{n-1})=f'(a)+...+\frac{f^{(n)}}{n!}(x-a)^{n-1}+o((x-a)^{n-1})$, odtiaľ opäť vyplýva rovnosť limít v bode a atď;
$\boxed{387.}$ $\boldsymbol{1.}$  $e^{\frac{x}{7}}=\sum_{k=0}^n \frac{x^{k}}{7^{k}.k!}+o(x^{n})$;
$\boldsymbol{3.}$ Maclavrinove polynómy stupňa 1, 2, ..., 5 sa zhodujú a majú tvar $T(x)=1$; Maclavrinove polynómy stupňa $n=6m, 6m+1,...,6m+5$ $(m \in \mathbb{N}$ sa zhodujú a majú tvar $T(x)=1+ \sum_{k=1}^m (-1)^{k}\frac{x^{6k}}{(2k!)}$, preto možno písať $\cos x^{3}=1+\sum_{k=1}^m (-1)^{k}.\frac{x^{6k}}{(2k!)}+o(x^{6m+5})$;
$\boldsymbol{5.}$ $\frac{1}{2+3x^{2}}=\frac{1}{2}. \frac{1}{1-(-\frac{3}{2}x^{2})}=\frac{1}{2}+\sum_{k=1}^m (-\frac{3}{2})^{k}x^{2k}+o(x^{2m+1})$,  Maclavrinov polynóm stupňa 1 má tvar $T(x)=\frac{1}{2}$ (využili sme vzorec V;  Maclavrinove polynómy stupňa $n=2m$ a $n=2m+1$ sa zhodujú);
$\boldsymbol{5.}$ $x^{3}\sin 3x= \sum_{k=1}^m (-1)^{k+1}\frac{3^{2k-1}x^{2k+2}}{(2k-1)!}+e(x^{2m+3})$ (teda Maclavrinov polynóm stupňa 1, 2, 3 majú rovnaký  tvar $T(x)=0$,  Maclavrinove polynómy stupňov  $n=2m+2, \, n=2m+3\quad (m\in \mathbb{N})$ sa zhodujú);
$\boxed{388.}$ $\boldsymbol{2.}$ $1+x-x^{2}+\frac{3}{2}x^{3} $ (najprv sme použili vzorec VI pre $\alpha=\frac{1}{3} $);
$\boldsymbol{3.}$ $-\frac{x^{3}}{8}$ (obidve strany rovnosti $\ln (1-\frac{x}{2})=-\frac{x}{2}+o(x)$ sme umocnili na tretiu);
$\boldsymbol{4.}$	$x+\frac{x^{2}}{2} -\frac{2x^{3}}{3}+\frac{x^{4}}{4} $ (položili sme  $ z=x^{2}+x $  a najprv použili vzorec IV; ak  $R(z)=o(z^{4})$, tak  $R(x^{2}+x)= o(z^{4}) $;
$\boldsymbol{5.}$ $\cos x-\ln (1+x)=(1-\frac{x^{2}}{2}+\frac{x^{4}}{24}+o(x^{4})). (x-\frac{x^{2}}{2}+\frac{x^{3}}{3}-\frac{x^{4}}{4}+\frac{x^{5}}{5}+ o(x^{5}))=x-\frac{x^{2}}{2}-\frac{x^{3}}{6}+\frac{3x^{5}}{40}+ o(x^{5}))$;
$\boxed{389.}$ $\boldsymbol{1.}$  $\sum_{k=0}^m\frac{e^{-2}}{k!}(x+1)^{2k} $ pre $n=2m$ aj pre $n=2m+1$;
$\boldsymbol{2.}$  $\sum_{k=0}^m\frac{(-1)^{k}e^{2}2^{k-2}}{k!}(k^{2}+3k+4)(x+1)^{k} $ $f(t+1)=e^{2}.(e^{-2t}-2te^{-2t}+t^{2}e^{-2t})=e^{2}[\sum_{k=0}^n\frac{(-1)^{k}2^{k}t^{k}}{k!} +o(t^{n})+\sum_{i=0}^{n-1}\frac{(-1)^{i+1}2^{i+1}t^{i+1}}{i!} +o(t^{n})+\sum_{j=0}^{n-2}\frac{(-1)^{j}2^{j}t^{j}}{j!} +o(t^{n})]$ $=e^{2}[\sum_{k=0}^n\frac{(-1)^{k}2^{k}t^{k}}{k!}+\sum_{k=1}^n\frac{(-1)^{k}2^{k}t^{k}}{(k-1)!}+\sum_{k=2}^n\frac{(-1)^{k}2^{k-2}t^{k}}{(k-2)!}+o(t^{n})]=$ (položili sme $i+1=k,j+2=k$)$=e^{2}[1-2t+\sum_{k=2}^n\frac{(-1)^{k}2^{k}t^{k}}{k!}-2t+\sum_{k=2}^n\frac{(-1)^{k}2^{k}t^{k}}{(k-1)!}+\sum_{k=2}^n\frac{(-1)^{k}2^{k}t^{k}}{(k-2)!}+o(t^{n})]$;
$\boldsymbol{3.}$   $\frac{x}{2}-\frac{x^{2}}{4}+\sum_{k=3}^n\frac{(-1)^{(k+1)}(k-1)x^{k}}{k(k-2)}$ pre  $n\geq 3$ Maclavrinov polynóm stupňa 1, resp. 2 je $\frac{x}{2}-\frac{x^{2}}{4}(\ln \sqrt{1+x}=\frac{1}{2}\ln (1+x))$;
$\boldsymbol{4.}$ $f(x)=1-x^{2}+\sum_{k=2}^m 2.(-1)^{k}x^{2k}+o(x^{2m+1})$;
$\boldsymbol{5.}$ $f(x)=-\frac{7}{6}+\sum_{k=1}^n (-\frac{5}{3})(\frac{10}{21})^{k}(x+\frac{1}{10})^{k}+o((x+\frac{1}{10})^{n})$  $(f(t-\frac{1}{10})=\frac{1}{2}-\frac{5}{3}.\frac{1}{1-\frac{10t}{21}})$;
$\boldsymbol{6.}$ $f(x)= \ln 6 - \sum_{k=1}^n \frac{1}{k}(\frac{1}{2^{k}}+\frac{1}{3^{k}})(x-1)^{k}+o(x-1)^{n})$ (pre $t<2$ platí $f(t+1)=\ln ((t-2)(t-3))=\ln (2-t)+\ln (3-t)=\ln 2 +\ln (1-\frac{t}{2}) +\ln 3+\ln (1-\frac{t}{3}))$;
$\boxed{390.}$ $\boldsymbol{1.}$ $\frac{1}{2} $  ($= \lim_{x \to 0}\frac{x^{3}/2+o(x^{3})}{x^{3}})$;
$\boldsymbol{3.}$ $\frac{1}{3} $;
$\boldsymbol{4.}$ $10 $ ($= \lim_{x \to 0}(-\frac{x^{3}/6+o(x^{3})}{x^{2}+o(x^{2})})$);
$\boldsymbol{5.}$ $\frac{4}{3} $;
$\boldsymbol{6.}$ $\frac{1}{2} $ ($\ln (\cos x +\frac{x^{2}}{2})= \ln (1+\cos x -1 +\frac{x^{2}}{2}))$, potom použiť vzorec IV; ak $R(z)=o(z^{2})$, tak $R( \cos x -1+\frac{x^{2}}{2})= o(x^{4}))$;
$\boxed{391.}$ $\boldsymbol{1.}$ $\frac{60!}{30!} $  ($f^{(60)}(0)$ možno vyjadriť pomocou koeficientu pri $x^{60}$ v Maclavrinovom polynóme stupňa $n\geq 60$ funkcie $f$);
$\boldsymbol{2.}$ $0$  ($\frac{1}{1+x+x^{2}}=\frac{1-x}{1-x^{3}}$ pre $x \ne 1$);
$\boxed{392.}$ $\boldsymbol{1.}$ $\bigtriangleup < \frac{e}{11!}$ (použili sme Lagrangeov tvar zvyšku);
$\boldsymbol{2.}$ $\bigtriangleup < \frac{\cos 0,5}{3840}$ (polynóm vpravo je Maclavrinovým polynómom stupňa 3 aj stupňa 4 funkcie $\sin$, preto sme odhadli zvyšky v obidvoch prípadoch a z odhadov vybrali ten menší);
$\boldsymbol{3.}$ $\bigtriangleup < 0,0015$;
$\boxed{393.}$ $\boldsymbol{2.}$ podľa vety 20 $e^{x}=1+x+\frac{e^{\upsilon (x)}}{2}x^{2}$, pritom $\upsilon (x)<x$;
$\boxed{394.}$ obr. 5, funkcia je nepárna;
$\boxed{395.}$ obr. 6;
$\boxed{396.}$ obr. 7;
$\boxed{397.}$ obr. 8;
$\boxed{Obr.5.}$  $y=3x-x^{3}$
$\boxed{Obr.6.}$  $y=x^{4}/(1+x)^{3}$
$\boxed{Obr.7.}$  $y=(x+1)^{3}/(x-1)^{2}$
$\boxed{Obr.8.}$  $y=(x^{2}+x-1)/(x^{2}-2x+1)$
$$*$$
$$*$$
$$*$$
$$*$$
(Pri popise obrázkov sme použili tieto označenia: n - bod, v ktorom sa funkčná hodnota rovná 0; m - bod, v ktorom funkcia nadobúda lokálne minimum; M - bod, v ktorom funkcia nadobúda lokálne maximum; i - inflexný bod; ak je os Ox (Oy) asymptotou, je k nej pripísaná jej rovnica y=0 (x=0), ostatné asymptoty sú vyznačené čiarkovane; s výnimkou obr. 17 a 23 sa jednotka dĺžky na Ox a na Oy zhodujú a jednotka dĺžky je označená len na Oy.)
$\boxed{398.}$ obr. 9; nepárna funkcia, $y_{-}'(-1)=y_{+}'(-1)=+\infty $; pozor: na intervale $\langle -\sqrt{\frac{3}{2}},0 \rangle$ $y$ nie je konvexná, hoci je konvexná na intervale $\langle -\sqrt{\frac{3}{2}},-1 \rangle$ a na intervale $\langle -1,0 \rangle$; podobná poznámka sa vzťahuje aj na interval $\langle 0,\sqrt{\frac{3}{2}} \rangle$;
$\boxed{399.}$ obr. 10; 
$\boxed{400.}$ obr. 11;	 nepárna fukcia;
$\boxed{401.}$ obr. 12;  párna funkcia, $y_{-}'(-1)=y_{-}'(1)=-\infty $, $y_{+}'(-1)=y_{+}'(1)=+\infty $; $y$ nie je konkávna na množine $\mathbb{R}$, hoci je konkávna na intervaloch $(-\infty ,-1\rangle , \langle -1,1 \rangle $ a $\langle 1, \infty)$;
$\boxed{402.}$ obr. 13;  $y_{-}'(0)=-\infty $, $y_{+}'(0)=+\infty $; $y$ nie je konkávna na intervale $\langle -2+\sqrt{3}, +\infty)$, hoci je konkávna na intervaloch $\langle -2+\sqrt{3},0 \rangle$ a $\langle 0,+\infty)$;
$\boxed{403.}$ obr. 14; funkcia s periódou $2\pi$;
$\boxed{404.}$ obr. 15; párna funkcia s periódou $2\pi$;
$\boxed{405.}$ obr. 16; $y_{-}'(0)=-\infty $, $y_{+}'(0)=+\infty $; $y$ nie je konkávna na intervale $\langle \frac{2-\sqrt{6}}{3},\frac{2+\sqrt{6}}{3} \rangle$;
$\boxed{Obr.9.}$  $y=x \sqrt{\vert}x^{2}-1 \vert)$
$\boxed{Obr.10.}$  $y=(x-2)/\sqrt{x^{2}+1}$
$\boxed{Obr.11.}$  $y=x/\sqrt[3]{x^{2}-1}$
$$*$$
$$*$$
$$*$$
$$*$$
$\boxed{406.}$ obr. 17;
$\boxed{407.}$ obr. 18;
$\boxed{408.}$ obr. 19; nepárna funkcia,  $y_{-}'(1)=y_{+}'(-1)=1 $,  $y_{+}'(1)=y_{-}'(-1)=-1 $; pozor $-1,1$ nie sú podľa nami používanej definície inflexnými bodmi (pri úprave výrazu pre $y'$ nezabudnite, že  $\sqrt{a^{2}}=\vert a \vert $, pri výpočte $y_{+}'(1),y_{+}'(-1)$ použite obdobu tvrdenia z pr. 384.1 pre jednostranné derivácie);
$\boxed{409.}$ obr. 20; $0 \notin D(y)$, existuje ale $\lim_{x \to 0-}y(x)=0$;
$\boxed{410.}$ obr. 21; 
$\boxed{411.}$ obr. 22; 0 je bod odstrániteľnej nespojitosti,  $\lim_{x \to 0}y(x)=1$;
$\boxed{413.}$ $(f.g)'(a)=f(a).g'(a)$;
$\boxed{414.}$ tvrdenie vyplýva zo vzťahu $\frac{f^{n}(x)-f^{n}(a)}{x-a}=\frac{f(x)-f(a)}{x-a}.(f^{n-1}(x)+f^{n-2}(x).f(a)+...+f^{n-1}(a))$;
$\boxed{417.}$ $\boldsymbol{1.}$ neplatí
$$*$$
$$*$$
$$*$$
$$*$$
$\boxed{Obr.12.}$  $y=(x+1)^{2/3}+(x-1)^{2/3}$
$\boxed{Obr.13.}$  $y=\sqrt[3]{x^{2}/(x+1)}$
$\boxed{Obr.14.}$  $y=\frac{1}{2}\sin 2x +\cos x$
($x+\sin x$ je neperiodická funkcia s periodickou deriváciou);
$\boldsymbol{2.}$ platí;
$\boxed{418.}$ $(f^{2})_{\pm}'(a)=2f(a).f_{\pm}'(a)$; nutná a postačujúca podmienka je $f(a)=0 $;
$\boxed{419.}$ $\boldsymbol{1.}$  $\sum_{j=1}^m$  ($\sum_{l=1}^n$ ($\varphi _{lj}'.\prod _{k=1,k\ne l}^n \varphi _{kj}))$;
$\boldsymbol{2.}$  $\sum_{j=1}^m$  ($\sum_{l=1}^n$ ($\varphi _{lj}'.\prod _{j=1}^m \sum_{j=1}^n \varphi _{kj}))$;
$\boxed{420.}$ $\boldsymbol{1.}$ $F'(x)=3x^{2}+3x+20$;
 $\boldsymbol{2.}$ $F'(x)=6x^{2}$;
$\boxed{421.}$ $\boldsymbol{1.}$ $f(x)=\vert x-a\vert$, $g(x)=x^{2}$;
$\boldsymbol{2.}$ $f(x)=( x-a)^{2}$, $g(x)=\vert x \vert$;	
$\boldsymbol{3.}$ $f(x)=2( x-a)+\vert x-a \vert $, $g(x)=2x - \vert x \vert$ (všimnite si, že jednostranné derivácie$ g\circ f$ možno v tomto prípade vypočítať pomocou $f_{+}'(a)$ a $g_{+}'(0)$, resp. $f_{-}'(a)$ a $g_{-}'(0))$;
$\boxed{422.}$ za daných predpokladov musí platiť
$$*$$
$$*$$
$$*$$
$$*$$
$\boxed{Obr.15.}$  $y=\cos x / \cos 2x$
$\boxed{Obr.16.}$  $y=x^{2/3}e^{-x}$
$f(a)=0$, $f'(a)\ne 0$, ďalej pozri pr. 412;
$\boxed{423.}$ 1. $f(x)=ax$ pre $x\ne \frac{1}{n}$, $x\ne n$ ($n \in \mathbb{N}$) (v tomto prípade využívame možnosť širšieho chápaania pojmu derivácia pre $f'(0))$, alebo 2. $f$ je nepárna funkcia definovaná pre $x\geq 0$ nasledovne: zoraďme spočitateľnú množinu ($\mathbb{Q}\cap \langle 0, \infty)) - \lbrace x \in \mathbb{R}:x=ap^{2}+ap, \, p\in \langle 0, \infty) \cap (\mathbb{Q}- \mathbb{N}) \rbrace $ (tá obsahuje aj všetky čísla tvaru $\frac{a}{n}$, kde $n \in \mathbb{N}$) do prostej postupnosti $\lbrace b_{n} \rbrace _{n=1} ^{\infty}$, potom položíme $f(x)=ax$ pre $x \in (0,\infty) - \mathbb{Q}$, $f(x)=ax^{2}+ax$ pre $x \in \langle 0,\infty )\cap (\mathbb{Q}- \mathbb{N})$, $f(n)=b_{n}$ $ n \in \mathbb{N}$ ( v tomto prípade nevyužívame možnosť širšieho chápania pojmu derivácia ani pre $f'(0)$ ani pre ($f^{-1})'(0))$;
$$*$$
$$*$$
$$*$$
$$*$$
$\boxed{Obr.17.}$  $y=\ln x / \sqrt{x}$
$\boxed{Obr.18.}$  $y=x \arctan x$
$\boxed{424.}$ $f''(0)=0$ (najprv treba odvodiť vzorec pre druhú deriváciu inverznej funkcie: ($f^{-1})''(x)=-\frac{f''(f^{-1}(x))}{(f')^{3}(f^{-1}(x))}$; stanovte predpoklady, za ktorých možno toto odvodenie vykonať);
$\boxed{425.}$ $\boldsymbol{1.}$ $y^{(n)}=\frac{3^{n}}{4}\cos (3x+\frac{n\pi}{2})+\frac{3}{4}\cos (x+ \frac{n\pi}{2}) $;
$\boldsymbol{2.}$ $y^{(n)}=2^{2n-3}\cos (4x+\frac{n\pi}{2}+2^{n-1}.\cos (2x+\frac{n\pi}{2})$;
$$*$$
$$*$$
$$*$$
$$*$$
$\boxed{Obr.19.}$  $y=\ln x / \arcsin (2x/(1+x^{2})$
$\boxed{Obr.20.}$  $y=(x+2)e^{1/x}$
$\boxed{Obr.21.}$  $y=\arccos((1-x)/(1-2x))$
$\boxed{Obr.22.}$  $y=x^{x}$
$\boldsymbol{3.}$ $y^{(n)}=72^{x}\ln ^{n-1} 72. ((2x-1)\ln 72+2n)$;
$\boldsymbol{4.}$ úpravu $\ln (x^{2}-3x+2)=\ln (x-2)+\ln(x-1)$ nemožno priamo použiť (porovnaj s pr. 26.2), treba si pomôcť nasledovne: pre všetky $x \in D((\ln f)'(x)=(\ln \vert f \vert )'(x)$; $y^{(n)}=(n-2)!(-1)^{n}.(\frac{x-2n}{(x-2)^{n}}+\frac{x-n}{(x-1)^{n}})$ pre $n\geq 2$; $y'=\ln (x^{2}-3x+2)+\frac{2x^{2}-3x}{x^{2}-3x+2}$;
$\boldsymbol{5.}$ $y^{(n)}=(\sqrt{2})^{n}e^{x} \sin (x+\frac{n\pi}{4})$ (treba použiť rovnosť $\sin x +\cos x=\sqrt{2} \sin (x+\frac{\pi}{4}) $ a indukciu);
$\boxed{426.}$ $f^{(n)}(a)=n\varphi ^{(n-1)}(a)$ (na výpočet $f^{(n)}(a)$ nemožno použiť vetu 7, existencia $\varphi ^{(n)}(a)$ nie je zaručená);
$\boxed{429.}$ pozri pr. 329;
$\boxed{430.}$ $a_{0}+...+a_{n}x^{n}=(a_{0}\frac{x}{1}+...+a_{n}\frac{x^{n+1}}{n+1})'$, polynóm v zátvorke vpravo má korene 0,  1;
$\boxed{431.}$ pozri pr. 332;
$\boxed{432.}$ pozri pr. 331, polynóm $(x^{2}-1)^{n}$ má n-násobné korene 1 a -1;
$\boxed{433.}$ sporom, použiť pr. 331;
$\boxed{435.}$ možno postupovať indukciou, pričom dôkaz pre dané n sa robí sporom; ak P je polynóm s n zmenami znamienok, tak $(x^{-k}P(x))'$ (kde $k\in \mathbb{N}$ je vhodne zvolené, pozri návod k pr. 434) má n-1 zmien znamienok, pritom $(x^{k+1}(x^{-k}P(x))'$ je polynóm, ktorý má rovnaké kladné korene ako $(x^{-k}P(x))'$ a tiež $n-1$ zmien znamienok;
$\boxed{437.}$ $f$ je polynóm stupňa $k$, kde $k\leq n-1$ (pozri návod k pr. 339);
$\boxed{438.}$ ak $f'(x_{0})=0$, tak $\frac{f(x)-f'(x_{0})}{f'(x_{0})} =\frac{f'(c(x))}{f'(x_{0})}(x-x_{0})$, pričom uvedený podiel derivácií je ohraničená funkcia;
$\boxed{441.}$ pozri pr. 343;
$\boxed{444.}$  možno použiť pr. 443 tak, ako sa používa Rolleho veta pri dôkaze Lagrangeovej vety o strednej hodnote;
$\boxed{445.}$ sporom	 keby to nebola pravda, tak by existoval interval $I\subset (a,b)$, na ktorom by funkcia $f$ mala nulovú deriváciu, čo je spor s jej injektívnosťou (treba si uvedomiť, že $f$ aj $f^{-1}$ sú monotónne spojité funkcie);
$\boxed{446.}$ treba použiť Cauchyho vetu pre funkcie $x^{2}.f(x),x^{2}$;
$\boxed{447.}$ (pozri tiež pr. 352)
$\boldsymbol{1.}$ ak $f(x):=\ln x$, $g(x):=\frac{x-1}{\sqrt{x}}$, tak $f(1)=g(1)$, $f'(x)< g'(x)$ pre $x>0, x\ne 1$, preto $f(x)< g(x)$ pre $x>1$,  $f(x)>g(x)$ pre $x\in (0,1)$;
$\boldsymbol{2.}$ najprv zlogaritmovať, potom položiť $x=\frac{1}{t}$;
$\boldsymbol{3.}$ môžeme predpokladať $x\leq y$; položme $t=\frac{x}{y}$, dostaneme ekvivalentnú nerovnosť $(1+t^{\alpha})^{1/\alpha}>(1+t^{\beta})^{1/\beta}$, $t\in (0,1 \rangle$, potom zlogaritmovať;
$\boxed{449.}$ $\boldsymbol{3.}$ funkcia $h:=f-g$ vyhovuje všetkým predpokladom tvrdenia z pr. 449.1;
$\boxed{450.}$ označme $O(c)$ to okolie bodu $c\in I$,v ktorom platí $\forall x,y \in O(c)$, $x<c<y:f(x)<f(c)<f(y)$; nech $a,b\in I$, $a<b$, potom $\langle a,b \rangle$ je kompatk, z otvoreného pokrytia $\lbrace O(c);c\in \langle a,b \rangle \rbrace $ vyberme konečné podpokrytie a z neho také konečné pokrytie $\lbrace O(c_{1}),...,O(c_{n})\rbrace$, že $O(c_{i})\cap O(c_{i+1})\ne \varnothing$ pre $i=1,...,n-1,$ $a<c_{1}<x_{1}<c_{2}<x_{2}<...<x_{n-1}<c_{n}<b$, pričom  $a\in O(c_{1})$, $b\in O(c_{n})$, $x_{i}\in O(c_{i})\cap O(c_{i+1})$ pre $i=1,...,n-1$; potom $f(a)<f(c_{1})<f(x_{1}<...<f(b))$; teda $f(a)<f(b))$, ak $a<b$;
$\boxed{451.}$ $\boldsymbol{1.}$ to, že $f$ je rastúca v bode 0, možno dokázať (aspoň) dvoma spôsobmi: a/ priamo z definície: pre $x\in (0,1)$ je $x+x^{2} \sin \frac{2}{x}\geq x-x^{2} = x(1-x)>0$, pre $x\in (-1,0)$ je $x+x^{2} \sin \frac{2}{x}< x-x^{2}<0$ využili sme nerovnosť $\vert \sin \frac{2}{x}\vert \leq 1$ pre $x\ne 0$); b/ na základe pr. 412, pretože $f'(0)=1$;
$\boldsymbol{2.}$ $f'$ je spojitá na $\mathbb{R}- \lbrace 0 \rbrace$ a v ľubovoľne malom okolí bodu 0 nadobúda kladné aj záporné hodnoty (stačí vypočítať $f'(\frac{2}{\pi/2+2k\pi})$ pre $k \in \mathbb{Z}$ a $f'(\frac{1}{k\pi})$ pre $k \in \mathbb{Z} - \lbrace 0 \rbrace $, preto $f$ nemôže byť rastúca na žiadnom okolí bodu 0 (porovnaj aj s pr.350);
$\boxed{452.}$ uvedieme dva návody: a/ ukázať, že $f$ je klesajúca v každom bode $x\in I$, ďalej porovnaj s pr.450; b/ ak $M$ je konečná, porovnaj s pr.354;  ak $M$ je nekonečná, tak má aspoň jeden hromadný bod, ten musí byť krajným bododm intervalu $I$ (označme krajné body intervalu $I$ ako a, b, $a<b$); ak je tým hromadným bodom len bod b, možno  $M$ zoradiť do rastúcej postupnosti  $\lbrace a_{n} \rbrace _{n=1}^{\infty}$, $\lim_{n \to \infty}a_n =b$, pričom $f$ je klesajúca na $\langle a_{n}, a_{n+1} \rangle$  $n\in \mathbb{N}$; úvahy pre prípad, že množina hromadných bodov množiny $M$ je $\lbrace a \rbrace$, resp. $\lbrace a,b \rbrace$ sú podobné;
$\boxed{453.}$ matematickou indukciou; $f(\lambda _{1}x_{1}+...+  \lambda _{n}x_{n}+  \lambda _{n+1}x_{n+1})=f((1-\lambda _{n+1})(\frac{\lambda  _{1}}{1-\lambda _{n+1}}x_{1}+...+\frac{\lambda _{n}}{1-\lambda _{n+1}}x_{n})+\lambda _{n+1}x_{n+1})$, pritom  $\lambda _{1}+...+\lambda _{n}=1-\lambda _{n+1}$;
$\boxed{454.}$ $\boldsymbol{1.}$ využiť Jensenovu nerovnosť pre funkciu $f(x)=x^{2}$ (zvoliť  $\lambda _{1}=...=\lambda _{n}=\frac{1}{n}$);
$\boldsymbol{2.}$ zlogaritmovať; funkcia - $\ln x$ je konvexná na $\mathbb{R}^{+}$;
$\boldsymbol{3.}$ vynásobiť obidve strany číslom $1/n^{r}$ a využiť Jensenovu nerovnosť pre funkciu $f(x)=x^{r}$, $r>1, x>0$;
$\boxed{456.}$ ak je polynóm P párnou funkciou, tak $P(x)=a_{0}+a_{2}x^{2}+...+a_{2m}x^{2m}$; súčet klaných násobkov konvexných funkcií je konvexná funkcia;
$\boxed{458.}$ stačí dokázať druhé z uvedených tvrdení; pre $x\in (c,b)$ je funkcia $F(x):=\frac{f(x)-f(c)}{x-c}$ neklesajúca (pozri návod k pr. 362.1) a zdola ohraničená (napr. číslom $=\frac{f(a)-f(c)}{a-c}$; pre $x\in (c,b)$ ležia body grafu funkcie $f$  nad alebo na spojnici bodov $(a,f(a)), (c,f(c))$, preto $\lim_{x \to c+}F(x)$  existuje a je konečná (z uvedeného tiež vyplýva, že $f_{-}'(c)\leq f_{+}'(c)$);
$\boxed{459.}$ stačí použiť pr. 362.2;
$\boxed{461.}$ inverzná k rastúcej konvexnej (konkávnej) je rastúca konvexná (konkávna), ku klesajúcej  konvexnej (konkávnej) je klesajúca konvexná (konkávna); pri  dôkaze použite definíciu konvexnosti prepísanú na tvar $\forall x,y,z\in I,\, x<z<y: f(z)\leq f(x) +\frac{f(y)-f(x)}{y-x}(z-x)$ (porovnaj s návodom k pr. 362.1) a inšpirujte sa obrázkom;
$\boxed{463.}$ $\boldsymbol{1.}$ lok. maximum $3\sqrt[3]{3}$ pre $x=-3$, lok. minimum $-\sqrt[3]{44}$ pre $x=2$ (stačí nájsť lokálne extrémy odmocnenca a využiť, že $f(z)=\sqrt[3]{z}$ je rastúca funkcia);
$\boldsymbol{2.}$  pre $n$ párne: lok. maximum $n^{n}e^{1-n}$ pre $x=n-1$, lok. minimum 0 pre  $x=-1$; pre $n$ nepárne: lok. maximum $n^{n}e^{1-n}$ pre $x=n-1$;
$\boldsymbol{3.}$ ak $n$ je  párne: lok. maximum  $1$ pre $x=0$; ak $n$ je  nepárne: nemá lok. extrémy;
$\boldsymbol{4.}$ lok. minimum 0 pre  $x=0$ (pri úprave výrazu pre $y'$ využite vzorce $\sin \alpha \pm \cos \alpha = \sqrt{2}\sin (\frac{\pi}{4}\pm \alpha )$; $y$ rastie pre $x\geq 0$, klesá pre $x\leq 0$);
$\boxed{464.}$ $\boldsymbol{2.}$  najprv vydeľte výrazom $(x+a)$, potom položte $z=\frac{x}{a}$  a vyšetrite priebeh funkcie $f(z)=\frac{\sqrt[n]{1+z^{n}}}{1+z}$ $z\geq 0$;
$\boxed{465.}$  $x=\frac{H}{2}$ (dostrek je $\sqrt{2gx}t$, pričom $t$ vypočítame z rovnice $\frac{gt^{2}}{2=H-x}$, tj. $t$ je čas, za ktorý teleso spadne voľným pádom z výšky $H-x$);
$\boxed{466.}$  $(a^{2/3}+b^{2/3})^{3/2}$;
$\boxed{467.}$  $\boldsymbol{1.}$ $-3$;
$\boldsymbol{2.}$ $\frac{1}{3}$;
$\boldsymbol{3.}$ $0$ (rozšírte členom x a využite, že $\lim_{x \to 0}\frac{\ln (1+x)}{x}=1)$;
$\boldsymbol{4.}$ $+\infty$ (rozšírte členom x, l'Hospitalovo pravidlo použite len na prvý člen v čitateli);
$\boldsymbol{5.}$ $0$ (v exponente vyňať $x^{n}$, l'Hospitalovo pravidlo použite len na druhý člen v zátvorke);
$\boldsymbol{6.}$ $0$ (upraviť na tvar $e^{f(x)}$, v $f(x)$ vyňať $\ln ^{2}x$ pred zátvorku, l'Hospitalovo pravidlo použite len na druhý člen v zátvorke);
$\boldsymbol{7.}$ $1$ pri úpravách využite, že $\lim_{x \to 0}x^{x}=1)$;
$\boldsymbol{8.}$ $-1$;
$\boldsymbol{9.}$ $+\infty$;
$\boldsymbol{10.}$ $-\frac{1}{6}$ (odpočítať a pripočítať $\sqrt{x^{2}+x+1}$; prvý rozdiel možno previesť na typ $\frac{0}{0}$ substitúciou $x=\frac{1}{t}$ a vyňatím $\frac{1}{t}$; druhý rozdiel možno úpraviť na tvar $\ln (1+\frac{x}{e^{x}}).\frac{\sqrt{x^{2}+x+1}}{x})$;
$\boldsymbol{11.}$ a (vyňať $(x+a)^{1/x}$ a upraviť na tvar a $(x+a)^{1/x}-x(\frac{x^{1/(x+a)}}{(x+a)^{1/x}}-1)$, prvý člen v zátvorke upraviť na tvar $e^{u}$ a využiť, že $\lim_{u \to 0}\frac{e^{u}-1}{u}=1)$;
$\boxed{468.}$ pri dôkaze neexistencie $\lim_{x \to +\infty}\frac{f(x)}{g(x)})$ využite, že v bodoch $k\pi - \frac{\pi}{4}$ má funkcia $\frac{f}{g}$ nevlastné jednostranné limity opačného znamienka, preto  $\frac{f}{g}$ nie je zhora ani zdola ohraničená v žiadnom okolí bodu  $+\infty$;
$\boxed{469.}$ $\boldsymbol{1.}$ $\sum_{k=0}^n (-1)^{k}(k+1)x^{3k+1}$ pre $n=3n+1, 3n+2, 3n+3$ ($n\in \mathbb{N} \cup \lbrace 0 \rbrace $);
$\boldsymbol{2.}$ $\sum_{k=1}^n c^{k-1}_{-1/2}.(-1)^{k-1}\frac{x^{2k}}{2^{2k-1}}+\sum_{k=0}^n c^{k}_{-1/2}.(-1)^{k}\frac{x^{2k+1}}{2^{2k}}$, kde $c_{\alpha}^{0}=1$, $c_{\alpha}^{k}=\frac{\alpha(\alpha -1).....(\alpha -k+1)}{k!}$, $k\in \mathbb{N}$ (pre $x \in (-2,2)$ platí $f(x)=x(2+x)(4-x^{2})^{-1/2}$);
$\boldsymbol{3.}$ $\sum_{k=0}^n \frac{e^{-27}3^{k}}{k!}(x+3)^{2k+1}$ pre $n=2n+1,2n+2$ ($n=0,1,...)$;
$\boldsymbol{4.}$ $ 1+ \sum_{k=1}^n \frac{(-1)^{k-1}3^{k}}{k.2^{k}\ln 2}(x-4)^{2k}$ pre $n=2n,2n+1$  ($n\in \mathbb{N}$), Taylorov polynóm stupňa 1 má tvar $T(x)=1$ (využite vzorec $\log _{a} b=\frac{\ln b}{\ln a}$);
$\boldsymbol{5.}$ $\sum_{k=1}^n c^{k-1}_{-1/3}.(-1)^{k-1}(x-1)^{2k}$, Taylorov polynóm stupňa 1 má tvar $T(x)=0$ (symbol $c^{i}_{\alpha}$ pozri v návode k pr.469.2);
$\boldsymbol{6.}$ $-\frac{\sin 2}{2}+\frac{\sin 2}{2}(x+1)+\sum_{k=0}^n \frac{(-1)^{k}2^{2k}}{(2k+1)!}(x+1)^{4k+3}+\sum_{k=0}^n \frac{(-1)^{k+1}2^{2k}}{(2k+1)!}(x+1)^{4k+2}$; ($\sin \alpha .\cos \beta =\frac{1}{2}(\sin (\alpha +\beta)+\sin (\alpha -\beta))$);
$\boxed{470.}$ hodnoty $f^{(k)}(0)$ potrebné na nájdenie Maclaurinovho polynómu funkcie $f$ možno vyjadriť z koeficientov Maclaurinovho polynómu funkcie $f'$;
$\boldsymbol{1.}$ $x+\sum_{k=1}^n \frac{(-1)^{k}.1.3.....(2k-1)}{(2k+1)2^{k}k!} x^{2k+1}$ pre $n=2n+1,2n+2$ ($n\in \mathbb{N}$), Maclaurinove polynómy stupňa 1 a 2 majú tvar $T(x)=x$;
$\boldsymbol{2.}$ $\sum_{k=0}^n \frac{(-1)^{k}x^{2k+1}}{2k+1}$ pre $n=2n+1,2n+2$ ($n=0,1,...)$;
$\boldsymbol{3.}$ $\frac{\pi}{2}-x- \sum_{k=1}^n \frac{1.3.....(2n-1)}{(2k+1)2^{k}k!}x^{2k+1}$ pre $n=2n+1,2n+2$ ($n\in \mathbb{N}$), Maclaurinove polynómy stupňa 1 resp. 2 má tvar $T(x)=\frac{\pi}{2}- x$;
$$*$$
$$*$$
$$*$$
$$*$$
$\boxed{Obr.23.}$  $y=e^{-2x}\sin ^{2} x$
$\boxed{Obr.24.}$  $y^{3}=6x^{2}-x^{3}$
$\boxed{Obr.25.}$  $y^{2}=x^{4}(x+1)$
$\boxed{471.}$  nech $e=\frac{p}{q}$ ($p\in \mathbb{Z}, q\in \mathbb{N}$); zvoľme $n\geq q$; platí $\frac{p}{q}=e=1+\frac{1}{1!}+...+\frac{1}{n!}+\frac{\vartheta}{n!}$, pričom $\vartheta \in(0,1)$; ak z tejto rovnosti vyjadríme $\vartheta$, zistíme, že $\vartheta \in \mathbb{Z}$, čo je spor;
$\boxed{472.}$ ak uvedenú rovnosť porovnáme s Taylorovým vzorcom $(n+1)$-vého stupňa so zvyškom v Peanovom tvare, dostaneme po úprave: $\vartheta (x).\frac{f^{(n)}(a+\vartheta (x)(x-a))-f^{(n)}(a)}{\vartheta (x)(x-a)}=\frac{1}{n+1}.f^{(n+1)}(a)+\frac{o(x-a)}{x-a}$;
$\boxed{473.}$ $\boldsymbol{1.}$ $\frac{19}{90}$;
$\boldsymbol{2.}$ $\frac{1}{2}$;
$\boldsymbol{3.}$ $e^{2/3}$ ($=\lim_{x \to 0}(1+\frac{2}{3}x^{3}+o(x^{3}))^{1/x^{3}}$);
$\boldsymbol{4.}$ $\frac{1}{2}$ (subst. $\frac{1}{x}=t$;
$\boldsymbol{5.}$ $\frac{1}{3}$ (pred zátvorku vyňať x, potom subst. $\frac{1}{x}=t$);
$\boldsymbol{6.}$ $-\frac{1}{4}$;
$\boxed{474.}$ hodnoty $f(0),f(1)$ zapíšte pomocou Taylorovho polynómu stupňa 1 so stredom v bode x a so zvyškom v Lagrangeovom tvare, potom jednu z týchto rovností odčítajte od druhej;
$\boxed{475.}$ $\boldsymbol{1.}$ pre $M_{2}=0$ dokážeme uvedené tvrdenie samostatne; postup pre $M_{2}\ne 0$: z rovnosti (*) $f(x+h)=f(x)+f'(x)h+f''(c)h^{2}/2$ vyjadríme $f'(x)$; odtiaľ dostaneme: pre každé $h>0$ platí $M_{1}\leq 2M_{0}/h+M_{2}h/2$; odtiaľ $M_{2}(h-M_{1}/M_{2})^{2}+4M_{0}-M_{1}^{2}M_{2}\geq 0$  pre $h>0$, potom položíme $h=M_{1}/M_{2}$;
$\boldsymbol{2.}$ postup je analogický ako v pr. 475.1, naviac využijeme ešte rovnosť $f(x-h)=f(x)-f'(x).h+f''(d)h^{2}/2$, ktorú odčítame od rovnosti (*);
$\boxed{476.}$ $\boldsymbol{1.}$ obr. 23; stačí zostrojiť graf na niektorom intervale dĺžky $\pi$ a potom využiť vzťah  $y(x+\pi)=e^{-2\pi}$ $y(x)\quad (x\in \mathbb{R}) $; 
$\boldsymbol{2.}$ obr. 24; $y_{-}'(0)=-\infty$, $y_{+}'(0)=+\infty$, $y'(6)=-\infty$; pozor: $y$ nie je konkávna na $(-\infty ,6)$; podľa nami používanej definície nie je bod 6 inflexným bodom;
$\boldsymbol{3.}$ obr. 25; dotyčnica v bode $(-1,0)$ má rovnicu $x=-1$;
$\boldsymbol{4.}$ obr. 26; (pri riešení nerovníc  $y'\ne 0, y''\ne 0$ 	nezabúdajte, že pred umocňovaním na druhú treba skontrolovať, či obidve strany nerovnice majú rovnaké znamienko, a že pred násobením číslom $\alpha $ treba zistiť znamienko tohto čísla);
$$*$$
$$*$$
$$*$$
$$*$$
$\boxed{Obr.26.}$  $y=1-x+\frac{x^{3}}{3+x}$
$\boldsymbol{5.}$ obr. 27; dotyčnica v bode $(1,0)$ má rovnicu $x=1$; ak $y=x\sqrt{1-x}/(1+x)$, tak $y''=(1+x)^{-3}(1-x)^{-3/2}(-x^{3}-6x^{2}+15x-12)/4$, pritom  $-x^{3}-6x^{2}+15x-12<0$ pre $x>-5$ (to yistíme, ak vyšetríme rast, klesanie a lokálne extrémy funkcie $-x^{3}-6x^{2}+15x-12$ na intervale $(-5,\infty$ );
$\boxed{477.}$ $\boldsymbol{1.}$ obr. 28;
$\boldsymbol{2.}$ obr. 29.$$*$$
$$*$$
$$*$$
$$*$$
$\boxed{Obr.27.}$  $y^{2}=x^{2}(1-x)(1+x)^{-2}$
$\boxed{Obr.28.}$
$\boxed{Obr.29.}$
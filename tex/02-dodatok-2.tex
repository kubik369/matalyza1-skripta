\chapter{Dodatok $2$; otvorené, uzavreté a kompaktné množiny}%\label{chapter:gramatiky}

Bod $a \in \mathbb{R}$ sa nazýva \textit{vnútorný bod množiny $a \subset \mathbb{R}$}, ak existuje také jeho okolie $O(a)$, že platí $O(a) \subset A$.

Neprázdna množina $a \subset \mathbb{R}$ sa nazýva \textit{otvorená}, ak každý jej prvok je jej vnútorným bodom. Ďalej sa dohodneme, že prázdnu množinu budeme pokladať za otvorenú.
Množina $A \subset \mathbb{R}$ sa nazýva \textit{uzavretá}, ak je $\mathbb{R} \setminus A$ otvorená.

\begin{veta}
Pre ľubovoľnú množinu $A \subset \mathbb{R}$ sú nasledujúce tvrdenia ekvivalentné:
\begin{enumerate}
\item $A$ je uzavretá množina;
\item ak $a \in \mathbb{R}$ je hromadný bod množiny $A$, tak $a \in A$;
\item ak všetky členy konvergentnej postupnosti ${\{a_n\}}_{n=1}^\infty$ sú prvkami množiny $A$, tak $\lim_{n \rightarrow \infty} a_n \in A$.
\end{enumerate}
\end{veta}

\textit{Poznámka:}
Často sa pojem uzavretej množiny definuje pomocou vlastnosti $2.$ alebo $3.$. Otvorená množina sa niekedy definuje pomocou uzavretej, a to tak, že sa najprv pomocou niektorej z vlastností $2.$,$3.$ zavedie pojem uzavretej množiny a za otvorené sa potom vy.hlásia tie množiny $A$, ktorých doplnky $\mathbb{R} \setminus A$ uzavreté.

\begin{enumerate}[resume]
	\item \useproblem[dodatok-2]{dodatok-1}
	\item \useproblem[dodatok-2]{dodatok-2}
	\item \useproblem[dodatok-2]{dodatok-3}
	\item \useproblem[dodatok-2]{dodatok-4}
	\item \useproblem[dodatok-2]{dodatok-5}
	\item \useproblem[dodatok-2]{dodatok-6}
\end{enumerate}

Systém $\{A_t: t \in I\}$ množín reálnych čísel (neprázdna množina I - nie nutne číselná - sa nazýva indexová) sa nazýva \textit{pokrytie množiny $A \subset \mathbb{R}$}, ak $A \subset \bigcup_{t \in I} A_t$  $(:= \{x \in \mathbb{R}; \exists t \in I: x \in A_t\})$. Ak je naviac každá z množín $A_t$, $t \in I$, otvorená, nazýva sa tento systém \textit{otvorené pokrytie množiny A}; termín \textit{konečné pokrytie} používame, ak je množina I konečná.

Množina $A \subset \mathbb{R}$ sa nazýva \textit{kompaktná množina (kompakt)}, ak z každého jej otvoreného pokrytia $\{a_t: t \in I\}$ možno vybrať konečné pokrytie (t.j. existuje konečná neprázdna množina $I' \subset I$ tak, že $A \subset \bigcup_{t \in I} A_t$).

\begin{veta}
Pre ľubovoľnú množinu $A \subset \mathbb{R}$ sú nasledujúce tvrdenia ekvivalentné:
\begin{enumerate}
\item $A$ je kompaktná množina;
\item $A$ je uzavretá a ohraničená množina;
\item ak všetky členy postupnosti ${\{a_n\}}_{n=1}^\infty$ sú prvkami množiny $A$, tak z ${\{a_n\}}_{n=1}^\infty$ možno vybrať konvergentnú podpostupnosť, ktorej milita je prvkom množiny $A$.

\textit{Poznámka:}
Často sa kompaktná množina definuje pomocou vlastnosti $3.$.
\end{enumerate}
\end{veta}

\begin{enumerate}[resume]
	\item \useproblem[dodatok-2]{dodatok-7}
	\item \useproblem[dodatok-2]{dodatok-8}
	\item \useproblem[dodatok-2]{dodatok-9}
	\item \useproblem[dodatok-2]{dodatok-10}
	\item \useproblem[dodatok-2]{dodatok-11}
	\item \useproblem[dodatok-2]{dodatok-12}
	\item \useproblem[dodatok-2]{dodatok-13}
\end{enumerate}
\chapter{Spojitosť funkcie}%\label{chapter:gramatiky}

\section{Definícia spojitosti v bode a na množine. Klasifikácia bodov nespojitosti}
Hovoríme, že funkcia $f$ je spojitá v bode $a \in D(f)$, ak platí
 $$\forall \varepsilon > 0 \exists \delta > 0 \forall x \in D(f):|x-a|<\delta \Rightarrow |f(x)-f(a)|< \varepsilon$$
 \textit{Poznámka:} Všimnime si, že bod $a$ nemusí byť hromadným bodom množiny $D(f)$. Môžu teda nastať dva prípady:
 \begin{itemize}
 \item ak $a \in D(f)$ je hromadným bodom množiny $D(f)$, tak podmienka vyššie je ekvivalentná s podmienkou $\lim_{x \rightarrow a} f(x)=f(a)$;
 \item ak $a \in D(f)$ nie je hromadným bodom množiny $D(f)$ (t.j. existuje $\eta > 0 $ tak, že $D(f) \cup (a-\eta,a+\eta)=\{ a\}$; taký bod sa nazýva izolovaným bodom množiny $D(f)$), je podmienka vyššie splnená (stačí položiť $\delta=\eta$ pre ľubovoľné $\varepsilon > 0$). Teda funkcia $f$ je splnená v každom izolovanom bode množiny $D(f)$.
 \end{itemize}
 
 \begin{veta}
 Ak funkcie $f,g,f-g,f \cdot g$ sú spojité v bode $a$. Ak naviac $g(a) \neq 0$, tak aj funkcia $\frac{f}{g}$ je spojitá v bode $a$.
 \end{veta}
 
 \begin{veta}
 Ak funkcia $f$ je spojitá v bode $a$, funkcia $g$ je spojitá v bode $f(a)$, tak funkcia $g \circ f$ je spojitá v bode $a$.
 
 Hovoríme, že funkcia $f$ je spojitá, ak je spojitá v každom bode svojho definičného oboru. Hovoríme, že funkcia $f$ je spojitá na množine $A$ ($\emptyset \neq a \subset D(f)$), ak je spojitá funkcia $f \setminus A$.
 \end{veta}
 
 \begin{veta}
 Každá základná elementárna funkcia je spojitá.
 \end{veta}
 
 (Z viet $1,2$ a $3$ vyplýva veta $4$ z kapitoly $2.$)
 
 \begin{enumerate}[resume]
	\item \useproblem[spojitost-funkcie]{s-funkcie-1}
	\item \useproblem[spojitost-funkcie]{s-funkcie-2}
	\item \useproblem[spojitost-funkcie]{s-funkcie-3}
	\item \useproblem[spojitost-funkcie]{s-funkcie-4}
\end{enumerate}

\textit{Riešenie 2:}
Dokázať spojitosť funkcie $f$ na množine $\mathbb{R} \setminus \{0\}$ je jednoduché: pre každé číslo $a<0$ možno nájsť také jeho okolie $O(a)$, na ktorom sa funkcia $f$ zhoduje s elementárnou funkciou $\frac{1-e^x}{x}$ tá je spojitá v bode $a$; preto $\lim_{x \rightarrow a}f(x)=\lim_{x \rightarrow a}\frac{1-e^x}{x}=\frac{1*e^a}{a}=f(a)$, čo znamená, že funkcia $f$ je spojitá v bode $a$.Analogicky sa postupuje v prípade $a>0$ \footnote{Uvedenú myšlienku možno ľahko zovšeobecniť: ak je daná funkcia $f$ a spojitá funkcia $g$ a existuje neprázdna otvorená množina $G \subset D(f) \cap D(g)$ taká, že $\frac{f}{G}=\frac{g}{G}$, tak $f$ je spojitá v každom bode množiny $G$.}. Prípad $a=0$ treba vyšetriť samostatne: vtedy $\lim_{x \rightarrow 0+}f(x)=\lim_{x \rightarrow 0+}\frac{1-e^x}{x}=-1,\lim_{x \rightarrow 0-}f(x)=\lim_{x \rightarrow 0-}(2x-1)=-1$, preto $\lim_{x \rightarrow 0}f(x)=-1=f(0)$. Funkcia $f$ je teda spojitá aj v bode $0$, preto $f$ je spojitá funkcia.

\begin{enumerate}[resume]
	\item \useproblem[spojitost-funkcie]{s-funkcie-5}
	\item \useproblem[spojitost-funkcie]{s-funkcie-6}
	\item \useproblem[spojitost-funkcie]{s-funkcie-7}
	\item \useproblem[spojitost-funkcie]{s-funkcie-8}
	\item \useproblem[spojitost-funkcie]{s-funkcie-9}
	\item \useproblem[spojitost-funkcie]{s-funkcie-10}
	\item \useproblem[spojitost-funkcie]{s-funkcie-11}
\end{enumerate}

Číslo $a \in \mathbb{R}$, ktoré je hromadným bodom definičného oboru $D(f)$ funkcie $f$, sa nazýva bodom nespojitosti funkcie $f$, ak je splnená niektorá z nasledujúcich podmienok:
\begin{itemize}
\item $a \notin D(f)$;
\item neexistuje $\lim_{x \rightarrow a}f(x)$;
\item $a \in D(f)$ a existuje $\lim_{x \rightarrow a}f(x)$, ale neplatí $\lim_{x \rightarrow a}f(x)=f(a)$.
\end{itemize}
Ak $a \in \mathbb{R}$ je bod nespojitosti funkcie $f$ a existuje konečná $\lim_{x \rightarrow a}f(x)$, nazýva sa bodom odstrániteľnej nespojitosti. Ak neexistuje vlastná $\lim_{x \rightarrow a}f(x)$ nazýva sa $a$ bodom neodstrániteľnej nespojitosti.

Nech $a \in \mathbb{R}$ je bod nespojitosti funkcie $f$, pričom $a$ je hromadným bodom množín $D(f) \cap (-\infty,a)$ aj $D(f)\cap (a,+\infty)$. Ak v bode $a$ existujú vlastné jednostranné limity a $\lim_{x \rightarrow a-}f(x) \neq \lim_{x \rightarrow a+}f(x)$ nazýva sa $a$ bodom nespojitosti $1.$ druhu (rozdiel $\lim_{x \rightarrow] a+}f(x)-\lim_{x \rightarrow a-}f(x)$ sa potom nazýva skokom funkcie $f$ v bode $a$). Ak aspoň jedna z jednostranných limít v bode $a$ neexistuje alebo je nevlastná, nayzva sa $a$ bodom nespojitosti $2.$ druhu.

\textit{Poznámka:}
Klasifikácia bodov nespojitosti nie je v matematickej literatúre jednotná; napr. niekedy sa hromadné body množiny $D(f)$, ktoré nie sú prvkami $D(f)$, nepovažujú za body nespojitosti funkcie $f$; v definícii bodu nespojitosti $1.$ druhu s aniekedy nepovažuje splnenie podmienky $\lim_{x \rightarrow a+}f(x) \neq \lim_{x \rightarrow a-}f(x)$ atď.

\begin{enumerate}[resume]
	\item \useproblem[spojitost-funkcie]{s-funkcie-12}
	\item \useproblem[spojitost-funkcie]{s-funkcie-13}
\end{enumerate}

\section{Vlastnosti spojitých funkcií I}

Funkcia $f$ definovaná na intervale $I$ sa nazýva darbouxovská na $I$, ak pre Každé $a,b \in I,a<b$ platí: na intervale $\langle a,b \rangle$ nabudúce funkcia $f$ všetky hodnoty medzi $f(a)$ a $f(b)$; t.j. ak platí 
$$\forall a,b \in I,a<b:f(\langle a,b \rangle)\supset \{y \in \mathbb{R}; \min \{f(a),f(b)\} \leq y \leq \max \{f(a),f(b)\}\}.$$

\begin{veta}
Ak je funkcia $f$ spojitá na intervale $I$, tak je darbouxovská na $I$.
\end{veta}

\begin{enumerate}[resume]
	\item \useproblem[spojitost-funkcie]{s-funkcie-14}
\end{enumerate}

\textit{Riešenie 1:}
Funkcia daná predpisom $f(x)=x^5+x^4+x^3+x^2-x-1$ je spojitá, a teda aj darbouxovská na $\mathbb{R}$, nadobúda preto na intervale $\langle 0,1 \rangle$ všetky hodnoty medzi $f(0)=-1$ a $f(1)=2$. Pretože $-1<0<2$, musí existovať $x \in (0,1)$, v ktorom $f(x)=0$.

\begin{enumerate}[resume]
	\item \useproblem[spojitost-funkcie]{s-funkcie-15}
	\item \useproblem[spojitost-funkcie]{s-funkcie-16}
	\item \useproblem[spojitost-funkcie]{s-funkcie-17}
	\item \useproblem[spojitost-funkcie]{s-funkcie-18}
	\item \useproblem[spojitost-funkcie]{s-funkcie-19}
\end{enumerate}

\begin{veta}
Ak je funkcia $f$ spojitá na kompaktnej množine $A$ (alebo špeciálne na uzavretom ohraničenom intervale), tak je na $A$ Ohraničená a existujú $\max_{x \in A}f(x)$ a $\min_{x \in A}f(x)$.
\end{veta}

\begin{enumerate}[resume]
	\item \useproblem[spojitost-funkcie]{s-funkcie-20}
	\item \useproblem[spojitost-funkcie]{s-funkcie-21}
	\item \useproblem[spojitost-funkcie]{s-funkcie-22}
	\item \useproblem[spojitost-funkcie]{s-funkcie-23}
	\item \useproblem[spojitost-funkcie]{s-funkcie-24}
	\item \useproblem[spojitost-funkcie]{s-funkcie-25}
	\item \useproblem[spojitost-funkcie]{s-funkcie-26}
\end{enumerate}

\section{Rovnomerná spojitosť. Vlastnosti spojitých funkcií II}
Funkcia $f$ sa nazýva rovnomerne spojitá na množine $A \subset D(f)$ $(A \neq \emptyset)$, ak platí 
$$\forall \varepsilon > 0 \exists \delta > 0 \forall x,y\in A: |x-y|<\delta \Rightarrow |f(x)-f(y)|<\varepsilon.$$
Funkcia $f$ rovnomerne spojitá na svojom definičnom obore sa nazýva rovnomerne spojitá.

\begin{enumerate}[resume]
	\item \useproblem[spojitost-funkcie]{s-funkcie-27}
	\item \useproblem[spojitost-funkcie]{s-funkcie-28}
\end{enumerate}

\textit{Riešenie 2:}
Predpokladajme, že $x,y \in \langle -2,5 \rangle,|x-y|<\delta$, a pokúsme sa na základe toho zhora odhadnúť výraz $|f(x)-f(y)|$. Postupne dostaneme $|f(x)-f(y)|=|x^2-2x-1-(y^2-2y-1)|=|(x^2-y^2)+2(y-x)|\leq |x^2-y^2|+2|x-y|=|x-y||x+y|+2|x-y|=|x-y|(|x+y|+2)\leq |x-y|(|x|+|y|+2)\leq 12|x-y|<\delta$.

Zistili sme teda: ak $x,y \in \langle -2,5 \rangle,|x-y|<\delta$, tak $|f(x)-f(y)|<12\delta$. To znamená, že funkcia $f$ je rovnomerne spojitá na intervale $\langle -2,5 \rangle$, pre dané $\varepsilon > 0$ stačí totiž položiť $\delta = \frac{\varepsilon}{12}$.

\textit{Riešenie 5:}
Ak si predstavíme graf funkcie $f(x)=\sin x^2,x \geq 0$, zistíme, že v smere k $+\infty$ funkcia $f$ na intervaloch stále menšej dĺžky nadobúda všetky hodnoty medzi $1$ a $-1$, t.j. že jej graf sa smerom k $+\infty$ "zhusťuje". To nás vedie k domnienke, že $f$ nie je na $\langle 0,\infty)$ rovnomerne spojitá. Všimnime si preto podrobnejšie dvojice bodov $(x_n,y_n)=(\sqrt{-\frac{\pi}{2}+2\pi n},\sqrt{\frac{\pi}{2}+2\pi n}),n\in \mathbb{N}$; zrejme $f(x_n)=-1,f(y_n)=1$. Pretože $\lim_{n \rightarrow \infty}|y_n-x_n|=0$, existuje pre každé $\delta>0$ dvojica $(x_n,y_n)$ také, že $|x_n-y_n|<\delta$. Tým sme dokázali 
$$\exists \varepsilon >0 (\varepsilon=2)\forall \delta>0\exists x_n \geq 0,y_n \geq 0: |x_n-y_n|<\delta \wedge |f(x_n)-f(y_n)| \geq \varepsilon,$$
čo je negácia tvrdenia "$f$ je rovnomerne spojitá na $\langle 0,\infty )$". Teda $\sin x^2$ nie je rovnomerne spojitá na intervale $\langle 0, \infty )$.

\begin{enumerate}[resume]
	\item \useproblem[spojitost-funkcie]{s-funkcie-29}
	\item \useproblem[spojitost-funkcie]{s-funkcie-30}
\end{enumerate}

\begin{veta}
Funkcia spojitá na kompaktnej množine $K$ je rovnomerne spojitá na $K$.
\end{veta}

\begin{enumerate}[resume]
	\item \useproblem[spojitost-funkcie]{s-funkcie-31}
\end{enumerate}

\textit{Riešenie 2:}
Na funkciu $f(x)=\frac{\sin x}{x},x \in (0,\pi \rangle$ nemôžeme vetu $6$, pretože $D(f)=(0,\pi \rangle$ nie je kompaktná množina. Pomôžeme si nasledovne: funkciu $f$ možno spojite dodefinovať v čísle $0$, pretože $0$ je bodom odstrániteľnej nespojitosti. Teda $f=f_1/(0,\pi \rangle$, kde funkcia $f_1$ je určená predpisom $f_1(x) = \left\{ \begin{array}{r@{\quad}c}
    f(x), & $ak $ x \in (0,\pi \rangle \\
    1, &  $ak $ x=0 \\ \end{array} \right.
    $ . Na kompakte $\langle 0,\pi \rangle$ je $f_1$ spojitá, a teda podľa vety $6$ aj rovnomerne spojitá. Pretože zúženie rovnomerne spojitej funkcie je funkcia rovnomerne spojitá, je funkcia $\frac{\sin x}{x}$ rovnomerne spojitá na intervale $(0,\pi \rangle$.
    
\begin{enumerate}[resume]
	\item \useproblem[spojitost-funkcie]{s-funkcie-32}
	\item \useproblem[spojitost-funkcie]{s-funkcie-33}
	\item \useproblem[spojitost-funkcie]{s-funkcie-34}
	\item \useproblem[spojitost-funkcie]{s-funkcie-35}
	\item \useproblem[spojitost-funkcie]{s-funkcie-36}
\end{enumerate}

\section{Ďalšie príklady}
\begin{enumerate}[resume]
	\item \useproblem[spojitost-funkcie]{s-funkcie-37}
	\item \useproblem[spojitost-funkcie]{s-funkcie-38}
	\item \useproblem[spojitost-funkcie]{s-funkcie-39}
	\item \useproblem[spojitost-funkcie]{s-funkcie-40}
	\item \useproblem[spojitost-funkcie]{s-funkcie-41}
	\item \useproblem[spojitost-funkcie]{s-funkcie-42}
	\item \useproblem[spojitost-funkcie]{s-funkcie-43}
	\item \useproblem[spojitost-funkcie]{s-funkcie-44}
	\item \useproblem[spojitost-funkcie]{s-funkcie-45}
	\item \useproblem[spojitost-funkcie]{s-funkcie-46}
	\item \useproblem[spojitost-funkcie]{s-funkcie-47}
	\item \useproblem[spojitost-funkcie]{s-funkcie-48}
	\item \useproblem[spojitost-funkcie]{s-funkcie-49}
	\item \useproblem[spojitost-funkcie]{s-funkcie-50}
	\item \useproblem[spojitost-funkcie]{s-funkcie-51}
	\item \useproblem[spojitost-funkcie]{s-funkcie-52}
	\item \useproblem[spojitost-funkcie]{s-funkcie-53}
	\item \useproblem[spojitost-funkcie]{s-funkcie-54}
	\item \useproblem[spojitost-funkcie]{s-funkcie-55}
	\item \useproblem[spojitost-funkcie]{s-funkcie-56}
	\item \useproblem[spojitost-funkcie]{s-funkcie-57}
	\item \useproblem[spojitost-funkcie]{s-funkcie-58}
\end{enumerate}
\epigraph{
  \enquote{Prosím také, aby čtenář nepovažoval tyto vědomě neúplné poznámky za
nějaký (sebe elementárnejší) úvod do matematickej logiky. Šlo mně jenom o to,
abychom si ujasnili způsob vjadřovaní, kterého ostatně nadaný matematik užívá
intuitivně s neomylnou správností.}
}{\textit{Vojtěch Jarník, Diferenciální počet (II)}}

V matematike sa stretávame s tvrdeniami, o ktorých má
zmysel povedať, že sú pravdivé alebo nepravdivé. Také tvrdenia sa nazývajú
\emph{výroky}.

Z výrokov A, B môžeme pomocou operácií negácie ($\neg$), konjukcie
($\land$), disjunkcie ($\lor$), implikácie ($\Rightarrow$) a ekvivalencie
($\iff$) vytvoriť nové, tzv. \emph{zložené výroky}:

\begin{center}
  \begin{tabular}{|m{0.15\linewidth} | p{0.6\linewidth}|}
    \hline
    $\neg A$ & negácia výroku $A$ \\
    \hline
    \bigskip
    $A \land B$
      & \enquote{$A$ a súčasne $B$},
        \enquote{$A$ a $B$};

        namiesto $A \land B$ a často píše $A, B$ \\
    \hline
    \bigskip
    $A \lor B$
      & \enquote{$A$ alebo $B$},
        \enquote{alebo $A$ alebo $B$} \\
    \hline
    \bigskip
    $A \Rightarrow B$
      & \enquote{ak $A$, tak $B$},
        \enquote{nech $A$, potom $B$},

        \enquote{$A$ je postačujúca podmienka pre $B$},

        \enquote{$B$ je nutná podmienka pre $A$} \\
    \hline
    \bigskip
    $A \iff B$
      & \enquote{$A$ vtedy a len vtedy, keď $B$},

        \enquote{$A$ práve vtedy, keď $B$},

        \enquote{$A$ je nutná a postačujúca podmienka pre $B$} \\
    \hline
  \end{tabular}
\end{center}

Pravdivosť týchto zložených výrokov závisí len na pravdivosti výrokov $A, B$,
a to následovne (1 označuje, že daný výrok je pravdivý; 0, že je nepravdivý):
\begin{center}
  \begin{tabular}{|c|c|}
    \hline
    $A$ & $\neg A$ \\
    \hline
    1 & 0 \\
    0 & 1 \\
    \hline
  \end{tabular}
  \quad
  \begin{tabular}{|c|c|c|c|c|c|}
    \hline
    $A$ & $B$ & $A \land B$ & $A \lor B$ & $A \Rightarrow B$ & $A \iff B$ \\
    \hline
    1 & 1 & 1 & 1 & 1 & 1 \\
    1 & 0 & 0 & 1 & 0 & 0 \\
    0 & 1 & 0 & 1 & 1 & 0 \\
    0 & 0 & 0 & 0 & 1 & 1 \\
    \hline
  \end{tabular}
\end{center}

Pre negáciu zložených výrokov platia tieto pravidlá:
\begin{center}
  \begin{tabular}{c | c}
    výrok & negácia \\
    \hline
    $\neg A$ & $A$ \\
    $A \land B$ & $\neg A \lor \neg B$ \\
    $A \lor B$ & $\neg A \land \neg B$ \\
    $A \Rightarrow B$ & $A \land \neg B$ \\
    $A \iff B$ & $(A \land \neg B) \lor (B \land \neg A)$ \\
  \end{tabular}
\end{center}

Často budeme využívať, že pre ľubovoľné výroky $A, B$ je:
\begin{enumerate}
  \item
    implikácia rovnocenná s obmenou (kontrapozíciou)
    \[
      A \Rightarrow B \iff \neg B \Rightarrow \neg A
    \]
  \item
    ekvivalencia rovnocenná s výrokom
    \[
      (A \iff B) \iff (A \Rightarrow B) \land (B \Rightarrow A)
    \]
\end{enumerate}

Každá definícia v matematike má podobu ekvivalencie. Je však zvykom nevyjadrovať
v definícií symbol \enquote{$\iff$} slovami \enquote{vtedy a len vtedy}, ale
slovom \enquote{ak}. Teda napr. definíciu v slovnej podobe \enquote{funkcia sa
nazýva rastúca, ak platí ...} treba chápať ako ekvivalenciu \enquote{funkcia je
rastúca $\iff$ ...}.

\subsubsection*{Výrokové formy. Všeobecné a existenčné výroky a ich negovanie}

Výroková forma je výraz, ktorý síce nie je výrokom, ale obsahuje premenné,
ktorých vhodným nahradením vznikne výrok (napr. výroková forma $x^2 + y \geq 0
\land y < 0$ obsahuje dve také premenné). Tvorenie zložených výrokových foriem
je analogické tvoreniu zložených výrokov.

Nech $\varphi (x)$ je výroková forma, v ktorej je možno dosadzovať za jedinú
premennú $x$, a $A$ je ľubovoľná neprázdna množina obsahujúca len také prvky,
ktorých dosadením do $\varphi (x)$ vznikne výrok.

Výrok \enquote{pre všetky $x \in A$ platí $\varphi (x)$} -- symbolicky ho
zapisujeme
\[
  \forall x \in A_1: \varphi (x)
\]
sa nazýva všobecný symbol $\forall$ zastupujúci slová \enquote{pre všetky} je
tzv. všeobecný (alebo veľký) kvantifikátor. (Častá slovná podoba všeobecného
výroku $\forall x \in A : \varphi (x)$ je \enquote{nech $x \in A$, potom
(platí) $\varphi (x)$} alebo \enquote{ak $x \in A$, tak (platí) $\varphi
(x)$}.)

Výrok \enquote{existuje $x \in A$ také, že platí $\varphi (x)$} -- symbolicky
\[
  \exists x \in A : \varphi (x)
\]
sa nazýva existenčný výrok; symbol $\exists$ vyjadrujúci slovo
\enquote{existuje} (\enquote{pre niektoré}) je tzv. existenčný (malý)
kvantifikátor.

Pre negáciu všeobecných a existenčných kvantifikátorov platia tieto pravidlá:
\begin{center}
  \begin{tabular}{c | c}
    výrok & negácia \\
    \hline
    $\forall x \in A : \varphi (x)$ & $\exists x \in A : \neg \varphi (x)$ \\
    $\exists x \in A : \varphi (x)$ & $\forall x \in A : \neg \varphi (x)$ \\
  \end{tabular}
\end{center}

Ak uplatníme všeobecný (alebo existenčný) kvantifikátor na premennú $x_1$ vo
výrokovej forme $\varphi(x_1, x_2, \ldots, x_n)$ -- v ktorej možno dosadzovať za
$n$ premenných $x_1, \ldots, x_n$ kde $(n \geq 2)$ -- dostaneme výrokovú formu
$\forall x_1 \in A: \varphi (x_1, \ldots, x_n) (\exists x_1 \in A: \varphi
(x_1, \ldots, x_n))$, v ktorej možno dosadzovať už len za $n - 1$ premenných
$x_2, \ldots, x_n$ za premennú viazanú kvantifikátorom už dosadzovať nemožno,
predpokladáme samozrejme, že $A \neq B$ a že pre každé $a \in A$ je výraz
$\varphi (a, x_2, \ldots, x_n)$ opäť výrokovou formou). Na premenné $x_2$ opäť
možeme uplatniť kvantifikátor atď. Ak v poradí posledný použitý -- t.j. pri
zápise prvý -- kvantifikátor je veľký (malý), dostaneme takýmto postupom
všeobecný (existenčný) výrok.

Dôkaz ekvivalencie $A \iff B$ spravidla prevádzame na dôkaz dvoch implikácií
$A \Rightarrow B$ a $B \Rightarrow A$

\subsubsection*{Podrobnejšie o dôkazoch všeobecných výrokov:}
Ako príklad priameho dôkazu všobecného výroku nám poslúži dôkaz tvrdenia
\enquote{ak celé číslo $n$ nie je deliteľné 3, tak $n^2 - 1$ je deliteľné 3}
(symbolicky $\forall n \in \mathbb{Z}: (3 \nmid n \Rightarrow 3 \mid
(n^2 - 1))$), ktorý prebieha nasledovne: ak $n$ nie je deliteľné 3, tak
$n - 1$ alebo $n + 1$ je deliteľné 3, teda $n^2 - 1$ je deliteľné 3. Symbolicky
možno tento dôkaz zapísať takto:
\[
3 \nmid n
\Rightarrow (3 \mid (n - 1) \lor 3 \mid (n + 1))
\Rightarrow 3 \mid (n^2 - 1)
\]

Keby sme v uvedených úvahách namiesto $n$ písali napr. číslo 5, dostali by sme
dôkaz implikácie $3 \nmid 5 \Rightarrow 3 \mid (5^2 - 1)$. To je pre tento typ
dôkazu charakteristické; priamy dôkaz výroku $\forall x \in A: \varphi (x)$ je
vlastne schémou, z ktorej možno dostať dôkaz každého jednotlivého výroku
$\varphi (a)$, ak v nej za $x$ dosadíme prvok $a \in A$.

Zvláštnym typom priameho dôkazu všeobecného výroku je dôkaz matematickou
indukciou. Výroky v tvare $\forall n \in N: \varphi (n)$ sa spravidla nedokazujú
bezprostredne, namiesto toho stačí dokázať výroky
\begin{enumerate}[label=(\Alph*)]
  \item platí $\varphi (1)$
  \item $\forall n \in \mathbb{N}: \varphi (n) \Rightarrow \varphi (n + 1)$
\end{enumerate}
(výrok $\varphi (n)$ v implikácií $\varphi (n) \Rightarrow \varphi (n + 1)$ sa
nazýva indukčný predpoklad).Z (A) a (B) možno totiž reťazcom úsudkov odvodiť
ľubovoľný jednotlivý výrok $\varphi (n)$ takto:

platí $\varphi (1)$

\underline{platí $\varphi (1) \Rightarrow \varphi (2)$}

\hspace{1em} teda platí $\varphi (2)$;

\hspace{3.2em} \underline{platí $\varphi(2) \Rightarrow \varphi(3)$}

\hspace{4.4em} teda platí $\varphi(3)$;

\hspace{5cm} $\ldots$

\hspace{6cm} platí $\varphi (n - 1)$

\hspace{6cm} \underline{platí $\varphi (n - 1) \Rightarrow \varphi (n)$}

\hspace{7.2cm} teda platí $\varphi (n)$

Takýmto spôsobom možno dokazovať aj tvrdenia platné pre všetky prirodzené čísla
väčšie než dané číslo $p$, pre všetky celé čísla, pre všetky párne čísla a pod.

Pri nepriamom dôkaze všeobecného výroku odvodzujeme z jeho negácie nepravdivé
tvrdenie.

\medskip

\subsubsection*{Dôkazy existenčných výrokov}
Výrok $\exists ~ x \in A: \varphi (x)$
môžeme dokázať tak, že priamo určíme (skonštruujeme) prvok $a \in A$, pre ktorý
je výrok $\varphi (a)$ pravdivý. Taký dôkaz sa nazýva konštruktívny. Často však
odvodzujeme existenčný výrok z iných existenčných viet bez toho, že by sme
skutočne určili prvok s požadovanou vlastnosťou.

Existenčné výroky možno dokazovať aj nepriamo, teda odvodením nepravdivého
tvrdenia z ich negácie.

\begin{comment}
\begin{enumerate}
  \item Určte negácie následujúcich výrokov:

  \item Namiesto bodiek doplňte slová \enquote{je nutné}, \enquote{stačí},
        \enquote{je nutné a stačí} tak, aby vznikol pravdivý výrok:
  \begin{enumerate}[label=\arabic*.]
    \item aby súčet dvoch celých čísel bol deliteľný 2, $\ldots$ , aby každý
          sčítanec bol deliteľný 2;
    \item aby celé číslo bolo deliteľné 100, $\ldots$ , aby bolo deliteľné 10;
    \item aby celé číslo bolo deliteľné 100, $\ldots$ , aby bolo deliteľné
          1000;
    \item aby platila nerovnosť $\frac{1}{x} < 1, \ldots$ , aby bolo $x > 1$;
    \item aby platilo $\frac{1}{x} < 1, \ldots$ , aby bolo $x > 1$ alebo
          $x < 0$;
  \end{enumerate}
\end{enumerate}
\end{comment}

\begin{enumerate}
  \item \useproblem[uvod]{uvod-1}
  \item \useproblem[uvod]{uvod-2}
\end{enumerate}

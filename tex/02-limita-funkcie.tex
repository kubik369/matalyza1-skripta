\chapter{Limita funkcie}%\label{chapter:gramatiky}

\section{Okolia a hromadné body}
Nech $a \in \mathbb{R};$ každý interval $(a-\varepsilon,a+\varepsilon)$, kde $\varepsilon > 0$, sa nazýva \textit{okolie bodu $a$}. Číslo $\varepsilon$ sa nazýva \textit{polomer okolia} $(a-\varepsilon,a+\varepsilon);$ ak chceme zdôrazniť, že dané okolie bodu $a$ má polomer $\varepsilon$, hovoríme o $\varepsilon-$okolí bodu $a$. Okolím bodu $\infty$ sa nazýva každý interval $(K,\infty)$, kde $K \in \mathbb{R};$ okolím bodu $- \infty$ je každý interval $(- \infty,K)$, kde $K \in \mathbb{R}.$ Okolie bodu $b \in \mathbb{R^*}$ ($\mathbb{R^*}$ sa nazýva rozšírená množina reálnych čísel a pozostáva zo všetkých reálnych čísel a symbolov $+\infty,-\infty$)\footnote{namiesto symbolu $+\infty$ sa často používa symbol $\infty$, niektorí autori však zavádzajú symbol $\infty$ s iným významom, k tomu pozri poznámku na konci odseku $2.5$} budeme označovať $O(b)$, symbol $O_\varepsilon (b)$ budeme používať pre $\varepsilon-$okolie bodu $b \in \mathbb{R}$.

Bod $a \in \mathbb{R^*}$ sa nazýva hromadný bod množiny $M \subset \mathbb{R}$, ak každé jeho okolie $O(a)$ obsahuje aspoň jeden prvok množiny $M$, rôzny od bodu $a$, t.j. ak platí 
$$\forall O(a): (O(a) \setminus \{ a \}) \cap M \neq \emptyset $$
(tento výrok možno zapísať aj v tvare $\forall O(a) \exists x \in M; x \neq a \wedge x \in O(a);$ je ekvivalentný s výrokom: každé okolie bodu $a$ obsahuje nekonečne veľa prvkov množiny $M$).

Množinu všetkých hromadných bodov množiny $M$ budeme označovať $M'$.

\begin{enumerate}[resume]
	\item \useproblem[limita-funkcie]{limita-1}
	\item \useproblem[limita-funkcie]{limita-2}
	\item \useproblem[limita-funkcie]{limita-3}
	\item \useproblem[limita-funkcie]{limita-4}
	\item \useproblem[limita-funkcie]{limita-5}
\end{enumerate}

\section{Definícia limity}

Nech $a \in \mathbb{R^*}$ je hromadný bod definičného oboru funkcie $f$. Bod $b \in \mathbb{R^*}$ sa nazýva \textit{limita funkcie $f$ v bode $a$} ak platí
$$\forall O(b) \exists O(a) \forall x \in (O(a)\setminus \{ a \}) ) \cap D(f): f(x) \in O(b).$$
(Ak $b \in \mathbb{R}$, hovoríme o \textit{vlastnej (alebo konečnej) limite}; ak $b=\infty$ alebo $b=-\infty$, o \textit{nevlastnej limite}; ak $a=\infty$ alebo $a=-\infty$, používane názov \textit{limita v nevlastnom bode}.) Zapisujeme $\lim_{x \rightarrow a} f(x)=b$ alebo $f(x) \rightarrow b$ pre $x \rightarrow a$.

Všimnime si teraz jednotlivé prípady, ktoré zahŕňa uvedená definícia; začneme postupnosťami:
\begin{itemize}
\item Nech $b \in \mathbb{R}$; hovoríme, že postupnosť ${\{a_n\}}_{n=1}^\infty$ konverguje k (číslu) $b$, ak $\lim_{n \rightarrow \infty} a_n=b$, t.j. ak platí 
$$\forall \varepsilon > 0 \exists n_0 \in \mathbb{N} \forall n \in \mathbb{N},n>n_0: |a_n-b|<\varepsilon.\footnote{Výroky $$\forall \varepsilon >0 \exists n_0 \in \mathbb{N} \forall n \in \mathbb{N}, n>n_0:|a_N-b<\varepsilon|$$ a $$\forall \varepsilon >0 \exists n_0 \in \mathbb{R} \forall n \in \mathbb{N}, n>n_0:|a_N-b<\varepsilon|$$ sú ekvivalentné. Pretože v definícii konečnej limity postupností je zvykom žiadať $n_0 \in \mathbb{N}$ (a nie $n_0 \in \mathbb{R}$), budeme ju v takej podobe používať aj my (hoci - ako uvidíme v príklade $105$ - sa tým niekedy komplikuje vyjadrenie závislosti čísla $n_0$ na čísle $\varepsilon$). Analogická poznámka sa vzťahuje aj na definície nevlastných limít postupností.}$$
\item Hovoríme, že postupnosť ${\{a_n\}}_{n=1}^\infty$ diverguje k $+\infty$, ak $\lim_{n \rightarrow \infty} a_N = +\infty$, t.j. ak platí $$\forall K \in \mathbb{R} \exists n_0 \in \mathbb{N} \forall n \in \mathbb{N}, n>n_0: a_n>K.$$
\item  Hovoríme, že postupnosť ${\{a_n\}}_{n=1}^\infty$ diverguje k $-\infty$, ak $\lim_{n \rightarrow \infty} a_N = -\infty$, t.j. ak platí $$\forall K \in \mathbb{R} \exists n_0 \in \mathbb{N} \forall n \in \mathbb{N}, n>n_0: a_n<K.$$
Postupnosti, ktoré majú vlastnú limitu, sa nazývajú \textit{konvergentné}; ak postupnosť nemá limitu alebo má nevlastnú limitu, nazýva sa \textit{divergentná}.
\end{itemize}

\begin{enumerate}[resume]
	\item \useproblem[limita-funkcie]{limita-6}
\end{enumerate}

\textit{Riešenie (b):} 

Musíme dokázať pravdivosť výroku (*)
 $$\forall \varepsilon > 0 \exists n_0 \in \mathbb{N} \forall n \in \mathbb{N}, n>n_0: |\frac{n^2+3n+1}{n^2+2}-\frac{1}{2}|<\varepsilon$$
 Nech je teda dané číslo $\varepsilon > 0$; zistíme, ktoré čísla $n \in \mathbb{N}$ vyhovujú nerovnici $$|\frac{n^2+3n+1}{n^2+2}-\frac{1}{2}|<\varepsilon.$$
 Postupnými úpravami dostaneme ekvivalentné vzťahy v $\mathbb{N}:$
 $$|\frac{3n}{2n^2+2}|<\varepsilon,$$
 $$2 \varepsilon n^2-3n+2 \varepsilon >0.$$
 (Nájdeme najprv všetky reálne rešenia poslednej nerovnice, z nich potom vyberieme tie, ktoré ležia v $\mathbb{N}$. Diskriminant $D$ je rovný $9-16 \varepsilon^2$, preto pre reálne riešenia platí:
\begin{itemize}
\item ak $\varepsilon > \frac{3}{4}$, t.j. ak $D<0$, je riešením každé reálne číslo;
\item ak $\varepsilon \in (0,\frac{3}{4} \rangle$, je $D \geq 0$, preto riešeniami sú všetky prvky množiny $(-\infty,\frac{3-\sqrt{9-16\varepsilon^2}}{2}) \cup (\frac{3+\sqrt{9-16\varepsilon^2}}{2},\infty)$.)
\end{itemize} 
Preto:
\begin{itemize}
\item ak $\varepsilon > \frac{3}{4}$, je riešením nerovnice $|\frac{n^2+3n+1}{n^2+2}-\frac{1}{2}|<\varepsilon$ každé číslo $n \in \mathbb{N}$;
\item ak $\varepsilon \in (0,\frac{3}{4} \rangle$, sú riešeniami nerovnice $|\frac{n^2+3n+1}{n^2+2}-\frac{1}{2}|<\varepsilon$ všetky tie $n \in \mathbb{N}$, pre ktoré platí $n>\frac{3+\sqrt{9-16\varepsilon^2}}{2\varepsilon}$.
Teraz už vidíme, že daný výrok (*) je pravdivý: stačí položiť $n_0=1$ pre $\varepsilon>\frac{3}{4}$ a $n_0=[\frac{3+\sqrt{9-16\varepsilon^2}}{2\varepsilon}]$ pre $\varepsilon \in (0,\frac{3}{4} \rangle$ ([.] označujú celú časť; keby sme v (a) mali podmienku $n_0 \in \mathbb{R}$, stačilo by pre $\varepsilon \in (0,\frac{3}{4} \rangle$ položiť $n_0=\frac{3+\sqrt{9-16\varepsilon^2}}{2\varepsilon}$). 
\end{itemize}

\textit{Poznámka:}

Ak je nerovnosť $|a_n-b|<\varepsilon$ splnená pre všetky $n>n_0$ a platí $n_1>n_0$, tak nerovnosti $|a_n-b|<\varepsilon$ iste vyhovujú všetky čísla $n>n_1$. Z tohto samozrejmého tvrdenia vyplýva, že v závere riešenia príkladu $105 (b)$ by stačilo položiť $n_0>1$ pre $\varepsilon > \frac{3}{4}$ a $n_0 \geq [\frac{3+\sqrt{9-16\varepsilon^2}}{2\varepsilon}]$ pre $\varepsilon \in (0,\frac{3}{4}\rangle$.

\begin{enumerate}[resume]
	\item \useproblem[limita-funkcie]{limita-7}
	\item \useproblem[limita-funkcie]{limita-8}
	\item \useproblem[limita-funkcie]{limita-9}
	\item \useproblem[limita-funkcie]{limita-10}
	\item \useproblem[limita-funkcie]{limita-11}
\end{enumerate}

Ak $x \in \mathbb{R^*}$, nastáva práve jedna z troch možností:
\begin{itemize}
\item $x \in \mathbb{R}$,
\item $x = +\infty$,
\item $x = -\infty$.
\end{itemize}

Ak v definícii limity funkcie rozlíšime pre body $a,b \in \mathbb{R^*}$ tieto možnosti, dostaneme nasledujúcich deväť špeciálnych prípadov (v nich už a,b označujú len reálne čísla):


\begin{multicols}{3}
\begin{enumerate}
    \item $\lim_{x \rightarrow a} f(x)=b$;
    \item $\lim_{x \rightarrow a} f(x)=+\infty$;
    \item $\lim_{x \rightarrow a} f(x)=-\infty$;
    \item $\lim_{x \rightarrow \infty} f(x)=b$;
    \item $\lim_{x \rightarrow \infty} f(x)=+\infty$;
    \item $\lim_{x \rightarrow \infty} f(x)=-\infty$;
    \item $\lim_{x \rightarrow -\infty} f(x)=b$;
    \item $\lim_{x \rightarrow -\infty} f(x)=+\infty$;
    \item $\lim_{x \rightarrow -\infty} f(x)=-\infty$.
\end{enumerate}
\end{multicols}

\begin{enumerate}[resume]
	\item \useproblem[limita-funkcie]{limita-12}
\end{enumerate}

\textit{Riešenie $1$:}
V tomto prípade sú okolia $O(a)$ a $O(b)$ jednoznačne určené svojimi polomermi $\delta,\epsilon$; definíciu limity možno potom prepísať do tvaru 
$$\forall \epsilon > 0 \exists \delta > 0 \forall x \in D(f), x \neq a, |x-a|<\delta:|f(x)-b|<\epsilon$$
alebo - čo je to isté - do tvaru
$$\forall \epsilon > 0 \exists \delta > 0 \forall x \in D(f), x \neq a, |x-a|<\delta \Rightarrow |f(x)-b|<\epsilon$$
(Samozrejme predpokladáme, že $a$ je hromadný bod množiny $D(f)$.)

\begin{veta}
\textbf{Cauchyho-Bolzanovo kritérium konvergencie}

Nech $a \in \mathbb{R^*}$ je hromadný bod definičného oboru funkcie $f$. Funkcia $f$ má v bode $a$ konečnú limitu práve vtedy, keď platí 
$$\forall \varepsilon > 0 \exists O(a) \forall x,y \in (O(a) \setminus \{ a \}) \cap D(f): |f(x)-f(y)|<\varepsilon.$$
\end{veta}

\begin{enumerate}[resume]
	\item \useproblem[limita-funkcie]{limita-13}
\end{enumerate}

\textit{Riešenie 1:}
Treba dokázať pravdivosť tvrdenia (*)
$$\forall \varepsilon > 0 \exists \delta > 0 \forall x \geq 0, x \neq 3, |x-3|< \delta : |\sqrt{x}-\sqrt{3}|< \varepsilon.$$
Predpokladajme, že $|x-3|<\delta$ a skúsme na základe toho zhora odhadnúť výraz $|\sqrt{x}-\sqrt{3}|$. Pretože $|\sqrt{x}-\sqrt{3}|=\frac{|x-3|}{\sqrt{x}+\sqrt{3}}$ a $\sqrt{x}+\sqrt{3} \geq \sqrt{3}$ platí $\sqrt{x}-\sqrt{3} \leq \frac{1}{\sqrt{3}}|x-3|<\frac{\delta}{\sqrt{3}}$. Teraz už vidíme, že tvrdenie (*) platí: ak je dané $\varepsilon > 0$ a chceme, aby platilo $|\sqrt{x}-\sqrt{3}|<\varepsilon$, stačí položiť $\delta=\varepsilon \sqrt{3}$ (alebo $\delta \leq \varepsilon \sqrt{3}$).

\begin{enumerate}[resume]
	\item \useproblem[limita-funkcie]{limita-14}
	\item \useproblem[limita-funkcie]{limita-15}
	\item \useproblem[limita-funkcie]{limita-16}
\end{enumerate}

\section{Vety o limitách I}
\begin{veta}
\textbf{o limite skalárneho násobku, súčtu, rozdielu, súčinu a podielu}

Nech sú dané funkcie $f,g$, nech $a \in \mathbb{R^*}$ je hromadný bod množiny $D(f) \cap D(g)$. Ak existujú konečné $\lim_{x \rightarrow a} f(x)=A$, $\lim_{x \rightarrow a} g(x)=B$, tak existuje aj $\lim_{x \rightarrow a} cf(x)$ ($c \in \mathbb{R}$ je konštanta), $\lim_{x \rightarrow a} (f(x)+g(x)),\lim_{x \rightarrow a} (f(x)-g(x)),\lim_{x \rightarrow a} (f(x) \cdot g(x))$ a platí 
\end{veta}

\begin{multicols}{2}
\begin{enumerate}[label=]
    \item $\lim_{x \rightarrow a} cf(x)=cA$;
    \item $\lim_{x \rightarrow a} (f(x)+g(x))=A+B$;
    \item $\lim_{x \rightarrow a} (f(x)-g(x))=A-B$;
    \item $\lim_{x \rightarrow a} (f(x) \cdot g(x))=A \cdot B$;
    \item $(=c \cdot \lim_{x \rightarrow a} f(x))$;
    \item $(=\lim_{x \rightarrow a} f(x)+\lim_{x \rightarrow a} g(x))$;
    \item $(=\lim_{x \rightarrow a} f(x)-\lim_{x \rightarrow a} g(x))$;
    \item $(=\lim_{x \rightarrow a} f(x) \cdot \lim_{x \rightarrow a} g(x))$.
\end{enumerate}
\end{multicols}
Ak naviac $B \neq 0$, tak existuje aj $\lim_{x \rightarrow a} \frac{f(x)}{g(x)}$ a platí 

\begin{multicols}{2}
\begin{enumerate}[label=]
\item $\lim_{x \rightarrow a} \frac{f(x)}{g(x)}=\frac{A}{B}$
\item $(=\frac{\lim_{x \rightarrow a} f(x)}{\lim_{x \rightarrow a} g(x)})$.
\end{enumerate}
\end{multicols}

\begin{veta}
\textbf{o limite zloženej funkcie}

Nech sú dané funkcie $f,g$ nech $a \in \mathbb{R^*}$ je hromadný bod množiny $D(f \circ g)$. Ak $\lim_{x \rightarrow a} g(x)=A$ $(A \in \mathbb{R^*})$, pričom je splnená podmienka 
$$\exists O(a)\forall x \in O(a): x \neq a \Rightarrow g(x) \neq A,$$
a $\lim_{x \rightarrow A} f(x)=B$  $(B \in \mathbb{R^*})$, tak $\lim_{x \rightarrow a} f(g(x))=B$
\end{veta}

\textit{Poznámka:}
Ak $A$ je hromadným bodom $D(f)$, ale $A \notin D(f)$, tak uvedená veta platí aj vtedy, keď nie je splnená podmienka (*).


Niektoré limity možno nájsť len na základe definície, v ostatných prípadoch je však oveľa efektívnejšie použiť vety o limitách. Pritom je potrebné osvojiť si zdôvodňovanie jednotlivých krokov výpočtu, inak sa nenaučíme odlišovať správne postupy od nesprávnych. Na ilustráciu podrobne popíšme nasledujúci výpočet

\begin{center}
$$\lim_{x \rightarrow \infty} \frac{2x^2+3x+5}{3x^2-7}=\lim_{x \rightarrow \infty} \frac{2+\frac{3}{x}+\frac{5}{x^2}}{3-\frac{7}{x^2}}=\frac{\lim_{x \rightarrow \infty} (2+\frac{3}{x}+\frac{5}{x^2})}{\lim_{x \rightarrow \infty} (3-\frac{7}{x^2})}=$$
$$=\frac{\lim_{x \rightarrow \infty} 2+\lim_{x \rightarrow \infty}\frac{3}{x}+\lim_{x \rightarrow \infty} \frac{5}{x^2}}{\lim_{x \rightarrow \infty}3-\lim_{x \rightarrow \infty}\frac{7}{x^2}}=\frac{2+0+0}{3-0}=\frac{2}{3}.$$
\end{center}

\textit{1. krok:}

Na intervale $(0,\infty)$ iste platí $\frac{2x^2+3x+5}{3x^2-7}=\frac{2+\frac{3}{x}+\frac{5}{x^2}}{3-\frac{7}{x^2}}$ (zlomok vľavo stačí rozšíriť výrazom $\frac{1}{x^2}$), preto: ak existuje limita na pravej strane prvej rovnosti, tak existuje aj limita na jej ľavej strane a tieto limity sa rovnajú \footnote{Táto elementárna, ale veľmi častá úvaha sa vo všeobecnosti formuluje takto: Nech $a \in \mathbb{R^*}$ je hromadný bod $D(f)$, nech existuje $O(a)$ tak, že $D(f) \cap (O(a) \setminus \{ a\})=D(g) \cap (O(a) \setminus \{ a\})$ a úre všetky $x \in D(f) \cap (O(a) \setminus \{ a\})$ platí $f(x)=g(x)$. Ak existuje $\lim_{x \rightarrow a} g(x)=b$ $(\in \mathbb{R^*})$, tak platí $\lim_{x \rightarrow a} f(x)=b$.}; ďalej sa teda snažíme zistiť, či existuje $\lim_{x \rightarrow \infty}\frac{2+\frac{3}{x}+\frac{5}{x^2}}{3-\frac{7}{x^2}}$;

\textit{2. krok:}

tu sme použili vetu o limite podielu (zatiaľ ovšem len "na čestné slovo"); presnejšie povedané: ak ukážeme, že limita v čitateli aj v menovateli existujú a sú konečné, pričom limita v menovateli je nenulová, tak podľa vety o limite podielu bude platiť druhá rovnosť;

\textit{3. krok:}

v čitateli sme použili vetu o limite súčtu (tú možno indukciou rozšítiť na ľubovoľný konečný počet sčítancov), v menovateli vetu o limite rozdielu (obidve zatiaľ tiež len "na čestné slovo");

\textit{4. krok:}

teraz už ľahko overíme, že druhá a tretia rovnosť skutočne platia, pretože $\lim_{x \rightarrow \infty} 2$, $\lim_{x \rightarrow \infty} \frac{3}{x}$, $\lim_{x \rightarrow \infty} \frac{5}{x^2}$, $\lim_{x \rightarrow \infty} 3$ a $\lim_{x \rightarrow \infty} \frac{7}{x^2}$ existujú a sú konečné (to ľahko dokážeme priamo z definície), bolo použitie viet o limite súčtu a rozdielu v tretej rovnosti oprávnené (a preto $\lim_{x \rightarrow \infty} (2+\frac{3}{x}+\frac{5}{x^2})=2$, $\lim_{x \rightarrow \infty} (3-\frac{7}{x^2})=3$); rovnako oprávnené bolo použitie vety o limite podielu v druhom kroku (limita v čitateli aj v menovateli - ako sme sa práve presvedčili - skutočne existujú, sú konečné a limita v menovateli je naviac nenulová).

Z platnosti prvej, druhej a tretej rovnosti vyplýva $\lim_{x \rightarrow \infty} \frac{2x^2+3x+5}{3x^2-7}=\frac{2}{3}$.

Takéto zdôvodnenie (vykonané ovšem len v duchu alebo ústne) by malo byť súčasťou výpočtu každej limity; po získaní istej prace budú zápisy aj argumentácia podstatne stručnejšie

\begin{veta}
Ak $f$ je elementárna funkcia a bod $a \in D(f)$ je hromadný bod množiny $D(f)$ je hromadný bod množiny $D(f)$, tak $\lim_{x \rightarrow a} f(x)=f(a)$.
\end{veta}

(Toto tvrdenie veľmi úzko súvisí s pojmom spojitosti (pozri kap. 3).)

\begin{enumerate}[resume]
	\item \useproblem[limita-funkcie]{limita-17}
	\item \useproblem[limita-funkcie]{limita-18}
\end{enumerate}

\textit{Riešenie 1:}
Funkcie $P(x)=x^2-5x+6$ a $Q(x)=x^3-6x^2+10x-3$ sú elementárne, preto $\lim_{x \rightarrow 3} P(x)=P(3)=0$, $\lim_{x \rightarrow 3} Q(x)=Q(3)=0$. Pretože $\lim_{x \rightarrow 3} Q(x)=0$, nemôžeme použiť vetu o limite podielu. (Alebo inak povedané: funkcia $R=P \setminus Q$ je elementárna, ale $Q(3)=0$, preto $3 \notin D(R)$ a limitu funkcie $R$ v bode $3$ teda nemožno nájsť "dosadením".) Z rovností $P(3)=0$, $Q(3)=0$ vyplýva, že číslo $3$ teda polynómov $P$ aj $Q$, preto $P(x)$ aj $Q(x)$ musia byť deliteľné koreňovým činiteľom $(x-3)$. Po vyňatí člena $(x-3)$ dostaneme $P(x)=(x-3)(x-2)$, $Q(x)=(x-3)(x^2-3x+1)$. Pre $x \in D(R)$ teda platí $\frac{x^2-5x+6}{x^3-6x^2+10x-3}=\frac{(x-3)(x-2)}{(x-3)(x^2-3x+1)}=\frac{x-2}{x^2-3x+1}$, pritom $\lim_{x \rightarrow 3} \frac{x-2}{x^2-3x+1}=1$ (elementárna funkcia $R_1(x)=\frac{x-2}{x^2-3x+1}=1$ je definocaná aj v bode $3$, preto $\lim_{x \rightarrow 3} R_1(x)$ už možno nájsť "dosadením").

Teda $$\lim_{x \rightarrow 3} \frac{x^2-5x+6}{x^3-6x^2+10x-3}=\lim_{x \rightarrow 3} \frac{(x-3)(x-2)}{(x-3)(x^2-3x+1)}=\lim_{x \rightarrow 3} \frac{x-2}{x^2-3x+1}=1.$$

\begin{enumerate}[resume]
	\item \useproblem[limita-funkcie]{limita-19}
	\item \useproblem[limita-funkcie]{limita-20}
	\item \useproblem[limita-funkcie]{limita-21}
\end{enumerate}

\textit{Riešenie (a):}
Elementárna funkcia $f(x)=\frac{\sqrt[3]{x-6}+2}{\sqrt{x^2-3}-1}$ nie je definovaná v bode $-2$, preto $\lim_{x \rightarrow -2} f(x)$ nemožno nájsť "dosadením". Pre výpočet limity bude výhodnejší iný zápis predpisu funkcie $f$; dosadíme ho použitím vzorca $A^2-B^2=(A-B)(A+B),A^3+B^3=(A+B)(A^2-AB+B^2)$ (obidva sú špeciálnym prípadom rovnost $A^n-B^n=(A-B)(A^{n-1}+A^{n-2}B+...+AB^{n-2}+B^{n-1})$ ). Podľa prvého z nich $x^2-4=(\sqrt{x^2-3}-1)(\sqrt{x^2-3}+1)$, podľa druhého $x+2=(\sqrt[3]{x-6}+2)(\sqrt[3]{(x-6)^2}-2\sqrt[3]{x-6}+4)$.

Pre všetky $x \in D(f)$ preto platí $f(x)=\frac{\sqrt[3]{x-6}+2}{\sqrt{x^2-3}-1}=\frac{x+2}{x^2-4}\cdot\frac{\sqrt{x^2-3}+1}{\sqrt[3]{(x-6)^2}-2\sqrt[3]{x-6}+4}$ (zlomok sme rozšírili výrazom $(\sqrt{x^2-3}+1)(\sqrt[3]{(x-6)^2}-2\sqrt[3]{x-6}+4)$). Teraz môžeme použiť vetu o limite súčinu: $\lim_{x \rightarrow -2} \frac{\sqrt{x^2-3}+1}{\sqrt[3]{(x-6)^2}-2\sqrt[3]{x-6}+4}=\frac{1}{6}$ (ide o elementárnu funkciu definovanú v bode $-2$, preto stačí dosadiť), $\lim_{x \rightarrow -2} \frac{x+2}{x^2-4}=lim_{x \rightarrow -2} \frac{1}{x-2}=-\frac{1}{4}$ . Teda $lim_{x \rightarrow -2} \frac{\sqrt[3]{x-6}+2}{\sqrt{x^2-3}-1}=lim_{x \rightarrow -2}  \frac{x+2}{x^2-4} \cdot \frac{\sqrt{x^2-3}+1}{\sqrt[3]{(x-6)^2}-2\sqrt[3]{x-6}+4}$    $(=lim_{x \rightarrow -2} \frac{x+2}{(x+2)(x-2)}\cdot lim_{x \rightarrow -2} \frac{\sqrt{x^2-3}+1}{\sqrt[3]{(x-6)^2}-2\sqrt[3]{x-6}+4})=(lim_{x \rightarrow -2} \frac{1}{x-2})\cdot \frac{1}{6}=-\frac{1}{24}$.

\begin{enumerate}[resume]
	\item \useproblem[limita-funkcie]{limita-22}
\end{enumerate}

\textit{Riešenie (a):}
Uvedieme dva rôzne návody:
\begin{itemize}
\item $\lim_{x \rightarrow 2} \frac{\sqrt{x+2}-\sqrt[3]{x^2+4}}{x-2}=\lim_{x \rightarrow 2} \frac{\sqrt[6]{(x+2)^3}-\sqrt[6]{(x^2+4)^3}}{x-2}$
\item $\lim_{x \rightarrow 2} \frac{\sqrt{x+2}-\sqrt[3]{x^2+4}}{x-2}=\lim_{x \rightarrow 2} \frac{\sqrt{x+2}-2+2-\sqrt[3]{x^2+4}}{x-2}=\lim_{x \rightarrow 2} \frac{\sqrt{x+2}-2}{x-2}+\lim_{x \rightarrow 2} \frac{2-\sqrt[3]{x^2+4}}{x-2}$
\end{itemize}

(číslo $2$, ktoré sme pripočítali a odpočítali, je spoločnou funkcnou hodnotou funkcií $\sqrt{x+2}$ a $\sqrt[3]{x^2+4}$ v bode $2$); ďalší postup je potom rovnaký ako v príklade $120$.

\begin{enumerate}[resume]
	\item \useproblem[limita-funkcie]{limita-23}
\end{enumerate}

\textit{Riešenie (b):}
Aby sme sa v limitovanom výraze zbavili odmocniny, položíme $\sqrt[n]{1+x}=t$ (odtiaľ $(x=t^n-1)$). Použiť túto substitúciu neznamená nič iné, ako napísať funkciu $y=\frac{\sqrt[n]{1+x}-1}{x}$ v tvare superpozície funkcií $y=\frac{t-1}{t^n-1}$ a $t=\sqrt[n]{1+x}$. Výpočet limity sa potom zakladá na vete o limite zloženej funkcie: limita vnútornej zložky (predstavujúcej substitúciu) je $(\lim_{x \rightarrow 0} \sqrt[n]{1+x}=)1$; podmienka (*) a vety $3$ je splnená, pretože $\sqrt[n]{1+x}$ je prostá funkcia. Hľadaná limita sa preto rovná limite vonkajšej zložky v bode $1$; teda 
$$\lim_{x \rightarrow 0} \frac{\sqrt[n]{1+x}-1}{x}=\lim_{t \rightarrow 1} \frac{t-1}{t^n-1}=\lim_{t \rightarrow 1} \frac{t-1}{(t-1)(t^{n-1}+t^{n-2}+...+1)}=\frac{1}{n}.$$

\begin{enumerate}[resume]
	\item \useproblem[limita-funkcie]{limita-24}
	\item \useproblem[limita-funkcie]{limita-25}
	\item \useproblem[limita-funkcie]{limita-26}
\end{enumerate}

\begin{veta}
$\lim_{x \rightarrow 0} \frac{sin x}{x}=1$.
\end{veta}

\begin{enumerate}[resume]
	\item \useproblem[limita-funkcie]{limita-27}
	\item \useproblem[limita-funkcie]{limita-28}
	\item \useproblem[limita-funkcie]{limita-29}
	\item \useproblem[limita-funkcie]{limita-30}
\end{enumerate}

\textit{Riešenie (d):}
Takéto limity sa pohodlnejšie počítajú v bode $0$, použijeme preto substitúciu $x-\frac{\pi}{3}=t$ (táto funkcia je prostá, podmienka (*) z vety o limite zloženej funkcie je teda splnená) a dostaneme
$\lim_{x \rightarrow \frac{\pi}{3}} \frac{sin (x-\frac{\pi}{3})}{1-2 cos x}=\lim_{t \rightarrow 0} \frac{sin t}{1-2 cos (t+\frac{\pi}{3})}=\lim_{t \rightarrow 0} \frac{sin t}{1-2(\frac{1}{2}cos t -\frac{\sqrt{3}}{2}sin t)}=\lim_{t \rightarrow 0} \frac{sin t}{\sqrt{3}sin t + 1 - cos t}=\lim_{t \rightarrow 0} \frac{\frac{sin t}{t}}{\sqrt{3}\frac{sin t}{t}+\frac{1-cos t}{t^2}\cdot t}=\frac{1}{\sqrt{3}}$ (pritom sme využili rovnosť $\lim_{t \rightarrow 0} \frac{1-cos t}{t^2}=\frac{1}{2}$; pozri príklad $127.1$).

\begin{enumerate}[resume]
	\item \useproblem[limita-funkcie]{limita-31}
\end{enumerate}

\section{Vety o limitách II (porovnávacie vety, vety o nevlastných limitách, jednostranné limity)}

\begin{veta}
Nech sú dané sunkcie $f,g,h$ nech $a \in \mathbb{R^*}$ je hromadný bod množiny $D(g)$ a nech pre niektoré jeho prstencové okolie \footnote{Prstencovým okolím bodu $a$ sa nazýva množiny $O(a) \setminus \{ a\}$, kde $O(a)$ je okolie bodu $a$. (Teda každé okolie bodov $+\infty,-\infty$ je súčasne aj ich prstencovým okolím.)} $O^*(a)$ platí $O^*(a) \cap D(f)=O^*(a)\cap D(g)=O^*(a) \cap D(h)=D$. Ak $\lim_{x \rightarrow a} f(x)=\lim_{x \rightarrow a} h(x)=b$  $( \in \mathbb{R})$ a pre všetky $x \in D$ je $f(x) \leq g(x) \leq h(x)$, tak existuje aj $\lim_{x \rightarrow a} g(x)$ a rovná sa $b$.
\end{veta}

\begin{veta}
Nech sú dané funkcie $f,g$, nech $a \in \mathbb{R^*}$ je hromadný bod množiny $D(f) \cap D(g)$. Ak $\lim_{x \rightarrow a} f(x)=0$ a funkcia $g$ je ohraničená v niektorom prstencovom okolí bodu $a$ (t.t. na niektorej z množín $(O(a) \setminus \{ a\})\cap D(g)$, tak $\lim_{x \rightarrow a} f(x) \cdot g(x)=0$.
\end{veta}

\begin{enumerate}[resume]
	\item \useproblem[limita-funkcie]{limita-32}
	\item \useproblem[limita-funkcie]{limita-33}
\end{enumerate}

\textit{Riešenie (a):}
Pre $n \geq 3$ platí $\frac{2}{1}\cdot \frac{2}{2}\cdot \frac{2}{3}\cdot ... \cdot \frac{2}{n} \leq 2 \cdot 1 \cdot \frac{2}{3} \cdot ... \cdot \frac{2}{3}=2 \cdot (\frac{2}{3})^{n-2}=\frac{9}{2} \cdot (\frac{2}{3})^n$. Teda pre $n \geq 3$ platí $0 \leq \frac{2^n}{n!}\leq \frac{9}{2}\cdot (\frac{2}{3})^n$, pritom $\lim_{n \rightarrow \infty} 0=\lim_{n \rightarrow \infty} \frac{9}{2} \cdot (\frac{2}{3})^n=0$ (pozri príklad $105.5$). Preto (podľa vety $6$) $\lim_{n \rightarrow \infty} \frac{2^n}{n!}=0$.

\begin{enumerate}[resume]
	\item \useproblem[limita-funkcie]{limita-34}
	\item \useproblem[limita-funkcie]{limita-35}
	\item \useproblem[limita-funkcie]{limita-36}
\end{enumerate}

\textit{Riešenie (a):}
Tvrdenie \footnote{Uvedenú rovnosť možno veľmi ľahko dokázať pomocou vety $3$ a vety $4$ $(\lim_{n \rightarrow \infty} \frac{1}{n}=0, \lim_{u \rightarrow 0} a^u=1)$. Tu uvedený postup nevyužívajúci vetu $4$ sa používa práve pri dôkaze skutočnosti, že tvrdenie vety $4$ platí pre exponenciálne funkcie} zrejme platí pre $a=1$. Predpokladajme teraz, že $a>1$, a označme $\omega (n)=\sqrt[n]{a}-1$. Potom iste $\omega (n)\geq 0$ a umocnením obidvoch strán rovnosti $\sqrt[n]{a}=1+\omega n$ na n-tú dostaneme 
$$a=1+n \omega(n)+{n \choose 2}\omega^2 (n)+...+\omega^n (n)\geq 1+n\omega (n).$$
Odtiaľ $$\omega (n) \leq \frac{a-1}{n}, n\in \mathbb{N}.$$
Teda $$0\leq \omega (n) \leq \frac{a-1}{n}, n \in \mathbb{N},$$
pritom $\lim_{n \rightarrow \infty} 0=\lim_{n \rightarrow \infty} \frac{a-1}{n}=0,$ preto (podľa vety $6$) $\lim_{n \rightarrow \infty}\omega (n)=0.$ Z rovnosti $\sqrt[n]{a}=1+\omega (n)$ potom (podľa vety o limite súčtu) vyplýva $\lim_{n \rightarrow \infty} (1+\omega (n))=1$. Zostal ešte prípad $0<a<1$; tu už bude dôkaz jednoduchý: ak $0<a<1$, tak $b=\frac{1}{a}>1$. Podľa predchádzajúceho teda $\lim_{n \rightarrow \infty} \sqrt[n]{b}=1$. Z rovnosti $\sqrt[n]{a}=\frac{1}{\sqrt[n]{b}}$ potom (podľa vety o limite podielu) vyplýva $\lim_{n \rightarrow \infty} \sqrt[n]{a}=\lim_{n \rightarrow \infty} \frac{1}{\sqrt[n]{b}}=1$.

\begin{enumerate}[resume]
	\item \useproblem[limita-funkcie]{limita-37}
\end{enumerate}

\begin{veta}
Nech sú dané funkcie $f,g$, nech $a \in \mathbb{R^*}$ je hromadný bod množiny $D(f)$ a nech pre niektoré jeho rýche okolie $O^*(a)$ a platí $O^*(a)\cap D(g)=:D$. Ak $\lim_{x \rightarrow a} g(x)=\infty$  $(-\infty)$ a pre všetky $x \in D$ platí $f(x)\geq g(x)$  $(f(x)\leq g(x))$ tak existuje aj $\lim_{x \rightarrow a} f(x)$ a rovná sa $\infty$  $(-\infty)$.
\end{veta}

\begin{veta}
Nech sú dané funkcie $f,g$, nech $a \in \mathbb{R^*}$ je hromadný bod množiny $D(f)\cap D(g)$, nech existuje $\lim_{x \rightarrow a} f(x)=:A$,  $\lim_{x \rightarrow a} g(x)=:B$. Potom
\begin{enumerate}
\item ak $A=+\infty$, $B \in \mathbb{R}$ alebo $B=+\infty$, tak $\lim_{x \rightarrow a}(f(x)+g(x))=+\infty$;
\item ak $A=+\infty$, $B >0$ alebo $B=+\infty$, tak $\lim_{x \rightarrow a}(f(x)\cdot g(x))=+\infty$.
\end{enumerate}
\end{veta}

\begin{veta}
Nech $a \in \mathbb{R^*}$ je hromadný bod definičného oboru funkcie $f$. Ak $\lim_{x \rightarrow a} f(x)=+\infty$   $(-\infty)$, tak $\lim_{\frac{1}{f(x)}=0}$.
\end{veta}

\textit{Poznámka:}
Predchádzajúce tvrdenie (spolu s ďalšími, analogickými) si ľahko zapamätáme pomocou nasledujúcich rovností (ktoré ovšem považujeme len za mnemotechnickú pomôcku):
\begin{multicols}{2}
\begin{enumerate}
    \item $A +/- \infty = +/- \infty$;
    \item $+/- \infty +/- \infty = +/- \infty$;
    \item $\frac{1}{+/- \infty} = 0$;
    \item $A \cdot (+/- \infty) = \left\{ \begin{array}{r@{\quad}c}
    +/- \infty, & $ak$ A>0 \\
    -/+ \infty, & $ak$ A<0 \\ \end{array} \right.
    $;  
    \item $\infty \cdot(+/- \infty)=+/- \infty$.
\end{enumerate}
\end{multicols}
($A$ označuje reálne číslo.)
Všimnime si, že žiadna z uvedených viet sa nevzťahuje na limity funkcií typu $+\inf\ -\infty, 0 \cdot ( +/- \infty),\frac{ +/- \infty}{ +/- \infty}, \frac{0}{0}$. Také funkcie budeme nazývať neurčitými výrazmi \footnote{neskôr tento pojem ešte zovšeobecníme (pozri poznámku pod čiarou v úvode odstavca $2.5$ a poznámku na konci toho istého odstavca)}; práve im je venovaná väčšina príkladov na výpočet limít.

\begin{veta}
Nech je daná funkcia $f$, nech $a \in \mathbb{R^*}$ he hromadný bod množiny $D:=\{ x \in D(f): f(x)\neq 0 \}$, nech $\lim_{x \rightarrow a} f(x)=0$. Ak existuje také prstencové okolie $O^*(a)$ bodu $a$, že pre všetky $x \in D \cap O^*(a)$ platí $f(x)>0$  $(f(x)<0)$, tak existuje $\lim_{x \rightarrow a}\frac{1}{f(x)}$ a rovná sa $+\infty)(-\infty)$
\end{veta}

\begin{enumerate}[resume]
	\item \useproblem[limita-funkcie]{limita-38}
	\item \useproblem[limita-funkcie]{limita-39}
	\item \useproblem[limita-funkcie]{limita-40}
	\item \useproblem[limita-funkcie]{limita-41}
	\item \useproblem[limita-funkcie]{limita-42}
\end{enumerate}

Nech je daná funkcia $f$, nech $a \mathbb{R}$ je hromadný bod množiny $D^+:=D(f)\cap (a,\infty)$ (množiny $D^-:=D(f)\cap (-\infty,a)$); označme $f'$ zúženie funkcie na množinu $D^+$ (na množinu $D^-$). Ak existuje limita funkcie $f'$ v bode $a$, nazývame ju \textit{limitou funkcie $f$ v bode $a$ sprava (zľava)} a označujeme $\lim_{x \rightarrow a+} f(x)$ $(\lim_{x \rightarrow a-} f(x))$. Pre limity sprava a zľava sa používa súhrnný názov \textit{jednostranné limity}.

\begin{veta}
Nech je daná funkcia $f$, nech $a$ je hromadný bod množín $D(f) \cap (-\infty,a)$ a $D(f) \cap (a,\infty)$. Potom $\lim_{x \rightarrow a} f(x)=\lim_{x \rightarrow -a} f(x)$; pritom $\lim_{x \rightarrow a} f(x)$ sa rovná spoločnej hodnote týchto jednostranných limít.
\end{veta}

\begin{enumerate}[resume]
	\item \useproblem[limita-funkcie]{limita-43}
	\item \useproblem[limita-funkcie]{limita-44}
	\item \useproblem[limita-funkcie]{limita-45}
	\item \useproblem[limita-funkcie]{limita-46}
	\item \useproblem[limita-funkcie]{limita-47}
\end{enumerate}

\section{Limity mocninovo-exponenciálnych funkcií}
\chapter{Limita funkcie}%\label{chapter:gramatiky}

\section{Okolia a hromadné body}
Nech $a \in \mathbb{R};$ každý interval $(a-\varepsilon,a+\varepsilon)$, kde $\varepsilon > 0$, sa nazýva \textit{okolie bodu $a$}. Číslo $\varepsilon$ sa nazýva \textit{polomer okolia} $(a-\varepsilon,a+\varepsilon);$ ak chceme zdôrazniť, že dané okolie bodu $a$ má polomer $\varepsilon$, hovoríme o $\varepsilon-$okolí bodu $a$. Okolím bodu $\infty$ sa nazýva každý interval $(K,\infty)$, kde $K \in \mathbb{R};$ okolím bodu $- \infty$ je každý interval $(- \infty,K)$, kde $K \in \mathbb{R}.$ Okolie bodu $b \in \mathbb{R^*}$ ($\mathbb{R^*}$ sa nazýva rozšírená množina reálnych čísel a pozostáva zo všetkých reálnych čísel a symbolov $+\infty,-\infty$)\footnote{namiesto symbolu $+\infty$ sa často používa symbol $\infty$, niektorí autori však zavádzajú symbol $\infty$ s iným významom, k tomu pozri poznámku na konci odseku $2.5$} budeme označovať $O(b)$, symbol $O_\varepsilon (b)$ budeme používať pre $\varepsilon-$okolie bodu $b \in \mathbb{R}$.

Bod $a \in \mathbb{R^*}$ sa nazýva hromadný bod množiny $M \subset \mathbb{R}$, ak každé jeho okolie $O(a)$ obsahuje aspoň jeden prvok množiny $M$, rôzny od bodu $a$, t.j. ak platí 
$$\forall O(a): (O(a) \setminus \{ a \}) \cap M \neq \emptyset $$
(tento výrok možno zapísať aj v tvare $\forall O(a) \exists x \in M; x \neq a \wedge x \in O(a);$ je ekvivalentný s výrokom: každé okolie bodu $a$ obsahuje nekonečne veľa prvkov množiny $M$).

Množinu všetkých hromadných bodov množiny $M$ budeme označovať $M'$.

\begin{enumerate}[resume]
	\item \useproblem[limita-funkcie]{limita-1}
	\item \useproblem[limita-funkcie]{limita-2}
	\item \useproblem[limita-funkcie]{limita-3}
	\item \useproblem[limita-funkcie]{limita-4}
	\item \useproblem[limita-funkcie]{limita-5}
\end{enumerate}

\section{Definícia limity}

Nech $a \in \mathbb{R^*}$ je hromadný bod definičného oboru funkcie $f$. Bod $b \in \mathbb{R^*}$ sa nazýva \textit{limita funkcie $f$ v bode $a$} ak platí
$$\forall O(b) \exists O(a) \forall x \in (O(a)\setminus \{ a \}) ) \cap D(f): f(x) \in O(b).$$
(Ak $b \in \mathbb{R}$, hovoríme o \textit{vlastnej (alebo konečnej) limite}; ak $b=\infty$ alebo $b=-\infty$, o \textit{nevlastnej limite}; ak $a=\infty$ alebo $a=-\infty$, používane názov \textit{limita v nevlastnom bode}.) Zapisujeme $\lim_{x \rightarrow a} f(x)=b$ alebo $f(x) \rightarrow b$ pre $x \rightarrow a$.

Všimnime si teraz jednotlivé prípady, ktoré zahŕňa uvedená definícia; začneme postupnosťami:
\begin{itemize}
\item Nech $b \in \mathbb{R}$; hovoríme, že postupnosť ${\{a_n\}}_{n=1}^\infty$ konverguje k (číslu) $b$, ak $\lim_{n \rightarrow \infty} a_n=b$, t.j. ak platí 
$$\forall \varepsilon > 0 \exists n_0 \in \mathbb{N} \forall n \in \mathbb{N},n>n_0: |a_n-b|<\varepsilon.\footnote{Výroky $$\forall \varepsilon >0 \exists n_0 \in \mathbb{N} \forall n \in \mathbb{N}, n>n_0:|a_N-b<\varepsilon|$$ a $$\forall \varepsilon >0 \exists n_0 \in \mathbb{R} \forall n \in \mathbb{N}, n>n_0:|a_N-b<\varepsilon|$$ sú ekvivalentné. Pretože v definícii konečnej limity postupností je zvykom žiadať $n_0 \in \mathbb{N}$ (a nie $n_0 \in \mathbb{R}$), budeme ju v takej podobe používať aj my (hoci - ako uvidíme v príklade $105$ - sa tým niekedy komplikuje vyjadrenie závislosti čísla $n_0$ na čísle $\varepsilon$). Analogická poznámka sa vzťahuje aj na definície nevlastných limít postupností.}$$
\item Hovoríme, že postupnosť ${\{a_n\}}_{n=1}^\infty$ diverguje k $+\infty$, ak $\lim_{n \rightarrow \infty} a_N = +\infty$, t.j. ak platí $$\forall K \in \mathbb{R} \exists n_0 \in \mathbb{N} \forall n \in \mathbb{N}, n>n_0: a_n>K.$$
\item  Hovoríme, že postupnosť ${\{a_n\}}_{n=1}^\infty$ diverguje k $-\infty$, ak $\lim_{n \rightarrow \infty} a_N = -\infty$, t.j. ak platí $$\forall K \in \mathbb{R} \exists n_0 \in \mathbb{N} \forall n \in \mathbb{N}, n>n_0: a_n<K.$$
Postupnosti, ktoré majú vlastnú limitu, sa nazývajú \textit{konvergentné}; ak postupnosť nemá limitu alebo má nevlastnú limitu, nazýva sa \textit{divergentná}.
\end{itemize}

\begin{enumerate}[resume]
	\item \useproblem[limita-funkcie]{limita-6}
\end{enumerate}

\textit{Riešenie (b):} 

Musíme dokázať pravdivosť výroku (*)
 $$\forall \varepsilon > 0 \exists n_0 \in \mathbb{N} \forall n \in \mathbb{N}, n>n_0: |\frac{n^2+3n+1}{n^2+2}-\frac{1}{2}|<\varepsilon$$
 Nech je teda dané číslo $\varepsilon > 0$; zistíme, ktoré čísla $n \in \mathbb{N}$ vyhovujú nerovnici $$|\frac{n^2+3n+1}{n^2+2}-\frac{1}{2}|<\varepsilon.$$
 Postupnými úpravami dostaneme ekvivalentné vzťahy v $\mathbb{N}:$
 $$|\frac{3n}{2n^2+2}|<\varepsilon,$$
 $$2 \varepsilon n^2-3n+2 \varepsilon >0.$$
 (Nájdeme najprv všetky reálne rešenia poslednej nerovnice, z nich potom vyberieme tie, ktoré ležia v $\mathbb{N}$. Diskriminant $D$ je rovný $9-16 \varepsilon^2$, preto pre reálne riešenia platí:
\begin{itemize}
\item ak $\varepsilon > \frac{3}{4}$, t.j. ak $D<0$, je riešením každé reálne číslo;
\item ak $\varepsilon \in (0,\frac{3}{4} \rangle$, je $D \geq 0$, preto riešeniami sú všetky prvky množiny $(-\infty,\frac{3-\sqrt{9-16\varepsilon^2}}{2}) \cup (\frac{3+\sqrt{9-16\varepsilon^2}}{2},\infty)$.)
\end{itemize} 
Preto:
\begin{itemize}
\item ak $\varepsilon > \frac{3}{4}$, je riešením nerovnice $|\frac{n^2+3n+1}{n^2+2}-\frac{1}{2}|<\varepsilon$ každé číslo $n \in \mathbb{N}$;
\item ak $\varepsilon \in (0,\frac{3}{4} \rangle$, sú riešeniami nerovnice $|\frac{n^2+3n+1}{n^2+2}-\frac{1}{2}|<\varepsilon$ všetky tie $n \in \mathbb{N}$, pre ktoré platí $n>\frac{3+\sqrt{9-16\varepsilon^2}}{2\varepsilon}$.
Teraz už vidíme, že daný výrok (*) je pravdivý: stačí položiť $n_0=1$ pre $\varepsilon>\frac{3}{4}$ a $n_0=[\frac{3+\sqrt{9-16\varepsilon^2}}{2\varepsilon}]$ pre $\varepsilon \in (0,\frac{3}{4} \rangle$ ([.] označujú celú časť; keby sme v (a) mali podmienku $n_0 \in \mathbb{R}$, stačilo by pre $\varepsilon \in (0,\frac{3}{4} \rangle$ položiť $n_0=\frac{3+\sqrt{9-16\varepsilon^2}}{2\varepsilon}$). 
\end{itemize}

\textit{Poznámka:}

Ak je nerovnosť $|a_n-b|<\varepsilon$ splnená pre všetky $n>n_0$ a platí $n_1>n_0$, tak nerovnosti $|a_n-b|<\varepsilon$ iste vyhovujú všetky čísla $n>n_1$. Z tohto samozrejmého tvrdenia vyplýva, že v závere riešenia príkladu $105 (b)$ by stačilo položiť $n_0>1$ pre $\varepsilon > \frac{3}{4}$ a $n_0 \geq [\frac{3+\sqrt{9-16\varepsilon^2}}{2\varepsilon}]$ pre $\varepsilon \in (0,\frac{3}{4}\rangle$.

\begin{enumerate}[resume]
	\item \useproblem[limita-funkcie]{limita-7}
	\item \useproblem[limita-funkcie]{limita-8}
	\item \useproblem[limita-funkcie]{limita-9}
	\item \useproblem[limita-funkcie]{limita-10}
	\item \useproblem[limita-funkcie]{limita-11}
\end{enumerate}

Ak $x \in \mathbb{R^*}$, nastáva práve jedna z troch možností:
\begin{itemize}
\item $x \in \mathbb{R}$,
\item $x = +\infty$,
\item $x = -\infty$.
\end{itemize}

Ak v definícii limity funkcie rozlíšime pre body $a,b \in \mathbb{R^*}$ tieto možnosti, dostaneme nasledujúcich deväť špeciálnych prípadov (v nich už a,b označujú len reálne čísla):


\begin{multicols}{3}
\begin{enumerate}
    \item $\lim_{x \rightarrow a} f(x)=b$;
    \item $\lim_{x \rightarrow a} f(x)=+\infty$;
    \item $\lim_{x \rightarrow a} f(x)=-\infty$;
    \item $\lim_{x \rightarrow \infty} f(x)=b$;
    \item $\lim_{x \rightarrow \infty} f(x)=+\infty$;
    \item $\lim_{x \rightarrow \infty} f(x)=-\infty$;
    \item $\lim_{x \rightarrow -\infty} f(x)=b$;
    \item $\lim_{x \rightarrow -\infty} f(x)=+\infty$;
    \item $\lim_{x \rightarrow -\infty} f(x)=-\infty$.
\end{enumerate}
\end{multicols}

\begin{enumerate}[resume]
	\item \useproblem[limita-funkcie]{limita-12}
\end{enumerate}

\textit{Riešenie $1$:}
V tomto prípade sú okolia $O(a)$ a $O(b)$ jednoznačne určené svojimi polomermi $\delta,\epsilon$; definíciu limity možno potom prepísať do tvaru 
$$\forall \epsilon > 0 \exists \delta > 0 \forall x \in D(f), x \neq a, |x-a|<\delta:|f(x)-b|<\epsilon$$
alebo - čo je to isté - do tvaru
$$\forall \epsilon > 0 \exists \delta > 0 \forall x \in D(f), x \neq a, |x-a|<\delta \Rightarrow |f(x)-b|<\epsilon$$
(Samozrejme predpokladáme, že $a$ je hromadný bod množiny $D(f)$.)

\begin{veta}
\textbf{Cauchyho-Bolzanovo kritérium konvergencie}

Nech $a \in \mathbb{R^*}$ je hromadný bod definičného oboru funkcie $f$. Funkcia $f$ má v bode $a$ konečnú limitu práve vtedy, keď platí 
$$\forall \varepsilon > 0 \exists O(a) \forall x,y \in (O(a) \setminus \{ a \}) \cap D(f): |f(x)-f(y)|<\varepsilon.$$
\end{veta}

\begin{enumerate}[resume]
	\item \useproblem[limita-funkcie]{limita-13}
\end{enumerate}

\textit{Riešenie 1:}
Treba dokázať pravdivosť tvrdenia (*)
$$\forall \varepsilon > 0 \exists \delta > 0 \forall x \geq 0, x \neq 3, |x-3|< \delta : |\sqrt{x}-\sqrt{3}|< \varepsilon.$$
Predpokladajme, že $|x-3|<\delta$ a skúsme na základe toho zhora odhadnúť výraz $|\sqrt{x}-\sqrt{3}|$. Pretože $|\sqrt{x}-\sqrt{3}|=\frac{|x-3|}{\sqrt{x}+\sqrt{3}}$ a $\sqrt{x}+\sqrt{3} \geq \sqrt{3}$ platí $\sqrt{x}-\sqrt{3} \leq \frac{1}{\sqrt{3}}|x-3|<\frac{\delta}{\sqrt{3}}$. Teraz už vidíme, že tvrdenie (*) platí: ak je dané $\varepsilon > 0$ a chceme, aby platilo $|\sqrt{x}-\sqrt{3}|<\varepsilon$, stačí položiť $\delta=\varepsilon \sqrt{3}$ (alebo $\delta \leq \varepsilon \sqrt{3}$).

\begin{enumerate}[resume]
	\item \useproblem[limita-funkcie]{limita-14}
	\item \useproblem[limita-funkcie]{limita-15}
	\item \useproblem[limita-funkcie]{limita-16}
\end{enumerate}

\section{Vety o limitách I}
\begin{veta}
\textbf{o limite skalárneho násobku, súčtu, rozdielu, súčinu a podielu}

Nech sú dané funkcie $f,g$, nech $a \in \mathbb{R^*}$ je hromadný bod množiny $D(f) \cap D(g)$. Ak existujú konečné $\lim_{x \rightarrow a} f(x)=A$, $\lim_{x \rightarrow a} g(x)=B$, tak existuje aj $\lim_{x \rightarrow a} cf(x)$ ($c \in \mathbb{R}$ je konštanta), $\lim_{x \rightarrow a} (f(x)+g(x)),\lim_{x \rightarrow a} (f(x)-g(x)),\lim_{x \rightarrow a} (f(x) \cdot g(x))$ a platí 
\end{veta}

\begin{multicols}{2}
\begin{enumerate}[label=]
    \item $\lim_{x \rightarrow a} cf(x)=cA$;
    \item $\lim_{x \rightarrow a} (f(x)+g(x))=A+B$;
    \item $\lim_{x \rightarrow a} (f(x)-g(x))=A-B$;
    \item $\lim_{x \rightarrow a} (f(x) \cdot g(x))=A \cdot B$;
    \item $(=c \cdot \lim_{x \rightarrow a} f(x))$;
    \item $(=\lim_{x \rightarrow a} f(x)+\lim_{x \rightarrow a} g(x))$;
    \item $(=\lim_{x \rightarrow a} f(x)-\lim_{x \rightarrow a} g(x))$;
    \item $(=\lim_{x \rightarrow a} f(x) \cdot \lim_{x \rightarrow a} g(x))$.
\end{enumerate}
\end{multicols}
Ak naviac $B \neq 0$, tak existuje aj $\lim_{x \rightarrow a} \frac{f(x)}{g(x)}$ a platí 

\begin{multicols}{2}
\begin{enumerate}[label=]
\item $\lim_{x \rightarrow a} \frac{f(x)}{g(x)}=\frac{A}{B}$
\item $(=\frac{\lim_{x \rightarrow a} f(x)}{\lim_{x \rightarrow a} g(x)})$.
\end{enumerate}
\end{multicols}

\begin{veta}
\textbf{o limite zloženej funkcie}

Nech sú dané funkcie $f,g$ nech $a \in \mathbb{R^*}$ je hromadný bod množiny $D(f \circ g)$. Ak $\lim_{x \rightarrow a} g(x)=A$ $(A \in \mathbb{R^*})$, pričom je splnená podmienka 
$$\exists O(a)\forall x \in O(a): x \neq a \Rightarrow g(x) /neq A,$$
a $\lim_{x \rightarrow A} f(x)=B$  $(B \in \mathbb{R^*})$, tak $\lim_{x \rightarrow a} f(g(x))=B$
\end{veta}

\textit{Poznámka:}
Ak $A$ je hromadným bodom $D(f)$, ale $A \notin D(f)$, tak uvedená veta platí aj vtedy, keď nie je splnená podmienka (*).


Niektoré limity možno nájsť len na základe definície, v ostatných prípadoch je však oveľa efektívnejšie použiť vety o limitách. Pritom je potrebné osvojiť si zdôvodňovanie jednotlivých krokov výpočtu, inak sa nenaučíme odlišovať správne postupy od nesprávnych. Na ilustráciu podrobne popíšme nasledujúci výpočet




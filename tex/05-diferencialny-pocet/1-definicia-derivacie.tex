Hovoríme, že funkcia $f$ (definovaná v okolí bodu $a$ \footnote{t.j. pre
niektoré $\varepsilon>0$ platí $\interval[open]{a-\varepsilon}{a+\varepsilon}
\subset D(f)$; pojmy derivácie a nevlastnej derivácie a nevlastnej derivácie v
bode $a$ by bolo možné zaviesť aj za slabšieho predpokladu \enquote{$a \in D(f)$
je hromadný bod množiny $D(f)$} (odstránili by sa tým aj niektoré ťažkosti so
zavedením pojmu derivácie ako funkcie), definícia predpokladajúca, že $a$ je
vnútorný bod množiny $D(f)$, je však v literatúre najčastejšia}) má deriváciu v
bode $a,(a \in \mathbb{R})$, ak existuje konečná $\lim\limits_{x \rightarrow
a}\frac{f(x)-f(a)}{x-a}$. Ak existuje nevlastná $\lim\limits_{x \rightarrow
a}\frac{f(x)-f(a)}{x-a}$, hovoríme, že funkcia $f$ má nevlastnú (alebo
nekonečnú) deriváciu v bode $a$. Hodnotu $\lim\limits_{x \rightarrow
a}\frac{f(x)-f(a)}{x-a}$ v obidvoch týchto prípadoch označujeme $f'(a)$. Ak nás
nezaujíma, či je $\lim\limits_{x \rightarrow a}\frac{f(x)-f(a)}{x-a}$ konečná
alebo nekonečná. používame spoločný názov vlastná alebo nevlastná derivácia v
bode $a$. Pojmy derivácie sprava a nevlastnej derivácie sprava, resp. derivácie
zľava a nevlastnej derivácie zľava dostaneme, ak v predchádzajúcich definíciách
$\lim\limits_{x \rightarrow a}\frac{f(x)-f(a)}{x-a}$ nahradíme limitou
$\lim\limits_{x \rightarrow a+}\frac{f(x)-f(a)}{x-a}$, resp. $\lim\limits_{x
\rightarrow a-}\frac{f(x)-f(a)}{x-a}$ (v takom prípade stačí predpokladať, že
definičný obor funkcie $f$ obsahuje interval $\interval[open
right]{a}{a+\varepsilon}, resp. (a-\varepsilon,a\rangle$ pre niektoré
$\varepsilon>0$). Hodnoty $\lim\limits_{x \rightarrow a+}\frac{f(x)-f(a)}{x-a}$,
resp. $\lim\limits_{x \rightarrow a-}\frac{f(x)-f(a)}{x-a}$ označujeme
$f'_+(a)$,resp. $f'_-(a)$.

\begin{veta}
Funkcia $f$ (definovaná v okolí bodu $a \in \mathbb{R}$) má vlastnú alebo
nevlastnú deriváciu $f'(a)$ práve vtedy, keď existujú $f'_+(a),f'_-(a)$ a platí
$f'_+(a)=f'_-(a)$. Hodnota $f'(a)$ sa pritom rovná spoločnej hodnote $f'_+(a)$ a
$f'_-(a)$.
\end{veta}

Pojem derivácie ako funkcie sa vo všeobecnosti definuje nasledovne: Nech $M$ je
množina všetkých bodov definičného oboru $D(f)$, v ktorých má funkcia $f$
deriváciu. Funkcia $f':M \rightarrow \mathbb{R}$, ktorá každému bodu $a$  $M$
priradí hodnotu $f'(a)$ derivácie funkcie $f$ v bode $a$, sa nazýva derivácia
funkcie $f$. V špeciálnych prípadoch, ktoré ilustruje nasledujúca poznámka, sa
niekedy pojem derivácie ako funkcie chápe širšie: ak definičným oborom funkcie
$f$ je niektorý z intervalov $\interval{a}{b}, \interval[open right]{a}{b},
\interval[open left]{a}{b}$, pričom v jeho koncovom bode existuje jednostranná
derivácia (v prvom prípade prichádzajú do úvahy body $a,b$, v druhom bod $a$, v
trečom bod $b$), tak funkciu $f'$ považujeme za definovanú aj v tomto bode, jej
funkčnou hodnotou je hodnota príslušnej jednostrannej derivácie.

\begin{enumerate}[resume]
  \item \useproblem[diferencialny-pocet]{diferencialny-pocet-279}
  \item \useproblem[diferencialny-pocet]{diferencialny-pocet-280}
  \item \useproblem[diferencialny-pocet]{diferencialny-pocet-281}
  \item \useproblem[diferencialny-pocet]{diferencialny-pocet-282}
  \item \useproblem[diferencialny-pocet]{diferencialny-pocet-283}
  \item \useproblem[diferencialny-pocet]{diferencialny-pocet-284}
  \item \useproblem[diferencialny-pocet]{diferencialny-pocet-285}
\end{enumerate}

\begin{veta}
Ak funkcie $f,g$ majú derivácie v bode $a$, tak aj funkcie $c \cdot f$ ($c$ je
reálna konštanta), $f+g,f-g,f\cdot g$ majú v bode $a$ derivácie a platí:
\begin{enumerate}
\item $(c\cdot f)'(a)=c\cdot f'(a)$,
\item $(f+g)'(a)=f'(a)+g'(a)$,
\item $(f-g)'(a)=f'(a)-g'(a)$,
\item $(f\cdot g)'(a)=f'(a)\cdot g'(a)$.
\end{enumerate}
Ak naviac $g(a)\neq 0$ majú v bode $a$ deriváciu aj funkcie
$\frac{1}{g},\frac{f}{g}$ a platí
\begin{enumerate}
\item $(\frac{1}{g})'(a)=-\frac{g'(a)}{g^2(a)}$,
\item $(\frac{f}{g})'(a)=\frac{f'(a)\cdot g(a)-f(a)\cdot g'(a)}{g^2 (a)}$.
\end{enumerate}
\end{veta}

\begin{veta}
Ak funkcia $f$ má deriváciu v bode $a$, funkcia $g$ v bode $f(a)$ a zložená
funkcia $h=g \circ f$ je definovaná v okolí bodu $a$, tak $h$ má v bode $a$
deriváciu a platí $$h'(a)=f'(a)\cdot g'(f(a)).$$ (Analogické vety možno dokázať
aj pre jednostranné derivácie.)
\end{veta}

V nasledujúcej tabuľke sú derivácie základných elementárnych funkcií ($c$ je
reálna konštanta):
\begin{itemize}
\item $c'=0$,
\item $(x^n)'=n\cdot x^{n-1},(n \in \mathbb{R} \setminus \{0\})$,
\item $(\sin x)'=\cos x$,
\item $(\cos x)'=-\sin x$,
\item $(\tan x)'=\frac{1}{\cos^2 x}$,
\item $(a^x)'=a^x \cdot \ln a$, špeciálne $(e^x)'=e^x$,
\item $(\log_{a} x)'=\frac{1}{x\cdot \ln a},x>0$, špeciálne $(\ln x)'=\frac{1}{x},x>0$,
\item $(\arcsin x)'=\frac{1}{\sqrt{1-x^2}}$,
\item $(\arccos x)'=-\frac{1}{\sqrt{1-x^2}}$,
\item $(\arctan x)'=\frac{1}{1+x^2}$,
\item $(arcctan x)'=-\frac{1}{1+x^2}$.
\end{itemize}

Pre deriváciu hyperbolických funkcií platia nasledujúce vzorce
\begin{multicols}{2}
\begin{itemize}
  \item $(\sinh x)'=\cosh x$,
  \item $(\cosh x)'=\sinh x$,
  \item $(\tanh x)'=\frac{1}{ch^2 x}$,
  \item $(\coth x)'=-\frac{1}{sh^2 x}$.
\end{itemize}
\end{multicols}

Ak pri hľadaní derivácie funkcie využívame len znalosť derivácií základných
elementárnych funkcií a vety o derivácii súčtu, rozdielu, súčinu, podielu a
zloženej funkcie, nazýva sa taký postup tabuľkovým derivovaním.

\begin{enumerate}[resume]
  \item \useproblem[diferencialny-pocet]{diferencialny-pocet-286}
  \item \useproblem[diferencialny-pocet]{diferencialny-pocet-287}
  \item \useproblem[diferencialny-pocet]{diferencialny-pocet-288}
  \item \useproblem[diferencialny-pocet]{diferencialny-pocet-289}
  \item \useproblem[diferencialny-pocet]{diferencialny-pocet-290}
  \item \useproblem[diferencialny-pocet]{diferencialny-pocet-291}
  \item \useproblem[diferencialny-pocet]{diferencialny-pocet-292}
  \item \useproblem[diferencialny-pocet]{diferencialny-pocet-293}
  \item \useproblem[diferencialny-pocet]{diferencialny-pocet-294}
  \item \useproblem[diferencialny-pocet]{diferencialny-pocet-295}
  \item \useproblem[diferencialny-pocet]{diferencialny-pocet-296}
  \item \useproblem[diferencialny-pocet]{diferencialny-pocet-297}
  \item \useproblem[diferencialny-pocet]{diferencialny-pocet-298}
  \item \useproblem[diferencialny-pocet]{diferencialny-pocet-299}
  \item \useproblem[diferencialny-pocet]{diferencialny-pocet-300}
  \item \useproblem[diferencialny-pocet]{diferencialny-pocet-301}
  \item \useproblem[diferencialny-pocet]{diferencialny-pocet-302}
  \item \useproblem[diferencialny-pocet]{diferencialny-pocet-303}
  \item \useproblem[diferencialny-pocet]{diferencialny-pocet-304}
  \item \useproblem[diferencialny-pocet]{diferencialny-pocet-305}
  \item \useproblem[diferencialny-pocet]{diferencialny-pocet-306}
\end{enumerate}

\begin{veta}
Ak funkcia $f$ má v bode $a$ deriváciu, tak $f$ je v tomto bode spojitá.
\end{veta}

\begin{enumerate}[resume]
  \item \useproblem[diferencialny-pocet]{diferencialny-pocet-307}
  \item \useproblem[diferencialny-pocet]{diferencialny-pocet-308}
  \item \useproblem[diferencialny-pocet]{diferencialny-pocet-309}
\end{enumerate}

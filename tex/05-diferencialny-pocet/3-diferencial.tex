Hovoríme, že funkcia $f$ (definovaná v okolí bodu $a$ má v bode $a$
diferenciál (je diferencovateľná v bode $a$), ak existuje reálna konštanta $A$
taká, že pre funkciu $\omega$, definovanú vzťahom
\[
  f(x)=f(a)+A(x-a)+\omega(x)
\]
platí $\lim\limits_{x \rightarrow a}\frac{\omega(x)}{x-a}=0$. Funkcia definovaná
predpisom $y=A(x-a)$ sa v takom prípade označuje $df(a)$ a nazýva sa diferenciál
funkcie $f$ v bode $a$. Funkcia $df(a)$ sa zvyčajnezapisuje v tvare $df(a)=A$
$dx(a)$ \footnote{písmeno $a$ sa v zápisoch často vynecháva,preto sa možno
stretnúť aj so zápisom $df=A$ $dx$}, kde symbol $dx(a)$ (označujúci diferenciál
funkcie $g(x)=x$ v bode $a$,t.j. funkciu danú predpisom $y=x-a$) sa nazýva
diferenciál nezávislej premennej.
\begin{veta}
Funkcia $f$ je diferencovateľná v bode $a$ práve vtedy, keď $f$ má v bode $a$
deriváciu; pritom platí $A=f'(a)$, kde $A$ je konštanta z definície diferenciálu
funkcie $f$ v bode $a$.
\end{veta}

Graf funkcie $y=f(a)+f'(a)(x-a)$ je dotyčnicou v bodu $(a,f(a))$ ku grafu
funkcir $f$. Ak je funkcia $f$ spojitá v bode $a$, pričom $f'(a)$ je nevlastná,
je dotyčnicou v bode $(a,f(a))$ ku grafu funkcie $f$ priamka $x=a$.

\begin{enumerate}[resume]
  \item \useproblem[diferencialny-pocet]{diferencialny-pocet-314}
  \item \useproblem[diferencialny-pocet]{diferencialny-pocet-315}
  \item \useproblem[diferencialny-pocet]{diferencialny-pocet-316}
  \item \useproblem[diferencialny-pocet]{diferencialny-pocet-317}
  \item \useproblem[diferencialny-pocet]{diferencialny-pocet-318}
  \item \useproblem[diferencialny-pocet]{diferencialny-pocet-319}
  \item \useproblem[diferencialny-pocet]{diferencialny-pocet-320}
\end{enumerate}

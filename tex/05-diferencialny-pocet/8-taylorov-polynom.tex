Nech funkcia $f$ je $n$-krát diferencovateľná v bode $a\in\mathbb{R}$. Potom
Taylorovým polynómom stupňa $n$ funkcie $f$ v bode $a$ sa nazýva polynóm (v
premennej $x$)
\[
  f(a)+\frac{f'(a)}{1!}(x-a)+...+\frac{f^{(n)}(a)}{n!}(x-a)^n
\]
Ak špeciálne $a=0$, používa sa namiesto názvu Taylorov polynóm označenie
Maclaurinov polynóm.

\begin{veta}
Ak funkcia $f$ je $n$-krát diferencovateľná v bode $a\in\mathbb{R}$ a $T_n$ je
jej Taylorov polynóm stupňa $n$ v bode $a$, tak
\[
  \lim_{x \rightarrow a}\frac{f(x)-T_n(x)}{(x-a)^n}=0
\]
(Rozdiel $f-T_n$ sa nazýva zvyšok Taylorovho polynómu stupňa $n$ funkcie $f$ v
bode $a$.)
\end{veta}

Nech funkcia $g$ je definovaná v niektorom rýdzom okolí bodu $a\in\mathbb{R^*}$
a nenadobúda tam nulové hodnoty. Znakom $c(g)$ budeme označovať triedu všetkých
funkcií $f$ takých, že
\begin{itemize}
\item ich definičný obor obsahuje niektoré rýdze okolie bodu $a$
\item platí $\lim\limits_{x\rightarrow a}\frac{f(x)}{g(x)}=0$
\end{itemize}

Namiesto zápisu $f\in c(g)$ sa používa zápis $f=c(g)$. (Pokiaľ by z kontextu
nebolo jasné, ktorého $a\in\mathbb{R^*}$ sa vzťah $f=c(g)$ týka, používa sa
zápis $f=c(g)$ $(x\rightarrow a).$) Zápis $f=h+c(g)$ treba chápať nasledovne:
funkcia $f$ je súčtom funkcie $h$ a niektorej funkcie z triedy $c(g)$.

Ak funkcia $f$ je $n$-krát diferencovateľná v bode $a\in\mathbb{R}$, patrí podľa
vety $19$ zvyšok jej Taylorovho polynómu stupňa $n$ v bode $a$ do triedy
$c((x-a)^n)$, možno teda písať
\[
  f(x)=f(a)+\frac{f'(a)}{1!}(x-a)+...+\frac{f^{(n)}(a)}{n!}c((x-a)^n)
\]
tento zápis sa nazýva Taylorovým vzorcom so zvyškom v Peanovom tvare. V
nasledujúcich príkladoch budeme používať tieto tvrdenia (v zápisoch všade
vynechávame $x\rightarrow 0;m,n$ sú prirodzené čísla):
\begin{itemize}
\item ak $f(x)=c(x^n)$ a $g(x)=c(x^n)$, tak $f(x)+g(x)=c(x^n)$
\item ak $m>n$ a $f(x)=c(x^m)$, tak $f(x)=c(x^n)$
\item ak $f(x)=c(x^n)$, tak $f^m(x)=c(x^{m\cdot n})$
\item ak $f(x)=c(x^n)$, tak $x^m\cdot f(x)=c(x^{m+n})$
\item ak $f(x)=c(x^n)$ a $g(x)=c(x^m)$, tak $f(x)\cdot g(x)=c(x^{m+n})$
\end{itemize}

Uvedené implikácie budeme zapisovať nasledovne:
\begin{itemize}
\item $c(x^n)+c(x^n)=c(x^n)$
\item $c(x^m)=c(x^n)$ pre $m>n$
\item $c^m(x^n)=c(x^{m\cdot n})$
\item $x^m\cdot c(x^n)=c(x^{m+n})$
\item $c(x^n)\cdot c(x^m)=c(x^{m+})$
\end{itemize}

Používanie týchto zápisov vyžaduje istú opatrnosť, pretože ide o symbolické
vyjadrenie implikácií, možno (na rozdiel od skutočných rovností) tieto
\enquote{rovnosti} čítať sprava doľava.

\begin{enumerate}[resume]
  \item \useproblem[diferencialny-pocet]{diferencialny-pocet-385}
  \item \useproblem[diferencialny-pocet]{diferencialny-pocet-386}
\end{enumerate}

V nasledujúcich príkladoch využijeme tvrdenie z príkladu $386$ a znalosť
Maclaurinových polynómov týchto funkcií (možno ich zostrojiť výpočtom
príslušných derivácií):
\begin{itemize}
\item
  $e^x=1+x+\frac{x^2}{2!}+...+\frac{x^n}{n!}+c(x^n)$
\item
  $\sin
  x=x-\frac{x^3}{3!}+...+(-1)^{n-1}\frac{x^{2n-1}}{(2n-1)!}+c(x^{2n})$\footnote{pretože
  $\sin^{(2n)}(0)=0$, majú Maclaurinov polynóm stupňa $2n-1$ a Maclaurinov
  polynóm stupňa $2n$ rovnaký tvar, preto aj zvyšok Maclaurinovho polynómu
  stupňa $2n-1$ funkcie $\sin$ patrí do triedy $c(x^{2n})$; podobná poznáma
  platí o funkcii $\cos$}
\item
  $\cos x=1-\frac{x^2}{2!}+...+(-1)^{n}\frac{x^{2n}}{(2n)!}+c(x^{2n+1})$
\item
  $\ln (1+x)=x-\frac{x^2}{2}+...+(-1)^{n-1}\frac{x^n}{n}+c(x^n)$
\item
  $\frac{1}{1-x}=1+x+x^2+...+\frac{x^n}{n!}+c(x^n)$
\item
  $(1+x)^{\alpha}=1+x+\frac{\alpha(\alpha-1)}{2!}x^2+...+\frac{(\alpha-1) ...
  (\alpha-n+1)}{n!}x^n+c(x^n),(\alpha\in\mathbb{R})$
\end{itemize}

\begin{enumerate}[resume]
  \item \useproblem[diferencialny-pocet]{diferencialny-pocet-387}
  \item \useproblem[diferencialny-pocet]{diferencialny-pocet-388}
  \item \useproblem[diferencialny-pocet]{diferencialny-pocet-389}
  \item \useproblem[diferencialny-pocet]{diferencialny-pocet-390}
  \item \useproblem[diferencialny-pocet]{diferencialny-pocet-391}
\end{enumerate}

\begin{veta}
Nech je daná funkcia $f$, nech funkcia $f^{(n)}$ je definovaná a spojitá v
niektorom okolí $O(a)$ bodu $a\in\mathbb{R}$, nech pre každé $x\in O(a)\setminus
\{a\}$ existuje vlastná $f^{(n+1)}(x)$. Nech $T_n$ je Taylorov polynóm stupňa
$n$ funkcie $f$ v bode $a$. Potom
\begin{itemize}
  \item
    pre každé $x\in O(a),x>a$ $(x\in O(a),x<a)$ existuje číslo $\vartheta (x)\in
    (a,x)$  $(\vartheta (x)\in (x,a))$ také, že:
    \[
      f(x)-T_n(x)=\frac{f^{(n+1)}(\vartheta (x))}{(n+1)!}(x-a)^{n+1}
    \]
    (tzv. Lagrangeov tvar zvyšku)
  \item
    pre každé $x\in O(a),x>a$  $(x\in O(a),x<a)$ existuje číslo $\vartheta
    (x)\in (a,x)$  $\vartheta (x)\in (a,x)$  $(\vartheta (x)\in (x,a))$ také,
    že:
    \[
      f(x)-T_n(x)=\frac{f^{(n+1)}(\vartheta (x))}{n!}(x-a)(x-\vartheta (x))^n
    \]
    (tzv. Cauchyho tvar zvyšku)
\end{itemize}
\end{veta}

\begin{enumerate}[resume]
  \item \useproblem[diferencialny-pocet]{diferencialny-pocet-392}
  \item \useproblem[diferencialny-pocet]{diferencialny-pocet-393}
\end{enumerate}

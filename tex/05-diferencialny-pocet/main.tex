\chapter[Diferenciálny počet]{Diferenciálny počet funkcií jednej premennej}
\loadallproblems[diferencialny-pocet]{./tex/05-diferencialny-pocet/ulohy.tex}

\section{Definícia derivácie}
Hovoríme, že funkcia $f$ (definovaná v okolí bodu $a$ \footnote{t.j. pre
niektoré $\varepsilon>0$ platí $\interval[open]{a-\varepsilon}{a+\varepsilon}
\subset D(f)$; pojmy derivácie a nevlastnej derivácie a nevlastnej derivácie v
bode $a$ by bolo možné zaviesť aj za slabšieho predpokladu \enquote{$a \in D(f)$
je hromadný bod množiny $D(f)$} (odstránili by sa tým aj niektoré ťažkosti so
zavedením pojmu derivácie ako funkcie), definícia predpokladajúca, že $a$ je
vnútorný bod množiny $D(f)$, je však v literatúre najčastejšia}) má deriváciu v
bode $a,(a \in \mathbb{R})$, ak existuje konečná $\lim\limits_{x \rightarrow
a}\frac{f(x)-f(a)}{x-a}$. Ak existuje nevlastná $\lim\limits_{x \rightarrow
a}\frac{f(x)-f(a)}{x-a}$, hovoríme, že funkcia $f$ má nevlastnú (alebo
nekonečnú) deriváciu v bode $a$. Hodnotu $\lim\limits_{x \rightarrow
a}\frac{f(x)-f(a)}{x-a}$ v obidvoch týchto prípadoch označujeme $f'(a)$. Ak nás
nezaujíma, či je $\lim\limits_{x \rightarrow a}\frac{f(x)-f(a)}{x-a}$ konečná
alebo nekonečná. používame spoločný názov vlastná alebo nevlastná derivácia v
bode $a$. Pojmy derivácie sprava a nevlastnej derivácie sprava, resp. derivácie
zľava a nevlastnej derivácie zľava dostaneme, ak v predchádzajúcich definíciách
$\lim\limits_{x \rightarrow a}\frac{f(x)-f(a)}{x-a}$ nahradíme limitou
$\lim\limits_{x \rightarrow a+}\frac{f(x)-f(a)}{x-a}$, resp. $\lim\limits_{x
\rightarrow a-}\frac{f(x)-f(a)}{x-a}$ (v takom prípade stačí predpokladať, že
definičný obor funkcie $f$ obsahuje interval $\interval[open
right]{a}{a+\varepsilon}, resp. (a-\varepsilon,a\rangle$ pre niektoré
$\varepsilon>0$). Hodnoty $\lim\limits_{x \rightarrow a+}\frac{f(x)-f(a)}{x-a}$,
resp. $\lim\limits_{x \rightarrow a-}\frac{f(x)-f(a)}{x-a}$ označujeme
$f'_+(a)$,resp. $f'_-(a)$.

\begin{veta}
Funkcia $f$ (definovaná v okolí bodu $a \in \mathbb{R}$) má vlastnú alebo
nevlastnú deriváciu $f'(a)$ práve vtedy, keď existujú $f'_+(a),f'_-(a)$ a platí
$f'_+(a)=f'_-(a)$. Hodnota $f'(a)$ sa pritom rovná spoločnej hodnote $f'_+(a)$ a
$f'_-(a)$.
\end{veta}

Pojem derivácie ako funkcie sa vo všeobecnosti definuje nasledovne: Nech $M$ je
množina všetkých bodov definičného oboru $D(f)$, v ktorých má funkcia $f$
deriváciu. Funkcia $f':M \rightarrow \mathbb{R}$, ktorá každému bodu $a$  $M$
priradí hodnotu $f'(a)$ derivácie funkcie $f$ v bode $a$, sa nazýva derivácia
funkcie $f$. V špeciálnych prípadoch, ktoré ilustruje nasledujúca poznámka, sa
niekedy pojem derivácie ako funkcie chápe širšie: ak definičným oborom funkcie
$f$ je niektorý z intervalov $\interval{a}{b}, \interval[open right]{a}{b},
\interval[open left]{a}{b}$, pričom v jeho koncovom bode existuje jednostranná
derivácia (v prvom prípade prichádzajú do úvahy body $a,b$, v druhom bod $a$, v
trečom bod $b$), tak funkciu $f'$ považujeme za definovanú aj v tomto bode, jej
funkčnou hodnotou je hodnota príslušnej jednostrannej derivácie.

\begin{enumerate}[resume]
  \item \useproblem[diferencialny-pocet]{diferencialny-pocet-279}
  \item \useproblem[diferencialny-pocet]{diferencialny-pocet-280}
  \item \useproblem[diferencialny-pocet]{diferencialny-pocet-281}
  \item \useproblem[diferencialny-pocet]{diferencialny-pocet-282}
  \item \useproblem[diferencialny-pocet]{diferencialny-pocet-283}
  \item \useproblem[diferencialny-pocet]{diferencialny-pocet-284}
  \item \useproblem[diferencialny-pocet]{diferencialny-pocet-285}
\end{enumerate}

\begin{veta}
Ak funkcie $f,g$ majú derivácie v bode $a$, tak aj funkcie $c \cdot f$ ($c$ je
reálna konštanta), $f+g,f-g,f\cdot g$ majú v bode $a$ derivácie a platí:
\begin{enumerate}
\item $(c\cdot f)'(a)=c\cdot f'(a)$,
\item $(f+g)'(a)=f'(a)+g'(a)$,
\item $(f-g)'(a)=f'(a)-g'(a)$,
\item $(f\cdot g)'(a)=f'(a)\cdot g'(a)$.
\end{enumerate}
Ak naviac $g(a)\neq 0$ majú v bode $a$ deriváciu aj funkcie
$\frac{1}{g},\frac{f}{g}$ a platí
\begin{enumerate}
\item $(\frac{1}{g})'(a)=-\frac{g'(a)}{g^2(a)}$,
\item $(\frac{f}{g})'(a)=\frac{f'(a)\cdot g(a)-f(a)\cdot g'(a)}{g^2 (a)}$.
\end{enumerate}
\end{veta}

\begin{veta}
Ak funkcia $f$ má deriváciu v bode $a$, funkcia $g$ v bode $f(a)$ a zložená
funkcia $h=g \circ f$ je definovaná v okolí bodu $a$, tak $h$ má v bode $a$
deriváciu a platí $$h'(a)=f'(a)\cdot g'(f(a)).$$ (Analogické vety možno dokázať
aj pre jednostranné derivácie.)
\end{veta}

V nasledujúcej tabuľke sú derivácie základných elementárnych funkcií ($c$ je
reálna konštanta):
\begin{itemize}
\item $c'=0$,
\item $(x^n)'=n\cdot x^{n-1},(n \in \mathbb{R} \setminus \{0\})$,
\item $(\sin x)'=\cos x$,
\item $(\cos x)'=-\sin x$,
\item $(\tan x)'=\frac{1}{\cos^2 x}$,
\item $(a^x)'=a^x \cdot \ln a$, špeciálne $(e^x)'=e^x$,
\item $(\log_{a} x)'=\frac{1}{x\cdot \ln a},x>0$, špeciálne $(\ln x)'=\frac{1}{x},x>0$,
\item $(\arcsin x)'=\frac{1}{\sqrt{1-x^2}}$,
\item $(\arccos x)'=-\frac{1}{\sqrt{1-x^2}}$,
\item $(\arctan x)'=\frac{1}{1+x^2}$,
\item $(arcctan x)'=-\frac{1}{1+x^2}$.
\end{itemize}

Pre deriváciu hyperbolických funkcií platia nasledujúce vzorce
\begin{multicols}{2}
\begin{itemize}
  \item $(\sinh x)'=\cosh x$,
  \item $(\cosh x)'=\sinh x$,
  \item $(\tanh x)'=\frac{1}{ch^2 x}$,
  \item $(\coth x)'=-\frac{1}{sh^2 x}$.
\end{itemize}
\end{multicols}

Ak pri hľadaní derivácie funkcie využívame len znalosť derivácií základných
elementárnych funkcií a vety o derivácii súčtu, rozdielu, súčinu, podielu a
zloženej funkcie, nazýva sa taký postup tabuľkovým derivovaním.

\begin{enumerate}[resume]
  \item \useproblem[diferencialny-pocet]{diferencialny-pocet-286}
  \item \useproblem[diferencialny-pocet]{diferencialny-pocet-287}
  \item \useproblem[diferencialny-pocet]{diferencialny-pocet-288}
  \item \useproblem[diferencialny-pocet]{diferencialny-pocet-289}
  \item \useproblem[diferencialny-pocet]{diferencialny-pocet-290}
  \item \useproblem[diferencialny-pocet]{diferencialny-pocet-291}
  \item \useproblem[diferencialny-pocet]{diferencialny-pocet-292}
  \item \useproblem[diferencialny-pocet]{diferencialny-pocet-293}
  \item \useproblem[diferencialny-pocet]{diferencialny-pocet-294}
  \item \useproblem[diferencialny-pocet]{diferencialny-pocet-295}
  \item \useproblem[diferencialny-pocet]{diferencialny-pocet-296}
  \item \useproblem[diferencialny-pocet]{diferencialny-pocet-297}
  \item \useproblem[diferencialny-pocet]{diferencialny-pocet-298}
  \item \useproblem[diferencialny-pocet]{diferencialny-pocet-299}
  \item \useproblem[diferencialny-pocet]{diferencialny-pocet-300}
  \item \useproblem[diferencialny-pocet]{diferencialny-pocet-301}
  \item \useproblem[diferencialny-pocet]{diferencialny-pocet-302}
  \item \useproblem[diferencialny-pocet]{diferencialny-pocet-303}
  \item \useproblem[diferencialny-pocet]{diferencialny-pocet-304}
  \item \useproblem[diferencialny-pocet]{diferencialny-pocet-305}
  \item \useproblem[diferencialny-pocet]{diferencialny-pocet-306}
\end{enumerate}

\begin{veta}
Ak funkcia $f$ má v bode $a$ deriváciu, tak $f$ je v tomto bode spojitá.
\end{veta}

\begin{enumerate}[resume]
  \item \useproblem[diferencialny-pocet]{diferencialny-pocet-307}
  \item \useproblem[diferencialny-pocet]{diferencialny-pocet-308}
  \item \useproblem[diferencialny-pocet]{diferencialny-pocet-309}
\end{enumerate}


\section{Derivácia inverznej funkcie}
\begin{veta}
Nech funkcia $f:\mathbb{I}\rightarrow\mathbb{R}$, rýdzomonotónna na intervale
$I$, má deriváciu v bode $a$. Potom inverzná funkcia $f^{-1}$ má vlastnú alebo
nevlastnú deriváciu v bode $f(a)$, pričom platí
\begin{enumerate}
\item ak $f'(a)\neq 0$, tak $(f^{-1})'(f(a))=\frac{1}{f'(a)}$
\item ak $f'(a)=0$ a $f$ je rastúca, tak $(f^{-1})'(f(a))=+\infty$
\item ak $f'(a)=0$ a $f$ je klesajúca, tak $(f^{-1})'(f(a))=-\infty$
\end{enumerate}
\end{veta}

\begin{enumerate}[resume]
  \item \useproblem[diferencialny-pocet]{diferencialny-pocet-310}
  \item \useproblem[diferencialny-pocet]{diferencialny-pocet-311}
  \item \useproblem[diferencialny-pocet]{diferencialny-pocet-312}
  \item \useproblem[diferencialny-pocet]{diferencialny-pocet-313}
\end{enumerate}

\section{Diferenciál}
Hovoríme, že funkcia $f$ (definovaná v okolí bodu $a$ má v bode $a$
diferenciál (je diferencovateľná v bode $a$), ak existuje reálna konštanta $A$
taká, že pre funkciu $\omega$, definovanú vzťahom
\[
  f(x)=f(a)+A(x-a)+\omega(x)
\]
platí $\lim\limits_{x \rightarrow a}\frac{\omega(x)}{x-a}=0$. Funkcia definovaná
predpisom $y=A(x-a)$ sa v takom prípade označuje $df(a)$ a nazýva sa diferenciál
funkcie $f$ v bode $a$. Funkcia $df(a)$ sa zvyčajnezapisuje v tvare $df(a)=A$
$dx(a)$ \footnote{písmeno $a$ sa v zápisoch často vynecháva,preto sa možno
stretnúť aj so zápisom $df=A$ $dx$}, kde symbol $dx(a)$ (označujúci diferenciál
funkcie $g(x)=x$ v bode $a$,t.j. funkciu danú predpisom $y=x-a$) sa nazýva
diferenciál nezávislej premennej.
\begin{veta}
Funkcia $f$ je diferencovateľná v bode $a$ práve vtedy, keď $f$ má v bode $a$
deriváciu; pritom platí $A=f'(a)$, kde $A$ je konštanta z definície diferenciálu
funkcie $f$ v bode $a$.
\end{veta}

Graf funkcie $y=f(a)+f'(a)(x-a)$ je dotyčnicou v bodu $(a,f(a))$ ku grafu
funkcir $f$. Ak je funkcia $f$ spojitá v bode $a$, pričom $f'(a)$ je nevlastná,
je dotyčnicou v bode $(a,f(a))$ ku grafu funkcie $f$ priamka $x=a$.

\begin{enumerate}[resume]
  \item \useproblem[diferencialny-pocet]{diferencialny-pocet-314}
  \item \useproblem[diferencialny-pocet]{diferencialny-pocet-315}
  \item \useproblem[diferencialny-pocet]{diferencialny-pocet-316}
  \item \useproblem[diferencialny-pocet]{diferencialny-pocet-317}
  \item \useproblem[diferencialny-pocet]{diferencialny-pocet-318}
  \item \useproblem[diferencialny-pocet]{diferencialny-pocet-319}
  \item \useproblem[diferencialny-pocet]{diferencialny-pocet-320}
\end{enumerate}


\section{Derivácie vyšších rádov}
Derivácie vyšších rádov definujeme rekurentne: Nech funkcia $f$ je definovaná v okolí bodu $a\in\mathbb{R}$; označme $f^{(0)}:=f$. Hovoríme,že funkcia $f$ má n-tú deriváciu v bode $a$, reps. nevlastnú n-tú deriváciu v bode $a$, ak existuje derivácia, resp. nevlastná derivácia funkcie $f^{(n-1)}$ v bode $a$. n-tú deriváciu v bode $a$ aj nevlastnú n-tú deriváciu v bode $a$   označujeme $f^{(n)}(a)$. (Analogicky sa definujú vlastné a nevlastné jednostranné n-té derivácie v bode $a$.) Derivácia funkcie $f^{(n-1)}$ sa nazýva n-tá derivácia funkcie $f$ a označuje sa $f^{(n)}$. Okrem označení $f^{(1)},f^{(2)},f^{(3)},f^{(4)},f^{(5)},...$ sa používajú aj označenia $f',f'',f''',f''''$ (alebo $f^{VI}$),$f^{V},...$ .
Platia nasledujúce vzťahy:

\begin{itemize}
    \item $(x^m)^{(n)}=
        \begin{cases}
            \frac{m!x^{m-n}}{(m-n)!},& $ak $ n\leq m \\
            0, &  $ak $ n>m
        \end{cases}
        $,$(m\in\mathbb{N})$
    \item $(\sin x)^{(n)}=\sin (x+\frac{n\pi}{2})$
    \item $(\cos x)^{(n)}=\cos (x+\frac{n\pi}{2})$
    \item $(a^x)^{(n)}=a^x\ln^n a$
    \item $(e^x)^{(n)}=e^x$
    \item $(\log_a x)^{(n)}=\frac{(-1)^{n-1}(n-1)!}{x^n\ln^n a},x>0$
    \item $(\ln x)^{(n)}=\frac{(-1)^{n-1}(n-1)!}{x^n},x>0$
\end{itemize}

\begin{veta}
\textit{Leibnitzov vzorec}
Ak funkcie $f,g$ majú n-tú deriváciu v bode $a$, tak existuje $(f\cdot
g)^{(n)}(a)$ a platí
\[
    (f\cdot g)^{(n)}(a)=\sum_{k=0}^n {n \choose k} f^{k}(a)\cdot g^{(n-k)}(a)
\]
\end{veta}

\begin{enumerate}[resume]
	\item \useproblem[diferencialny-pocet]{diferencialny-pocet-321}
	\item \useproblem[diferencialny-pocet]{diferencialny-pocet-322}
	\item \useproblem[diferencialny-pocet]{diferencialny-pocet-323}
	\item \useproblem[diferencialny-pocet]{diferencialny-pocet-324}
	\item \useproblem[diferencialny-pocet]{diferencialny-pocet-325}
	\item \useproblem[diferencialny-pocet]{diferencialny-pocet-326}
	\item \useproblem[diferencialny-pocet]{diferencialny-pocet-327}
	\item \useproblem[diferencialny-pocet]{diferencialny-pocet-328}
\end{enumerate}


\section{Základné vety diferenciálneho počtu}
\begin{veta}
\textit{(Rolle).}
Nech funkcia $f:\interval{a}{b} \rightarrow \mathbb{R}$ vyhovuje nasledujúcim
podmienkam:
\begin{itemize}
\item $f$ je spojitá na intervale $\interval{a}{b}$
\item
  v každom bode $x\in \interval[open]{a}{b}$ existuje vlastná alebo nevlastná
  $f'(x)$
\item $f(a)=f(b)$
\end{itemize}
Potom existuje bod $c\in \interval[open]{a}{b}$, v ktorom $f'(c)=0$.
\end{veta}

\begin{enumerate}[resume]
  \item \useproblem[diferencialny-pocet]{diferencialny-pocet-329}
  \item \useproblem[diferencialny-pocet]{diferencialny-pocet-330}
  \item \useproblem[diferencialny-pocet]{diferencialny-pocet-331}
  \item \useproblem[diferencialny-pocet]{diferencialny-pocet-332}
  \item \useproblem[diferencialny-pocet]{diferencialny-pocet-333}
  \item \useproblem[diferencialny-pocet]{diferencialny-pocet-334}
  \item \useproblem[diferencialny-pocet]{diferencialny-pocet-335}
\end{enumerate}

\begin{veta}
\textit{(Lagrangeova veta o strednej hodnote.)}
Nech funkcia $f:\interval{a}{b} \rightarrow\mathbb{R}$ vyhovuje nasledujúcim
podmienkam:
\begin{itemize}
\item $f$ je spojitá;
\item
  v každom bode $x\in \interval[open]{a}{b}$ existuje vlastná alebo nevlastná
  $f'(x)$. Potom existuje bod $c\in \interval[open]{a}{b}$, v ktorom
  $f'(c)=\frac{f(b)-f(a)}{b-a}.$
\end{itemize}
\end{veta}

\begin{enumerate}[resume]
  \item \useproblem[diferencialny-pocet]{diferencialny-pocet-336}
  \item \useproblem[diferencialny-pocet]{diferencialny-pocet-337}
  \item \useproblem[diferencialny-pocet]{diferencialny-pocet-338}
  \item \useproblem[diferencialny-pocet]{diferencialny-pocet-339}
  \item \useproblem[diferencialny-pocet]{diferencialny-pocet-340}
  \item \useproblem[diferencialny-pocet]{diferencialny-pocet-341}
  \item \useproblem[diferencialny-pocet]{diferencialny-pocet-342}
  \item \useproblem[diferencialny-pocet]{diferencialny-pocet-343}
  \item \useproblem[diferencialny-pocet]{diferencialny-pocet-344}
  \item \useproblem[diferencialny-pocet]{diferencialny-pocet-345}
\end{enumerate}

\begin{veta}
\textit{(Cauchyho veta o strednej hodnote).}
Nech funkcie $f: \interval{a}{b} \rightarrow \mathbb{R}$ a $g: \interval{a}{b}
\rightarrow \mathbb{R}$ spĺňajú nasledujúce podmienky:
\begin{itemize}
\item
  $f$ a $g$ sú spojité na intervale $\interval{a}{b}$
\item
  v každom bode $x\in \interval[open]{a}{b}$ existujú vlastná alebo nevlastná
  $f'(x)$ a vlastná $g'(x)$
\end{itemize}
Potom existuje bod $c\in \interval[open]{a}{b}$, v ktorom platí
\[
  (f(b)-f(a))g'(c)=(g(b)-g(a))f'(c)
\]
Ak sú naviac splnené predpoklaady
\begin{itemize}
\item $(f'(x))^2+(g'(x))^2>0$ Pre každé $x\in \interval[open]{a}{b}$
\item $g(b)\neq g(a)$
\end{itemize}
možno uvedenú rovnosť písať v tvare
\[
  \frac{f(b)-f(a)}{g(b)-g(a)}=\frac{f'(c)}{g'(c)}
\]
\end{veta}

\begin{enumerate}[resume]
  \item \useproblem[diferencialny-pocet]{diferencialny-pocet-346}
  \item \useproblem[diferencialny-pocet]{diferencialny-pocet-347}
  \item \useproblem[diferencialny-pocet]{diferencialny-pocet-348}
\end{enumerate}


\section[Vyšetrovanie vlastností funkcií]{Vyšetrovanie vlastností funkcií pomocou diferenciálneho počtu}
\subsection{Monotónnosť}
\begin{veta}
Nech funkcia $f: I \rightarrow\mathbb{R}$ je spojitá na intervale $I$ a má
deriváciu v každom jeho vnútornom bode. Potom
\begin{itemize}
\item
  ak $f'>0,(f'\geq 0)$ vnútri intervalu $I$ (t.j. ak pre každý vnútorný bod
  $x$ intervalu $I$ platí $f'(x)>0,(f'(x)\geq 0)$), tak $f$ je rastúca
  (neklesajúca) na $I$
\item
  ak $f'<0,(f'\leq 0)$ vnútri intervalu $I$, tak $f$ je klesajúca (nerastúca) na
  $I$
\end{itemize}
\end{veta}

\begin{enumerate}[resume]
  \item \useproblem[diferencialny-pocet]{diferencialny-pocet-349}
  \item \useproblem[diferencialny-pocet]{diferencialny-pocet-350}
  \item \useproblem[diferencialny-pocet]{diferencialny-pocet-351}
\end{enumerate}

\begin{veta}
Nech funkcie $f,g$ sú $n$-krát diferencovateľné na intervale $I$, nech v bode
$a\in I$ platí $f(a)=g(a),f'(a)=g'(a),...,f^{(n-1)}(a)=g^{(n-1)}(a)$ (teda ak
$n=1$, predpokladáme len $f(a)=g(a)$). Potom
\begin{enumerate}
\item
  ak $f^{(n)}(x)>g^{(n)}(x)$, pre všetky $x\in I \cap \interval[open]{a}{\infty}$, tak
  $f(x)>g(x)$ Pre všetky $x\in I \cap \interval[open]{a}{\infty}$ (pritom samozrejme
  predpokladáme, že $I \cap \interval[open]{a}{\infty} \neq \emptyset$ )
\item
  \begin{itemize}
    \item
      ak $f^{(n)}(x)>g^{(n)}(x)$ pre všetky $x\in I \cap
      \interval[open]{-\infty}{a}$ a $n$ je párne, tak $f(x)>g(x)$ pre všetky
      $x\in I \cap \interval[open]{-\infty}{a}$;
    \item
      ak $f^{(n)}(x)>g^{(n)}(x)$ pre všetky $x\in I \cap
      \interval[open]{-\infty}{a}$ a $n$ je nepárne, tak $f(x)<g(x)$ pre všetky
      $x\in I \cap \interval[open]{-\infty}{a}$ (pritom predpokladáme $I \cap
      \interval[open]{-\infty}{a}\neq \emptyset$)
  \end{itemize}
\end{enumerate}
\end{veta}

\begin{enumerate}[resume]
  \item \useproblem[diferencialny-pocet]{diferencialny-pocet-352}
  \item \useproblem[diferencialny-pocet]{diferencialny-pocet-353}
  \item \useproblem[diferencialny-pocet]{diferencialny-pocet-354}
  \item \useproblem[diferencialny-pocet]{diferencialny-pocet-355}
  \item \useproblem[diferencialny-pocet]{diferencialny-pocet-356}
\end{enumerate}

\subsection{Konvexnosť a konkávnosť. Inflexné body}
Funkcia $f$ sa nazýva rýdzo konvexná (konvexná) na intervale $I \subset D(f)$,
ak platí $(*)$
\[
  \forall x,y\in I,x\neq y \forall p,q>0,p+q=1;f(px+qy)<pf(x)+qf(y)
\]
\[
  (\forall x,y\in I,x\neq y \forall p,q>0,p+q=1;f(px+qy)\leq pf(x)+qf(y))
\]
Funkcia $f$ sa nazýva rýdzo konkávna (konkávna) na intervale $I \subset D(f)$,
ak platí
\[
  \forall x,y\in I,x\neq y \forall p,q>0,p+q=1;f(px+qy)>pf(x)+qf(y)
\]
\[
  (\forall x,y\in I,x\neq y \forall p,q>0,p+q=1;f(px+qy)\geq pf(x)+qf(y))
\]
(Geometricky možno výrok $(*)$ interpretovať takto: pre ľubovoľné $2$ čísla
$x,y\in I,x<y$, leží úsečka spájajúca body $(x,f(x))$ nad grafom funkcie
$f/(x,y)$.)

\begin{veta}
Nech funkcia $f$ je spojitá na intervale $I$ a dvakrát diferencovateľná v každom
jeho vnútornom bode. Potom
\begin{itemize}
\item
  $f''>0,(f''\geq 0)$ vnútri intervalu $I$, tak $f$ je rýdzo konvexná (konvexná)
  na $I$
\item
  $f''<0,(f''\leq 0)$ vnútri intervalu $I$, tak $f$ je rýdzo konkávna (konkávna)
  na $I$
\end{itemize}
Vnútorný bod $a$ množiny $D(f)$ sa nazýva inflexný bod funkcie $f$, ak $f$ má v
bode $a$ deriváciu a existuje $\varepsilon >0$ tak, že funkcia $f$ je rýdzo
konvexná na jednej z množín $\interval[open left]{a-\varepsilon}{a}$,
$\interval[open right]{a}{a + \varepsilon}$ a rýdzo konkávna na druhej z nich.
\end{veta}

\begin{veta}
Nech funkcia $f$ je trikrát diferencovateľná v bode $a$ a dvakrát
diferencovateľná v niektorom jeho okolí. Ak $f''(a)=0,f''(a)\neq 0$, tak $a$ je
inflexný bod funkcie $f$.
\end{veta}

\textit{Poznámka:}
Existujú aj iné definície rýdzej konvexnosti, rýdzej konkávnosti a inflexného
bodu, ktoré nie sú ekvivalentné tu uvedeným. Všetky v matematickej literatúre
používané definície týchto pojmov sú však volené tak, že vety $13$ a $14$
zostanú v platnosti.

\begin{enumerate}[resume]
  \item \useproblem[diferencialny-pocet]{diferencialny-pocet-357}
  \item \useproblem[diferencialny-pocet]{diferencialny-pocet-358}
  \item \useproblem[diferencialny-pocet]{diferencialny-pocet-359}
  \item \useproblem[diferencialny-pocet]{diferencialny-pocet-360}
  \item \useproblem[diferencialny-pocet]{diferencialny-pocet-361}
  \item \useproblem[diferencialny-pocet]{diferencialny-pocet-362}
  \item \useproblem[diferencialny-pocet]{diferencialny-pocet-363}
\end{enumerate}

\textit{Poznámka:}
Na vlastnostiach odvodených v príklade $362$ sa zakladajú nasledujúce definície,
odlišné od definícií používaných v tomto odstavci:
\begin{itemize}
\item
  Funkcia $f$ diferencovateľná na intervale $I$ sa nazýva rýdzo konvexná (rýdzo
  konkávna) na $I$, ak pre každé $a\in I$ platí: na množine $I \setminus \{a\}$
  leží graf funkcie $f$ nad (pod) svojou dotyčnicou v bode $(a,f(a))$.
\item
  Bod $a$ sa nazýva inflexný bod funkcie $f$, ak existuje konečná $f'(a)$ a graf
  funkcie $f$ prechádza v bode $a$ z jednej strany svojej dotyčnice na druhú.
\end{itemize}
Prvá z týchto definícií predstavuje užšie chápanie pojmov rýdzo konvexná a rýdzo
konkávna funkcia (to sme tu nedokazovali, ale môžete to skúsiť sami), druhá
naopak širšie chápanie pojmu inflexný bod (to ukazujú príklady $362$ a $363$).

\subsection{Extrémy}
Nech definičným oborom funkcie $f$ je interval $I$. Hovoríme, že funkcia $f$ má
v bode $a$ $I$ lokálne maximum (lokálne minimum), ak existuje okolie $O(a)$ bodu
$a$ tak, že platí:
\[
  \forall x\in (O(a)\setminus \{a\})\cap I:f(x)\leq f(a)
\]
\[
  (\forall x\in (O(a)\setminus \{a\})\cap I:f(x)\geq f(a))
\]
Definíciu ostrého lokálneho maxima (ostrého lokálneho minima) dostaneme, ak v
predchádzajúcej definícii zameníme znak $\leq$ ($\geq$) znakom $<(>)$. Lokálne
maximá a lokálne minimá sa súhrnne nazývajú lokálnymi extrémami.

\begin{veta}
Ak funkcia $f$ má lokálny extrém vo vnútornom bode $a$ svojho definičného oboru,
tak buď neexistuje vlastná ani nevlastná $f'(a)$, alebo $f'(a)=0$. Bod $a$ sa
nazýva stacionálny bod funkcie $f$, ak $f'(a)=0$. Pri hľadaní lokálnych extrémov
funkcie $f:I\rightarrow\mathbb{R}$, treba teda vyšetriť:
\begin{enumerate}
\item všetky jej stacionárne body
\item všetky body $a\in I$, v ktorých neexistuhe $f'(a)$
\item všetky body $a\in I$, ktoré nie sú vnútornými bodmi intervalu $I$
\end{enumerate}
\end{veta}

\begin{veta}
Ak funkcia $f$ je dvakrát diferencovateľná vo vnútornom bode $a$ množiny $D(f)$
a platí $f'(a)=0,f''(a)>0$  $(f'(a)=0,f''(a)<0)$, tak $f$ má v bode $a$ ostré
lokálne minimum (ostré lokálne maximum).
\end{veta}

\begin{veta}
Nech funkcia $f$ je $n$-krát $(n\geq 2)$ diferencovateľná vo vnútornom bode $a$
množiny $D(f)$, nech $f'(a)=...=f^{(n-1)}(a)=0,f^{(n)}\neq 0$.

Ak $n$ je párne a $f^{(n)}(a)>0$  $(f^{(n)}(a)<0)$, tak funkcia $f$ má v bode
$a$ ostré lokálne minimum (ostré lokálne maximum).

Ak $n$ je nepárne, nemá funkcia $f$ v bode $a$ lokálny extrém.
\end{veta}

\begin{enumerate}[resume]
  \item \useproblem[diferencialny-pocet]{diferencialny-pocet-364}
  \item \useproblem[diferencialny-pocet]{diferencialny-pocet-365}
  \item \useproblem[diferencialny-pocet]{diferencialny-pocet-366}
  \item \useproblem[diferencialny-pocet]{diferencialny-pocet-367}
  \item \useproblem[diferencialny-pocet]{diferencialny-pocet-368}
  \item \useproblem[diferencialny-pocet]{diferencialny-pocet-369}
  \item \useproblem[diferencialny-pocet]{diferencialny-pocet-370}
  \item \useproblem[diferencialny-pocet]{diferencialny-pocet-371}
  \item \useproblem[diferencialny-pocet]{diferencialny-pocet-372}
  \item \useproblem[diferencialny-pocet]{diferencialny-pocet-373}
  \item \useproblem[diferencialny-pocet]{diferencialny-pocet-374}
  \item \useproblem[diferencialny-pocet]{diferencialny-pocet-375}
  \item \useproblem[diferencialny-pocet]{diferencialny-pocet-376}
\end{enumerate}


\section{L' Hospitalovo pravidlo}
\begin{veta}
Nech
\begin{enumerate}
\item
  funkcia $f,g$ sú diferencovateľné v niektorom prstencovom okolí $O^*(a)$ bodu
  $a\in\mathbb{R^*}$
\item
  $\forall x\in O^*(a):g'(x)\neq 0$
\item
  $\lim\limits_{x\rightarrow a}f(x)=\lim\limits_{x\rightarrow a}g(x)=0$ alebo
  $\lim\limits_{x\rightarrow a}|g(x)|=+\infty$
\item
  existuje vlastná alebo nevlastná $\lim\limits_{x\rightarrow a}\frac{f'(x)}{g'(x)}$.
  Potom existuje aj $\lim\limits_{x\rightarrow a}\frac{f(x)}{g(x)}$ a platí
  $\lim\limits_{x\rightarrow a}\frac{f(x)}{g(x)}=\lim\limits_{x\rightarrow
  a}\frac{f'(x)}{g'(x)}$

(Analogické tvrdenia možno sformulovať aj pre jednostranné limity.)
\end{enumerate}
\end{veta}

\begin{enumerate}[resume]
  \item \useproblem[diferencialny-pocet]{diferencialny-pocet-377}
  \item \useproblem[diferencialny-pocet]{diferencialny-pocet-378}
  \item \useproblem[diferencialny-pocet]{diferencialny-pocet-379}
  \item \useproblem[diferencialny-pocet]{diferencialny-pocet-380}
  \item \useproblem[diferencialny-pocet]{diferencialny-pocet-381}
  \item \useproblem[diferencialny-pocet]{diferencialny-pocet-382}
  \item \useproblem[diferencialny-pocet]{diferencialny-pocet-383}
  \item \useproblem[diferencialny-pocet]{diferencialny-pocet-384}
\end{enumerate}


\section{Taylorov polynóm}
Nech funkcia $f$ je $n$-krát diferencovateľná v bode $a\in\mathbb{R}$. Potom
Taylorovým polynómom stupňa $n$ funkcie $f$ v bode $a$ sa nazýva polynóm (v
premennej $x$)
\[
  f(a)+\frac{f'(a)}{1!}(x-a)+...+\frac{f^{(n)}(a)}{n!}(x-a)^n
\]
Ak špeciálne $a=0$, používa sa namiesto názvu Taylorov polynóm označenie
Maclaurinov polynóm.

\begin{veta}
Ak funkcia $f$ je $n$-krát diferencovateľná v bode $a\in\mathbb{R}$ a $T_n$ je
jej Taylorov polynóm stupňa $n$ v bode $a$, tak
\[
  \lim_{x \rightarrow a}\frac{f(x)-T_n(x)}{(x-a)^n}=0
\]
(Rozdiel $f-T_n$ sa nazýva zvyšok Taylorovho polynómu stupňa $n$ funkcie $f$ v
bode $a$.)
\end{veta}

Nech funkcia $g$ je definovaná v niektorom rýdzom okolí bodu $a\in\mathbb{R^*}$
a nenadobúda tam nulové hodnoty. Znakom $c(g)$ budeme označovať triedu všetkých
funkcií $f$ takých, že
\begin{itemize}
\item ich definičný obor obsahuje niektoré rýdze okolie bodu $a$
\item platí $\lim\limits_{x\rightarrow a}\frac{f(x)}{g(x)}=0$
\end{itemize}

Namiesto zápisu $f\in c(g)$ sa používa zápis $f=c(g)$. (Pokiaľ by z kontextu
nebolo jasné, ktorého $a\in\mathbb{R^*}$ sa vzťah $f=c(g)$ týka, používa sa
zápis $f=c(g)$ $(x\rightarrow a).$) Zápis $f=h+c(g)$ treba chápať nasledovne:
funkcia $f$ je súčtom funkcie $h$ a niektorej funkcie z triedy $c(g)$.

Ak funkcia $f$ je $n$-krát diferencovateľná v bode $a\in\mathbb{R}$, patrí podľa
vety $19$ zvyšok jej Taylorovho polynómu stupňa $n$ v bode $a$ do triedy
$c((x-a)^n)$, možno teda písať
\[
  f(x)=f(a)+\frac{f'(a)}{1!}(x-a)+...+\frac{f^{(n)}(a)}{n!}c((x-a)^n)
\]
tento zápis sa nazýva Taylorovým vzorcom so zvyškom v Peanovom tvare. V
nasledujúcich príkladoch budeme používať tieto tvrdenia (v zápisoch všade
vynechávame $x\rightarrow 0;m,n$ sú prirodzené čísla):
\begin{itemize}
\item ak $f(x)=c(x^n)$ a $g(x)=c(x^n)$, tak $f(x)+g(x)=c(x^n)$
\item ak $m>n$ a $f(x)=c(x^m)$, tak $f(x)=c(x^n)$
\item ak $f(x)=c(x^n)$, tak $f^m(x)=c(x^{m\cdot n})$
\item ak $f(x)=c(x^n)$, tak $x^m\cdot f(x)=c(x^{m+n})$
\item ak $f(x)=c(x^n)$ a $g(x)=c(x^m)$, tak $f(x)\cdot g(x)=c(x^{m+n})$
\end{itemize}

Uvedené implikácie budeme zapisovať nasledovne:
\begin{itemize}
\item $c(x^n)+c(x^n)=c(x^n)$
\item $c(x^m)=c(x^n)$ pre $m>n$
\item $c^m(x^n)=c(x^{m\cdot n})$
\item $x^m\cdot c(x^n)=c(x^{m+n})$
\item $c(x^n)\cdot c(x^m)=c(x^{m+})$
\end{itemize}

Používanie týchto zápisov vyžaduje istú opatrnosť, pretože ide o symbolické
vyjadrenie implikácií, možno (na rozdiel od skutočných rovností) tieto
\enquote{rovnosti} čítať sprava doľava.

\begin{enumerate}[resume]
  \item \useproblem[diferencialny-pocet]{diferencialny-pocet-385}
  \item \useproblem[diferencialny-pocet]{diferencialny-pocet-386}
\end{enumerate}

V nasledujúcich príkladoch využijeme tvrdenie z príkladu $386$ a znalosť
Maclaurinových polynómov týchto funkcií (možno ich zostrojiť výpočtom
príslušných derivácií):
\begin{itemize}
\item
  $e^x=1+x+\frac{x^2}{2!}+...+\frac{x^n}{n!}+c(x^n)$
\item
  $\sin
  x=x-\frac{x^3}{3!}+...+(-1)^{n-1}\frac{x^{2n-1}}{(2n-1)!}+c(x^{2n})$\footnote{pretože
  $\sin^{(2n)}(0)=0$, majú Maclaurinov polynóm stupňa $2n-1$ a Maclaurinov
  polynóm stupňa $2n$ rovnaký tvar, preto aj zvyšok Maclaurinovho polynómu
  stupňa $2n-1$ funkcie $\sin$ patrí do triedy $c(x^{2n})$; podobná poznáma
  platí o funkcii $\cos$}
\item
  $\cos x=1-\frac{x^2}{2!}+...+(-1)^{n}\frac{x^{2n}}{(2n)!}+c(x^{2n+1})$
\item
  $\ln (1+x)=x-\frac{x^2}{2}+...+(-1)^{n-1}\frac{x^n}{n}+c(x^n)$
\item
  $\frac{1}{1-x}=1+x+x^2+...+\frac{x^n}{n!}+c(x^n)$
\item
  $(1+x)^{\alpha}=1+x+\frac{\alpha(\alpha-1)}{2!}x^2+...+\frac{(\alpha-1) ...
  (\alpha-n+1)}{n!}x^n+c(x^n),(\alpha\in\mathbb{R})$
\end{itemize}

\begin{enumerate}[resume]
  \item \useproblem[diferencialny-pocet]{diferencialny-pocet-387}
  \item \useproblem[diferencialny-pocet]{diferencialny-pocet-388}
  \item \useproblem[diferencialny-pocet]{diferencialny-pocet-389}
  \item \useproblem[diferencialny-pocet]{diferencialny-pocet-390}
  \item \useproblem[diferencialny-pocet]{diferencialny-pocet-391}
\end{enumerate}

\begin{veta}
Nech je daná funkcia $f$, nech funkcia $f^{(n)}$ je definovaná a spojitá v
niektorom okolí $O(a)$ bodu $a\in\mathbb{R}$, nech pre každé $x\in O(a)\setminus
\{a\}$ existuje vlastná $f^{(n+1)}(x)$. Nech $T_n$ je Taylorov polynóm stupňa
$n$ funkcie $f$ v bode $a$. Potom
\begin{itemize}
  \item
    pre každé $x\in O(a),x>a$ $(x\in O(a),x<a)$ existuje číslo $\vartheta (x)\in
    (a,x)$  $(\vartheta (x)\in (x,a))$ také, že:
    \[
      f(x)-T_n(x)=\frac{f^{(n+1)}(\vartheta (x))}{(n+1)!}(x-a)^{n+1}
    \]
    (tzv. Lagrangeov tvar zvyšku)
  \item
    pre každé $x\in O(a),x>a$  $(x\in O(a),x<a)$ existuje číslo $\vartheta
    (x)\in (a,x)$  $\vartheta (x)\in (a,x)$  $(\vartheta (x)\in (x,a))$ také,
    že:
    \[
      f(x)-T_n(x)=\frac{f^{(n+1)}(\vartheta (x))}{n!}(x-a)(x-\vartheta (x))^n
    \]
    (tzv. Cauchyho tvar zvyšku)
\end{itemize}
\end{veta}

\begin{enumerate}[resume]
  \item \useproblem[diferencialny-pocet]{diferencialny-pocet-392}
  \item \useproblem[diferencialny-pocet]{diferencialny-pocet-393}
\end{enumerate}


\section[Zostrojovanie grafov]{Použitie diferenciálneho počtu pri zostrojovaní grafov funkcií}
Pri zostrojovaní grafu funkcie $f$ postupujeme spravidla nasledovne:
\begin{itemize}
\item
  určíme $D(f)$
\item
  nájdeme všetky hodnoty $x$, pre ktoré $f(x)=0$ (t.j. priesečníky grafu funkcie
  $f$ s osou $Ox$)
\item
  vyšetríme spojitosť funkcie $f$ a jej správanie sa v bodoch nespokojnosti
\item
  zistíme, na ktorých intervaloch je $f$ monotónna, a nájdeme body, v ktorých
  nadobúda lokálne extrémy
\item
  vyšetríme konvexnosť a konkávnosť funkcie $f$, nájdeme inflexné body
\item
  nájdeme asymptoty grafu funkcie (definíciu asymptoty pozri ďalej)
\end{itemize}
Zostrojenie grafu funkcie $f$ môžu uľahčiť niektoré jej špeciálne vlastnosti:
pri párnej alebo nepárnej funkcii stačí zostrojiť graf funkcie $f/D(f)\cap
\interval[open right]{0}{+\infty}$, v prípade periodickej funkcie $f$ graf
funkcie $f/\interval{a}{a + T} \cap D(f)$ a $T$ je niektorá perióda funkcie $f$.

(Nech je daná funkcia $f$, nech $a\in\mathbb{R}$ je hromadný bod množiny
$D(f)\cap (a,+\infty)$) (množiny $D(f)\cap (-\infty,a)$). Ak existuje
$\lim\limits_{x\rightarrow a+}f(x)$  $(\lim\limits_{x\rightarrow a-}f(x))$ a je nevlastná,
nazýva sa priamka $x=ay$ asymptotou bez smernice grafu funkcie $f$.

Nech bod $+\infty$ je hromadný bod definičného oboru funkcie $f$. Priamka
$y=kx+q$ sa nazýva asymptota so smernicou grafu funkcie $f$ v bode $+\infty$, ak
$\lim\limits_{x\rightarrow \infty}(f(x)-kx-q)=0$.

Analogicky sa definuje asymptota so smernicou grafu funkcie $f$ v bode
$-\infty$.

\begin{veta}
Nech bod $+\infty$ je hromadný bod definičného oboru funkcie $f$. Priamka
$y=kx+q$ je asymptotou so smernicou grafu funkcie $f$ v bode $+\infty$ práve
vtedy, keď:
\begin{itemize}
\item
  existuje konečná $\lim\limits_{x\rightarrow \infty} \frac{f(x)}{x}$ a platí
  $\lim\limits_{x\rightarrow \infty} \frac{f(x)}{x}=k$
\item
  existuje konečná $\lim\limits_{x\rightarrow \infty} (f(x)-kx)$ a rovná sa
  číslu $q$

(Analogická veta platí pre asymptotu so smernicou grafu funkcie $f$ v bode $-\infty$.)
\end{itemize}
\end{veta}

Zostrojte grafy nasledujúcich funkcií:
\begin{enumerate}[resume]
  \begin{multicols}{2}
    \item \useproblem[diferencialny-pocet]{diferencialny-pocet-394}
    \item \useproblem[diferencialny-pocet]{diferencialny-pocet-395}
    \item \useproblem[diferencialny-pocet]{diferencialny-pocet-396}
    \item \useproblem[diferencialny-pocet]{diferencialny-pocet-397}
    \item \useproblem[diferencialny-pocet]{diferencialny-pocet-398}
    \item \useproblem[diferencialny-pocet]{diferencialny-pocet-399}
    \item \useproblem[diferencialny-pocet]{diferencialny-pocet-400}
    \item \useproblem[diferencialny-pocet]{diferencialny-pocet-401}
    \item \useproblem[diferencialny-pocet]{diferencialny-pocet-402}
    \item \useproblem[diferencialny-pocet]{diferencialny-pocet-403}
    \item \useproblem[diferencialny-pocet]{diferencialny-pocet-404}
    \item \useproblem[diferencialny-pocet]{diferencialny-pocet-405}
    \item \useproblem[diferencialny-pocet]{diferencialny-pocet-406}
    \item \useproblem[diferencialny-pocet]{diferencialny-pocet-407}
    \item \useproblem[diferencialny-pocet]{diferencialny-pocet-408}
    \item \useproblem[diferencialny-pocet]{diferencialny-pocet-409}
    \item \useproblem[diferencialny-pocet]{diferencialny-pocet-410}
    \item \useproblem[diferencialny-pocet]{diferencialny-pocet-411}
  \end{multicols}
\end{enumerate}


\section{Ďalšie príklady}
\begin{enumerate}[resume]
	\item \useproblem[diferencialny-pocet]{diferencialny-pocet-412}
	\item \useproblem[diferencialny-pocet]{diferencialny-pocet-413}
	\item \useproblem[diferencialny-pocet]{diferencialny-pocet-414}
	\item \useproblem[diferencialny-pocet]{diferencialny-pocet-415}
	\item \useproblem[diferencialny-pocet]{diferencialny-pocet-416}
	\item \useproblem[diferencialny-pocet]{diferencialny-pocet-417}
	\item \useproblem[diferencialny-pocet]{diferencialny-pocet-418}
	\item \useproblem[diferencialny-pocet]{diferencialny-pocet-419}
	\item \useproblem[diferencialny-pocet]{diferencialny-pocet-420}
	\item \useproblem[diferencialny-pocet]{diferencialny-pocet-421}
	\item \useproblem[diferencialny-pocet]{diferencialny-pocet-422}
	\item \useproblem[diferencialny-pocet]{diferencialny-pocet-423}
	\item \useproblem[diferencialny-pocet]{diferencialny-pocet-424}
	\item \useproblem[diferencialny-pocet]{diferencialny-pocet-425}
	\item \useproblem[diferencialny-pocet]{diferencialny-pocet-426}
	\item \useproblem[diferencialny-pocet]{diferencialny-pocet-427}
	\item \useproblem[diferencialny-pocet]{diferencialny-pocet-428}
	\item \useproblem[diferencialny-pocet]{diferencialny-pocet-429}
	\item \useproblem[diferencialny-pocet]{diferencialny-pocet-430}
	\item \useproblem[diferencialny-pocet]{diferencialny-pocet-431}
	\item \useproblem[diferencialny-pocet]{diferencialny-pocet-432}
	\item \useproblem[diferencialny-pocet]{diferencialny-pocet-433}
	\item \useproblem[diferencialny-pocet]{diferencialny-pocet-434}
	\item \useproblem[diferencialny-pocet]{diferencialny-pocet-435}
	\item \useproblem[diferencialny-pocet]{diferencialny-pocet-436}
	\item \useproblem[diferencialny-pocet]{diferencialny-pocet-437}
	\item \useproblem[diferencialny-pocet]{diferencialny-pocet-438}
	\item \useproblem[diferencialny-pocet]{diferencialny-pocet-439}
	\item \useproblem[diferencialny-pocet]{diferencialny-pocet-440}
	\item \useproblem[diferencialny-pocet]{diferencialny-pocet-441}
	\item \useproblem[diferencialny-pocet]{diferencialny-pocet-442}
	\item \useproblem[diferencialny-pocet]{diferencialny-pocet-443}
	\item \useproblem[diferencialny-pocet]{diferencialny-pocet-444}
	\item \useproblem[diferencialny-pocet]{diferencialny-pocet-445}
	\item \useproblem[diferencialny-pocet]{diferencialny-pocet-446}
	\item \useproblem[diferencialny-pocet]{diferencialny-pocet-447}
	\item \useproblem[diferencialny-pocet]{diferencialny-pocet-448}
	\item \useproblem[diferencialny-pocet]{diferencialny-pocet-449}
	\item \useproblem[diferencialny-pocet]{diferencialny-pocet-450}
	\item \useproblem[diferencialny-pocet]{diferencialny-pocet-451}
	\item \useproblem[diferencialny-pocet]{diferencialny-pocet-452}
	\item \useproblem[diferencialny-pocet]{diferencialny-pocet-453}
	\item \useproblem[diferencialny-pocet]{diferencialny-pocet-454}
	\item \useproblem[diferencialny-pocet]{diferencialny-pocet-455}
	\item \useproblem[diferencialny-pocet]{diferencialny-pocet-456}
	\item \useproblem[diferencialny-pocet]{diferencialny-pocet-457}
	\item \useproblem[diferencialny-pocet]{diferencialny-pocet-458}
	\item \useproblem[diferencialny-pocet]{diferencialny-pocet-459}
	\item \useproblem[diferencialny-pocet]{diferencialny-pocet-460}
	\item \useproblem[diferencialny-pocet]{diferencialny-pocet-461}
	\item \useproblem[diferencialny-pocet]{diferencialny-pocet-462}
	\item \useproblem[diferencialny-pocet]{diferencialny-pocet-463}
	\item \useproblem[diferencialny-pocet]{diferencialny-pocet-464}
	\item \useproblem[diferencialny-pocet]{diferencialny-pocet-465}
	\item \useproblem[diferencialny-pocet]{diferencialny-pocet-466}
	\item \useproblem[diferencialny-pocet]{diferencialny-pocet-467}
	\item \useproblem[diferencialny-pocet]{diferencialny-pocet-468}
	\item \useproblem[diferencialny-pocet]{diferencialny-pocet-469}
	\item \useproblem[diferencialny-pocet]{diferencialny-pocet-470}
	\item \useproblem[diferencialny-pocet]{diferencialny-pocet-471}
	\item \useproblem[diferencialny-pocet]{diferencialny-pocet-472}
	\item \useproblem[diferencialny-pocet]{diferencialny-pocet-473}
	\item \useproblem[diferencialny-pocet]{diferencialny-pocet-474}
	\item \useproblem[diferencialny-pocet]{diferencialny-pocet-475}
	\item \useproblem[diferencialny-pocet]{diferencialny-pocet-476}
	\item \useproblem[diferencialny-pocet]{diferencialny-pocet-477}
	\item \useproblem[diferencialny-pocet]{diferencialny-pocet-478}
\end{enumerate}


\subsection{Monotónnosť}
\begin{veta}
Nech funkcia $f: I \rightarrow\mathbb{R}$ je spojitá na intervale $I$ a má
deriváciu v každom jeho vnútornom bode. Potom
\begin{itemize}
\item
  ak $f'>0,(f'\geq 0)$ vnútri intervalu $I$ (t.j. ak pre každý vnútorný bod
  $x$ intervalu $I$ platí $f'(x)>0,(f'(x)\geq 0)$), tak $f$ je rastúca
  (neklesajúca) na $I$
\item
  ak $f'<0,(f'\leq 0)$ vnútri intervalu $I$, tak $f$ je klesajúca (nerastúca) na
  $I$
\end{itemize}
\end{veta}

\begin{enumerate}[resume]
  \item \useproblem[diferencialny-pocet]{diferencialny-pocet-349}
  \item \useproblem[diferencialny-pocet]{diferencialny-pocet-350}
  \item \useproblem[diferencialny-pocet]{diferencialny-pocet-351}
\end{enumerate}

\begin{veta}
Nech funkcie $f,g$ sú $n$-krát diferencovateľné na intervale $I$, nech v bode
$a\in I$ platí $f(a)=g(a),f'(a)=g'(a),...,f^{(n-1)}(a)=g^{(n-1)}(a)$ (teda ak
$n=1$, predpokladáme len $f(a)=g(a)$). Potom
\begin{enumerate}
\item
  ak $f^{(n)}(x)>g^{(n)}(x)$, pre všetky $x\in I \cap \interval[open]{a}{\infty}$, tak
  $f(x)>g(x)$ Pre všetky $x\in I \cap \interval[open]{a}{\infty}$ (pritom samozrejme
  predpokladáme, že $I \cap \interval[open]{a}{\infty} \neq \emptyset$ )
\item
  \begin{itemize}
    \item
      ak $f^{(n)}(x)>g^{(n)}(x)$ pre všetky $x\in I \cap
      \interval[open]{-\infty}{a}$ a $n$ je párne, tak $f(x)>g(x)$ pre všetky
      $x\in I \cap \interval[open]{-\infty}{a}$;
    \item
      ak $f^{(n)}(x)>g^{(n)}(x)$ pre všetky $x\in I \cap
      \interval[open]{-\infty}{a}$ a $n$ je nepárne, tak $f(x)<g(x)$ pre všetky
      $x\in I \cap \interval[open]{-\infty}{a}$ (pritom predpokladáme $I \cap
      \interval[open]{-\infty}{a}\neq \emptyset$)
  \end{itemize}
\end{enumerate}
\end{veta}

\begin{enumerate}[resume]
  \item \useproblem[diferencialny-pocet]{diferencialny-pocet-352}
  \item \useproblem[diferencialny-pocet]{diferencialny-pocet-353}
  \item \useproblem[diferencialny-pocet]{diferencialny-pocet-354}
  \item \useproblem[diferencialny-pocet]{diferencialny-pocet-355}
  \item \useproblem[diferencialny-pocet]{diferencialny-pocet-356}
\end{enumerate}

\subsection{Konvexnosť a konkávnosť. Inflexné body}
Funkcia $f$ sa nazýva rýdzo konvexná (konvexná) na intervale $I \subset D(f)$,
ak platí $(*)$
\[
  \forall x,y\in I,x\neq y \forall p,q>0,p+q=1;f(px+qy)<pf(x)+qf(y)
\]
\[
  (\forall x,y\in I,x\neq y \forall p,q>0,p+q=1;f(px+qy)\leq pf(x)+qf(y))
\]
Funkcia $f$ sa nazýva rýdzo konkávna (konkávna) na intervale $I \subset D(f)$,
ak platí
\[
  \forall x,y\in I,x\neq y \forall p,q>0,p+q=1;f(px+qy)>pf(x)+qf(y)
\]
\[
  (\forall x,y\in I,x\neq y \forall p,q>0,p+q=1;f(px+qy)\geq pf(x)+qf(y))
\]
(Geometricky možno výrok $(*)$ interpretovať takto: pre ľubovoľné $2$ čísla
$x,y\in I,x<y$, leží úsečka spájajúca body $(x,f(x))$ nad grafom funkcie
$f/(x,y)$.)

\begin{veta}
Nech funkcia $f$ je spojitá na intervale $I$ a dvakrát diferencovateľná v každom
jeho vnútornom bode. Potom
\begin{itemize}
\item
  $f''>0,(f''\geq 0)$ vnútri intervalu $I$, tak $f$ je rýdzo konvexná (konvexná)
  na $I$
\item
  $f''<0,(f''\leq 0)$ vnútri intervalu $I$, tak $f$ je rýdzo konkávna (konkávna)
  na $I$
\end{itemize}
Vnútorný bod $a$ množiny $D(f)$ sa nazýva inflexný bod funkcie $f$, ak $f$ má v
bode $a$ deriváciu a existuje $\varepsilon >0$ tak, že funkcia $f$ je rýdzo
konvexná na jednej z množín $\interval[open left]{a-\varepsilon}{a}$,
$\interval[open right]{a}{a + \varepsilon}$ a rýdzo konkávna na druhej z nich.
\end{veta}

\begin{veta}
Nech funkcia $f$ je trikrát diferencovateľná v bode $a$ a dvakrát
diferencovateľná v niektorom jeho okolí. Ak $f''(a)=0,f''(a)\neq 0$, tak $a$ je
inflexný bod funkcie $f$.
\end{veta}

\textit{Poznámka:}
Existujú aj iné definície rýdzej konvexnosti, rýdzej konkávnosti a inflexného
bodu, ktoré nie sú ekvivalentné tu uvedeným. Všetky v matematickej literatúre
používané definície týchto pojmov sú však volené tak, že vety $13$ a $14$
zostanú v platnosti.

\begin{enumerate}[resume]
  \item \useproblem[diferencialny-pocet]{diferencialny-pocet-357}
  \item \useproblem[diferencialny-pocet]{diferencialny-pocet-358}
  \item \useproblem[diferencialny-pocet]{diferencialny-pocet-359}
  \item \useproblem[diferencialny-pocet]{diferencialny-pocet-360}
  \item \useproblem[diferencialny-pocet]{diferencialny-pocet-361}
  \item \useproblem[diferencialny-pocet]{diferencialny-pocet-362}
  \item \useproblem[diferencialny-pocet]{diferencialny-pocet-363}
\end{enumerate}

\textit{Poznámka:}
Na vlastnostiach odvodených v príklade $362$ sa zakladajú nasledujúce definície,
odlišné od definícií používaných v tomto odstavci:
\begin{itemize}
\item
  Funkcia $f$ diferencovateľná na intervale $I$ sa nazýva rýdzo konvexná (rýdzo
  konkávna) na $I$, ak pre každé $a\in I$ platí: na množine $I \setminus \{a\}$
  leží graf funkcie $f$ nad (pod) svojou dotyčnicou v bode $(a,f(a))$.
\item
  Bod $a$ sa nazýva inflexný bod funkcie $f$, ak existuje konečná $f'(a)$ a graf
  funkcie $f$ prechádza v bode $a$ z jednej strany svojej dotyčnice na druhú.
\end{itemize}
Prvá z týchto definícií predstavuje užšie chápanie pojmov rýdzo konvexná a rýdzo
konkávna funkcia (to sme tu nedokazovali, ale môžete to skúsiť sami), druhá
naopak širšie chápanie pojmu inflexný bod (to ukazujú príklady $362$ a $363$).

\subsection{Extrémy}
Nech definičným oborom funkcie $f$ je interval $I$. Hovoríme, že funkcia $f$ má
v bode $a$ $I$ lokálne maximum (lokálne minimum), ak existuje okolie $O(a)$ bodu
$a$ tak, že platí:
\[
  \forall x\in (O(a)\setminus \{a\})\cap I:f(x)\leq f(a)
\]
\[
  (\forall x\in (O(a)\setminus \{a\})\cap I:f(x)\geq f(a))
\]
Definíciu ostrého lokálneho maxima (ostrého lokálneho minima) dostaneme, ak v
predchádzajúcej definícii zameníme znak $\leq$ ($\geq$) znakom $<(>)$. Lokálne
maximá a lokálne minimá sa súhrnne nazývajú lokálnymi extrémami.

\begin{veta}
Ak funkcia $f$ má lokálny extrém vo vnútornom bode $a$ svojho definičného oboru,
tak buď neexistuje vlastná ani nevlastná $f'(a)$, alebo $f'(a)=0$. Bod $a$ sa
nazýva stacionálny bod funkcie $f$, ak $f'(a)=0$. Pri hľadaní lokálnych extrémov
funkcie $f:I\rightarrow\mathbb{R}$, treba teda vyšetriť:
\begin{enumerate}
\item všetky jej stacionárne body
\item všetky body $a\in I$, v ktorých neexistuhe $f'(a)$
\item všetky body $a\in I$, ktoré nie sú vnútornými bodmi intervalu $I$
\end{enumerate}
\end{veta}

\begin{veta}
Ak funkcia $f$ je dvakrát diferencovateľná vo vnútornom bode $a$ množiny $D(f)$
a platí $f'(a)=0,f''(a)>0$  $(f'(a)=0,f''(a)<0)$, tak $f$ má v bode $a$ ostré
lokálne minimum (ostré lokálne maximum).
\end{veta}

\begin{veta}
Nech funkcia $f$ je $n$-krát $(n\geq 2)$ diferencovateľná vo vnútornom bode $a$
množiny $D(f)$, nech $f'(a)=...=f^{(n-1)}(a)=0,f^{(n)}\neq 0$.

Ak $n$ je párne a $f^{(n)}(a)>0$  $(f^{(n)}(a)<0)$, tak funkcia $f$ má v bode
$a$ ostré lokálne minimum (ostré lokálne maximum).

Ak $n$ je nepárne, nemá funkcia $f$ v bode $a$ lokálny extrém.
\end{veta}

\begin{enumerate}[resume]
  \item \useproblem[diferencialny-pocet]{diferencialny-pocet-364}
  \item \useproblem[diferencialny-pocet]{diferencialny-pocet-365}
  \item \useproblem[diferencialny-pocet]{diferencialny-pocet-366}
  \item \useproblem[diferencialny-pocet]{diferencialny-pocet-367}
  \item \useproblem[diferencialny-pocet]{diferencialny-pocet-368}
  \item \useproblem[diferencialny-pocet]{diferencialny-pocet-369}
  \item \useproblem[diferencialny-pocet]{diferencialny-pocet-370}
  \item \useproblem[diferencialny-pocet]{diferencialny-pocet-371}
  \item \useproblem[diferencialny-pocet]{diferencialny-pocet-372}
  \item \useproblem[diferencialny-pocet]{diferencialny-pocet-373}
  \item \useproblem[diferencialny-pocet]{diferencialny-pocet-374}
  \item \useproblem[diferencialny-pocet]{diferencialny-pocet-375}
  \item \useproblem[diferencialny-pocet]{diferencialny-pocet-376}
\end{enumerate}

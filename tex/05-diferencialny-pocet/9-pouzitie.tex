Pri zostrojovaní grafu funkcie $f$ postupujeme spravidla nasledovne:
\begin{itemize}
\item
  určíme $D(f)$
\item
  nájdeme všetky hodnoty $x$, pre ktoré $f(x)=0$ (t.j. priesečníky grafu funkcie
  $f$ s osou $Ox$)
\item
  vyšetríme spojitosť funkcie $f$ a jej správanie sa v bodoch nespokojnosti
\item
  zistíme, na ktorých intervaloch je $f$ monotónna, a nájdeme body, v ktorých
  nadobúda lokálne extrémy
\item
  vyšetríme konvexnosť a konkávnosť funkcie $f$, nájdeme inflexné body
\item
  nájdeme asymptoty grafu funkcie (definíciu asymptoty pozri ďalej)
\end{itemize}
Zostrojenie grafu funkcie $f$ môžu uľahčiť niektoré jej špeciálne vlastnosti:
pri párnej alebo nepárnej funkcii stačí zostrojiť graf funkcie $f/D(f)\cap
\interval[open right]{0}{+\infty}$, v prípade periodickej funkcie $f$ graf
funkcie $f/\interval{a}{a + T} \cap D(f)$ a $T$ je niektorá perióda funkcie $f$.

(Nech je daná funkcia $f$, nech $a\in\mathbb{R}$ je hromadný bod množiny
$D(f)\cap (a,+\infty)$) (množiny $D(f)\cap (-\infty,a)$). Ak existuje
$\lim\limits_{x\rightarrow a+}f(x)$  $(\lim\limits_{x\rightarrow a-}f(x))$ a je nevlastná,
nazýva sa priamka $x=ay$ asymptotou bez smernice grafu funkcie $f$.

Nech bod $+\infty$ je hromadný bod definičného oboru funkcie $f$. Priamka
$y=kx+q$ sa nazýva asymptota so smernicou grafu funkcie $f$ v bode $+\infty$, ak
$\lim\limits_{x\rightarrow \infty}(f(x)-kx-q)=0$.

Analogicky sa definuje asymptota so smernicou grafu funkcie $f$ v bode
$-\infty$.

\begin{veta}
Nech bod $+\infty$ je hromadný bod definičného oboru funkcie $f$. Priamka
$y=kx+q$ je asymptotou so smernicou grafu funkcie $f$ v bode $+\infty$ práve
vtedy, keď:
\begin{itemize}
\item
  existuje konečná $\lim\limits_{x\rightarrow \infty} \frac{f(x)}{x}$ a platí
  $\lim\limits_{x\rightarrow \infty} \frac{f(x)}{x}=k$
\item
  existuje konečná $\lim\limits_{x\rightarrow \infty} (f(x)-kx)$ a rovná sa
  číslu $q$

(Analogická veta platí pre asymptotu so smernicou grafu funkcie $f$ v bode $-\infty$.)
\end{itemize}
\end{veta}

Zostrojte grafy nasledujúcich funkcií:
\begin{enumerate}[resume]
  \begin{multicols}{2}
    \item \useproblem[diferencialny-pocet]{diferencialny-pocet-394}
    \item \useproblem[diferencialny-pocet]{diferencialny-pocet-395}
    \item \useproblem[diferencialny-pocet]{diferencialny-pocet-396}
    \item \useproblem[diferencialny-pocet]{diferencialny-pocet-397}
    \item \useproblem[diferencialny-pocet]{diferencialny-pocet-398}
    \item \useproblem[diferencialny-pocet]{diferencialny-pocet-399}
    \item \useproblem[diferencialny-pocet]{diferencialny-pocet-400}
    \item \useproblem[diferencialny-pocet]{diferencialny-pocet-401}
    \item \useproblem[diferencialny-pocet]{diferencialny-pocet-402}
    \item \useproblem[diferencialny-pocet]{diferencialny-pocet-403}
    \item \useproblem[diferencialny-pocet]{diferencialny-pocet-404}
    \item \useproblem[diferencialny-pocet]{diferencialny-pocet-405}
    \item \useproblem[diferencialny-pocet]{diferencialny-pocet-406}
    \item \useproblem[diferencialny-pocet]{diferencialny-pocet-407}
    \item \useproblem[diferencialny-pocet]{diferencialny-pocet-408}
    \item \useproblem[diferencialny-pocet]{diferencialny-pocet-409}
    \item \useproblem[diferencialny-pocet]{diferencialny-pocet-410}
    \item \useproblem[diferencialny-pocet]{diferencialny-pocet-411}
  \end{multicols}
\end{enumerate}

\begin{veta}
\textit{(Rolle).}
Nech funkcia $f:\interval{a}{b} \rightarrow \mathbb{R}$ vyhovuje nasledujúcim
podmienkam:
\begin{itemize}
\item $f$ je spojitá na intervale $\interval{a}{b}$
\item
  v každom bode $x\in \interval[open]{a}{b}$ existuje vlastná alebo nevlastná
  $f'(x)$
\item $f(a)=f(b)$
\end{itemize}
Potom existuje bod $c\in \interval[open]{a}{b}$, v ktorom $f'(c)=0$.
\end{veta}

\begin{enumerate}[resume]
  \item \useproblem[diferencialny-pocet]{diferencialny-pocet-329}
  \item \useproblem[diferencialny-pocet]{diferencialny-pocet-330}
  \item \useproblem[diferencialny-pocet]{diferencialny-pocet-331}
  \item \useproblem[diferencialny-pocet]{diferencialny-pocet-332}
  \item \useproblem[diferencialny-pocet]{diferencialny-pocet-333}
  \item \useproblem[diferencialny-pocet]{diferencialny-pocet-334}
  \item \useproblem[diferencialny-pocet]{diferencialny-pocet-335}
\end{enumerate}

\begin{veta}
\textit{(Lagrangeova veta o strednej hodnote.)}
Nech funkcia $f:\interval{a}{b} \rightarrow\mathbb{R}$ vyhovuje nasledujúcim
podmienkam:
\begin{itemize}
\item $f$ je spojitá;
\item
  v každom bode $x\in \interval[open]{a}{b}$ existuje vlastná alebo nevlastná
  $f'(x)$. Potom existuje bod $c\in \interval[open]{a}{b}$, v ktorom
  $f'(c)=\frac{f(b)-f(a)}{b-a}.$
\end{itemize}
\end{veta}

\begin{enumerate}[resume]
  \item \useproblem[diferencialny-pocet]{diferencialny-pocet-336}
  \item \useproblem[diferencialny-pocet]{diferencialny-pocet-337}
  \item \useproblem[diferencialny-pocet]{diferencialny-pocet-338}
  \item \useproblem[diferencialny-pocet]{diferencialny-pocet-339}
  \item \useproblem[diferencialny-pocet]{diferencialny-pocet-340}
  \item \useproblem[diferencialny-pocet]{diferencialny-pocet-341}
  \item \useproblem[diferencialny-pocet]{diferencialny-pocet-342}
  \item \useproblem[diferencialny-pocet]{diferencialny-pocet-343}
  \item \useproblem[diferencialny-pocet]{diferencialny-pocet-344}
  \item \useproblem[diferencialny-pocet]{diferencialny-pocet-345}
\end{enumerate}

\begin{veta}
\textit{(Cauchyho veta o strednej hodnote).}
Nech funkcie $f: \interval{a}{b} \rightarrow \mathbb{R}$ a $g: \interval{a}{b}
\rightarrow \mathbb{R}$ spĺňajú nasledujúce podmienky:
\begin{itemize}
\item
  $f$ a $g$ sú spojité na intervale $\interval{a}{b}$
\item
  v každom bode $x\in \interval[open]{a}{b}$ existujú vlastná alebo nevlastná
  $f'(x)$ a vlastná $g'(x)$
\end{itemize}
Potom existuje bod $c\in \interval[open]{a}{b}$, v ktorom platí
\[
  (f(b)-f(a))g'(c)=(g(b)-g(a))f'(c)
\]
Ak sú naviac splnené predpoklaady
\begin{itemize}
\item $(f'(x))^2+(g'(x))^2>0$ Pre každé $x\in \interval[open]{a}{b}$
\item $g(b)\neq g(a)$
\end{itemize}
možno uvedenú rovnosť písať v tvare
\[
  \frac{f(b)-f(a)}{g(b)-g(a)}=\frac{f'(c)}{g'(c)}
\]
\end{veta}

\begin{enumerate}[resume]
  \item \useproblem[diferencialny-pocet]{diferencialny-pocet-346}
  \item \useproblem[diferencialny-pocet]{diferencialny-pocet-347}
  \item \useproblem[diferencialny-pocet]{diferencialny-pocet-348}
\end{enumerate}

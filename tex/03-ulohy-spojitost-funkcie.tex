\begin{defproblem}{s-funkcie-1}
Pomocou kvantifikátorov zapíšte tvrdenie "funkcia $f$ nie je spojitá v bode $a \in D(f)$"!
  \end{defproblem}

  \begin{defproblem}{s-funkcie-2}
  Vyšetrite spojitosť nasledujúcich funkcií v daných bodoch:
  \begin{enumerate}
  \item $f(x) = \left\{ \begin{array}{r@{\quad}c}
    \frac{\sin x}{x},& $ak $ x \neq 0 \\
    1, &  $ak $ x = 0 \\ \end{array} \right.
    $ v bode $0$;
  \item $f(x) = \left\{ \begin{array}{r@{\quad}c}
    \frac{\sin x}{|x|},& $ak $ x \neq 0 \\
    1, &  $ak $ x = 0 \\ \end{array} \right.
    $ v bode $0$;
  \item $f(x) = \left\{ \begin{array}{r@{\quad}c}
    \frac{1}{(1+x)^2},& $ak $ x \neq 0 \\
    1, &  $ak $ x = 0 \\ \end{array} \right.
    $ v bode $-1$;
  \item $f(x)=sgn x$ v bode $0$;
  \item $f(x)=x \cdot sgn x$ v bode $0$;
  \end{enumerate}

  \end{defproblem}

  \begin{defproblem}{s-funkcie-3}
  Nasledujúce funkcie nie sú definované v bode $a$. Určite hodnotu     $f(a)$ tak, aby takto dodefinovaná funkcia $f$ bola spojitá v bode $a$:
  \begin{enumerate}
  \item $f(x)=\frac{\sqrt{1+x}-1}{\sqrt[3]{1+x}-x}$, $a=0$;
  \item $f(x)=(1+x)^{\frac{1}{2}x}$, $a=0$;
  \item $f(x)=e^{-\frac{1}{x^2}}$, $a=0$;
  \item $f(x)=\frac{1}{1+e^{\frac{1}{x-1}}},x<1$, $a=1$;
  \item $f(x)=(x+3) \cdot \sin \frac{1}{x+3}$, $a=-3$.
\end{enumerate}
  \end{defproblem}

  \begin{defproblem}{s-funkcie-4}
  Vyšetrite spojitosť nasledujúcich funkcií:
  \begin{enumerate}
  \item $f(x)=\frac{x^2-4x+7}{x^3+5x^2+6x}$;
  \item $f(x) = \left\{ \begin{array}{r@{\quad}c}
    \frac{1-e^x}{x},& $ak $ x \leq 0 \\
    2x-1, &  $ak $ x \geq 0 \\ \end{array} \right.
    $;
  \item $f(x)=sgn (x \cdot (1-x^2))$;
  \item $f(x)=x \cdot [x]$;
  \item $f(x)=\lim_{n \rightarrow \infty} \frac{1}{1+x^n},x \geq 0$;
  \item $f(x)=\lim_{n \rightarrow \sqrt[n]{1+x^{2n}}}$;
  \item $f(x)=\sqrt{-\sin^2 x}$.
  \end{enumerate}
  \end{defproblem}

  \begin{defproblem}{s-funkcie-5}
  Nech funkcie $f,g$ sú definované na $\mathbb{R}$ a funkcia $f+g$ je spojitá v bode $a$. Potom nastane práve jedna z nasledujúcich možností:
  \begin{enumerate}
  \item funkcia $f$ aj funkcia $g$ sú spojité v bode $a$;
  \item ani funkcia $f$, ani funkcia $g$ nie je spojitá v bode $a$.
  \end{enumerate}
  Dokážte: obidve uvedené možnosti dokumentujte na príkladoch!
  \end{defproblem}

  \begin{defproblem}{s-funkcie-6}
  Nech $f: \mathbb{R} \rightarrow \mathbb{R}$ je spojitá funkcia. Potom funkcia $f \cdot \zeta$ je spojitá práve v bodoch množiny $\{x\in \mathbb{R}: f(x)=0\}$. Dokážte!
  \end{defproblem}

  \begin{defproblem}{s-funkcie-7}
  Možno tvrdiť, že funkcia $g \circ f$ nie je spojitá v bode $x_{0}$, ak $f$ je spojitá v bode $x_{0}$ a $g$ nie je spojitá v bode $f(x_0) \in D(g)$?
  \end{defproblem}

  \begin{defproblem}{s-funkcie-8}
  Nech funkcie $f,g$ sú spojité v bode $a$ $(\in D(f)\cap D(g))$. Potom aj funkcie $|f|,\max \{f,g\},\min \{f,g\}$ sú spojité v bode $a$. Dokážte!
  \end{defproblem}

  \begin{defproblem}{s-funkcie-9}
  Vyšetrite spojitosť Dirichletovej funkcie $\zeta (x)$!
  \end{defproblem}

  \begin{defproblem}{s-funkcie-10}
  Dokážte, že Riemannova funkcia $f$, definovaná predpisom $f(x)=0$, ak $x \in \mathbb{R} \setminus \mathbb{Q}$ a $f(x)=\frac{1}{n}$, ak $x=\frac{m}{n}$, kde $\mathbb{N}, m \in \mathbb{Z}$ sú nesúdeliteľné čísla, je spojitá v každom iracionálnom čísle a nespojitá v každom racionálnom čísle.
  \end{defproblem}

  \begin{defproblem}{s-funkcie-11}
  Nech funkcia $f$ je spojitá na intervale $\langle a,b \rangle$. Potom $$\sup \{f(x); x \in \langle a,b \rangle\}= \sup \{f(x); x \in \langle a,b \rangle \cap \mathbb{Q}\}.$$ Dokážte!
  \end{defproblem}

  \begin{defproblem}{s-funkcie-12}
  \begin{enumerate}
  \item $f(x)=\frac{x}{(1-x)^2}$;
  \item $f(x)=\frac{x^2-1}{x^3-3x+2}$;
  \item $f(x)=\frac{\sin x}{x}$;
  \item $f(x)=\frac{\frac{1}{x}-\frac{1}{x+1}}{\frac{1}{x-1}-\frac{1}{x}}$;
  \item $f(x)=\sqrt{\frac{1-\cos \pi x}{4-x^2}}$;
  \item $f(x)=\cos^2 \frac{1}{x}$;
  \item $f(x)=\sqrt{x}\cdot \arctan \frac{1}{x}$;
  \item $f(x)=\frac{1}{\ln x}$;
  \item $f(x) = \left\{ \begin{array}{r@{\quad}c}
    x,& $ak$ |x| \leq 1 \\
    1, &  $ak$ |x| \geq 1 \\ \end{array} \right.
    $;
  \item $f(x)=x \cdot [\frac{1}{x}]$;
  \item $g(g(g(x)))$, kde $g(x)=\frac{1}{1-x}$;
  \item
    $$f(x)=g(\varphi(x))$$
    $$g(x) =
      \left\{
        \begin{array}{r@{\quad}c}
          \sin \pi x, & \text{ak } x \in \mathbb{Q} \\
          0,          & \text{ak } x \in \mathbb{R} \setminus \mathbb{Q} \\
        \end{array}
      \right.
    $$
  \item
    $$f(x)=g(\varphi(x))$$
    $$g(x) =
      \left\{
        \begin{array}{r@{\quad}c}
          x,   & \text{ak } 0 \leq x \leq 1 \\
          2-x, & \text{ak } 1<x<2  \\
        \end{array}
      \right.
    $$
    $$\varphi(x) =
      \left\{
        \begin{array}{r@{\quad}c}
          x,   & \text{ak } x \in \mathbb{Q} \\
          2-x, &  \text{ak } x \in \mathbb{R} \setminus \mathbb{Q} \\
        \end{array}
      \right.
    $$
  \end{enumerate}
  \end{defproblem}

  \begin{defproblem}{s-funkcie-13}
  \begin{enumerate}
  \item Dokážte, že monotónna funkcia definovaná na $\mathbb{R}$ môže mať len body nespojitosti $1.$ druhu.
  \item Dokážte, že množina bodov nespojitosti neklesajúcej funkcie $f$ definovanej na intervale $I$ je spočítateľná.

  (\textit{Návod:} Stačí dokázať, že množina $S$ všetkých bodov nespojitosti funkcie $f$ ležia vnútri intervalu $I$ - každý z nich je bodom nespojitosti $1.$ druhu - je spočítateľná. Ak každému bodu $a \in S$ priradíme interval ($\lim_{x \rightarrow a-}f(x),\lim_{x \rightarrow a+}f(x)$)), dostaneme systém po dvochdisjunktných intervalov; tento systém je spočítateľný - pozri príklad $99$. )

  \end{enumerate}
  \end{defproblem}

  \begin{defproblem}{s-funkcie-14}
  Dokážte, že existuje
  \begin{enumerate}
  \item $x \in (0,1)$, pre ktoré platí $x^5+x^4+x^3+x^2-x-1=0$;
  \item $x \in \mathbb{R}$, pre ktoré platí $x=\cos x$;
  \item aspoň jedno riešenie rovnice $P(x)=0$, kde $P$ je polynóm nepárneho stupňa.
  \end{enumerate}
  \end{defproblem}

  \begin{defproblem}{s-funkcie-15}
  Nech funkcia $f$ je spojitá na intervale $(a,b)$, nech $x_1,x_2,...,x_n \in (a,b)$. Potom existuje $o \in (a,b)$ tak, že
  $$f(c)=\frac{1}{n}(f(x_1)+...+f(x_n)).$$ Dokážte!
  \end{defproblem}

  \begin{defproblem}{s-funkcie-16}
  Nech $f:\langle 0,1 \rangle \rightarrow \langle 0,1 \rangle$ je spojitá funkcia. Potom pre niektoré $c \in \langle 0,1 \rangle$ platí $f(c)=c$. Dokážte!
  \end{defproblem}

  \begin{defproblem}{s-funkcie-17}
  Ak pre spojitú unkciu $f: \mathbb{R} \rightarrow \mathbb{R}$ platí $f(\mathbb{Q}= \{0\})$, tak aj $f(\mathbb{R})=\{0\}$. Dokážte!
  \end{defproblem}

  \begin{defproblem}{s-funkcie-18}
  Nech funkcia $f: \mathbb{R} \rightarrow \mathbb{R}$ má túto vlastnosť: pre každý interval $I \subset \mathbb{R}$ je $I \setminus f(\mathbb{R} \neq \emptyset)$. Potom je $f$ konštantná na $\mathbb{R}$ alebo nespojitá v každom bode $x \in \mathbb{R}$. Dokážte!
  \end{defproblem}

  \begin{defproblem}{s-funkcie-19}
  Ak je funkcia $f$ prostá a spojitá na intervale $I$, tak je tam rýdyomonotónna. Dokážte!
  \end{defproblem}

  \begin{defproblem}{s-funkcie-20}
  Nech spojitá funkcia $f$ nadobúda na intervale $\langle a,b \rangle$ len kladné hodnoty. Potom existuje $\mu > 0$ tak, že $f(x) \geq \mu$ platí pre každé $x \in \langle a,b \rangle$. Dokážte!
  \end{defproblem}

  \begin{defproblem}{s-funkcie-21}
  Ak funkcia $f: (a,b) \rightarrow \mathbb{R}$ $(a,b\in \mathbb{R^*})$ je spojitá a existujú konečné $\lim_{x \rightarrow a}f(x),\lim_{x \rightarrow b}f(x)$, tal $f$ je ohraničená. Dokážte!
  \end{defproblem}

  \begin{defproblem}{s-funkcie-22}
  Ak $P$ je polynóm párneho stupňa, tak existuje $\max_{x \in \mathbb{R}} P(x)$. Dokážte! Aký je koeficient pri člene s najvyššou mocninou, ak existuje $\min_{x \in \mathbb{R}} P(x)$?
  \end{defproblem}

  \begin{defproblem}{s-funkcie-23}
  Ak $f$ je spojitá na intervale $\langle0,\infty)$ a $\lim_{x \rightarrow \infty} f(x)=f(0)$, tak existuje $\max_{x \geq 0} f(x)$ aj $\min_{x \geq 0}f(x)$. Dokážte!
  \end{defproblem}

  \begin{defproblem}{s-funkcie-24}
  Vetu $5$ možno dokázať na základe nasledujúcich faktov:
  \begin{enumerate}
  \item ak funkcia $f$ je spojitá na kompaktnej množine $A$, tak $f(A)$ je kompaktná množina;
  \item každá kompakná množina je ohraničená a obsahuje svoje supremum a infimum.
  \end{enumerate}
  Dokážte tieto tvrdenia!
  \end{defproblem}

  \begin{defproblem}{s-funkcie-25}
  Nech $f: \mathbb{R} \rightarrow \mathbb{R}$ má len body nespojitosti $1.$ druhu. Potom $f$ je ohraničená na každom ohraničenom intervale. Dokážte!
  \end{defproblem}

  \begin{defproblem}{s-funkcie-26}
  Existuje spojitá a ohraničená funkcia $f:(0,1 \rangle \rightarrow \mathbb{R}$ taká, že neexistuje $\max_{x \in (0,1 \rangle} f(x)$ ani $\min_{x \in (0,1 \rangle} f(x)$?
  \end{defproblem}

  \begin{defproblem}{s-funkcie-27}
  Pomocou kvamtifikátorov zapíšeme tvrdenie "Funkcia $f$ je spojitá na intervale $(a,b)$, ale nie je tam rovnomerne spojitá."!
  \end{defproblem}

  \begin{defproblem}{s-funkcie-28}
  Rozhodnite, či je funkcia $f$ rovnomerne spojitá na množine $A$. Svoje tvrdenie dokážte na základe definície!
  \begin{enumerate}
  \item $f(x)=5x-3$,$A=\mathbb{R}$;
  \item $f(x)=x^2-2x-1$,$A=\langle -2,5 \rangle$;
  \item $f(x)=\cos \frac{1}{x}$,$A=(0,1)$;
  \item $f(x)=\frac{1}{x}$,$A=(\frac{1}{10},1)$;
  \item $f(x)=\sin x^2$,$A=\langle 0,\infty)$;
  \item $f(x)=\sqrt{x}$,$A=\langle 0,\infty)$;
  \item $f(x)=x+\sin x$,$A=\mathbb{R}$.
  \end{enumerate}
  \end{defproblem}

  \begin{defproblem}{s-funkcie-29}
  Funkcia $f$ je rovnomerne spojitá na ohraničenom intervale $I$ práve vtedy, keď pre ľubovoľné $\varepsilon >0$ existuje spojitá po častiach lineárna funkcia $\varphi$ taká, že pre všetky $x \in I$ platí $|f(x)-\varphi(x)|< \varepsilon$. Dokážte! (Spojitá funkcia $\varphi$ definovaná na ohraničenom intervale $I$ sa nazýva po častiach lineárna, ak existuje konečný počet po dvoch disjunktných intervlov $I_1,I_2,...,I_n$ tak, že $I_1 \cup I_2 \cup ... \cup I_n=I$ a funkcie $\frac{\varphi}{I_i}$ $(i=1,...,n)$ sú lineárne; t.j. grafom je "lomená čiara".)
  \end{defproblem}

  \begin{defproblem}{s-funkcie-30}
  Pre funkciu $F$ nájdite spojitú po častiach lineárnu funkciu $f$ takú, aby pre všetky $x \in \langle a,b \rangle$ platilo $|F(x)-f(x)|<0,1$ ak:
  \begin{enumerate}
  \item $F(x)=x^2, \langle a,b \rangle = \langle -1,1 \rangle$;
  \item $F(x)=\frac{1}{x}, \langle a,b \rangle = \langle \frac{2}{3},2 \rangle$.
  \end{enumerate}
  \end{defproblem}

  \begin{defproblem}{s-funkcie-31}
  Rozhodnite, či je funkcia $f$ rovnomerne spojitá na množine $A$:
  \begin{enumerate}
  \item $f(x)=\frac{x}{4-x^2}$, $A=\langle -1,1 \rangle$;
  \item $f(x)=\frac{\sin x}{x}$, $A=(0, \pi \rangle$;
  \item $f(x)=\sqrt[3]{x}$, $A=\langle 0,\infty)$;
  \item  $f(x) = \left\{ \begin{array}{r@{\quad}c}
    x\cdot \sin \frac{1}{x}, & $ak $ x \in (0,\pi \rangle \\
    0, &  $ak $ x=0 \\ \end{array} \right.
    $, $A= \langle 0,\pi \rangle$;
  \item $f(x)=x \cdot \sin x$, $A= \langle 0,\infty)$;
  \item $f(x)=\frac{x^6-1}{\sqrt{1-x^4}}$, $A=(-1,1)$;
  \item $f(x)=\ln x$, $A=\langle 1, \infty)$.
  \end{enumerate}
  \end{defproblem}

\begin{defproblem}{s-funkcie-32}
  Ak funkcia $f$ je spojitá na intervale
  $\interval[open right]{0}{\infty}$ a $\exists$ konečná
  $\lim\limits_{x \rightarrow \infty} f(x)$, tak $f$ je rovnomerne spojitá na
  $\interval[open right]{0}{\infty}$. Dokážte!
\end{defproblem}

\begin{defproblem}{s-funkcie-33}
  Každá spojitá periodická funkcia $f:\mathbb{R} \rightarrow \mathbb{R}$ je
  rovnomerne spojitá. Dokážte!
\end{defproblem}

\begin{defproblem}{s-funkcie-34}
  Nech funkcia $f$ je definovaná na ohraničenom intervale $(a,b)$, nech
  \begin{enumerate}
  \item $\lim_{x \rightarrow b-}f(x)=+\infty$;
  \item $\lim_{x \rightarrow b-}f(x)$ neexistuje.
  \end{enumerate}
  Potom $f$ nie je rovnomerne spojitá na $(a,b)$. Dokážte!
\end{defproblem}

\begin{defproblem}{s-funkcie-35}
  Spojitá funkcia $f$ definovaná na ohraničenom intervale
  $\interval[open]{a}{b}$ je rovnomerne spojitá na $\interval[open]{a}{b}$
  práve vtedy, keď existujú konečné
  $\lim\limits_{x \rightarrow a^+}f(x)$, $\lim\limits_{x \rightarrow b^-}f(x)$. Dokážte!
\end{defproblem}

\begin{defproblem}{s-funkcie-36}
\begin{enumerate}
\item Ak $f$ je rovnomerne spojitá na ohraničenom intervale $(a,b)$, tak $f$ je na $(a,b)$ ohraničená. Dokážte!
  \item Uveďte príklad funkcie, ktorá je spojitá a ohraničená na ohraničenom intervale, ale nie je tam rovnomerne spojitá!
\end{enumerate}
\end{defproblem}

\begin{defproblem}{s-funkcie-37}
Vyšetrite spojitosť nasledujúcich funkcií, určte charakter bodov nespojitosti:
\begin{enumerate}
\item $f(a)=(-1)^{[x^2]}$;
\item $f(a)=[\frac{1}{x}] sgn \sin [\frac{\pi}{x}]$;
\item $f(a)=\frac{x+1}{\arctan\frac{1}{x}}$;
\item $f(a)=\lim_{n \rightarrow \infty} (x \cdot \arctan(n \cdot \cot x))$;
\item $f(a)=\lim_{n \rightarrow \infty} \frac{x+x^2e^{nx}}{1+e^{nx}}$;
\item $f(a)=\lim_{t \rightarrow \infty}\frac{\ln (1+e^{xt})}{1+e^t}$.
\end{enumerate}
\end{defproblem}

\begin{defproblem}{s-funkcie-38}
Určte číslo $A$ tak, aby funkcia $f_1(x) = \left\{ \begin{array}{r@{\quad}c}
  f(x), & $ak $ x \in D(f) \\
  A, &  $ak $ x=0 \\ \end{array} \right.
  $ bola spojitá v bode $0$:
  \begin{enumerate}
  \item $f(x)=\frac{1}{x^2}^{-\frac{1}{x^2}}$;
  \item $f(x)=x^x,x>0$;
  \item $f(x)=x \cdot \ln^2 x$.
  \end{enumerate}
\end{defproblem}

\begin{defproblem}{s-funkcie-39}
  Nech $f: \mathbb{R} \rightarrow \mathbb{R}$ je spojitá funkcia. Potom funkcia:
  $$
  F(x) = \left\{
    \begin{array}{r@{\quad}l}
      -c,   & \text{ak } f(x)<-c \\
      f(x), & \text{ak } |f(x)|\leq c \\
      0,    & \text{ak } f(x)>c \\ \end{array} \right.
  $$
    je spojitá. Dokážte!
\end{defproblem}

\begin{defproblem}{s-funkcie-40}
Zoraďme množinu $\mathbb{Q}$ do proste postupnosti ${\{q_n\}}_{n=1}^\infty$; nech ${\{a_n\}}_{n=1}^\infty$ je daná postupnosť reálnych čísel. Definujme funkciu $\psi$ (nazýva sa zovšeobecnená Riemannova funkcia) nasledovne:
$$\psi = \left\{ \begin{array}{r@{\quad}c}
  0, & $ak $ x \in \mathbb{R} \setminus \mathbb{Q} \\
  a_n, &  $ak $ x=q_n \\ \end{array} \right.$$
  Ak $\lim_{n \rightarrow \infty} a_n=0$, tak funkcia $\psi$ je spojitá v každom iracionálnom čísle. Dokážte! Platí aj obrátená implikácia?
\end{defproblem}

\begin{defproblem}{s-funkcie-41}
Vyšetrite spojitosť funkcie
$$
  f(x) =
  \left\{
    \begin{array}{r@{\quad}l}
      \frac{nx}{|x|}, & \text{ak } x=\frac{m}{n}, m\in \mathbb{Z} \text{ a } m\in \mathbb{N} \text{ sú nesúdeliteľné} \\
      f(x),           & \text{ak } |f(x)|\leq c \\
      |x|,            & \text{ak } x \in \mathbb{R} \setminus \mathbb{Q} \\
    \end{array}
  \right.
$$
\end{defproblem}

\begin{defproblem}{s-funkcie-42}
Nech $f: \langle a,b \rangle \rightarrow \mathbb{R}$ je spojitá funkcia. Vyšetrite spojitosť funkcie
$$g(x) = \left\{ \begin{array}{r@{\quad}c}
  \sup_{t \in \langle a,x)}f(t)-\inf_{t \in \langle a,x)}f(t), & $ak $ x \in (a,b \rangle \\
  0, &  $ak $ x=a \\ \end{array} \right.
  $$
\end{defproblem}

\begin{defproblem}{s-funkcie-43}
\begin{enumerate}
\item Nech je daná funkcia $f$, nech $x_0 \in D(f)$ a platí
  $$\forall \delta>0 \exists \varepsilon >0: |x-x_0|<\delta \Rightarrow |f(x)-f(x_0)|<\varepsilon.$$
  Vyplýva z týchto predpokladov spojitosť funkcie $f$ v bode $x_0$? Akú vlastnosť funkcie $f$ popisuje uvedená podmienka?
\item Nech je daná funkcia $f$, nech platí
$$\forall x_0 \in D(f) \forall \delta>0 \exists \varepsilon >0: |f(x)-f(x_0)|<\varepsilon \Rightarrow |x-x_0|<\delta.$$
Vyplýva z týchto predpokladov spojitosť funkcie $f$? Aká vlastnosť funkcie $f$ je popísaná uvedenou podmienkou?
\end{enumerate}
\end{defproblem}

\begin{defproblem}{s-funkcie-44}
Možno tvrdiť, že funkcia $g \circ f$ je nespojitá v bode $a$, ak $f$ nie je spojitá v bode $a \in D(f)$ a funkcia $g$ je
\begin{enumerate}
\item spojitá na $\mathbb{R}$;
\item prostá a spojitá na $\mathbb{R}$?
\end{enumerate}
\end{defproblem}

\begin{defproblem}{s-funkcie-45}
Zostrojte funkciu $f: \mathbb{R} \rightarrow \mathbb{R}$, ktorá je spojitá práve v bodoch množiny $M$, ak
\begin{enumerate}
\item $M=\{0\}$;
\item $M=\emptyset$;
\item $M=\mathbb{N}$;
\item $M=\{\frac{1}{n};n \in \mathbb{N}\}$.
\end{enumerate}
\end{defproblem}

\begin{defproblem}{s-funkcie-46}
Zistite, či existuje bijekcia $f: \mathbb{R} \rightarrow \mathbb{R}$, ktorá nie je spojitá v žiadnom bode $x \in \mathbb{R}$.
\end{defproblem}

\begin{defproblem}{s-funkcie-47}
Dokážte, že neexistuje funkcia $f: \mathbb{R} \rightarrow \mathbb{R}$ taká, že pre každé $a \in \mathbb{R}$ existuje vlastná $\lim_{x \rightarrow a}f(x)$ a platí $\lim_{x \rightarrow a}f(x) \neq f(a)$.
\end{defproblem}

\begin{defproblem}{s-funkcie-48}
\begin{enumerate}
\item Ak pre funkciu $f: \mathbb{R} \rightarrow \mathbb{R}$ platí $\lim_{x \rightarrow a}f(x)=0$ v každom bode $a \in \mathbb{R}$, tak množina $A:=\{x \in \mathbb{R};f(x)\neq 0\}$ je spojiteľná. Dokážte!
\item Neexistuje funkcia $f: \mathbb{R} \rightarrow \mathbb{R}$, ktorá má v každom bode $a \in \mathbb{R}$ nevlastnú limitu. Dokážte!
\end{enumerate}
\end{defproblem}

\begin{defproblem}{s-funkcie-49}
Ak polynóm $P$ párneho stupňa nadobúda aspoň jednu hodnotu, ktorá má oparné znamienko ako koeficient pri člene s najvyššou mocninou, tak $P$ má aspoň dva reálne korene. Dokážte!
\end{defproblem}

\begin{defproblem}{s-funkcie-50}
Ak je funkcia $f: \mathbb{R} \rightarrow \mathbb{Q}$ spojitá, tak je konštantná. Dokážte!
\end{defproblem}

\begin{defproblem}{s-funkcie-51}
Nech funkcia $f: \langle 0, \infty) \rightarrow \mathbb{R}$ je spojitá a ohraničená a neexistuje $\lim_{x \rightarrow \infty}f(x)$. Potom existuje $A \in \mathbb{R}$, pre ktoré má rovnica $f(x)=A$ nekonečne veľa riešení. Dokážte!
\end{defproblem}

\begin{defproblem}{s-funkcie-52}
Ak $f: \mathbb{R} \rightarrow \mathbb{R}$ je spojitá funkcia a pre každé $x \in \mathbb{R}$ platí $f(f(x))=x$, tak existuje riešenie rovnice $f(x)=x$. Dokážte!
\end{defproblem}

\begin{defproblem}{s-funkcie-53}
Ak $f: \mathbb{R} \rightarrow \mathbb{R}$ je spojitá periodická funkcia s periódou $T$, tak existuje také $a \in \mathbb{R}$, pre ktoré $f(a+\frac{T}{2})=f(a)$. Dokážte!
\end{defproblem}

\begin{defproblem}{s-funkcie-54}
  \begin{enumerate}
    \item
      Inverzná funkcia k rýdzomonotónnej funkcii definovanej na intervale je
      spojitá. Dokážte!
    \item
      Uveďte príklad prostej funkcie spojitej aj v bode $0$, ktorej inverzná
      funkcia nie je spojitá v bode $f(0)$;
    \item
      Uveďte príklad spojitej rýdzomonotónnej funkcie, ktorej inverzná funkcia
      nie je spojitá!
  \end{enumerate}
\end{defproblem}

\begin{defproblem}{s-funkcie-55}
Nech funkcia $f: \mathbb{R} \rightarrow \mathbb{R}$ je rovnomerne spojitá. Potom existujú čísla $a \geq 0,b \geq 0$ tak, že pre všetky $x \in \mathbb{R}$ platí $|f(x)| \leq a \cdot |x|+b$. Dokážte!
\end{defproblem}

\begin{defproblem}{s-funkcie-56}
\begin{enumerate}
\item Ak sú funkcie $f,g$ rovnomerne spojité na ohraničenom intervale $(a,b)$, tak sú tam rovnomerne spojité aj funkcie $f+g,f \cdot g$. Dokážte!
\item Uveďte príklad funkcií $f \cdot g$ rovnomerne spojitých na intervale, ktorých súčin tam nie je rovnomerne spojitý.
\end{enumerate}
\end{defproblem}

\begin{defproblem}{s-funkcie-57}
Rozhodnite o pravdivosti nasledujúceho tvrdenia: "Nech $g$ je spojitá nekonštantná funkcia definovaná na intervale $I$, nech funkcia $f$ je spojitá na intervale $g(I)$. Ak aspoň jedna z funkcií $f,g$ je rovnomerne spojitá, tak aj funkcia $f \circ g$ je rovnomerne spojitá."
\end{defproblem}

\begin{defproblem}{s-funkcie-58}
Popíšte funkcie, ktoré vyhovujú podmienke:
\begin{enumerate}
\item $\forall \varepsilon >0 \exists 0<\delta<\varepsilon \forall x,y \in \mathbb{R}:|x-y|<\delta \Rightarrow |f(x)-f(y)|<\varepsilon$;
\item $\forall \varepsilon >0 \exists \delta>0 \forall x,y \in \mathbb{R}:|x-y|<\delta \Rightarrow f(x)-f(y)<\varepsilon$;
\item $\forall \varepsilon >0 \exists \delta>0 \forall x,y \in \mathbb{R}:|x-y|<\delta \Rightarrow |f(x)-f(y)|<\varepsilon$.
\end{enumerate}
\end{defproblem}
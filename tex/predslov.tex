\chapter*{Predslov}

\setlength\epigraphrule{0pt}
\epigraph{
  \enquote{Človek, ktorý nedokáže hovoriť tak,
  aby mu ostatní rozumeli, je idiot.
  Chápeš ma?} \\
  \enquote{Nie, otec.}
}{}

Predslov je pre zúfalých autorov často poslednou príležitosťou napraviť,
čo sa ešte dá. Chápeme sa preto možnosti vysvetliť zachmúrenému čitateľovi,
ako sme to vlastne mysleli.

Skriptá \emph{Cvičenia z matematickej analýzy I} sú určené pre študentov prvého
ročníka odboru Matematika a tých študentov prvého ročníka odboru Učiteľstvo
všeobecnovzdelávacích predmetov, ktorých aprobačným predmetom je matematika;
obsahujú cvičenia k látke preberanej v prvom semestri.

Dvojitou čiarou sú v týchto skriptách označené definície a tie tvrdenia, ktoré
sa čitateľ môže - bez toho, že by ich dokazoval - odvolávať pri riešení za nimi
nasledujúcich úloh. Teda:
\begin{enumerate*}[label=\alph*)]
  \item pri riešení príkladu sa nemožno odvolávať na tvrdenia uvedené až za ním
  \item riešenie musí obsahovať dôkazy všetkých ostatných použitých tvrdení.
\end{enumerate*}

Jednoduchou čiarou sú označené riešenia príkladov uvedené priamo v texte.

V každej kapitole možno rozlíšiť dve časti; druhá z nich (doplňujúca) má vždy
názov \textit{Ďaľšie príklady}. Do prvej (hlavnej) časti každej kapitoly sme
zaradili úlohy, ktoré pokladáme za základné, odsek \textit{Ďaľšie príklady}
obsahuje aj trocha náročnejšie cvičenia. (Skoro v každej študíjnej skupine sa
totiž nájdu študenti, ktorí príklady preberané na cvičeniach už ovládajú alebo
si to aspoň myslia. Nech si teda ticho prepočítavajú úlohy z odseku
\textit{Ďaľšie príklady} a nedeprimujú svojím znudeným výrazom cvičiaceho. Tým
samozrejme nechceme povedať, že by tieto úlohy boli neprístupné pre ostatných
študentov.)

V poslednej časti skrípt sú uvedené výsledky všetkých výpočtových príkladov a
návody k mnohým dôkazovým. Najmä druhá z týchto skutočností by mohla pôsobiť
demoralizujúco, preto sa pri nej zastavme. Cieľom zbierky úloh nie je podľa
nášho názoru študenta odradiť, ale niečo ho naučiť. Či sa niečo naučí neborák,
štyri hodiny bezvýsledne riešiaci nejaké cvičenie, je otázne.

Samozrejme, že najnevhodnejším možným postupom je prečítať si zadanie a hneď si
vzadu nalistovať návod; človek sa tak iste veľa nenaučí. Morálne právo uchýliť
sa k návodu dáva až predchádzajúca neúspašná snaha o nájdenie riešenia. (Celý
tento odstavec sa napokon rovnako vzťahuje na riešené príklady.) Vzadu uvedené
návody nie sú samozrejme jediné možné, iné postupy môžu byť rovnako správne
(nehovoriac už o chybách a nedopatreniach, ktoré sa vo výsledkoch a návodoch
zákonite vyskytnú).

Na úlohu, ktorá je súčasťou príkladu pozostávajúceho z viacerých zadaní, sa
odvolávame vždy (aj v rámci toho istého príkladu) dvojčíslom: napr. na druhú
úlohu v príklade 4 sa odvolávame ako na pr. 4.2. Pokiaľ sme u niektorého
príkladu nepovažovali za vhodné zaradiť do poslednej časti skrípt návod na jeho
riešenie, má jeho číslo index $_0$. Pokiaľ je niektorý príklad formulovaný ako
tvrdenie, očakávame, že ho čitateľ dokáže, preto nie všetky dôkazové príklady
sa začínajú "Dokážte, že ...".

Napokon ešte jedna poznámka: goniometrické, exponenciálne a logaritmické
funkcie sú zväčša na cvičeniach potrebné skôr, než sa ich podarí na prednáške,
resp. v učebnici korektne zaviesť. Pri zostavovaní príkladov sme preto od
začiatku považovali tieto funkcie a ich vlastnosti (grafy, goniometrické vzorce
a pod.) za známe.

Ďakujeme na tomto mieste všetkým, ktorí príspeli k výslednej podobe popisu
týchto skrípt. Sú to predovšetkým recenzenti doc. RNDr. Jozef Vencko, CSc. a
RNDr. Igor Book, CSc., ďalej RNDr. Zuzana Ondrejková, RNDr. Ivan Kupka, Ing.
Jiří Kubáček, CSc., RNDr. Iveta Kundraciková a RNDr. Viera Čerňanová. Za
mimoriadnu ochotu, s ktorou nám vyšiel v ústrety, sme zaviazaní vedúcemu
Vydavateľského oddelenia RUK Ing. Antonínovi Skovajsíkovi.

Funkcia $f$ sa nazýva rovnomerne spojitá na množine $A \subset D(f)$ $(A \neq
\emptyset)$, ak platí
\[
  (\forall \varepsilon > 0)
    (\exists \delta > 0)
      (\forall x,y\in A):
        |x-y| < \delta \Rightarrow |f(x)-f(y)| < \varepsilon
\]
Funkcia $f$ rovnomerne spojitá na svojom definičnom obore sa nazýva rovnomerne
spojitá.

\begin{enumerate}[resume]
  \item \useproblem[spojitost-funkcie]{s-funkcie-247}
  \item \useproblem[spojitost-funkcie]{s-funkcie-248}
  \item \useproblem[spojitost-funkcie]{s-funkcie-249}
  \item \useproblem[spojitost-funkcie]{s-funkcie-250}
\end{enumerate}

\begin{veta}
Funkcia spojitá na kompaktnej množine $K$ je rovnomerne spojitá na $K$.
\end{veta}

\begin{enumerate}[resume]
  \item \useproblem[spojitost-funkcie]{s-funkcie-251}
  \item \useproblem[spojitost-funkcie]{s-funkcie-252}
  \item \useproblem[spojitost-funkcie]{s-funkcie-253}
  \item \useproblem[spojitost-funkcie]{s-funkcie-254}
  \item \useproblem[spojitost-funkcie]{s-funkcie-255}
  \item \useproblem[spojitost-funkcie]{s-funkcie-256}
\end{enumerate}

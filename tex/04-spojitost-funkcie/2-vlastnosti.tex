Funkcia $f$ definovaná na intervale $I$ sa nazýva darbouxovská na $I$, ak pre
Každé $a,b \in I,a<b$ platí: na intervale $\interval{a}{b}$ nabudúce funkcia $f$
všetky hodnoty medzi $f(a)$ a $f(b)$; t.j. ak platí
\[
  \forall a, b \in I, a < b:
    f(\left[ a, b \right] ) \supset X
\]
\[
  X = \{y \in \mathbb{R}; \min \{f(a),f(b)\} \leq y \leq \max \{f(a),f(b)\}\}
\]

\begin{veta}
Ak je funkcia $f$ spojitá na intervale $I$, tak je darbouxovská na $I$.
\end{veta}

\begin{enumerate}[resume]
  \item \useproblem[spojitost-funkcie]{s-funkcie-234}
  \item \useproblem[spojitost-funkcie]{s-funkcie-235}
  \item \useproblem[spojitost-funkcie]{s-funkcie-236}
  \item \useproblem[spojitost-funkcie]{s-funkcie-237}
  \item \useproblem[spojitost-funkcie]{s-funkcie-238}
  \item \useproblem[spojitost-funkcie]{s-funkcie-239}
\end{enumerate}

\begin{veta}
Ak je funkcia $f$ spojitá na kompaktnej množine $A$ (alebo špeciálne na
uzavretom ohraničenom intervale), tak je na $A$ ohraničená a existujú
$\max\limits_{x \in A}f(x)$ a $\min\limits_{x \in A}f(x)$.
\end{veta}

\begin{enumerate}[resume]
  \item \useproblem[spojitost-funkcie]{s-funkcie-240}
  \item \useproblem[spojitost-funkcie]{s-funkcie-241}
  \item \useproblem[spojitost-funkcie]{s-funkcie-242}
  \item \useproblem[spojitost-funkcie]{s-funkcie-243}
  \item \useproblem[spojitost-funkcie]{s-funkcie-244}
  \item \useproblem[spojitost-funkcie]{s-funkcie-245}
  \item \useproblem[spojitost-funkcie]{s-funkcie-246}
\end{enumerate}

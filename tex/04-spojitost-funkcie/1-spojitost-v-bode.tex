Hovoríme, že funkcia $f$ je spojitá v bode $a \in D(f)$, ak platí
\[
  (\forall \varepsilon > 0)
    (\exists \delta > 0)
      (\forall x \in D(f)):
        |x-a| < \delta \Rightarrow |f(x)-f(a)| < \varepsilon
\]
\textit{Poznámka:} Všimnime si, že bod $a$ nemusí byť hromadným bodom množiny
$D(f)$. Môžu teda nastať dva prípady:
\begin{itemize}
\item
  ak $a \in D(f)$ je hromadným bodom množiny $D(f)$, tak podmienka vyššie je
  ekvivalentná s podmienkou $\lim\limits_{x \rightarrow a} f(x)=f(a)$;
\item
  ak $a \in D(f)$ nie je hromadným bodom množiny $D(f)$ (t.j. existuje $\eta > 0
  $ tak, že $D(f) \cup \interval[open]{a-\eta}{a+\eta}=\{ a\}$; taký bod sa
  nazýva izolovaným bodom množiny $D(f)$), je podmienka vyššie splnená (stačí
  položiť $\delta=\eta$ pre ľubovoľné $\varepsilon > 0$). Teda funkcia $f$ je
  splnená v každom izolovanom bode množiny $D(f)$.
\end{itemize}

\begin{veta}
Ak funkcie $f,g,f-g,f \cdot g$ sú spojité v bode $a$. Ak naviac $g(a) \neq 0$,
tak aj funkcia $\frac{f}{g}$ je spojitá v bode $a$.
\end{veta}

\begin{veta}
Ak funkcia $f$ je spojitá v bode $a$, funkcia $g$ je spojitá v bode $f(a)$, tak
funkcia $g \circ f$ je spojitá v bode $a$. Hovoríme, že funkcia $f$ je spojitá,
ak je spojitá v každom bode svojho definičného oboru. Hovoríme, že funkcia $f$
je spojitá na množine $A$ ($\emptyset\neq a\subset D(f)$), ak je spojitá funkcia
$f \setminus A$.
\end{veta}

\begin{veta}
Každá základná elementárna funkcia je spojitá.
\end{veta}
(Z viet $1,2$ a $3$ vyplýva veta $4$ z kapitoly $2.$)
\begin{enumerate}[resume]
  \item \useproblem[spojitost-funkcie]{s-funkcie-221}
  \item \useproblem[spojitost-funkcie]{s-funkcie-222}
  \item \useproblem[spojitost-funkcie]{s-funkcie-223}
  \item \useproblem[spojitost-funkcie]{s-funkcie-224}
  \item \useproblem[spojitost-funkcie]{s-funkcie-225}
  \item \useproblem[spojitost-funkcie]{s-funkcie-226}
  \item \useproblem[spojitost-funkcie]{s-funkcie-227}
  \item \useproblem[spojitost-funkcie]{s-funkcie-228}
  \item \useproblem[spojitost-funkcie]{s-funkcie-229}
  \item \useproblem[spojitost-funkcie]{s-funkcie-230}
  \item \useproblem[spojitost-funkcie]{s-funkcie-231}
\end{enumerate}

Číslo $a \in \mathbb{R}$, ktoré je hromadným bodom definičného oboru $D(f)$
funkcie $f$, sa nazýva bodom nespojitosti funkcie $f$, ak je splnená niektorá z
nasledujúcich podmienok:
\begin{itemize}
\item $a \notin D(f)$;
\item neexistuje $\lim\limits_{x \rightarrow a}f(x)$;
\item $a \in D(f)$ a existuje $\lim\limits_{x \rightarrow a}f(x)$, ale neplatí
      $\lim\limits_{x \rightarrow a}f(x)=f(a)$.
\end{itemize}

Ak $a \in \mathbb{R}$ je bod nespojitosti funkcie $f$ a existuje konečná
$\lim\limits_{x \rightarrow a}f(x)$, nazýva sa bodom odstrániteľnej
nespojitosti. Ak neexistuje vlastná $\lim\limits_{x \rightarrow a}f(x)$ nazýva
sa $a$ bodom neodstrániteľnej nespojitosti.

Nech $a \in \mathbb{R}$ je bod nespojitosti funkcie $f$, pričom $a$ je hromadným
bodom množín $D(f) \cap (-\infty,a)$ aj $D(f)\cap (a,+\infty)$. Ak v bode $a$
existujú vlastné jednostranné limity a $\lim\limits_{x \rightarrow a-}f(x) \neq
\lim\limits_{x \rightarrow a+}f(x)$ nazýva sa $a$ bodom nespojitosti $1.$ druhu
(rozdiel $\lim\limits_{x \rightarrow] a+}f(x)-\lim\limits_{x \rightarrow
a-}f(x)$ sa potom nazýva skokom funkcie $f$ v bode $a$). Ak aspoň jedna z
jednostranných limít v bode $a$ neexistuje alebo je nevlastná, nayzva sa $a$
bodom nespojitosti $2.$ druhu.

\textit{Poznámka:}
Klasifikácia bodov nespojitosti nie je v matematickej literatúre jednotná; napr.
niekedy sa hromadné body množiny $D(f)$, ktoré nie sú prvkami $D(f)$, nepovažujú
za body nespojitosti funkcie $f$; v definícii bodu nespojitosti $1.$ druhu s
aniekedy nepovažuje splnenie podmienky $\lim\limits_{x \rightarrow a+}f(x) \neq
\lim\limits_{x \rightarrow a-}f(x)$ atď.

\begin{enumerate}[resume]
  \item \useproblem[spojitost-funkcie]{s-funkcie-232}
  \item \useproblem[spojitost-funkcie]{s-funkcie-233}
\end{enumerate}

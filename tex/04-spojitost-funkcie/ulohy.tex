\begin{defproblem}{s-funkcie-221}
Pomocou kvantifikátorov zapíšte tvrdenie \enquote{funkcia $f$ nie je spojitá v
bode $a \in D(f)$}!
\end{defproblem}

\begin{defproblem}{s-funkcie-222}
Vyšetrite spojitosť nasledujúcich funkcií v daných bodoch:
\begin{tasks}
\task $f(x) =
  \begin{cases}
    \frac{\sin x}{x}& $ak $ x \neq 0 \\
    1 &  $ak $ x = 0
  \end{cases}
  $ v bode $0$
\task $f(x) =
  \begin{cases}
    \frac{\sin x}{|x|} & $ak $ x \neq 0 \\
  1 &  $ak $ x = 0
  \end{cases}
  $ v bode $0$
\task $f(x) =
  \begin{cases}
    \frac{1}{(1+x)^2} & $ak $ x \neq 0 \\
    1 &  $ak $ x = 0
  \end{cases}
  $ v bode $-1$
\task $f(x)=\sgn x$ v bode $0$
\task $f(x)=x \cdot \sgn x$ v bode $0$
\end{tasks}

\end{defproblem}

\begin{defproblem}{s-funkcie-223}
Nasledujúce funkcie nie sú definované v bode $a$. Určite hodnotu $f(a)$ tak, aby
takto dodefinovaná funkcia $f$ bola spojitá v bode $a$:
\begin{tasks}
  \task $f(x)=\frac{\sqrt{1+x}-1}{\sqrt[3]{1+x}-x}$, $a=0$
  \task $f(x)=(1+x)^{\frac{1}{2}x}$, $a=0$
  \task $f(x)=e^{-\frac{1}{x^2}}$, $a=0$
  \task $f(x)=\frac{1}{1+e^{\frac{1}{x-1}}},x<1$, $a=1$
  \task $f(x)=(x+3) \cdot \sin \frac{1}{x+3}$, $a=-3$
\end{tasks}
\end{defproblem}

\begin{defproblem}{s-funkcie-224}
Vyšetrite spojitosť nasledujúcich funkcií:
\begin{tasks}(2)
\task $f(x)=\frac{x^2-4x+7}{x^3+5x^2+6x}$
\task $f(x) =
  \begin{cases}
    \frac{1-e^x}{x},& $ak $ x \leq 0 \\
    2x-1, &  $ak $ x \geq 0
  \end{cases}
  $
\task $f(x)=\sgn (x \cdot (1-x^2))$
\task $f(x)=x \cdot [x]$
\task $f(x)=\lim\limits_{n \rightarrow \infty} \frac{1}{1+x^n},x \geq 0$
\task $f(x)=\lim\limits_{n \rightarrow \sqrt[n]{1+x^{2n}}}$
\task $f(x)=\sqrt{-\sin^2 x}$
\end{tasks}

\begin{solution}
  \textbf{b):}
  Dokázať spojitosť funkcie $f$ na množine $\mathbb{R} \setminus \{0\}$ je
  jednoduché: pre každé číslo $a<0$ možno nájsť také jeho okolie $O(a)$, na
  ktorom sa funkcia $f$ zhoduje s elementárnou funkciou $\frac{1-e^x}{x}$ tá je
  spojitá v bode $a$; preto
  \[
    \lim\limits_{x \rightarrow a}f(x)
    = \lim_{x \rightarrow a}\frac{1-e^x}{x}=\frac{1*e^a}{a}=f(a)
  \]
  čo znamená, že funkcia $f$ je spojitá v bode $a$. Analogicky sa postupuje v
  prípade $a>0$. Prípad $a=0$ treba vyšetriť samostatne: vtedy
  \[
    \lim_{x \rightarrow 0+}f(x)=\lim_{x \rightarrow 0+}\frac{1-e^x}{x}=-1
  \]
  \[
    \lim_{x \rightarrow 0-}f(x)=\lim_{x \rightarrow 0-}(2x-1)=-1
  \]
  preto $\lim\limits_{x \rightarrow 0}f(x)=-1=f(0)$. Funkcia $f$ je teda spojitá
  aj v bode $0$, preto $f$ je spojitá funkcia.
\end{solution}
\end{defproblem}

\begin{defproblem}{s-funkcie-225}
Nech funkcie $f,g$ sú definované na $\mathbb{R}$ a funkcia $f+g$ je spojitá v
bode $a$. Potom nastane práve jedna z nasledujúcich možností:
\begin{enumerate}
\item funkcia $f$ aj funkcia $g$ sú spojité v bode $a$;
\item ani funkcia $f$, ani funkcia $g$ nie je spojitá v bode $a$.
\end{enumerate}
Dokážte: obidve uvedené možnosti dokumentujte na príkladoch!
\end{defproblem}

\begin{defproblem}{s-funkcie-226}
Nech $f: \mathbb{R} \rightarrow \mathbb{R}$ je spojitá funkcia. Potom funkcia $f
\cdot \zeta$ je spojitá práve v bodoch množiny $\{x\in \mathbb{R}: f(x)=0\}$.
Dokážte!
\end{defproblem}

\begin{defproblem}{s-funkcie-227}
Možno tvrdiť, že funkcia $g \circ f$ nie je spojitá v bode $x_{0}$, ak $f$ je
spojitá v bode $x_{0}$ a $g$ nie je spojitá v bode $f(x_0) \in D(g)$?
\end{defproblem}

\begin{defproblem}{s-funkcie-228}
Nech funkcie $f,g$ sú spojité v bode $a$ $(\in D(f)\cap D(g))$. Potom aj funkcie
$|f|,\max \{f,g\},\min \{f,g\}$ sú spojité v bode $a$. Dokážte!
\end{defproblem}

\begin{defproblem}{s-funkcie-229}
Vyšetrite spojitosť Dirichletovej funkcie $\zeta (x)$!
\end{defproblem}

\begin{defproblem}{s-funkcie-230}
Dokážte, že Riemannova funkcia $f$, definovaná predpisom $f(x)=0$, ak $x \in
\mathbb{R} \setminus \mathbb{Q}$ a $f(x)=\frac{1}{n}$, ak $x=\frac{m}{n}$, kde
$\mathbb{N}, m \in \mathbb{Z}$ sú nesúdeliteľné čísla, je spojitá v každom
iracionálnom čísle a nespojitá v každom racionálnom čísle.
\end{defproblem}

\begin{defproblem}{s-funkcie-231}
Nech funkcia $f$ je spojitá na intervale $\interval{a}{b}$. Potom
\[
  \sup \{f(x); x \in \interval{a}{b}\}= \sup \{f(x);
  x \in \interval{a}{b} \cap \mathbb{Q}\}
\]
Dokážte!
\end{defproblem}

\begin{defproblem}{s-funkcie-232}
\begin{tasks}(2)
\task $f(x)=\frac{x}{(1-x)^2}$
\task $f(x)=\frac{x^2-1}{x^3-3x+2}$
\task $f(x)=\frac{\sin x}{x}$
\task $f(x)=\frac{\frac{1}{x}-\frac{1}{x+1}}{\frac{1}{x-1}-\frac{1}{x}}$
\task $f(x)=\sqrt{\frac{1-\cos \pi x}{4-x^2}}$
\task $f(x)=\cos^2 \frac{1}{x}$
\task $f(x)=\sqrt{x}\cdot \arctan \frac{1}{x}$
\task $f(x)=\frac{1}{\ln x}$
\task $f(x) = \begin{cases}
  x & $, ak $ |x| \leq 1 \\
  1 &  $, ak $ |x| \geq 1
  \end{cases}
  $
\task $f(x)=x \cdot [\frac{1}{x}]$
\task $g(g(g(x)))$, kde $g(x)=\frac{1}{1-x}$
\task!
  $ f(x)=g(\varphi(x)) $

  $
    g(x) =
    \begin{cases}
        \sin \pi x & \text{, ak } x \in \mathbb{Q} \\
        0          & \text{, ak } x \in \mathbb{R} \setminus \mathbb{Q}
    \end{cases}
  $
\task!
  $f(x)=g(\varphi(x))$

  $g(x) =
    \begin{cases}
      x,   & \text{, ak } 0 \leq x \leq 1 \\
      2-x, & \text{, ak } 1 < x < 2
    \end{cases}
  $

  $\varphi(x) =
    \begin{cases}
        x   & \text{, ak } x \in \mathbb{Q} \\
        2-x &  \text{, ak } x \in \mathbb{R} \setminus \mathbb{Q}
    \end{cases}
  $
\end{tasks}
\end{defproblem}

\begin{defproblem}{s-funkcie-233}
\begin{tasks}
\task
  Dokážte, že monotónna funkcia definovaná na $\mathbb{R}$ môže mať len body
  nespojitosti $1.$ druhu.
\task
  Dokážte, že množina bodov nespojitosti neklesajúcej funkcie $f$ definovanej na
  intervale $I$ je spočítateľná.

(\textit{Návod:} Stačí dokázať, že množina $S$ všetkých bodov nespojitosti
funkcie $f$ ležia vnútri intervalu $I$ - každý z nich je bodom nespojitosti $1.$
druhu - je spočítateľná. Ak každému bodu $a \in S$ priradíme interval ($\lim_{x
\rightarrow a-}f(x),\lim_{x \rightarrow a+}f(x)$)), dostaneme systém po
dvochdisjunktných intervalov; tento systém je spočítateľný - pozri príklad $99$.
)
\end{tasks}
\end{defproblem}

\begin{defproblem}{s-funkcie-234}
Dokážte, že existuje
\begin{tasks}
\task $x \in \interval[open]{0}{1}$, pre ktoré platí $x^5+x^4+x^3+x^2-x-1=0$
\task $x \in \mathbb{R}$, pre ktoré platí $x=\cos x$
\task aspoň jedno riešenie rovnice $P(x)=0$, kde $P$ je polynóm nepárneho stupňa
\end{tasks}

\begin{solution}
  \textbf{a):}
  Funkcia daná predpisom
  \[
    f(x)=x^5+x^4+x^3+x^2-x-1
  \]
  je spojitá, a teda aj darbouxovská na $\mathbb{R}$, nadobúda preto na
  intervale $\interval{0}{1}$ všetky hodnoty medzi $f(0)=-1$ a $f(1)=2$. Pretože
  $-1<0<2$, musí existovať $x \in (0,1)$, v ktorom $f(x)=0$.
\end{solution}
\end{defproblem}

\begin{defproblem}{s-funkcie-235}
Nech funkcia $f$ je spojitá na intervale $(a,b)$, nech $x_1,x_2,...,x_n \in
(a,b)$. Potom existuje $o \in (a,b)$ tak, že:
$$f(c)=\frac{1}{n}(f(x_1)+...+f(x_n))$$
Dokážte!
\end{defproblem}

\begin{defproblem}{s-funkcie-236}
Nech $f: \interval{0}{1} \rightarrow \interval{0}{1}$ je spojitá funkcia.
Potom pre niektoré $c \in \interval{0}{1}$ platí $f(c)=c$. Dokážte!
\end{defproblem}

\begin{defproblem}{s-funkcie-237}
Ak pre spojitú unkciu $f: \mathbb{R} \rightarrow \mathbb{R}$ platí
$f(\mathbb{Q}= \{0\})$, tak aj $f(\mathbb{R})=\{0\}$. Dokážte!
\end{defproblem}

\begin{defproblem}{s-funkcie-238}
Nech funkcia $f: \mathbb{R} \rightarrow \mathbb{R}$ má túto vlastnosť: pre každý
interval $I \subset \mathbb{R}$ je $I \setminus f(\mathbb{R} \neq \emptyset)$.
Potom je $f$ konštantná na $\mathbb{R}$ alebo nespojitá v každom bode $x \in
\mathbb{R}$. Dokážte!
\end{defproblem}

\begin{defproblem}{s-funkcie-239}
Ak je funkcia $f$ prostá a spojitá na intervale $I$, tak je tam rýdyomonotónna.
Dokážte!
\end{defproblem}

\begin{defproblem}{s-funkcie-240}
Nech spojitá funkcia $f$ nadobúda na intervale $\interval{a}{b}$ len kladné
hodnoty. Potom existuje $\mu > 0$ tak, že $f(x) \geq \mu$ platí pre každé $x \in
\interval{a}{b}$. Dokážte!
\end{defproblem}

\begin{defproblem}{s-funkcie-241}
Ak funkcia $f: \interval[open]{a}{b} \rightarrow \mathbb{R}$ $(a,b\in
\mathbb{R^*})$ je spojitá a existujú konečné $\lim\limits_{x \rightarrow
a}f(x),\lim\limits_{x \rightarrow b}f(x)$, tak $f$ je ohraničená. Dokážte!
\end{defproblem}

\begin{defproblem}{s-funkcie-242}
Ak $P$ je polynóm párneho stupňa, tak existuje $\max\limits_{x \in \mathbb{R}}
P(x)$. Dokážte! Aký je koeficient pri člene s najvyššou mocninou, ak existuje
$\min\limits_{x \in \mathbb{R}} P(x)$?
\end{defproblem}

\begin{defproblem}{s-funkcie-243}
Ak $f$ je spojitá na intervale $\interval[open right]{0}{\infty}$ a
$\lim\limits_{x \rightarrow \infty} f(x)=f(0)$, tak existuje $\max\limits_{x
\geq 0} f(x)$ aj $\min\limits_{x \geq 0}f(x)$. Dokážte!
\end{defproblem}

\begin{defproblem}{s-funkcie-244}
Vetu $5$ možno dokázať na základe nasledujúcich faktov:
\begin{tasks}
\task
  ak funkcia $f$ je spojitá na kompaktnej množine $A$, tak $f(A)$ je
  kompaktná množina
\task
  každá kompakná množina je ohraničená a obsahuje svoje supremum a infimum.
\end{tasks}
Dokážte tieto tvrdenia!
\end{defproblem}

\begin{defproblem}{s-funkcie-245}
Nech $f: \mathbb{R} \rightarrow \mathbb{R}$ má len body nespojitosti $1.$ druhu.
Potom $f$ je ohraničená na každom ohraničenom intervale. Dokážte!
\end{defproblem}

\begin{defproblem}{s-funkcie-246}
Existuje spojitá a ohraničená funkcia $f:\interval[open left]{0}{1} \rightarrow
\mathbb{R}$ taká, že neexistuje $\max\limits_{x \in \interval[open left]{0}{1}}
f(x)$ ani $\min\limits_{x \in \interval[open left]{0}{1}} f(x)$?
\end{defproblem}

\begin{defproblem}{s-funkcie-247}
Pomocou kvamtifikátorov zapíšeme tvrdenie \enquote{Funkcia $f$ je spojitá na intervale
$\interval[open]{a}{b}$, ale nie je tam rovnomerne spojitá.}!
\end{defproblem}

\begin{defproblem}{s-funkcie-248}
Rozhodnite, či je funkcia $f$ rovnomerne spojitá na množine $A$. Svoje tvrdenie
dokážte na základe definície!
\begin{tasks}(2)
\task $f(x)=5x-3$,$A=\mathbb{R}$
\task $f(x)=x^2-2x-1$,$A= \interval{-2}{5}$
\task $f(x)=\cos \frac{1}{x}$,$A=\interval[open]{0}{1}$
\task $f(x)=\frac{1}{x}$,$A=\interval[open]{\frac{1}{10}}{1}$
\task $f(x)=\sin x^2$,$A=\interval[open right]{0}{\infty}$
\task $f(x)=\sqrt{x}$,$A=\interval[open right]{0}{\infty}$
\task $f(x)=x+\sin x$,$A=\mathbb{R}$
\end{tasks}

\begin{solution}
  \textbf{b):}
  Predpokladajme, že $x,y \in \interval{-2}{5}, |x-y|<\delta$, a pokúsme sa na
  základe toho zhora odhadnúť výraz $|f(x)-f(y)|$. Postupne dostaneme
  \begin{align*}
    |f(x)-f(y)|&=|x^2-2x-1-(y^2-2y-1)| = \\
       &= |(x^2-y^2)+2(y-x)|\leq
    |x^2-y^2|+2|x-y|=  \\
    &=|x-y||x+y|+2|x-y| = \\
    &=|x-y|(|x+y|+2)\leq  \\
    & \leq|x-y|(|x|+|y|+2)\leq 12|x-y|<\delta
  \end{align*}

  Zistili sme teda:
  \[
    x,y \in \interval{-2}{5}:
    |x-y|<\delta
    \Rightarrow
    |f(x)-f(y)|<12\delta
  \]
  To znamená, že funkcia $f$ je rovnomerne spojitá na
  intervale $\interval{-2}{5}$, pre dané $\varepsilon > 0$ stačí totiž
  položiť $\delta = \frac{\varepsilon}{12}$.

  \textbf{e):}
  Ak si predstavíme graf funkcie $f(x)=\sin x^2,x \geq 0$, zistíme, že v smere k
  $+\infty$ funkcia $f$ na intervaloch stále menšej dĺžky nadobúda všetky
  hodnoty medzi $1$ a $-1$, t.j. že jej graf sa smerom k $+\infty$
  \enquote{zhusťuje}. To nás vedie k domnienke, že $f$ nie je na $\interval[open
  right]{0}{\infty}$ rovnomerne spojitá. Všimnime si preto podrobnejšie dvojice
  bodov
  \[
    (x_n,y_n)=(\sqrt{-\frac{\pi}{2}+2\pi n},\sqrt{\frac{\pi}{2}+2\pi
    n}),n\in \mathbb{N}
  \]
  Zrejme $f(x_n)=-1,f(y_n)=1$. Pretože $\lim_{n \rightarrow \infty}|y_n-x_n|=0$,
  existuje pre každé $\delta>0$ dvojica $(x_n,y_n)$ také, že $|x_n-y_n|<\delta$.
  Tým sme dokázali
  \begin{align*}
    \exists \varepsilon >0 (\varepsilon=2)
    \forall\delta>0\exists x_n \geq 0,y_n \geq 0: \\
    |x_n-y_n|<\delta \wedge |f(x_n)-f(y_n)| \geq \varepsilon
  \end{align*}
  čo je negácia tvrdenia \enquote{$f$ je rovnomerne spojitá na $\interval[open
  right]{0}{\infty}$}. Teda $\sin x^2$ nie je rovnomerne spojitá na intervale
  $\interval[open right]{0}{\infty}$.
\end{solution}
\end{defproblem}

\begin{defproblem}{s-funkcie-249}
Funkcia $f$ je rovnomerne spojitá na ohraničenom intervale $I$ práve vtedy, keď
pre ľubovoľné $\varepsilon >0$ existuje spojitá po častiach lineárna funkcia
$\varphi$ taká, že pre všetky $x \in I$ platí $|f(x)-\varphi(x)|< \varepsilon$.
Dokážte! (Spojitá funkcia $\varphi$ definovaná na ohraničenom intervale $I$ sa
nazýva po častiach lineárna, ak existuje konečný počet po dvoch disjunktných
intervlov $I_1,I_2,...,I_n$ tak, že $I_1 \cup I_2 \cup ... \cup I_n=I$ a funkcie
$\frac{\varphi}{I_i}$ $(i=1,...,n)$ sú lineárne; t.j. grafom je "lomená čiara".)
\end{defproblem}

\begin{defproblem}{s-funkcie-250}
Pre funkciu $F$ nájdite spojitú po častiach lineárnu funkciu $f$ takú, aby pre
všetky $x \in \interval{a}{b}$ platilo $|F(x)-f(x)|<0,1$ ak:
\begin{tasks}(2)
\task $F(x)=x^2, \interval{a}{b} = \interval{-1}{1}$
\task $F(x)=\frac{1}{x}, \interval{a}{b} = \interval{\frac{2}{3}}{2}$
\end{tasks}
\end{defproblem}

\begin{defproblem}{s-funkcie-251}
Rozhodnite, či je funkcia $f$ rovnomerne spojitá na množine $A$:
\begin{tasks}(2)
\task $f(x)=\frac{x}{4-x^2}$, $A=\interval{-1}{1}$
\task $f(x)=\frac{\sin x}{x}$, $A=\interval[open left]{0}{\pi}$
\task $f(x)=\sqrt[3]{x}$, $A=\interval[open right]{0}{\infty}$
\task!  $f(x) =
  \begin{cases}
    x\cdot \sin \frac{1}{x} & $, ak $ x \in \interval[open left]{0}{\pi} \\
    0 &  $, ak $ x=0
  \end{cases}
    $, $A= \interval{0}{\pi}$
\task $f(x)=x \cdot \sin x$, $A= \interval[open right]{0}{\infty}$
\task! $f(x)=\frac{x^6-1}{\sqrt{1-x^4}}$, $A=\interval[open]{-1}{1}$
\task $f(x)=\ln x$, $A=\interval[open right]{1}{\infty}$
\end{tasks}

\begin{solution}
  \textbf{b):}
  Na funkciu $f(x)=\frac{\sin x}{x},x \in \interval[open left]{0}{\pi}$ nemôžeme
  vetu $6$, pretože $D(f)=\interval[open left]{0}{\pi}$ nie je kompaktná
  množina. Pomôžeme si nasledovne: funkciu $f$ možno spojite dodefinovať v čísle
  $0$, pretože $0$ je bodom odstrániteľnej nespojitosti. Teda
  $f=f_1/\interval[open left]{0}{\pi}$, kde funkcia $f_1$ je určená predpisom:
  \[
    f_1(x) =\begin{cases}
      f(x) & $, ak $ x \in \interval[open left]{0}{\pi} \\
      1    & $, ak $ x = 0
    \end{cases}
  \]
  Na kompakte $\interval{0}{\pi}$ je $f_1$ spojitá, a teda podľa vety $6$ aj
  rovnomerne spojitá. Pretože zúženie rovnomerne spojitej funkcie je funkcia
  rovnomerne spojitá, je funkcia $\frac{\sin x}{x}$ rovnomerne spojitá na
  intervale $\interval[open left]{0}{\pi}$.

\end{solution}
\end{defproblem}

\begin{defproblem}{s-funkcie-252}
  Ak funkcia $f$ je spojitá na intervale
  $\interval[open right]{0}{\infty}$ a $\exists$ konečná
  $\lim\limits_{x \rightarrow \infty} f(x)$, tak $f$ je rovnomerne spojitá na
  $\interval[open right]{0}{\infty}$. Dokážte!
\end{defproblem}

\begin{defproblem}{s-funkcie-253}
  Každá spojitá periodická funkcia $f:\mathbb{R} \rightarrow \mathbb{R}$ je
  rovnomerne spojitá. Dokážte!
\end{defproblem}

\begin{defproblem}{s-funkcie-254}
Nech funkcia $f$ je definovaná na ohraničenom intervale $(a,b)$, nech
\begin{tasks}(2)
  \task $\lim\limits_{x \rightarrow b-}f(x)=+\infty$
  \task $\lim\limits_{x \rightarrow b-}f(x)$ neexistuje
\end{tasks}
Potom $f$ nie je rovnomerne spojitá na $\interval[open]{a}{b}$.
Dokážte!
\end{defproblem}

\begin{defproblem}{s-funkcie-255}
  Spojitá funkcia $f$ definovaná na ohraničenom intervale
  $\interval[open]{a}{b}$ je rovnomerne spojitá na $\interval[open]{a}{b}$
  práve vtedy, keď existujú konečné
  $\lim\limits_{x \rightarrow a^+}f(x)$, $\lim\limits_{x \rightarrow b^-}f(x)$.
  Dokážte!
\end{defproblem}

\begin{defproblem}{s-funkcie-256}
\begin{tasks}
  \task
    Ak $f$ je rovnomerne spojitá na ohraničenom intervale
    $\interval[open]{a}{b}$, tak $f$ je na $\interval[open]{a}{b}$ ohraničená.
    Dokážte!
  \task
    Uveďte príklad funkcie, ktorá je spojitá a ohraničená na ohraničenom
    intervale, ale nie je tam rovnomerne spojitá!
\end{tasks}
\end{defproblem}

\begin{defproblem}{s-funkcie-257}
Vyšetrite spojitosť nasledujúcich funkcií, určte charakter bodov nespojitosti:
\begin{tasks}(2)
\task $f(a)=(-1)^{[x^2]}$
\task $f(a)=[\frac{1}{x}] \sgn \sin [\frac{\pi}{x}]$
\task $f(a)=\frac{x+1}{\arctan\frac{1}{x}}$
\task $f(a)=\lim\limits_{n \rightarrow \infty} \frac{x+x^2e^{nx}}{1+e^{nx}}$
\task $f(a)=\lim\limits_{t \rightarrow \infty}\frac{\ln (1+e^{xt})}{1+e^t}$
\task! $f(a)=\lim\limits_{n \rightarrow \infty} (x \cdot \arctan(n \cdot \cot x))$
\end{tasks}
\end{defproblem}

\begin{defproblem}{s-funkcie-258}
Určte číslo $A$ tak, aby funkcia
$f_1(x) = \begin{cases}
  f(x), & $ak $ x \in D(f) \\
  A, &  $ak $ x=0
\end{cases}
  $ bola spojitá v bode $0$:
\begin{tasks}(3)
  \task $f(x)=\frac{1}{x^2}^{-\frac{1}{x^2}}$
  \task $f(x)=x^x,x>0$
  \task $f(x)=x \cdot \ln^2 x$
\end{tasks}
\end{defproblem}

\begin{defproblem}{s-funkcie-259}
Nech $f: \mathbb{R} \rightarrow \mathbb{R}$ je spojitá funkcia. Potom funkcia:
\[
F(x) = \begin{cases}
    -c   & \text{, ak } f(x)<-c \\
    f(x) & \text{, ak } |f(x)|\leq c \\
    0    & \text{, ak } f(x)>c
\end{cases}
\]
je spojitá. Dokážte!
\end{defproblem}

\begin{defproblem}{s-funkcie-260}
Zoraďme množinu $\mathbb{Q}$ do proste postupnosti ${\{q_n\}}_{n=1}^\infty$;
nech ${\{a_n\}}_{n=1}^\infty$ je daná postupnosť reálnych čísel. Definujme
funkciu $\psi$ (nazýva sa zovšeobecnená Riemannova funkcia) nasledovne:
\[
  \psi = \begin{cases}
  0 & $, ak $ x \in \mathbb{R} \setminus \mathbb{Q} \\
  a_n &  $, ak $ x=q_n
\end{cases}
\]
  Ak $\lim\limits_{n \rightarrow \infty} a_n=0$, tak funkcia $\psi$ je spojitá v
  každom iracionálnom čísle. Dokážte! Platí aj obrátená implikácia?
\end{defproblem}

\begin{defproblem}{s-funkcie-261}
Vyšetrite spojitosť funkcie
\[
  f(x) =
  \begin{cases}
      \frac{nx}{|x|} & \text{, ak } x=\frac{m}{n}, m\in \mathbb{Z} \text{ a } m\in \mathbb{N} \text{ sú nesúdeliteľné} \\
      f(x)           & \text{, ak } |f(x)|\leq c \\
      |x|            & \text{, ak } x \in \mathbb{R} \setminus \mathbb{Q}
  \end{cases}
\]
\end{defproblem}

\begin{defproblem}{s-funkcie-262}
Nech $f: \interval{a}{b} \rightarrow \mathbb{R}$ je spojitá funkcia. Vyšetrite
spojitosť funkcie
\[
  g(x) = \begin{cases}
  \sup\limits_{t \in \interval[open right]{a}{x}}f(t)
    - \inf\limits_{t \in \interval[open right]{a}{x}}f(t)
      &$, ak $ x \in \interval[open left]{a}{b} \\
  0 &  $, ak $ x=a
\end{cases}
\]
\end{defproblem}

\begin{defproblem}{s-funkcie-263}
\begin{tasks}
\task
  Nech je daná funkcia $f$, nech $x_0 \in D(f)$ a platí
  \[
    (\forall \delta > 0)
      (\exists \varepsilon >0):
        |x - x_0| < \delta \Rightarrow |f(x)-f(x_0)| < \varepsilon
  \]
  Vyplýva z týchto predpokladov spojitosť funkcie
  $f$ v bode $x_0$? Akú vlastnosť funkcie $f$ popisuje uvedená podmienka?
\task
  Nech je daná funkcia $f$, nech platí
  \[
    (\forall x_0 \in D(f))
      (\forall \delta > 0)
        (\exists \varepsilon > 0):
          |f(x)-f(x_0)| < \varepsilon \Rightarrow |x-x_0| < \delta
  \]
  Vyplýva z týchto predpokladov spojitosť funkcie $f$? Aká vlastnosť funkcie $f$
  je popísaná uvedenou podmienkou?
\end{tasks}
\end{defproblem}

\begin{defproblem}{s-funkcie-264}
Možno tvrdiť, že funkcia $g \circ f$ je nespojitá v bode $a$, ak $f$ nie je spojitá v bode $a \in D(f)$ a funkcia $g$ je
\begin{tasks}
\task spojitá na $\mathbb{R}$
\task prostá a spojitá na $\mathbb{R}$
\end{tasks}
\end{defproblem}

\begin{defproblem}{s-funkcie-265}
Zostrojte funkciu $f: \mathbb{R} \rightarrow \mathbb{R}$, ktorá je spojitá práve
v bodoch množiny $M$, ak
\begin{tasks}(3)
\task $M=\{0\}$
\task $M=\emptyset$
\task $M=\mathbb{N}$
\task $M=\{\frac{1}{n};n \in \mathbb{N}\}$
\end{tasks}
\end{defproblem}

\begin{defproblem}{s-funkcie-266}
Zistite, či existuje bijekcia $f: \mathbb{R} \rightarrow \mathbb{R}$, ktorá nie
je spojitá v žiadnom bode $x \in \mathbb{R}$.
\end{defproblem}

\begin{defproblem}{s-funkcie-267}
Dokážte, že neexistuje funkcia $f: \mathbb{R} \rightarrow \mathbb{R}$ taká, že
pre každé $a \in \mathbb{R}$ existuje vlastná $\lim\limits_{x \rightarrow
a}f(x)$ a platí $\lim\limits_{x \rightarrow a}f(x) \neq f(a)$.
\end{defproblem}

\begin{defproblem}{s-funkcie-268}
\begin{tasks}
\task
  Ak pre funkciu $f: \mathbb{R} \rightarrow \mathbb{R}$ platí $\lim\limits_{x
  \rightarrow a}f(x)=0$ v každom bode $a \in \mathbb{R}$, tak množina $A:=\{x
  \in \mathbb{R};f(x)\neq 0\}$ je spojiteľná. Dokážte!
\task
  Neexistuje funkcia $f: \mathbb{R} \rightarrow \mathbb{R}$, ktorá má v každom
  bode $a \in \mathbb{R}$ nevlastnú limitu. Dokážte!
\end{tasks}
\end{defproblem}

\begin{defproblem}{s-funkcie-269}
Ak polynóm $P$ párneho stupňa nadobúda aspoň jednu hodnotu, ktorá má oparné
znamienko ako koeficient pri člene s najvyššou mocninou, tak $P$ má aspoň dva
reálne korene. Dokážte!
\end{defproblem}

\begin{defproblem}{s-funkcie-270}
Ak je funkcia $f: \mathbb{R} \rightarrow \mathbb{Q}$ spojitá, tak je konštantná.
Dokážte!
\end{defproblem}

\begin{defproblem}{s-funkcie-271}
Nech funkcia $f: \interval[open right]{0}{\infty} \rightarrow \mathbb{R}$ je
spojitá a ohraničená a neexistuje $\lim\limits_{x \rightarrow \infty}f(x)$.
Potom existuje $A \in \mathbb{R}$, pre ktoré má rovnica $f(x)=A$ nekonečne veľa
riešení. Dokážte!
\end{defproblem}

\begin{defproblem}{s-funkcie-272}
Ak $f: \mathbb{R} \rightarrow \mathbb{R}$ je spojitá funkcia a pre každé $x \in
\mathbb{R}$ platí $f(f(x))=x$, tak existuje riešenie rovnice $f(x)=x$. Dokážte!
\end{defproblem}

\begin{defproblem}{s-funkcie-273}
Ak $f: \mathbb{R} \rightarrow \mathbb{R}$ je spojitá periodická funkcia s
periódou $T$, tak existuje také $a \in \mathbb{R}$, pre ktoré
$f(a+\frac{T}{2})=f(a)$. Dokážte!
\end{defproblem}

\begin{defproblem}{s-funkcie-274}
\par
\needspace{2\baselineskip}
\begin{tasks}
  \task
    Inverzná funkcia k rýdzomonotónnej funkcii definovanej na intervale je
    spojitá. Dokážte!
  \task
    Uveďte príklad prostej funkcie spojitej aj v bode $0$, ktorej inverzná
    funkcia nie je spojitá v bode $f(0)$.
  \task
    Uveďte príklad spojitej rýdzomonotónnej funkcie, ktorej inverzná funkcia
    nie je spojitá!
\end{tasks}
\end{defproblem}

\begin{defproblem}{s-funkcie-275}
Nech funkcia $f: \mathbb{R} \rightarrow \mathbb{R}$ je rovnomerne spojitá. Potom
existujú čísla $a \geq 0,b \geq 0$ tak, že pre všetky $x \in \mathbb{R}$ platí
$|f(x)| \leq a \cdot |x|+b$. Dokážte!
\end{defproblem}

\begin{defproblem}{s-funkcie-276}
\begin{tasks}
\task
  Ak sú funkcie $f,g$ rovnomerne spojité na ohraničenom intervale $(a,b)$, tak
  sú tam rovnomerne spojité aj funkcie $f+g,f \cdot g$. Dokážte!
\task
  Uveďte príklad funkcií $f \cdot g$ rovnomerne spojitých na intervale, ktorých
  súčin tam nie je rovnomerne spojitý.
\end{tasks}
\end{defproblem}

\begin{defproblem}{s-funkcie-277}
Rozhodnite o pravdivosti nasledujúceho tvrdenia: \enquote{Nech $g$ je spojitá
nekonštantná funkcia definovaná na intervale $I$, nech funkcia $f$ je spojitá na
intervale $g(I)$. Ak aspoň jedna z funkcií $f,g$ je rovnomerne spojitá, tak aj
funkcia $f \circ g$ je rovnomerne spojitá.}
\end{defproblem}

\begin{defproblem}{s-funkcie-278}
Popíšte funkcie, ktoré vyhovujú podmienke:
\begin{tasks}
\task
  $
    (\forall \varepsilon > 0)
      (\exists 0 < \delta <\varepsilon)
        (\forall x,y \in \mathbb{R}){:}
          |x-y| < \delta \Rightarrow |f(x)-f(y)|<\varepsilon
  $
\task
  $
    (\forall \varepsilon > 0)
      (\exists \delta > 0)
        (\forall x,y \in \mathbb{R}):
          |x - y| < \delta \Rightarrow f(x)-f(y) < \varepsilon
  $
\task
  $
    (\forall \varepsilon > 0)
      (\exists \delta > 0)
        (\forall x,y \in \mathbb{R}):
          |x-y| < \delta \Rightarrow |f(x)-f(y)| < \varepsilon$
\end{tasks}
\end{defproblem}

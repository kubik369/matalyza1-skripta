\begin{defproblem}{limita-1}
Dokážte, že bod $a$ je hromadný bod množiny $A$, ak
	
\begin{enumerate}
	\item $a=0 \quad A={x \in \mathbb{R}; \sin{\frac{1}{x}} = 0}$
	\item $a=-\infty \quad A={x \in \mathbb{R}; \cos{x} = \frac{1}{2}}$ 
	\item $a=\frac{1}{9} \quad A={\frac{m}{10^n}; m,n \in \mathbb{N}}$
\end{enumerate}
	
(teda $A$ je množina všetkých kladných čísel, ktorých zápis v desiatkovej sústave má konečný počet nenulových cifier za desatinnou čiarkou).
	
\end{defproblem}

\begin{defproblem}{limita-2}
Nájdite všetky hromadné body množín
\begin{enumerate}
\item $A= \langle 0,1 \rangle$ ;
\item $B= \{ (-1)^n n; n\in \mathbb{N} \} $;
\item $C= (-2,\infty)$ ;
\item $D= \{\frac{m}{n}; m,n \in \mathbb{N} \}$;
\item $E= {\ \frac{m}{n}; m<n,m,n \in \mathbb{Z}}$;
\item $F= \langle 1,2 ) \ \mathbb{Q}$;
\end{enumerate}
\end{defproblem}

\begin{defproblem}{limita-3}
Pomocou symbolov $\forall,\exists$ zapíšte výroky:
\begin{enumerate}
\item Bod $\infty$ nie je hromadný bod množiny $M$;
\item Množina $M$ nemá hromadné body.
\end{enumerate}
\end{defproblem}

\begin{defproblem}{limita-4}
Uveďte príklad neprázdnej množiny $A \subset \mathbb{R}$ takej, že
\begin{enumerate}
\item $A'=\emptyset$;
\item $A'={\ 1, +\infty}$;
\item $A'={\ -\infty,+\infty}$;
\item $A'$ je nespočítateľná množina;
\item $A'$ je nekonečná a spočítateľná množina.
\end{enumerate}
\end{defproblem}

\begin{defproblem}{limita-5}
Nech $A \subset \mathbb{R}$ je zhora ohraničená neprázdna množina, nech $sup A \notin A$. Potom $sup A$ je hromadný bod množiny $A$. Dokážte!
\end{defproblem}

\begin{defproblem}{limita-6}
 Na základe definície dokážte nasledujúce tvrdenia:
 \begin{enumerate}
 \item $\lim_{n \rightarrow \infty} \frac{3n^2+1}{5n^2-1}=\frac{3}{5}$ (pre ktoré $n \in \mathbb{N}$ platí: 
 \begin{enumerate}
 \item $|\frac{3n^2+1}{5n^2-1}-\frac{3}{5}|<0,5;$
 \item $<0,005;$
 \item $<0,00005 ?);$
 \end{enumerate}
 
 \item $\lim_{n \rightarrow \infty} \frac{n^2+3n+1}{2n^2+2}=\frac{1}{2}$;
 \item $\lim_{n \rightarrow \infty} \frac{n^2}{n+8}=+\infty$ 
 
 (počínajúc ktorým prirodzeným číslom platí nerovnosť $\frac{n^2}{n+8}>10^3$?);
 \item $\lim_{n \rightarrow \infty} (\frac{5}{n}-n)=-\infty$;
 \item $\lim_{n \rightarrow \infty} q^n=0$  $(|q|<1).$
 \end{enumerate}
\end{defproblem}

\begin{defproblem}{limita-7}

Rozhodnite, či existujú limity nasldujúcich postupností (nezabúdajte, že svoje tvrdenie musíte dokázať):
\begin{enumerate}
\item 
 $$a_n = \left\{ \begin{array}{r@{\quad}c}
    1-\frac{1}{n}, & $ak n $\in \mathbb{N}\ $je párne$ \\
    1+\frac{1}{n^2}, & $ak n $\in \mathbb{N}\ $je nepárne$ \\ \end{array} \right.
    $$ 
\item $$a_n=\frac{cos \frac{n \pi}{2}}{n}$$
\end{enumerate}
\end{defproblem}

\begin{defproblem}{limita-8}
Postupnosť ${\{a_n\}}_{n=1}^\infty$ je daná vzťahom $a_n=n(1-(-1)^n)).$ Dokážte, že
\begin{enumerate}
\item číslo $0$ nie je limitou tejto postupnosti;
\item bod $+\infty$ nie je limitou tejto postupnosti;
\item žiadne $b \in \mathbb{R^*}$ nie je limitou tejto postupnosti.
\end{enumerate}
\end{defproblem}

\begin{defproblem}{limita-9}
Pri formulácii definície vlastnej limity postupnosti študent:
\begin{enumerate}
\item namiesto "pre ľubovoľné $\varepsilon > 0$" povedal "pre ľubovoľné $\varepsilon$". Existujú postupnosti, ktoré majú limitu pri takejto definícii?
\item definíciu napísal takto:
$$\forall \varepsilon > 0 \exists n_0 \in \mathbb{N} \forall n \in \mathbb{N}: |a_n-b|<\varepsilon.$$
Ktoré postupnosti by mali limitu pri takejto definícii?
\item namiesto "pre každé $\varepsilon > 0$" povedal "aspoň pre jedno $\varepsilon > 0$". Ukážte, že pri takejto definícii je číslo $7$ limitou postupnosti $2,2, ...$ 
\item namiesto "existuje $n_0 \in \mathbb{N}$" povedal "pre všetky $n_0 \in \mathbb{N}$". Ktoré postupnosti majú limitu pri takejto definícii?
\item definćiu napísal takto:
$$\forall \varepsilon > 0 \exists n_0 \in \mathbb{N} \forall n \in \mathbb{N},n>n_0:a_n-b<\varepsilon.$$
Ukážte, že pri takejto definícii je čáslo $5$ limitou postupnosti $1,1,1,...$ .
\end{enumerate}
\end{defproblem}

\begin{defproblem}{limita-10}
Je číslo $b \in \mathbb{R}$ limitou postupnosti ${\{a_n\}}_{n=1}^\infty$, ak existuje také prirodzené číslo $\mathbb{N^*}$, že pre ľubovoľné $\varepsilon>0$ a všetky $n \in \mathbb{N}, n>\mathbb{N^*}$ platí $|a_n-b|<\varepsilon$? 
\end{defproblem}

\begin{defproblem}{limita-11}
Nájdite všetky postupnosti ${\{x_n\}}_{n=1}^\infty$, ktoré vyhovujú podmienke
\begin{enumerate}
\item $$\exists \varepsilon>0 \forall n_0 \in \mathbb{N} \forall n \in \mathbb{N}, n>n_0: |x_n|<\varepsilon; $$
\item $$\forall \varepsilon>0 \forall n_0 \in \mathbb{N} \forall n \in \mathbb{N}, n>n_0: |x_n|<\varepsilon; $$
\item $$\exists \varepsilon>0 \exists n_0 \in \mathbb{N} \forall n \in \mathbb{N}, n>n_0: |x_n|<\varepsilon. $$
\end{enumerate}
\end{defproblem}

\begin{defproblem}{limita-12}
Prepíšte definíciu limity funkcie pre prípady $1-9$. (Všimnite si, že limity postupností sú samy špeciálnym prípadom limít $4-6$.)
\end{defproblem}

\begin{defproblem}{limita-13}
Na základe definície limity dokážte tieto tvrdenia:
\begin{enumerate}
\item $\lim_{x \rightarrow 3} \sqrt{x}=\sqrt{3}$;
\item $\lim_{x \rightarrow 1} \frac{1}{(1-x^2)^2}=+\infty$;
\item $\lim_{x \rightarrow -\infty}x^3=-\infty$;
\item $\lim_{x \rightarrow 8} \sqrt[3]{x}=2$;
\item $\lim_{x \rightarrow -2} x^2=4$.
\end{enumerate}
\end{defproblem}

\begin{defproblem}{limita-14}
Nech bod $0$ je hromadný bod definičného oboru funkcie $f$. Pomocou symbolov $\forall, \exists$ zapíšte tieto tvrdenia:
\begin{enumerate}
\item Číslo $4$ nie je limitou funkcie $f$ v bode $0$;
\item Funkcia $f$ nemá v bode $0$ limitu.
\end{enumerate}
\end{defproblem}

\begin{defproblem}{limita-15}
Ak existuje $\lim_{x \rightarrow a} f(x)=b$ ($a \in \mathbb{R^*},b \in \mathbb{R}$), tak existuje aj $\lim_{x \rightarrow a} |f(x)|$ a platí $\lim_{x \rightarrow a} |f(x)|=|b|$. Dokážte; platí aj opačná implikácia?
\end{defproblem}

\begin{defproblem}{limita-16}
\begin{enumerate}
\item Dokážte implikáciu v Cauchyho-Bolzanovom kritériu konvergencie (t.j. dokážte, že tvrdenie (*) z vety $1$ je nutná podmienka existencie vlastnej limity funkcie $f$ v bode $a$).
\item Dokážte, že Dirichletova funkcia ani funkcia $sin \frac{1}{x}$ nemajú limitu v bode $0$. (Neexistenciu konečných limít možno dokázať na základe píkladu $115.1$; neexistenciu nevlastných limít treba dokázať samostatne.)
\end{enumerate}
\end{defproblem}

\begin{defproblem}{limita-17}
Nájdite nasledujúce limity:

\begin{multicols}{2}
\begin{enumerate}
    \item $\lim_{{x \rightarrow \infty}} \frac{x^2-1}{2x^2-x+1}$;
    \item $\lim_{{x \rightarrow -\infty}} \frac{3x^3+5x^2-2}{2x^4-7}$;
    \item $\lim_{{x \rightarrow \infty}} \frac{(x-1)(x-2)...(x-5)}{(5x-1)^5}$;
    \item $\lim_{{x \rightarrow \infty}} (\frac{x^3}{3x^2-4}-\frac{x^2}{3x+2})$;
    \item $\lim_{{x \rightarrow \infty}} \frac{1}{n}[(x+\frac{a}{n})+(x+\frac{2a}{n})+...+(x+\frac{n-1}{n}a)]$;
    \item $\lim_{{x \rightarrow \infty}} \frac{(-2)^n+3^n}{(-2)^{n+1}+3^{n+1}}$.
\end{enumerate}
\end{multicols}
\end{defproblem}

\begin{defproblem}{limita-18}
Nájdite nasledujúce limity:

\begin{multicols}{2}
\begin{enumerate}
    \item $\lim_{{x \rightarrow \infty}} \frac{x^2-5x+6}{x^3-6x^2+10x-3}$;
    \item $\lim_{{x \rightarrow -\infty}} \frac{x^2-1}{2x^2-x-1}$;
    \item $\lim_{{x \rightarrow \infty}} \frac{x^2-1}{2x^2-x-1}$;
    \item $\lim_{{x \rightarrow \infty}} \frac{x^4-3x+2}{x^5-4x+3}$;
    \item $\lim_{{x \rightarrow \infty}} \frac{2x^3-5x^2-4x+12}{5x^2-4x-12}$;
    \item $\lim_{{x \rightarrow \infty}} \frac{(x^3-x-2)^{20}}{(x^3-12x+16)^{10}}$;
    \item $\lim_{{x \rightarrow \infty}} \frac{x^m-1}{x^n-1}$ $(m,n \in \mathbb{N})$;
    \item $\lim_{{x \rightarrow \infty}} (\frac{2}{2x-x^2}+\frac{1}{x^2-3x+2})$;
    \item $\lim_{{x \rightarrow \infty}} \frac{x+x^2+...+x^n-n}{x-1}$;
    \item $\lim_{{x \rightarrow \infty}} \frac{x^{100}-2x+1}{x^{50}-2x+1}$.
\end{enumerate}
\end{multicols}
\end{defproblem}

\begin{defproblem}{limita-19}
Zostrojte funkcie $f,g$ definované na $\mathbb{R}$ tak, aby neexistovali $\lim_{x \rightarrow 1} f(x)$ ani $\lim_{x \rightarrow 1} g(x)$ a exitovala konečná 
\begin{multicols}{2}
\begin{enumerate}
    \item $\lim_{{x \rightarrow 1}} (f(x)+g(x))$;
    \item $\lim_{{x \rightarrow 1}} (f(x) \cdot g(x))$.
\end{enumerate}
\end{multicols}
\end{defproblem}

\begin{defproblem}{limita-20}
Možno nájsť postupnosť ${\{a_n\}}_{n=1}^\infty$ takú, že existuje konečná $\lim_{n \rightarrow \infty} a_n$ a neexistuje $\lim_{n \rightarrow \infty} \frac{1}{a_n}$?
\end{defproblem}

\begin{defproblem}{limita-21}
\begin{multicols}{2}
\begin{enumerate}
    \item $\lim_{{x \rightarrow -2}} \frac{\sqrt[3]{x-6}+2}{\sqrt{x^2-3}-1}$;
    \item $\lim_{{x \rightarrow 4}} \frac{\sqrt{1+2x}-3}{\sqrt{x}-2}$;
    \item $\lim_{{x \rightarrow a}} \frac{\sqrt{x}-\sqrt{a}+\sqrt{x-a}}{\sqrt{x^2-a^2}}$ $(a>0)$;
    \item $\lim_{{x \rightarrow 3}} \frac{\sqrt{x+13}-2\sqrt{x+1}}{x^2-9}$;
    \item $\lim_{{x \rightarrow 0}} \frac{\sqrt[3]{27+x}-\sqrt[3]{27-x}}{x+2\sqrt[3]{x^4}}$;
    \item $\lim_{{x \rightarrow 1}} \frac{(1-\sqrt{x})(1-\sqrt[3]{x})...(1-\sqrt[n]{x})}{(1-x)^{n-1}}$.
\end{enumerate}
\end{multicols}
\end{defproblem}

\begin{defproblem}{limita-22}
\begin{enumerate}
\item $\lim_{x \rightarrow 2} \frac{\sqrt{x+2}-\sqrt[3]{x^2+4}}{x-2}$;
\item  $\lim_{x \rightarrow 7} \frac{\sqrt{x+2}-\sqrt[3]{x^2+20}}{\sqrt[4]{x+9}-2}$;
\item  $\lim_{x \rightarrow 0} \frac{\sqrt[3]{1+\frac{x}{3}}-\sqrt[4]{1+\frac{x}{4}}}{1-\sqrt{1-\frac{x}{2}}}$.
\end{enumerate}
\end{defproblem}

\begin{defproblem}{limita-23}
\begin{enumerate}
\item $\lim_{x \rightarrow -1} \frac{1+\sqrt[3]{x}}{1+\sqrt[5]{x}}$;
\item $\lim_{x \rightarrow 0} \frac{\sqrt[n]{1+x}-1}{x}$  $(n \in \mathbb{N})$;
\item $\lim_{x \rightarrow 1} \frac{\sqrt[m]{x}-1}{\sqrt[n]{x}-1}$  $(m,n \in \mathbb{N})$;
\item $\lim_{x \rightarrow 1} (\frac{3}{1-\sqrt{x}}-\frac{2}{1-\sqrt[3]{x}})$.
\end{enumerate}
\end{defproblem}

\begin{defproblem}{limita-24}
\begin{multicols}{2}
\begin{enumerate}
    \item $\lim_{{x \rightarrow \infty}} \frac{\sqrt{x+\sqrt{x+\sqrt{x}}}}{\sqrt{x+1}}$;
    \item $\lim_{{x \rightarrow \infty}} \frac{\sqrt{x}+\sqrt[3]{x}+\sqrt[4]{x}}{\sqrt{2x+1}}$;
    \item $\lim_{{x \rightarrow \infty}} \frac{\sqrt{x^2+6}+|x|}{\sqrt[6]{x^4+2}-|x|}$;
    \item $\lim_{{x \rightarrow \infty}} (\sqrt{(x+a)(x+b)}-x)$;
    \item $\lim_{{x \rightarrow \infty}} (\sqrt[3]{x^3+x^2+1}-\sqrt[3]{x^3-x^2+1})$;
    \item $\lim_{{x \rightarrow \infty}} (\sqrt[n]{(x+a_1)(x+a_2)...(x+a_n)}-x)$;
    \item $\lim_{{x \rightarrow \infty}} x(\sqrt{x^2+2x}-2\sqrt{x^2+x}+x)$;
    \item $\lim_{{x \rightarrow \infty}} \frac{n}{2}(\sqrt[3]{1+\frac{2}{n}}-1)$.
\end{enumerate}
\end{multicols}
\end{defproblem}

\begin{defproblem}{limita-25}
Dokážte túto modifikáciu vety o limite zloženej funkcie: Nech $a \in \mathbb{R^*}$ je hromadný bod množiny $D(f \circ g)$, nech $A \in \mathbb{R}\cap D(f)$ je hromadný bod $D(f)$. Ak $\lim_{x \rightarrow a} g(x)=A$, $\lim_{x \rightarrow A} f(x)=f(A)$, tak $\lim_{x \rightarrow a} f(g(x))=f(A)$.

(Výhodou tejto modifikácie je, pri jej použití netreba overovať podmienku (*) vystupujúcu vo vete $3$.)
\end{defproblem}

\begin{defproblem}{limita-26}
Existuje funkcie $f,g$ definované na $\mathbb{R}$ taká, že $\lim_{x \rightarrow 1} g(x)=2$, $\lim_{x \rightarrow 2} f(x)$ existuje a $\lim_{x \rightarrow 1} f(g(x))$ neexistuje?
\end{defproblem}

\begin{defproblem}{limita-27}
Nájdite limity:
\begin{multicols}{2}
\begin{enumerate}
    \item $\lim_{x \rightarrow 0} \frac{sin 5x}{x}$;
    \item $\lim_{x \rightarrow 0} \frac{sin (x+1)}{x+1}$;
    \item $\lim_{x \rightarrow 0}  \frac{sin mx}{sin nx} (m,n \neq 0)$;
    \item $\lim_{x \rightarrow 0} \frac{sin (x^3+2x)}{x}$;
    \item $\lim_{x \rightarrow 0} x \cdot ctg 3x$;
    \item $\lim_{x \rightarrow \infty} 2^n sin \frac{x}{2^n}$  $(x \neq 0)$.
\end{enumerate}
\end{multicols}
\end{defproblem}

\begin{defproblem}{limita-28}
\begin{multicols}{2}
\begin{enumerate}
    \item $\lim_{{x \rightarrow 0}} \frac{1-cos x}{x^2}$;
    \item $\lim_{{x \rightarrow 0}} \frac{tg x-sin x}{sin ^3 x}$;
    \item $\lim_{{x \rightarrow 0}}  \frac{sin 5x-sin 3x}{sin x}$;
    \item $\lim_{{x \rightarrow 0}} (\frac{2}{sin 2x sin x}-\frac{1}{sin ^2 x})$;
    \item $\lim_{{x \rightarrow 0}} \frac{cos x - com 3x}{x^2} $
    \item $\lim_{x \rightarrow 0} \frac{1+sin x - cos x}{1+ sin px - cos px}$  $(p \neq 0)$.
\end{enumerate}
\end{multicols}
\end{defproblem}

\begin{defproblem}{limita-29}
\begin{multicols}{2}
\begin{enumerate}
    \item $\lim_{{x \rightarrow a}} \frac{sin x - sin a}{x-a}$;
    \item $\lim_{{x \rightarrow a}} \frac{ctg x - ctg a}{x-a}$;
    \item $\lim_{{x \rightarrow 0}}  \frac{cos (a+2x)-2 cos (a+x)+cos a}{x^2}$;
    \item $\lim_{{x \rightarrow 0}} (\frac{2}{sin (a+x) sin (a+2x)-sin ^2 a}{x})$.
\end{enumerate}
\end{multicols}
\end{defproblem}

\begin{defproblem}{limita-30}
\begin{multicols}{2}
\begin{enumerate}
    \item $\lim_{{x \rightarrow}} \frac{\pi}{4} tg 2x tg(\frac{\pi}{4}-x)$;
    \item $\lim_{{x \rightarrow}} \frac{\pi}{6} \frac{2 sin^2 x +sin x -1}{2 sin^2 x - 3 sin x + 1}$;
    \item $\lim_{{x \rightarrow}} \frac{\pi}{4}  \frac{1-ctg^3 x}{2- ctg x - ctg^3 x}$;
    \item $\lim_{{x \rightarrow}} \frac{sin(x-\frac{\pi}{3})}{1-2cos x}$;
    \item $\lim_{{x \rightarrow 1}} (1-x) tg \frac{\pi x}{2}$
    \item $\lim_{x \rightarrow 0} \frac{arcsin x}{x}$.
\end{enumerate}
\end{multicols}
\end{defproblem}

\begin{defproblem}{limita-31}
\begin{multicols}{2}
\begin{enumerate}
    \item $\lim_{{x \rightarrow 0}} \frac{\sqrt{1+tg x}-\sqrt{1+sin x}}{x^3}$;
    \item $\lim_{{x \rightarrow 0}} \frac{x^2}{\sqrt{1+x sin x}-\sqrt{cos x}}$;
    \item $\lim_{{x \rightarrow 0}} \frac{\sqrt{cos x}-\sqrt[3]{cos x}}{sin^2 x}$;
    \item $\lim_{{x \rightarrow 0}} \frac{\sqrt{1-cos x}}{1-cos \sqrt{x}}$;
    \item $\lim_{{x \rightarrow 0}} \frac{\sqrt{x+4}-2}{sin 5x}$.
\end{enumerate}
\end{multicols}
\end{defproblem}

\begin{defproblem}{limita-32}
\begin{multicols}{2}
\begin{enumerate}
    \item $\lim_{{x \rightarrow \infty}} \frac{sin x}{x}$;
    \item $\lim_{{x \rightarrow \infty}} \frac{x^2+sin x}{2x^2-cos x}$;
    \item $\lim_{{x \rightarrow \infty}} (sin \sqrt{x+1}-sin \sqrt{x})$;
    \item $\lim_{{x \rightarrow 0}} \frac{2+ln (e+x sin \frac{1}{x})}{cos x + sin x}$;
    \item $\lim_{{x \rightarrow 1}} \frac{cos 2\pi x}{2+(e^{\sqrt{x-1}}-1)arctg \frac{x+2}{x-1}}$.
\end{enumerate}
\end{multicols}
\end{defproblem}

\begin{defproblem}{limita-33}
\begin{enumerate}
\item $\lim_{n \rightarrow \infty} \frac{2^n}{n!}=0$;
\item $\lim_{n \rightarrow \infty} \frac{n}{2^n}=0$;
\item $\lim_{n \rightarrow \infty} \frac{n^3}{5^n}=0$.
\end{enumerate}
\end{defproblem}

\begin{defproblem}{limita-34}
\begin{enumerate}
\item Nech $0<q<1$ a nech postupnosť ${\{a_n\}}_{n=1}^\infty$ kladných čísel spĺňa podmienku $\frac{a_{n+1}}{a_{n}} \leq q$ pre všetky $n \in \mathbb{N}$. Potom $\lim_{n \rightarrow] \infty} a_n=0$. Dokážte!
\item Rozhodnite o platnosti tvrdenia " Ak postupnosť ${\{a_n\}}_{n=1}^\infty$ kldných čísel spĺňa podmienku $\frac{a_{n+1}}{a_{n}} \leq 1$ pre všetky $n \in \mathbb{N}$, tak $\lim_{n \rightarrow \infty} a_n=0$." ! 
\item Nájdite limity:
\begin{itemize}
\item $\lim_{n \rightarrow \infty} \frac{4^n n!}{(3n)^n}$;
\item $\lim_{n \rightarrow \infty} \frac{1001 \cdot 1002 \cdot ... \cdot (1000+?)}{1 \cdot 3 \cdot ... \cdot (2n-1)}$.
\end{itemize}
\end{enumerate}
\end{defproblem}

\begin{defproblem}{limita-35}
Nájdite limity: 
\begin{enumerate}
\item $\lim_{n \rightarrow \infty} (\frac{7}{3n})^n$;
\item $\lim_{n \rightarrow \infty} (\frac{2n+3}{n^2})^n$.
\end{enumerate}
\end{defproblem}

\begin{defproblem}{limita-36}
\begin{enumerate}
\item $\lim_{n \rightarrow \infty} \sqrt[n]{a}=1$  $(a>0)$;
\item $\lim_{n \rightarrow \infty} \sqrt[n]{n}=1$.
\end{enumerate}
\end{defproblem}

\begin{defproblem}{limita-37}
\begin{multicols}{2}
\begin{enumerate}
    \item $\lim_{{x \rightarrow \infty}} \frac{n 3^n+1}{n!+1}$;
    \item $\lim_{{x \rightarrow \infty}} \sqrt[n]{\frac{5n+1}{n+5}}$;
    \item $\lim_{{x \rightarrow \infty}} \sqrt[n]{3^n-2^n}$;
    \item $\lim_{{x \rightarrow \infty}} \sqrt[n]{\frac{1}{2}-\frac{1}{2^n}}$.
\end{enumerate}
\end{multicols}
\end{defproblem}

\begin{defproblem}{limita-38}
\begin{multicols}{2}
\begin{enumerate}
    \item $\lim_{{x \rightarrow -\infty}} (\sqrt{(x+a)(x+b)}-x)$;
    \item $\lim_{{x \rightarrow -\infty}} x(\sqrt{x^2+1}-x)$;
    \item $\lim_{{x \rightarrow \infty}} x^2arcsin(\frac{x^2+1}{3x^2-2})$;
    \item $\lim_{{x \rightarrow -\infty}} \frac{x^4-5x}{x^2-3x+1}$;
    \item $\lim_{{x \rightarrow \infty}} \frac{\sqrt{x^3+\sqrt{x^3+1}}}{\sqrt[3]{x^2+\sqrt[3]{x^2+1}}}$;
    \item $\lim_{{x \rightarrow \infty}} (\frac{x}{10})^n$;
    \item $\lim_{{x \rightarrow \infty}} (2x+x sin x)$.
\end{enumerate}
\end{multicols}
\end{defproblem}

\begin{defproblem}{limita-39}
\begin{multicols}{2}
\begin{enumerate}
    \item $\lim_{{x \rightarrow 0}} \frac{1}{sin ^2 x}$;
    \item $\lim_{{x \rightarrow 0}} \frac{sin x}{x^3}$;
    \item $\lim_{{x \rightarrow 0}} \frac{1-cos x}{1-cos x^2}$;
    \item $\lim_{{x \rightarrow 0}} \frac{\sqrt{x+1}-1}{\sqrt{x^3+1}-1}$;
    \item $\lim_{{x \rightarrow \frac{\pi}{2}}} \frac{tg ^2 x}{(2x-\pi)^4}$;
    \item $\lim_{{x \rightarrow \infty}} \frac{1}{\sqrt[n]{2}-1}$.
\end{enumerate}
\end{multicols}
\end{defproblem}

\begin{defproblem}{limita-40}
Uveďte príklad funkcií $f,g$ definovaných v niektorom prstencovom okolí bodu $1$ takých, že $\lim_{x \rightarrow 1} f(x)=+\infty,\lim_{x \rightarrow 1} g(x)=-\infty$ a $\lim_{x \rightarrow 1} (f(x)+g(x))$:
\begin{enumerate}
\item je konečná;
\item je nevlastná;
\item neexistuje.
\end{enumerate}
\end{defproblem}

\begin{defproblem}{limita-41}
Uveďte príklady postupností nenulových čísel ${\{a_n\}}_{n=1}^\infty$ a ${\{b_n\}}_{n=1}^\infty$ takých, že $\lim_{n \rightarrow \infty} a_n=+\infty,\lim_{n \rightarrow \infty} b_n=0$ a $\lim_{n \rightarrow \infty} a_nb_n$:
\begin{enumerate}
\item $=0$;
\item $=+\infty$;
\item je konečná a nenulová;
\item neexistuje.
\end{enumerate}
\end{defproblem}

\begin{defproblem}{limita-42}
Nech $R$ je racionálna funkcia, t.j. funkcia daná predpisom $R(x)=\frac{a_0x^n+a_1x^{n-1}+...+a_n}{b_0x^m+b_1x^{m-1}+...+b_m},(a_0 \neq 0, b_0 \neq 0,m,n \in \mathbb{N} \cup \{ 0\})$. Čomu sa rovná $\lim_{x \rightarrow \infty} R(x)$ ? 
\end{defproblem}

\begin{defproblem}{limita-43}
Pomocou kvantifikátorov a nerovností zapíšte tvrdenia:
\begin{enumerate}
\item $\lim_{x \rightarrow a+} f(x)=b$;
\item $\lim_{x \rightarrow a-} f(x)=b$.
\end{enumerate}
$(a,b \in \mathbb{R})$
\end{defproblem}

\begin{defproblem}{limita-44}
\begin{multicols}{2}
\begin{enumerate}
    \item $\lim_{{x \rightarrow 0}} \frac{x^2-1}{x-1|},a=1$;
    \item $\lim_{{x \rightarrow 0}} \frac{\sqrt{1-cos 2x}}{x},a=0$;
    \item $\lim_{{x \rightarrow 0}} \frac{5}{(x-2)^3},a=2$;
    \item $\lim_{{x \rightarrow 0}} \frac{1}{2-2^{\frac{1}{x}},a=0}$.
\end{enumerate}
\end{multicols}
\end{defproblem}

\begin{defproblem}{limita-45}
Vyšetrite existenciu nasledujúcich limít:
\begin{enumerate}
\item $\lim_{x \rightarrow \frac{\pi}{2}} x tg x$;
\item $\lim_{x \rightarrow 0} x sgn x$;
\item $\lim_{x \rightarrow 0} \frac{sin x}{x^2}$.
\end{enumerate}
\end{defproblem}

\begin{defproblem}{limita-46}
Uveďte príklad funkcie $f$: $\mathbb{R} \setminus \{ 0\} \rightarrow \mathbb{R}$ takej, že 
\begin{enumerate}
\item $\lim_{x \rightarrow 0-} f(x)>\lim_{x \rightarrow 0+} f(x)$;
\item $\lim_{x \rightarrow 0-} f(x)$ neexistuje, $\lim_{x \rightarrow 0+} f(x)$ je nevlastná.
\end{enumerate}
\end{defproblem}

\begin{defproblem}{limita-47}
\begin{enumerate}
\item Nech $\lim_{x \rightarrow \infty} f(x)=\lim_{x \rightarrow -\infty} f(x)=b$  $(\in \mathbb{R^*})$. Potom existuje aj $\lim_{x \rightarrow 0} f(\frac{1}{x})$  a rovná sa $b$. Dokážte!
\item Nech: $f: \mathbb{R} \rightarrow \mathbb{R}$ je nepárna funkcia. Akú hodnotu musí mať $\lim_{x \rightarrow 0+} f(x)$, aby existovala $\lim_{x \rightarrow 0} f(x)$? (Funkcia $\psi$ sa nazýva párna (nepárna), ak vzhovuje nasledujúcim podmienkam:
\begin{enumerate}
\item $\forall x \in D(\varphi): -x \in D(\varphi)$;
\item $\forall x \in D(\varphi): \varphi(x)=\varphi(-x) (\forall x \in D(\varphi): \varphi(-x)=-\varphi(x))$.)
\end{enumerate}
\end{enumerate}
\end{defproblem}

\begin{defproblem}{limita-48}
Nech je daná funkcia $g$ a kladná funkcia $f$, nech $a \in \mathbb{R^*}$ je hromadný bod množiny $D(f)\cap D(g)$. Ak $\lim_{x \rightarrow a} g(x)=A \in \mathbb{R^+},\lim_{x \rightarrow a} g(x)=B \in \mathbb{R}$, tak existuje aj $\lim_{x \rightarrow a} f(x)^{g(x)}$ a rovná sa $A^B$. Dokážte!
\end{defproblem}

\begin{defproblem}{limita-49}
Nájdite $\lim_{x \rightarrow \frac{\pi}{2}} (sin x)^{tg x}$!
\end{defproblem}

\begin{defproblem}{limita-50}
Sformulujte a dokážte pravidlá pre výpočet limít typu
\begin{multicols}{2}
\begin{enumerate}
    \item $a^{+\infty}$, kde $a \in (0,1)$;
    \item $a^{-\infty}$, kde $a \in (0,1)$;
    \item $a^{+\infty}$, kde $a \in (1,\infty) \cup \{+\infty \}$;
    \item $a^{-\infty}$, kde $a \in (1,\infty) \cup \{+\infty \}$.
\end{enumerate}
\end{multicols}
\end{defproblem}

\begin{defproblem}{limita-51}
\begin{multicols}{2}
\begin{enumerate}
    \item $\lim_{{x \rightarrow 1}} (\frac{1+x}{2+x})^{\frac{1-\sqrt{x}}{1-x}}$;
    \item $\lim_{{x \rightarrow 2}} (\frac{\sqrt{x+2}-2}{x^2-4})^{\frac{1}{x}}$;
    \item $\lim_{{x \rightarrow \infty}} (\frac{2x^2+7}{x^2+3})^{\frac{3x^3-11}{4x^2-12}}$;
    \item $\lim_{{x \rightarrow 1}} (\frac{1+cos \pi x}{tg ^2 \pi x})^{x^2}$;
    \item $\lim_{{x \rightarrow \infty}} sin ^n \frac{2 \pi n}{3n+1}$;
    \item $\lim_{{x \rightarrow 0}} (1+cos x)^{-\frac{1}{x^2}}$;
    \item $\lim_{{x \rightarrow 0}} (1+\frac{1}{x^2})^{\frac{sin x}{x}}$;
    \item $\lim_{{x \rightarrow 0}} (1+cos x)^{\frac{1}{x}}$.
\end{enumerate}
\end{multicols}
\end{defproblem}

\begin{defproblem}{limita-52}
\begin{multicols}{2}
\begin{enumerate}
    \item $\lim_{{x \rightarrow 1}} (\frac{x^2+1}{x^2-2})^{x^2}$;
    \item $\lim_{{x \rightarrow 2}} \sqrt[x]{1-2x}$;
    \item $\lim_{{x \rightarrow \infty}} (\frac{2x-1}{x})^{\frac{1}{\sqrt[3]{x}-1}}$;
    \item $\lim_{{x \rightarrow 1}} (1+sin \pi x)^{ctg \pi x}$;
    \item $\lim_{{x \rightarrow \infty}} (\frac{sin x}{sin a})^{\frac{1}{x-a}}$;
    \item $\lim_{{x \rightarrow 0}} (\frac{1+tg x}{1+sin x})^{\frac{1}{sin x}}$;
    \item $\lim_{{x \rightarrow 0}} (sin \frac{1}{x}+cos \frac{1}{x})^x$;
    \item $\lim_{{x \rightarrow 0}} cos ^n \frac{x}{\sqrt{n}}$.
\end{enumerate}
\end{multicols}
\end{defproblem}

\begin{defproblem}{limita-53}
\begin{multicols}{2}
\begin{enumerate}
    \item $\lim_{{x \rightarrow 0}} \frac{ln(1+x)}{x}$;
    \item $\lim_{{x \rightarrow \infty}} x(ln(1+x)-ln x)$;
    \item $\lim_{{x \rightarrow a}} \frac{ln x - ln a}{x-a}$;
    \item $\lim_{{x \rightarrow 0}} \frac{sin 3x}{ln(1+sin 5x)}$;
    \item $\lim_{{x \rightarrow -\infty}} (\frac{ln (1+3^x)}{(1+2^x)}$;
    \item $\lim_{{x \rightarrow 0}} \frac{ln(nx+\sqrt{1-n^2x^2})}{ln(x+\sqrt{1-x^2})}$.
\end{enumerate}
\end{multicols}
\end{defproblem}

\begin{defproblem}{limita-54}
\begin{multicols}{2}
\begin{enumerate}
    \item $\lim_{{x \rightarrow 0}} \frac{e^x-1}{x}$;
    \item $\lim_{{x \rightarrow 0}} \frac{sh x}{x}$;
    \item $\lim_{{x \rightarrow 2}} \frac{2^x-x^2}{x-2}$;
    \item $\lim_{{x \rightarrow a}} \frac{x^x-x^a}{x-a},(a>0)$;
    \item $\lim_{{x \rightarrow 0}} \frac{ln^2 (1+5x)}{e^{x sin 4x}-1}$;
    \item $\lim_{{x \rightarrow 0}} \frac{e^{\alpha x}-e^{\beta x}}{x-a},(\alpha \neq \beta)$;
    \item $\lim_{{x \rightarrow \infty}}n(\sqrt[n]{x}-1),(x>0)$;
    \item $\lim_{{x \rightarrow 0}} ({2e^{\frac{x}{x+1}-1}})^{\frac{x^2+1}{x}}$.
\end{enumerate}
\end{multicols}
\end{defproblem}

\begin{defproblem}{limita-55}
\begin{multicols}{2}
\begin{enumerate}
    \item $\lim_{{x \rightarrow 0}} \frac{ln(x^2+e^x)}{ln(x^4+e^3x)}$;
    \item $\lim_{{x \rightarrow \infty}} \frac{ln(x^2+e^x)}{ln(x^4+e^3x)}$;
    \item $\lim_{{x \rightarrow -\infty}} \frac{ln(x^2+e^x)}{ln(x^4+e^3x)}$.
\end{enumerate}
\end{multicols}
\end{defproblem}

\begin{defproblem}{limita-56}
Dokážte, že nasledujúce postupnosti sú konvergentné:
\begin{enumerate}
\item $a_n=(1-\frac{1}{2})(1-\frac{1}{4})...(1-\frac{1}{2^n})$;
\item $a_n=1+\frac{1}{1!}+\frac{1}{2!}+...+\frac{1}{n!}$;
\item $a_n=\frac{1}{n+1}+\frac{1}{n+2}+...+\frac{1}{2n}$;
\item $a_n=\frac{1}{5+1}+\frac{1}{n5^2+2}+...+\frac{1}{5^n+n}$;
\item $a_n=\frac{10}{1}\cdot \frac{11}{3}\cdot ...\cdot \frac{n+9}{2n-1}$.
\end{enumerate}
\end{defproblem}

\begin{defproblem}{limita-57}
Dokážte, že nasledujúce postupnosti sú konvergentné a nájdite ich limity:
\begin{enumerate}
\item $a_1=\sqrt{2}, a_2=\sqrt{2+\sqrt{2}}, a_3=\sqrt{2+\sqrt{2+\sqrt{2}}},...$;
\item $a_1=\frac{1}{2}, a_n=\frac{1}{2}+\frac{{a_{n-1}}^{2}}{2}$;
\item $a_1>0, a_n=\frac{a_{n-1}}{2+a_{n-1}}$;
\item $a_1=1, a_n=\frac{a_{n-1}}{2+a_{n-1}}$;
\item $a_n=\underbrace{sin sin ... sin}_n x (x \in \mathbb{R})$
\end{enumerate}
\end{defproblem}

\begin{defproblem}{limita-58}
Dokážte, že postupnosť $a_{n}=(1+\frac{1}{n})^{n+1}$ je klesajúca a zdola ohraničená. Na základe toho dokážte nerovnosť 
$$(1+\frac{1}{n})^n<e<(1+\frac{1}{n})^{n+1}.$$
\end{defproblem}

\begin{defproblem}{limita-59}
Dokážte, že neexistujú nasledujúce limity:
\begin{multicols}{2}
\begin{enumerate}
    \item $\lim_{{x \rightarrow \infty}} \sin x$;
    \item $\lim_{{x \rightarrow a}} \zeta (x), (a \in \mathbb{R^*})$;
    \item $\lim_{{x \rightarrow 0}} \cos \frac{\pi}{x}$;
    \item $\lim_{{x \rightarrow -\infty}} f(x)$, ak $f$ je nekonečná periodická funkcia.
\end{enumerate}
\end{multicols}
\end{defproblem}

\begin{defproblem}{limita-60}
Nech $f$ je funkcia definovaná na $\mathbb{R}$, $a \in \mathbb{R^*}$. Rozhodnite, či tvrdenie $\lim_{x \rightarrow a} f(x)=b$ je ekvivalentné s niektorým z nasledujúcich výrokov:
\begin{enumerate}
\item pre každú postupnosť ${\{a_n\}}_{n=1}^\infty$ racionálnych čísel takú, že $a_n \neq a$ pre všetky $n \in \mathbb{N}$ a $\lim_{n \rightarrow \infty} a_n=a$, platí $\lim_{n \rightarrow \infty} f(a_n)=b$;
\item pre každú postupnosť ${\{a_n\}}_{n=1}^\infty$ takú, že $\lim_{n \rightarrow \infty} a_n=a$, pričom množina $\{ a_n : n\in \mathbb{N }\}$ je podmnožinou $Q \setminus \{ a\}$ alebo podmnožinou $\mathbb{R} \setminus (\mathbb{Q} \cup \{ a\})$ platí $\lim_{n \rightarrow \infty} f(a_n)=b$.
\end{enumerate}
\end{defproblem}

\begin{defproblem}{limita-61}
Nájdite limes superior a limes inferior nasledujúcich postupností:
\begin{multicols}{2}
\begin{enumerate}
    \item $a_n=(-1)^{n-1}(2+\frac{3}{n})$;
    \item $a_n=1+2 \cdot (-1)^{n+1}+3 \cdot (-1)^{\frac{n(n-1)}{2}}$;
    \item $a_n=\cos \frac{n \pi}{3}$;
    \item $a_n=(1+\frac{1}{n})^n(-1)^n+\sin \frac{n \pi}{4}$;
    \item $a_n=n^{(-1)^n}$;
    \item $a_n=\sqrt[n]{1+2^{n \cdot (-1)^n}}$;
\end{enumerate}
\end{multicols}
\begin{enumerate}[resume]
    \item $1,\frac{1}{10},\frac{2}{10},...,\frac{9}{10},\frac{1}{10^2},\frac{2}{10^2},...,\frac{99}{10^2},...,\frac{1}{10^n},...,\frac{10^n-1}{10^n},...$;
    \item $1,\frac{1}{2},\frac{2}{2},\frac{3}{2},\frac{1}{4}\frac{2}{4},...,\frac{5}{4},...,\frac{1}{2^n},...,\frac{2^n+1}{2^n},... $.
\end{enumerate}
\end{defproblem}

\begin{defproblem}{limita-62}
Zostrojte postupnosť ${\{a_n\}}_{n=1}^\infty$ tak, aby množina jej hromadných hodnôt bola:
\begin{enumerate}
\item $\{ 1\}$;
\item $\{ 0,1\}$;
\item daná konečná množina $\{ a_1,...,a_n\}$;
\item $\mathbb{N} \cup \{ +\infty \}$.
\end{enumerate}
\end{defproblem}

\begin{defproblem}{limita-63}
Nech ${\{a_n\}}_{n=1}^\infty$ je postupnosť taká, že $\{ a_n: n \in \mathbb{N} \}=\mathbb{Q}$ (taká postupnosť existuje, pretože $\mathbb{Q}$ je spočítateľná množina). Potom množina hromadných hodnôt postupnosti ${\{a_n\}}_{n=1}^\infty$ je $\mathbb{R^*}$. Dokážte!
\end{defproblem}

\begin{defproblem}{limita-64}
Nech ohraničená postupnosť ${\{a_n\}}_{n=1}^\infty$ má práve dve hromadné hodnoty: označme $a := \liminf_{n \rightarrow \infty} a_n,b := \limsup_{n \rightarrow \infty} a_n$. Dokážte: pre každé $\epsilon \in (0,\frac{b-a}{2})$ je množina $\mathbb{N_\epsilon} := \{n \in \mathbb{N}; a_n \in \langle a+\epsilon, b-\epsilon \rangle \}$ konečná.
\end{defproblem}

\begin{defproblem}{limita-65}
Nech pre ohraničenú postupnosť ${\{a_n\}}_{n=1}^\infty$ platí $$\forall \epsilon > 0: \limsup_{n \rightarrow \infty} a_n < \liminf_{n \rightarrow \infty} a_n+\epsilon.$$ Potom je postupnosť ${\{a_n\}}_{n=1}^\infty$ konvergentná. Dokážte! 
\end{defproblem}

\begin{defproblem}{limita-66}
\begin{enumerate}
\item Nech $a$ je hromadný bod množiny $A$ hromadných hodnôt postupnosti ${\{a_n\}}_{n=1}^\infty$. Potom $a$ je h romadnou hodnotou postupnosti ${\{a_n\}}_{n=1}^\infty$.
\item Existuje postupnosť, ktorej množina hromadných hodnôt je $A=\{ \frac{1}{n}; n \in \mathbb{N} \}$ ?
\end{enumerate}
\end{defproblem}

\begin{defproblem}{limita-67}
\begin{enumerate}
\item Nech sú dané ohraničené postupnosti ${\{a_n\}}_{n=1}^\infty$ a ${\{b_n\}}_{n=1}^\infty$. Potom
\begin{enumerate}
\item ak existuje konečná $\lim_{n \rightarrow \infty} a_n$, tak 
$$\liminf_{n \rightarrow \infty} (a_n+b_n)=\lim_{n \rightarrow \infty} a_n+\liminf_{n \rightarrow \infty} b_n;$$
\item $\liminf_{n \rightarrow \infty} a_n+\liminf_{n \rightarrow \infty} b_n \leq \liminf_{n \rightarrow \infty} (a_n+b_n)\leq \liminf_{n \rightarrow \infty} a_n+\limsup_{n \rightarrow \infty} b_n\leq \limsup_{n \rightarrow \infty} (a_n+b_n) \leq \limsup_{n \rightarrow \infty} a_n+\limsup_{n \rightarrow \infty} b_n$.
\end{enumerate}
\item Uveďte príklady postupností ${\{a_n\}}_{n=1}^\infty$ a ${\{b_n\}}_{n=1}^\infty$, pre ktoré budú jednotlivé nerovnosti v príklade $166.1b/$ ostré.
\end{enumerate}
\end{defproblem}

\begin{defproblem}{limita-68}
Uveďte príklad neprázdnej množiny $a \subset \mathbb{R}$, že $A' \neq \emptyset$ a platí
\begin{enumerate}
\item $A' \subsetneq A$;
\item $A'=A$;
\item $A \subsetneq A'$;
\item $A' \cap A = \emptyset$;
\item $A \not\subset A'\wedge A' \not\subset A \wedge A' \cap A \neq\emptyset$
\end{enumerate} 
\end{defproblem}

\begin{defproblem}{limita-69}
Ak $a \in \mathbb{R^*}$ je hromadný bod množiny $A'$, tak $a$ je aj hromadný bod množiny $A$ (teda $(A')' \subset A'$). Dokážte!
\end{defproblem}

\begin{defproblem}{limita-70}
Existuje množiny $A$ taká, že $A'=(0,1)$?
\end{defproblem}

\begin{defproblem}{limita-71}
Nech $a \in \mathbb{R^*}$ je hromadný bod množiny $A \cup B$. Potom $a$ je hromadný bod množiny $A$ alebo $a$ je hromadný bod množiny $B$ (teda $(A \cup B)' \subset A' \cup B'$). Dokážte!
\end{defproblem}

\begin{defproblem}{limita-72}
Nech je daná postupnosť ${\{a_n\}}_{n=1}^\infty$, definujme postupnosť ${\{b_n\}}_{n=1}^\infty$ predpisom 
$$b_n=\frac{1}{n}(a_1+a_2+...+a_n).$$
Ak $\lim_{n \rightarrow \infty} a_n=a (a \in \mathbb{R^*})$, tak existuje aj $\lim_{n \rightarrow \infty} b_n$ a platí $\lim_{n \rightarrow \infty} a_n=a$. Dokážte; ďalej ukážte, že obrátená implikácia vo všeobecnosti neplatí.
\end{defproblem}

\begin{defproblem}{limita-73}
Nájdite limity:
\begin{enumerate}
\item $\lim_{n \rightarrow \infty} \frac{1}{n}(1+\frac{1}{2}+...+\frac{1}{n})$;
\item $\lim_{n \rightarrow \infty} \frac{1}{n}(1+\frac{1}{\sqrt{2}}+...+\frac{1}{\sqrt{n}})$
\end{enumerate}
\end{defproblem}

\begin{defproblem}{limita-74}
\begin{enumerate}
\item Nech ${\{a_n\}}_{n=1}^\infty$ je postupnosť taká, že $\{ a_n: n \in \mathbb{N} \}=\mathbb{N}$. Potom existuje $\lim_{n \rightarrow \infty} a_n$ a rovná sa $+\infty$.
\item Postupnosť ${\{b_n\}}_{n=1}^\infty$ sa nazýva prerovnaním postupnosti ${\{a_n\}}_{n=1}^\infty$, ak existuje taká bijekcia $p: \mathbb{N} \rightarrow \mathbb{N}$, že $b_n=a_{p(n)},(n \in \mathbb{N})$. Ak $\lim_{n \rightarrow \infty} a_n=b,(b \in \mathbb{R^*})$, tak každé prerovnanie postupnosti ${\{a_n\}}_{n=1}^\infty$, má tiež limitu rovnú $b$. Dokážte!
\end{enumerate}
\end{defproblem}

\begin{defproblem}{limita-75}
Zostane tvrdenie z príkladu $173.(a)$  v platnosti, ak v ňom
\begin{enumerate}
\item vynecháme predpoklad "${\{a_n\}}_{n=1}^\infty$, je prostá postupnosť"?;
\item Predpoklad "$\{a_n: n \in \mathbb{N}\}=\mathbb{N}$" nahradíme predpokladom "$\{a_n: n \in \mathbb{N}\} \subset \mathbb{N}$"?
\end{enumerate}
\end{defproblem}

\begin{defproblem}{limita-76}
Nech ${\{a_n\}}_{n=1}^\infty$ je taká postupnosť, že existuje $\lim_{n \rightarrow \infty} |a_n|=b$ a neexistuje $\lim_{n \rightarrow \infty} a_n$. Dokážte, že
\begin{enumerate}
\item $b \neq 0$;
\item množiny $\mathbb{N^+}:= \{n \in \mathbb{N}: a_n>0 \}$ a $\mathbb{N^-}:= \{n \in \mathbb{N}: a_n<0 \}$ sú konečné;
\item množina $\mathbb{N} \setminus (\mathbb{N^+} \cup \mathbb{N^-})$ je konečná.
\end{enumerate}
\end{defproblem}

\begin{defproblem}{limita-77}
Na základe definície limity dokážte:
\begin{multicols}{2}
\begin{enumerate}
    \item $\lim_{n \rightarrow \infty} \frac{5 \cdot 3^n}{3^n-2}=5$;
    \item $\lim_{x \rightarrow \infty} arctg x=\frac{\pi}{2}$.
\end{enumerate}
\end{multicols}
\end{defproblem}

\begin{defproblem}{limita-78}
Uveďte príklad funkcie $f: \mathbb{R} \rightarrow \mathbb{R}$, Ktorá má limitu len v bode $0$.
\end{defproblem}

\begin{defproblem}{limita-79}
Dokážte, že postupnosť ${\{\sin n\}}_{n=1}^\infty$ nemá limitu.
\end{defproblem}

\begin{defproblem}{limita-80}
Nájdite limity:
\begin{multicols}{2}
\begin{enumerate}
    \item $\lim_{x \rightarrow 0} \frac{(1+mx)^n-(1+nx)^m}{x^2},(m,n \in \mathbb{N})$;
    \item $\lim_{x \rightarrow a} \frac{x^n-a^n-na^{n-1}(x-a)}{(x-a)^2}$;
    \item $\lim_{x \rightarrow 1} (\frac{3}{1-x^2}-\frac{1}{x-1})$;
    \item $\lim_{x \rightarrow 1} (\frac{m}{1-x^m}-\frac{n}{1-x^n}),(m,n \in \mathbb{N})$;
    \item $\lim_{x \rightarrow \infty} \frac{1}{n}[(x+\frac{a}{n})^2+(x+\frac{2a}{n})^2+...+(x+\frac{n-1}{n}a)^2]$;
    \item $\lim_{n\rightarrow \infty} \frac{(x+1)(x^2+1)...(x^n+1)}{[(nx)^n+1]^{\frac{n+1}{2}}}$;
    \item $\lim_{n \rightarrow \infty} \frac{(n+4)!-(n+2)!}{(n+3)!}$;
    \item $\lim_{n \rightarrow \infty} \frac{1+3+...+(2n-1)}{1+4+...+(3n-2)}$;
    \item $\lim_{n \rightarrow \infty} \frac{1-2+3-4+...+(-1)^{n-1}n}{n}$;
    \item $\lim_{x \rightarrow \infty} (\frac{5}{6}+\frac{13}{36}+...+\frac{2^n-3^n}{6^n})$.
\end{enumerate}
\end{multicols}
\end{defproblem}

\begin{defproblem}{limita-81}
Nech ${\{a_n\}}_{n=1}^\infty$ je konvergentná postupnosť. Potom existuje maximum alebo minimum množiny $\{a_n: n \in \mathbb{N}\}$. Dokážte!
\end{defproblem}

\begin{defproblem}{limita-82}
Dokážte, že neexistuje racionálna funkcia $\mathbb{R}$ s celočíselnými koeficientami taká, aby platilo 
$$\forall r \in \mathbb{Q} \exists k \in \mathbb{Z}: R(k)=r.$$
\end{defproblem}

\begin{defproblem}{limita-83}
Nájdite limity:
\begin{enumerate}
\item $\lim_{x \rightarrow 0 \frac{\sqrt[m]{1+\alpha x}-\sqrt[n]{1+\beta x}}{x}},(m,n \in \mathbb{Z})$;
\item $\lim_{x \rightarrow 0}  \frac{\sqrt[m]{1+\alpha x}\sqrt[n]{1+\beta x}-1}{x},(m,n \in \mathbb{Z})$;
\item $\lim_{x \rightarrow 0} (\sqrt{\frac{1}{x}+\sqrt{\frac{1}{x}+\sqrt{\frac{1}{x}}}}-\sqrt{\frac{1}{x}+\sqrt{\frac{1}{x}+\sqrt{\frac{1}{x}}}})$;
\end{enumerate}
\begin{multicols}{2}
\begin{enumerate}[resume]
    \item $\lim_{x \rightarrow -\infty} (x+\sqrt{\frac{x^3+2x^2}{x+1}})$;
    \item $\lim_{x \rightarrow -\infty} \arcsin (\sqrt{x^2+x}+x)$;
    \item $\lim_{x \rightarrow \infty} \frac{\sqrt{3x-1}-\sqrt[3]{125n^3+n}}{\sqrt[5]{n}-n}$;
    \item $\lim_{x \rightarrow \infty} x^{\frac{3}{2}}(\sqrt{x+2}-2\sqrt{x+1}+\sqrt{x})$;
    \item $\lim_{x \rightarrow \infty} (\sqrt[3]{x^3+3x^2}-\sqrt{x^2-2x})$;
    \item $\lim_{x \rightarrow \infty} \frac{(x-\sqrt{x^2-1})^n+(x+\sqrt{x^2-1})^n}{x^n}, (n \in \mathbb{N})$.
\end{enumerate}
\end{multicols}
\end{defproblem}

\begin{defproblem}{limita-84}
Nech $f,g: \mathbb{R} \rightarrow \mathbb{R}$ sú periodické funkcie a $\lim_{x \rightarrow \infty} (f(x)-g(x)=0)$. Potom $f=g$. Dokážte!
\end{defproblem}

\begin{defproblem}{limita-85}
\begin{multicols}{2}
\begin{enumerate}
    \item $\lim_{x \rightarrow 0} \frac{\cos 3x^3-1}{\sin ^6 2x}$;
    \item $\lim_{x \rightarrow 0} \frac{\sin (a+2x)-2\sin (a+x)+\sin a}{x^2}$;
    \item $\lim_{x \rightarrow 0} \frac{4\sin (\frac{\pi}{6}+x)\sin (\frac{\pi}{6}+2x)-1}{\sin x}$;
    
    \item $\lim_{x \rightarrow 2} \frac{\arctan (x^2-2x)}{\sin 3\pi x}$;
    \item $\lim_{x \rightarrow 0} \frac{\arctan \frac{1}{x}+2}{x^2}$;
    \item $\lim_{x \rightarrow 0} \frac{1-\cos x\sqrt{\cos 2x}\sqrt[3]{\cos 3x}}{x^2}$;
    \item $\lim_{x \rightarrow \frac{\pi}{3}} \frac{\tan^3 x-3 \tan c}{\cos (x+\frac{\pi}{6})}$;
    \item $\lim_{x \rightarrow 0} \frac{\arcsin (1-x)}{\sqrt{x}}$;
    \item $\lim_{x \rightarrow \frac{1}{2}} \frac{\arccos x}{(2x-1)^2}$;
    \item $\lim_{x \rightarrow \frac{\pi}{2}} \frac{1-\sin ^3 x}{\cos ^2 x}$;
    \item $\lim_{x \rightarrow 1} \frac{\sqrt{\sin \frac{\pi x}{2}}+\sqrt[3]{\sin \frac{3\pi x}{2}}}{\sqrt{x^2+1}-\sqrt{2x}}$;
    \item $\lim_{x \rightarrow 1} \frac{\cos \frac{\pi x}{2}}{x-1}\cdot \frac{\sqrt{x+3}-2}{\sqrt{x^3+3x}-\sqrt{3x^2+1}}$;
    \item $\lim_{x \rightarrow 0} \frac{1-\sqrt{\cos x}}{1-\cos \sqrt{x}}$;
    \item $\lim_{x \rightarrow \frac{\pi}{2}} \sqrt{\frac{(2x-\pi)\sin \frac{\pi}{2x-\pi}}{\cos 4x}}$;
    \item $\lim_{x \rightarrow \infty} (\sin \ln (x+1)-\sin \ln x)$.
\end{enumerate}
\end{multicols}
\end{defproblem}

\begin{defproblem}{limita-86}
\begin{enumerate}
\item Nech $q \in (0,1)$, nech pre postupnosť ${\{a_n\}}_{n=1}^\infty$ nezáporných čísel platí $\sqrt[n]{a_n}\leq q,(n \in \mathbb{N})$. Potom $\lim_{n \rightarrow \infty} a_n=0$. Dokážte!
\item Nájdite $\lim_{n \rightarrow \infty}\frac{n^{n-1}}{(2n^2+n+1)^{\frac{n+1}{2}}}$.
\end{enumerate}
\end{defproblem}

\begin{defproblem}{limita-87}
Dokážte, že
\begin{enumerate}
\item $\lim_{n \rightarrow \infty} \frac{n^k}{a^n}=0,(k \in \mathbb{R},a>1)$;
\item $\lim_{n \rightarrow \infty} \frac{a^n}{n!}=0$;
\item $\lim_{n \rightarrow \infty} \frac{1}{\sqrt[n]{n!}}=0$.
\end{enumerate}
\end{defproblem}

\begin{defproblem}{limita-88}
\begin{multicols}{2}
\begin{enumerate}
    \item $\lim_{n \rightarrow \infty} \sqrt[n]{\frac{2n^2-5n+3}{n^5+1}}$;
    \item $\lim_{n \rightarrow \infty} \sqrt[n]{a^n+b^n},(a,b>0)$;
    \item $\lim_{n \rightarrow \infty} \sqrt[n]{3^n+n2^n}$;
    \item $\lim_{n \rightarrow \infty} (1+11^n)^{\frac{1}{n+2}}$;
    \item $\lim_{n \rightarrow \frac{\pi}{2}} \frac{2^nn!}{n^n}$;
    \item $\lim_{n \rightarrow \infty} \frac{3^nn!}{n^n}$;
    \item $\lim_{n \rightarrow \infty} \frac{n!}{4^n}$;
    \item $\lim_{n \rightarrow \infty} \frac{n^3+3^n}{n+3^{n+1}}$;
    \item $\lim_{n \rightarrow \infty} \frac{2^{\frac{n}{2}}+(n+1)!}{n(3^n+n!)}$;
    \item $\lim_{n \rightarrow \infty} \frac{2}{1-\sqrt[n]{n}}$.
\end{enumerate}
\end{multicols}
\end{defproblem}

\begin{defproblem}{limita-89}
Nech $x_n=\sum_{k=1}^n \frac{1}{\sqrt{n^2+k}}$; ukážeme dva spôsoby výpočtu $\lim_{n \rightarrow \infty} x_n:$
\begin{enumerate}
\item odhadneme $x_n$ zhora a zdola:
$$x_n=\sum_{k=1}^n \frac{1}{\sqrt{n^2+n}}\leq \sum_{k=1}^n \frac{1}{\sqrt{n^2+k}}\leq \sum_{k=1}^n \frac{1}{\sqrt{n^2}}=1;$$
teda $$\frac{n}{\sqrt{n^2+n}}\leq x_n \leq 1, (n \in \mathbb{N}).$$
Pretože $\lim_{n \rightarrow \infty} \frac{n}{\sqrt{n^2+n}}=1=\lim_{n \rightarrow \infty} 1$, je podľa vety $6$ $\lim_{n \rightarrow \infty} x_n=1$.
\item Podľa vety o limite súčtu je 
$$\lim_{n \rightarrow \infty} \sum_{k=1}^n \frac{1}{\sqrt{n^2+k}}=\lim_{n \rightarrow \infty} \frac{n}{\sqrt{n^2+1}}+\lim_{n \rightarrow \infty} \frac{n}{\sqrt{n^2+2}}+...+\lim_{n \rightarrow \infty} \frac{n}{\sqrt{n^2+n}}=0+0+...+0=0.$$
Zrejme aspoň jeden z uvedených postupov je nesprávny. Ktorý to je a v čom spočíva chyba?
\end{enumerate}
\end{defproblem}

\begin{defproblem}{limita-90}
Sformulujte a dokážte pravidlá pre výpočet limít typu $0^{+\infty}$ a $0^{-\infty}$.
\end{defproblem}

\begin{defproblem}{limita-91}
Nájdite limity:
\begin{multicols}{2}
\begin{enumerate}
    \item $\lim_{n \rightarrow \infty} (\frac{3x^2-x+1}{2x^2+x+1})^{\frac{x^3}{1-x}}$;
    \item $\lim_{n \rightarrow \infty} \tan ^n (\frac{\pi}{4}+\frac{1}{n})$;
    \item $\lim_{n \rightarrow \infty} (\frac{2^x}{1+x^{x+1}})^{-x^2}$;
    \item $\lim_{n \rightarrow 0} \sqrt[x]{\cos \sqrt{x}}$;
    \item $\lim_{n \rightarrow 0} (1-\cos x)^{\frac{1}{x^2}}$;
    \item $\lim_{n \rightarrow \frac{\pi}{4}} (\tan (\frac{\pi}{8}+x))^{\tan 2x}$;
    \item $\lim_{n \rightarrow 0} (\frac{1+\sin x \cos ^ \alpha x}{1+\sin x \cos \beta x})^{\coth ^3 x}$;
    \item $\lim_{n \rightarrow \frac{\pi}{3}} (\cos x)^{\frac{1}{(3x-\pi)^3}}$;
    \item $\lim_{n \rightarrow \infty} (\frac{a_1x+b_1}{a_2x+b_2})^x,(a_1,a_2>0)$;
    \item $\lim_{n \rightarrow 1} (\frac{\sin (x-1)}{x-1})^{\frac{\sin (x-1)}{x-1-\sin (x-1)}}$;
    \item $\lim_{n \rightarrow \infty} \frac{(x+a)^{x+a}(x+b)^{x+b}}{(x+a+b)^{2x+a+b}}$.
\end{enumerate}
\end{multicols}
\end{defproblem}

\begin{defproblem}{limita-92}
Nájdite limity:
\begin{multicols}{2}
\begin{enumerate}
    \item $\lim_{n \rightarrow 1} (1-x)\log_x 2$;
    \item $\lim_{n \rightarrow \infty} \ln (1+2^x) \ln (1+\frac{3}{x})$;
    \item $\lim_{n \rightarrow \infty} \ln \frac{x+\sqrt{x^2+1}}{x+\sqrt{x^2-1}} \cdot \ln ^{-2} \frac{x+1}{x-1}$;
    \item $\lim_{n \rightarrow a} \frac{x^\alpha-a^\alpha}{x-a},(a>0,\alpha \in \mathbb{R})$;
    \item $\lim_{n \rightarrow a} \frac{x^x-a^a}{x-a},(a>0)$;
    \item $\lim_{n \rightarrow \infty} (\frac{\sqrt[n]{a}+\sqrt[n]{b}}{2})^n,(a.b>0)$;
    \item $\lim_{n \rightarrow \infty} n^2(\sqrt[n]{x}-\sqrt[n+1]{x}),(x>0)$;
    \item $\lim_{n \rightarrow 0} \frac{a^{x^2}-b^{x^2}}{(a^x-b^x)^2},(a,b>0)$;
    \item $\lim_{n \rightarrow 0} (\frac{a^{x^2}-b^{x^2}}{(a^x-b^x)^2})^{\frac{1}{x}},(a,b>0)$;
    \item $\lim_{n \rightarrow 0} \frac{\cos (xe^x)-\cos (xe^{-x})}{x^3}$;
    \item $\lim_{n \rightarrow 1} \frac{\sin ^2 \pi x^\alpha}{\sin ^2 \pi x^\beta},(\beta \neq 0)$;
    \item $\lim_{n \rightarrow 1} \frac{\sin ^2 \pi 2^{x}}{\ln (\cos \pi 2^{x})}$.
\end{enumerate}
\end{multicols}
\end{defproblem}

\begin{defproblem}{limita-93}
\begin{enumerate}
\item Dokážte, že:
\begin{multicols}{2}
\begin{enumerate}
    \item $\lim_{n \rightarrow \infty} \frac{x^{\alpha}}{a^x}=0,(a>1,\alpha>0)$;
    \item $\lim_{x \rightarrow \infty} \frac{\log_a x}{x^\alpha}=0,(a>1,\alpha >0)$.
\end{enumerate}
\end{multicols}
\item Nájdite:
\begin{multicols}{2}
\begin{enumerate}
    \item $\lim_{n \rightarrow 0} x^{\frac{1}{100}^{-\frac{1}{x^2}}}$;
    \item $\lim_{x \rightarrow 0} x \ln x$;
    \item $\lim_{x \rightarrow \infty} (\frac{1}{n})^{\tan \frac{1}{n}}$.
\end{enumerate}
\end{multicols}
\textit{Poznámka:}
Tvrdenie z príkladu $192.(a)$ si možno zapamätať v nasledujúcej symbolickej podobe $\log_a x\ll x^\alpha \ll a^x$ (pre $x$ dostatočne veľké, $\alpha >0, a>0$)
kde $f\ll g$ (pre dostatočne veľké x) znamená $\lim_{x \rightarrow \infty} \frac{f(x)}{g(x)}=0$.
Pre postupnosti platí (pozri aj príklad $186$):
$\log_a n\ll n^\alpha\ll a^n\ll n!$ (pre $n$ dostatočne veľké,  $\alpha >0, a>0$)
\end{enumerate}
\end{defproblem}

\begin{defproblem}{limita-94}
Nech $\lim_{x \rightarrow a} \frac{\alpha (x)}{\alpha_1 (x)}$ a $\lim_{x \rightarrow a} \frac{\beta (x)}{\beta_1(x)}$ sú konečné a nenulové. Potom $\lim_{x \rightarrow a} \frac{\alpha (x)}{\beta (x)}$ existuje práve vtedy, keď existuje $\lim_{x \rightarrow a} \frac{\alpha_1 (x)}{\beta_1 (x)}$. Dokážte!
\end{defproblem}

\begin{defproblem}{limita-95}
Nech $a$ je definovaná na $\mathbb{R}$ a $\lim_{x \rightarrow 0} f(x)=+\infty$. Zostrojte funkciu $g: \mathbb{R} \rightarrow \mathbb{R}$, pre ktorú existuje nenulová $\lim_{x \rightarrow 0} g(x)$ a neexistuje $\lim_{x \rightarrow 0} \frac{f(x)}{g(x)}$.
\end{defproblem}

\begin{defproblem}{limita-96}
Dokážte, že nasledujúce postupnosti sú konvergentné:
\begin{enumerate}
\item $a_n=\frac{(2n)!!}{(2n+1)!!}$, kde $(2n)!!=2\cdot4\cdot...\cdot2n,(2n+1)!!=1\cdot3\cdot...\cdot(2n+1)$;
\item $a_n=1+\frac{1}{2^2}+\frac{1}{3^2}+...+\frac{1}{n^2}$;
\item $a_n=(1+\frac{1}{2})\cdot(1+\frac{1}{4})\cdot...\cdot(1+\frac{1}{2^n})$.
\end{enumerate}
\end{defproblem}

\begin{defproblem}{limita-97}
Nech $b_1=1,b_n=\sqrt{1+\sqrt{2+...+\sqrt{n}}}$. Potom ${\{b_n\}}_{n=1}^\infty$ je konvergentná postupnosť. Dokážte!
\end{defproblem}

\begin{defproblem}{limita-98}
Nájdite $\lim_{n \rightarrow \infty} a_n$, ak
\begin{multicols}{2}
\begin{enumerate}
    \item $a_1=0,a_2=1,a_{n+2}=\frac{a_n+a_{n+1}}{2}$;
    \item $a_1=a,a_2=b,a_{n+2}=\frac{a_n+a_{n+1}}{2}$.
\end{enumerate}
\end{multicols}
\end{defproblem}

\begin{defproblem}{limita-99}
Nech $f$ je funkcia definovaná na $\mathbb{R}$, $a \in \mathbb{R^*},b \in \mathbb{R^*}$. Rozhodnite, či tvrdenie $\lim_{x \rightarrow f(x)=b}$ je ekvivalentné s niektorým z nasledujúcich výrokov:
\begin{enumerate}
\item pre každú monotónnu postupnosť ${\{a_n\}}_{n=1}^\infty$ takú, že $\lim_{x \rightarrow \infty} a_n=a$, pričom $a_n \neq a,(n \in \mathbb{N})$ platí $\lim_{x \rightarrow \infty} f(a_n)=b$;
\item z každej postupnosti ${\{a_n\}}_{n=1}^\infty$, ktorej limitou je $a$, pričom $a_n \neq a ,(n \in \mathbb{N})$, možno vybrať postupnosť ${\{a_{n(k)}\}}_{n=1}^\infty$ takú, že $\lim_{x \rightarrow \infty} f(a_{n(k)})=b$.
\end{enumerate}
\end{defproblem}

\begin{defproblem}{limita-100}
Nech $a$ je hromadný bod definičného oboru $D(f)$ funkcie $f$. Dokážte, že nasledujúce dve podmienky sú ekvivalentné:
\begin{enumerate}
\item neexistuje $\lim_{x \rightarrow \infty} f(x)=b$;
\item existujú postupnosti ${\{a_n\}}_{n=1}^\infty$, ${\{b_n\}}_{n=1}^\infty$ prvkov z $D(f) \setminus \{a \}$ také, že $\lim_{n \rightarrow \infty} a_n=lim_{n \rightarrow \infty} b_n=a$, $\lim_{x \rightarrow \infty} f(a_n)$ a $\lim_{x \rightarrow \infty} f(b_n)$ existujú a nerovnajú sa.
\end{enumerate}
\end{defproblem}

\begin{defproblem}{limita-101}
Nech je daná ohraničená postupnosť ${\{a_n\}}_{n=1}^\infty$. Dokážte, že 
\begin{multicols}{2}
\begin{enumerate}
    \item $\limsup_{n \rightarrow \infty} a_n=\inf \{\sup_{k \geq n} a_k: n \in \mathbb{N}\}$;
    \item $\liminf_{n \rightarrow \infty} a_n= \sup \{\inf_{k \geq n} a_k: n \in \mathbb{N}\}$.
\end{enumerate}
\end{multicols}
\end{defproblem}

\begin{defproblem}{limita-102}
Existuje postupnosť taká, že množina jej hromadných hodnôt je
\begin{enumerate}
\item $\langle 0,1 \rangle$;
\item $(0,1)$?
\end{enumerate}
\end{defproblem}

\begin{defproblem}{limita-103}
Nech ${\{a_n\}}_{n=1}^\infty$ je ohraničená postupnosť a $\lim_{n \rightarrow \infty} (a_{n+1}-a_n)=0$. Dokážte, že množina $H$ hromadných hodnôt postupnosti ${\{a_n\}}_{n=1}^\infty$ je množina $\{x \in \mathbb{R}: \liminf_{n \rightarrow \infty} a_n \leq x \leq \limsup_{n \rightarrow \infty} a_n\}$.
\end{defproblem}

\begin{defproblem}{limita-104}
Nech v postupnosti ${\{a_n\}}_{n=1}^\infty$ konvergujú podpostupnosti ${\{a_{2k}\}}_{k=1}^\infty$,${\{a_{2k-1}\}}_{k=1}^\infty$,${\{a_{3k}\}}_{k=1}^\infty$. Dokážte, že potom konverguje aj postupnosť ${\{a_n\}}_{n=1}^\infty$.
\end{defproblem}

\begin{defproblem}{limita-105}
Aké postupnosti vyhovujú podmienke
\begin{enumerate}
\item $\forall \varepsilon >0 \forall n_0 \in \mathbb{N} \exists n \in \mathbb{N},n>n_0: |a_n|< \varepsilon$;
\item $\forall \varepsilon >0 \exists n_0 \in \mathbb{N} \exists n \in \mathbb{N},n>n_0: |a_n|< \varepsilon$?
\end{enumerate}
\end{defproblem}

\begin{defproblem}{limita-106}
\begin{enumerate}
\item Nech pre postupnosť ${\{a_n\}}_{n=1}^\infty$ kladných čísel platí $\limsup_{n \rightarrow \infty} a_n \cdot \limsup_{n \rightarrow \infty} \frac{1}{a_n}=1$. Potom ${\{a_n\}}_{n=1}^\infty$ je konvergentná postupnosť. Dokážte!
\item Pre ktoré postupnosti ${\{a_n\}}_{n=1}^\infty$ kladných čísel platí vzťah $\liminf_{n \rightarrow \infty} a_n \cdot \limsup_{n \rightarrow \infty} \frac{1}{a_n}=1$?
\end{enumerate}
\end{defproblem}

\begin{defproblem}{limita-107}
Nech ${\{a_n\}}_{n=1}^\infty$, ${\{b_n\}}_{n=1}^\infty$ sú ohraničené postupnosti nezáporných čísel. Dokážte nasledujúce tvrdenia:
\begin{enumerate}
\item ak existuje konečná $\lim_{n \rightarrow \infty} a_n$, tak $\liminf_{n \rightarrow \infty} a_n \vdots b_n=\lim_{n \rightarrow \infty} a_n \cdot \limsup_{n \rightarrow \infty} b_n$;
\item $\liminf_{n \rightarrow \infty} a_n \cdot \liminf_{n \rightarrow \infty} b_n \leq \liminf_{n \rightarrow \infty} a_n \cdot b_n \leq \liminf_{n \rightarrow \infty} a_n \cdot \limsup_{n \rightarrow \infty} b_n \leq \limsup_{n \rightarrow \infty} a_n \cdot b_n \leq \limsup_{n \rightarrow \infty} a_n \cdot \limsup_{n \rightarrow \infty} b_n$.
\end{enumerate}
\end{defproblem}

\begin{defproblem}{limita-108}
Nech ${\{a_n\}}_{n=1}^\infty$ je postupnosť kladných čísel, nech $\liminf_{n \rightarrow \infty} a_n =0$. Dokážte, že existuje nekonečne veľa indexov $n$ takých, že platí $$\forall k \in \mathbb{N}: k<n \Rightarrow a_k>a_n$$
(t.j. $a_n$ je menšie než všetky predchádzajúce členy postupnosti ${\{a_n\}}_{n=1}^\infty$).
\end{defproblem}

\begin{defproblem}{limita-109}

\end{defproblem}

\begin{defproblem}{limita-110}

\end{defproblem}

\begin{defproblem}{limita-111}

\end{defproblem}

\begin{defproblem}{limita-112}

\end{defproblem}

\begin{defproblem}{limita-113}

\end{defproblem}

\begin{defproblem}{limita-114}

\end{defproblem}

\begin{defproblem}{limita-115}

\end{defproblem}

\begin{defproblem}{limita-116}

\end{defproblem}

\begin{defproblem}{limita-117}

\end{defproblem}

\begin{defproblem}{limita-118}

\end{defproblem}

\begin{defproblem}{limita-119}

\end{defproblem}

\begin{defproblem}{limita-120}

\end{defproblem}

\begin{defproblem}{limita-121}

\end{defproblem}

\begin{defproblem}{limita-122}

\end{defproblem}

\begin{defproblem}{limita-123}

\end{defproblem}

\begin{defproblem}{limita-124}

\end{defproblem}

\begin{defproblem}{limita-125}

\end{defproblem}

\begin{defproblem}{limita-126}

\end{defproblem}

\begin{defproblem}{limita-127}

\end{defproblem}

\begin{defproblem}{limita-128}

\end{defproblem}

\begin{defproblem}{limita-129}

\end{defproblem}

\begin{defproblem}{limita-130}

\end{defproblem}

\begin{defproblem}{limita-131}

\end{defproblem}

\begin{defproblem}{limita-132}

\end{defproblem}

\begin{defproblem}{limita-133}

\end{defproblem}

\begin{defproblem}{limita-134}

\end{defproblem}

\begin{defproblem}{limita-135}

\end{defproblem}

\begin{defproblem}{limita-136}

\end{defproblem}

\begin{defproblem}{limita-137}

\end{defproblem}

\begin{defproblem}{limita-138}

\end{defproblem}

\begin{defproblem}{limita-139}

\end{defproblem}

\begin{defproblem}{limita-140}

\end{defproblem}

\begin{defproblem}{limita-141}

\end{defproblem}

\begin{defproblem}{limita-142}

\end{defproblem}

\begin{defproblem}{limita-143}

\end{defproblem}

\begin{defproblem}{limita-144}

\end{defproblem}

\begin{defproblem}{limita-145}

\end{defproblem}

\begin{defproblem}{limita-146}

\end{defproblem}

\begin{defproblem}{limita-147}

\end{defproblem}

\begin{defproblem}{limita-148}

\end{defproblem}

\begin{defproblem}{limita-149}

\end{defproblem}

\begin{defproblem}{limita-150}

\end{defproblem}
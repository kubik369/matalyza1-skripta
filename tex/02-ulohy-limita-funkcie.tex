\begin{defproblem}{limita-1}
Dokážte, že bod $a$ je hromadný bod množiny $A$, ak
	
\begin{enumerate}
	\item $a=0 \quad A={x \in \mathbb{R}; \sin{\frac{1}{x}} = 0}$
	\item $a=-\infty \quad A={x \in \mathbb{R}; \cos{x} = \frac{1}{2}}$ 
	\item $a=\frac{1}{9} \quad A={\frac{m}{10^n}; m,n \in \mathbb{N}}$
\end{enumerate}
	
(teda $A$ je množina všetkých kladných čísel, ktorých zápis v desiatkovej sústave má konečný počet nenulových cifier za desatinnou čiarkou).
	
\end{defproblem}

\begin{defproblem}{limita-2}
Nájdite všetky hromadné body množín
\begin{enumerate}
\item $A= \langle 0,1 \rangle$ ;
\item $B= \{ (-1)^n n; n\in \mathbb{N} \} $;
\item $C= (-2,\infty)$ ;
\item $D= \{\frac{m}{n}; m,n \in \mathbb{N} \}$;
\item $E= {\ \frac{m}{n}; m<n,m,n \in \mathbb{Z}}$;
\item $F= \langle 1,2 ) \ \mathbb{Q}$;
\end{enumerate}
\end{defproblem}

\begin{defproblem}{limita-3}
Pomocou symbolov $\forall,\exists$ zapíšte výroky:
\begin{enumerate}
\item Bod $\infty$ nie je hromadný bod množiny $M$;
\item Množina $M$ nemá hromadné body.
\end{enumerate}
\end{defproblem}

\begin{defproblem}{limita-4}
Uveďte príklad neprázdnej množiny $A \subset \mathbb{R}$ takej, že
\begin{enumerate}
\item $A'=\emptyset$;
\item $A'={\ 1, +\infty}$;
\item $A'={\ -\infty,+\infty}$;
\item $A'$ je nespočítateľná množina;
\item $A'$ je nekonečná a spočítateľná množina.
\end{enumerate}
\end{defproblem}

\begin{defproblem}{limita-5}
Nech $A \subset \mathbb{R}$ je zhora ohraničená neprázdna množina, nech $sup A \notin A$. Potom $sup A$ je hromadný bod množiny $A$. Dokážte!
\end{defproblem}

\begin{defproblem}{limita-6}
 Na základe definície dokážte nasledujúce tvrdenia:
 \begin{enumerate}
 \item $\lim_{n \rightarrow \infty} \frac{3n^2+1}{5n^2-1}=\frac{3}{5}$ (pre ktoré $n \in \mathbb{N}$ platí: 
 \begin{enumerate}
 \item $|\frac{3n^2+1}{5n^2-1}-\frac{3}{5}|<0,5;$
 \item $<0,005;$
 \item $<0,00005 ?);$
 \end{enumerate}
 
 \item $\lim_{n \rightarrow \infty} \frac{n^2+3n+1}{2n^2+2}=\frac{1}{2}$;
 \item $\lim_{n \rightarrow \infty} \frac{n^2}{n+8}=+\infty$ 
 
 (počínajúc ktorým prirodzeným číslom platí nerovnosť $\frac{n^2}{n+8}>10^3$?);
 \item $\lim_{n \rightarrow \infty} (\frac{5}{n}-n)=-\infty$;
 \item $\lim_{n \rightarrow \infty} q^n=0$  $(|q|<1).$
 \end{enumerate}
\end{defproblem}

\begin{defproblem}{limita-7}

Rozhodnite, či existujú limity nasldujúcich postupností (nezabúdajte, že svoje tvrdenie musíte dokázať):
\begin{enumerate}
\item 
 $$a_n = \left\{ \begin{array}{r@{\quad}c}
    1-\frac{1}{n}, & $ak n $\in \mathbb{N}\ $je párne$ \\
    1+\frac{1}{n^2}, & $ak n $\in \mathbb{N}\ $je nepárne$ \\ \end{array} \right.
    $$ 
\item $$a_n=\frac{cos \frac{n \pi}{2}}{n}$$
\end{enumerate}
\end{defproblem}

\begin{defproblem}{limita-8}
Postupnosť ${\{a_n\}}_{n=1}^\infty$ je daná vzťahom $a_n=n(1-(-1)^n)).$ Dokážte, že
\begin{enumerate}
\item číslo $0$ nie je limitou tejto postupnosti;
\item bod $+\infty$ nie je limitou tejto postupnosti;
\item žiadne $b \in \mathbb{R^*}$ nie je limitou tejto postupnosti.
\end{enumerate}
\end{defproblem}

\begin{defproblem}{limita-9}
Pri formulácii definície vlastnej limity postupnosti študent:
\begin{enumerate}
\item namiesto "pre ľubovoľné $\varepsilon > 0$" povedal "pre ľubovoľné $\varepsilon$". Existujú postupnosti, ktoré majú limitu pri takejto definícii?
\item definíciu napísal takto:
$$\forall \varepsilon > 0 \exists n_0 \in \mathbb{N} \forall n \in \mathbb{N}: |a_n-b|<\varepsilon.$$
Ktoré postupnosti by mali limitu pri takejto definícii?
\item namiesto "pre každé $\varepsilon > 0$" povedal "aspoň pre jedno $\varepsilon > 0$". Ukážte, že pri takejto definícii je číslo $7$ limitou postupnosti $2,2, ...$ 
\item namiesto "existuje $n_0 \in \mathbb{N}$" povedal "pre všetky $n_0 \in \mathbb{N}$". Ktoré postupnosti majú limitu pri takejto definícii?
\item definćiu napísal takto:
$$\forall \varepsilon > 0 \exists n_0 \in \mathbb{N} \forall n \in \mathbb{N},n>n_0:a_n-b<\varepsilon.$$
Ukážte, že pri takejto definícii je čáslo $5$ limitou postupnosti $1,1,1,...$ .
\end{enumerate}
\end{defproblem}

\begin{defproblem}{limita-10}
Je číslo $b \in \mathbb{R}$ limitou postupnosti ${\{a_n\}}_{n=1}^\infty$, ak existuje také prirodzené číslo $\mathbb{N^*}$, že pre ľubovoľné $\varepsilon>0$ a všetky $n \in \mathbb{N}, n>\mathbb{N^*}$ platí $|a_n-b|<\varepsilon$? 
\end{defproblem}

\begin{defproblem}{limita-11}
Nájdite všetky postupnosti ${\{x_n\}}_{n=1}^\infty$, ktoré vyhovujú podmienke
\begin{enumerate}
\item $$\exists \varepsilon>0 \forall n_0 \in \mathbb{N} \forall n \in \mathbb{N}, n>n_0: |x_n|<\varepsilon; $$
\item $$\forall \varepsilon>0 \forall n_0 \in \mathbb{N} \forall n \in \mathbb{N}, n>n_0: |x_n|<\varepsilon; $$
\item $$\exists \varepsilon>0 \exists n_0 \in \mathbb{N} \forall n \in \mathbb{N}, n>n_0: |x_n|<\varepsilon. $$
\end{enumerate}
\end{defproblem}

\begin{defproblem}{limita-12}
Prepíšte definíciu limity funkcie pre prípady $1-9$. (Všimnite si, že limity postupností sú samy špeciálnym prípadom limít $4-6$.)
\end{defproblem}

\begin{defproblem}{limita-13}
Na základe definície limity dokážte tieto tvrdenia:
\begin{enumerate}
\item $\lim_{x \rightarrow 3} \sqrt{x}=\sqrt{3}$;
\item $\lim_{x \rightarrow 1} \frac{1}{(1-x^2)^2}=+\infty$;
\item $\lim_{x \rightarrow -\infty}x^3=-\infty$;
\item $\lim_{x \rightarrow 8} \sqrt[3]{x}=2$;
\item $\lim_{x \rightarrow -2} x^2=4$.
\end{enumerate}
\end{defproblem}

\begin{defproblem}{limita-14}
Nech bod $0$ je hromadný bod definičného oboru funkcie $f$. Pomocou symbolov $\forall, \exists$ zapíšte tieto tvrdenia:
\begin{enumerate}
\item Číslo $4$ nie je limitou funkcie $f$ v bode $0$;
\item Funkcia $f$ nemá v bode $0$ limitu.
\end{enumerate}
\end{defproblem}

\begin{defproblem}{limita-15}
Ak existuje $\lim_{x \rightarrow a} f(x)=b$ ($a \in \mathbb{R^*},b \in \mathbb{R}$), tak existuje aj $\lim_{x \rightarrow a} |f(x)|$ a platí $\lim_{x \rightarrow a} |f(x)|=|b|$. Dokážte; platí aj opačná implikácia?
\end{defproblem}

\begin{defproblem}{limita-16}
\begin{enumerate}
\item Dokážte implikáciu v Cauchyho-Bolzanovom kritériu konvergencie (t.j. dokážte, že tvrdenie (*) z vety $1$ je nutná podmienka existencie vlastnej limity funkcie $f$ v bode $a$).
\item Dokážte, že Dirichletova funkcia ani funkcia $sin \frac{1}{x}$ nemajú limitu v bode $0$. (Neexistenciu konečných limít možno dokázať na základe píkladu $115.1$; neexistenciu nevlastných limít treba dokázať samostatne.)
\end{enumerate}
\end{defproblem}

\begin{defproblem}{limita-17}
Nájdite nasledujúce limity:

\begin{multicols}{2}
\begin{enumerate}
    \item $\lim_{{x \rightarrow \infty}} \frac{x^2-1}{2x^2-x+1}$;
    \item $\lim_{{x \rightarrow -\infty}} \frac{3x^3+5x^2-2}{2x^4-7}$;
    \item $\lim_{{x \rightarrow \infty}} \frac{(x-1)(x-2)...(x-5)}{(5x-1)^5}$;
    \item $\lim_{{x \rightarrow \infty}} (\frac{x^3}{3x^2-4}-\frac{x^2}{3x+2})$;
    \item $\lim_{{x \rightarrow \infty}} \frac{1}{n}[(x+\frac{a}{n})+(x+\frac{2a}{n})+...+(x+\frac{n-1}{n}a)]$;
    \item $\lim_{{x \rightarrow \infty}} \frac{(-2)^n+3^n}{(-2)^{n+1}+3^{n+1}}$.
\end{enumerate}
\end{multicols}
\end{defproblem}

\begin{defproblem}{limita-18}
Nájdite nasledujúce limity:

\begin{multicols}{2}
\begin{enumerate}
    \item $\lim_{{x \rightarrow \infty}} \frac{x^2-5x+6}{x^3-6x^2+10x-3}$;
    \item $\lim_{{x \rightarrow -\infty}} \frac{x^2-1}{2x^2-x-1}$;
    \item $\lim_{{x \rightarrow \infty}} \frac{x^2-1}{2x^2-x-1}$;
    \item $\lim_{{x \rightarrow \infty}} \frac{x^4-3x+2}{x^5-4x+3}$;
    \item $\lim_{{x \rightarrow \infty}} \frac{2x^3-5x^2-4x+12}{5x^2-4x-12}$;
    \item $\lim_{{x \rightarrow \infty}} \frac{(x^3-x-2)^{20}}{(x^3-12x+16)^{10}}$;
    \item $\lim_{{x \rightarrow \infty}} \frac{x^m-1}{x^n-1}$ $(m,n \in \mathbb{N})$;
    \item $\lim_{{x \rightarrow \infty}} (\frac{2}{2x-x^2}+\frac{1}{x^2-3x+2})$;
    \item $\lim_{{x \rightarrow \infty}} \frac{x+x^2+...+x^n-n}{x-1}$;
    \item $\lim_{{x \rightarrow \infty}} \frac{x^{100}-2x+1}{x^{50}-2x+1}$.
\end{enumerate}
\end{multicols}
\end{defproblem}

\begin{defproblem}{limita-19}
Zostrojte funkcie $f,g$ definované na $\mathbb{R}$ tak, aby neexistovali $\lim_{x \rightarrow 1} f(x)$ ani $\lim_{x \rightarrow 1} g(x)$ a exitovala konečná 
\begin{multicols}{2}
\begin{enumerate}
    \item $\lim_{{x \rightarrow 1}} (f(x)+g(x))$;
    \item $\lim_{{x \rightarrow 1}} (f(x) \cdot g(x))$.
\end{enumerate}
\end{multicols}
\end{defproblem}

\begin{defproblem}{limita-20}
Možno nájsť postupnosť ${\{a_n\}}_{n=1}^\infty$ takú, že existuje konečná $\lim_{n \rightarrow \infty} a_n$ a neexistuje $\lim_{n \rightarrow \infty} \frac{1}{a_n}$?
\end{defproblem}

\begin{defproblem}{limita-21}
\begin{multicols}{2}
\begin{enumerate}
    \item $\lim_{{x \rightarrow -2}} \frac{\sqrt[3]{x-6}+2}{\sqrt{x^2-3}-1}$;
    \item $\lim_{{x \rightarrow 4}} \frac{\sqrt{1+2x}-3}{\sqrt{x}-2}$;
    \item $\lim_{{x \rightarrow a}} \frac{\sqrt{x}-\sqrt{a}+\sqrt{x-a}}{\sqrt{x^2-a^2}}$ $(a>0)$;
    \item $\lim_{{x \rightarrow 3}} \frac{\sqrt{x+13}-2\sqrt{x+1}}{x^2-9}$;
    \item $\lim_{{x \rightarrow 0}} \frac{\sqrt[3]{27+x}-\sqrt[3]{27-x}}{x+2\sqrt[3]{x^4}}$;
    \item $\lim_{{x \rightarrow 1}} \frac{(1-\sqrt{x})(1-\sqrt[3]{x})...(1-\sqrt[n]{x})}{(1-x)^{n-1}}$.
\end{enumerate}
\end{multicols}
\end{defproblem}

\begin{defproblem}{limita-22}
\begin{enumerate}
\item $\lim_{x \rightarrow 2} \frac{\sqrt{x+2}-\sqrt[3]{x^2+4}}{x-2}$;
\item  $\lim_{x \rightarrow 7} \frac{\sqrt{x+2}-\sqrt[3]{x^2+20}}{\sqrt[4]{x+9}-2}$;
\item  $\lim_{x \rightarrow 0} \frac{\sqrt[3]{1+\frac{x}{3}}-\sqrt[4]{1+\frac{x}{4}}}{1-\sqrt{1-\frac{x}{2}}}$.
\end{enumerate}
\end{defproblem}

\begin{defproblem}{limita-23}
\begin{enumerate}
\item $\lim_{x \rightarrow -1} \frac{1+\sqrt[3]{x}}{1+\sqrt[5]{x}}$;
\item $\lim_{x \rightarrow 0} \frac{\sqrt[n]{1+x}-1}{x}$  $(n \in \mathbb{N})$;
\item $\lim_{x \rightarrow 1} \frac{\sqrt[m]{x}-1}{\sqrt[n]{x}-1}$  $(m,n \in \mathbb{N})$;
\item $\lim_{x \rightarrow 1} (\frac{3}{1-\sqrt{x}}-\frac{2}{1-\sqrt[3]{x}})$.
\end{enumerate}
\end{defproblem}

\begin{defproblem}{limita-24}
\begin{multicols}{2}
\begin{enumerate}
    \item $\lim_{{x \rightarrow \infty}} \frac{\sqrt{x+\sqrt{x+\sqrt{x}}}}{\sqrt{x+1}}$;
    \item $\lim_{{x \rightarrow \infty}} \frac{\sqrt{x}+\sqrt[3]{x}+\sqrt[4]{x}}{\sqrt{2x+1}}$;
    \item $\lim_{{x \rightarrow \infty}} \frac{\sqrt{x^2+6}+|x|}{\sqrt[6]{x^4+2}-|x|}$;
    \item $\lim_{{x \rightarrow \infty}} (\sqrt{(x+a)(x+b)}-x)$;
    \item $\lim_{{x \rightarrow \infty}} (\sqrt[3]{x^3+x^2+1}-\sqrt[3]{x^3-x^2+1})$;
    \item $\lim_{{x \rightarrow \infty}} (\sqrt[n]{(x+a_1)(x+a_2)...(x+a_n)}-x)$;
    \item $\lim_{{x \rightarrow \infty}} x(\sqrt{x^2+2x}-2\sqrt{x^2+x}+x)$;
    \item $\lim_{{x \rightarrow \infty}} \frac{n}{2}(\sqrt[3]{1+\frac{2}{n}}-1)$.
\end{enumerate}
\end{multicols}
\end{defproblem}

\begin{defproblem}{limita-25}
Dokážte túto modifikáciu vety o limite zloženej funkcie: Nech $a \in \mathbb{R^*}$ je hromadný bod množiny $D(f \circ g)$, nech $A \in \mathbb{R}\cap D(f)$ je hromadný bod $D(f)$. Ak $\lim_{x \rightarrow a} g(x)=A$, $\lim_{x \rightarrow A} f(x)=f(A)$, tak $\lim_{x \rightarrow a} f(g(x))=f(A)$.

(Výhodou tejto modifikácie je, pri jej použití netreba overovať podmienku (*) vystupujúcu vo vete $3$.)
\end{defproblem}

\begin{defproblem}{limita-26}
Existuje funkcie $f,g$ definované na $\mathbb{R}$ taká, že $\lim_{x \rightarrow 1} g(x)=2$, $\lim_{x \rightarrow 2} f(x)$ existuje a $\lim_{x \rightarrow 1} f(g(x))$ neexistuje?
\end{defproblem}

\begin{defproblem}{limita-27}
Nájdite limity:
\begin{multicols}{2}
\begin{enumerate}
    \item $\lim_{x \rightarrow 0} \frac{sin 5x}{x}$;
    \item $\lim_{x \rightarrow 0} \frac{sin (x+1)}{x+1}$;
    \item $\lim_{x \rightarrow 0}  \frac{sin mx}{sin nx} (m,n \neq 0)$;
    \item $\lim_{x \rightarrow 0} \frac{sin (x^3+2x)}{x}$;
    \item $\lim_{x \rightarrow 0} x \cdot ctg 3x$;
    \item $\lim_{x \rightarrow \infty} 2^n sin \frac{x}{2^n}$  $(x \neq 0)$.
\end{enumerate}
\end{multicols}
\end{defproblem}

\begin{defproblem}{limita-28}
\begin{multicols}{2}
\begin{enumerate}
    \item $\lim_{{x \rightarrow 0}} \frac{1-cos x}{x^2}$;
    \item $\lim_{{x \rightarrow 0}} \frac{tg x-sin x}{sin ^3 x}$;
    \item $\lim_{{x \rightarrow 0}}  \frac{sin 5x-sin 3x}{sin x}$;
    \item $\lim_{{x \rightarrow 0}} (\frac{2}{sin 2x sin x}-\frac{1}{sin ^2 x})$;
    \item $\lim_{{x \rightarrow 0}} \frac{cos x - com 3x}{x^2} $
    \item $\lim_{x \rightarrow 0} \frac{1+sin x - cos x}{1+ sin px - cos px}$  $(p \neq 0)$.
\end{enumerate}
\end{multicols}
\end{defproblem}

\begin{defproblem}{limita-29}
\begin{multicols}{2}
\begin{enumerate}
    \item $\lim_{{x \rightarrow a}} \frac{sin x - sin a}{x-a}$;
    \item $\lim_{{x \rightarrow a}} \frac{ctg x - ctg a}{x-a}$;
    \item $\lim_{{x \rightarrow 0}}  \frac{cos (a+2x)-2 cos (a+x)+cos a}{x^2}$;
    \item $\lim_{{x \rightarrow 0}} (\frac{2}{sin (a+x) sin (a+2x)-sin ^2 a}{x})$.
\end{enumerate}
\end{multicols}
\end{defproblem}

\begin{defproblem}{limita-30}
\begin{multicols}{2}
\begin{enumerate}
    \item $\lim_{{x \rightarrow}} \frac{\pi}{4} tg 2x tg(\frac{\pi}{4}-x)$;
    \item $\lim_{{x \rightarrow}} \frac{\pi}{6} \frac{2 sin^2 x +sin x -1}{2 sin^2 x - 3 sin x + 1}$;
    \item $\lim_{{x \rightarrow}} \frac{\pi}{4}  \frac{1-ctg^3 x}{2- ctg x - ctg^3 x}$;
    \item $\lim_{{x \rightarrow}} \frac{sin(x-\frac{\pi}{3})}{1-2cos x}$;
    \item $\lim_{{x \rightarrow 1}} (1-x) tg \frac{\pi x}{2}$
    \item $\lim_{x \rightarrow 0} \frac{arcsin x}{x}$.
\end{enumerate}
\end{multicols}
\end{defproblem}

\begin{defproblem}{limita-31}
\begin{multicols}{2}
\begin{enumerate}
    \item $\lim_{{x \rightarrow 0}} \frac{\sqrt{1+tg x}-\sqrt{1+sin x}}{x^3}$;
    \item $\lim_{{x \rightarrow 0}} \frac{x^2}{\sqrt{1+x sin x}-\sqrt{cos x}}$;
    \item $\lim_{{x \rightarrow 0}} \frac{\sqrt{cos x}-\sqrt[3]{cos x}}{sin^2 x}$;
    \item $\lim_{{x \rightarrow 0}} \frac{\sqrt{1-cos x}}{1-cos \sqrt{x}}$;
    \item $\lim_{{x \rightarrow 0}} \frac{\sqrt{x+4}-2}{sin 5x}$.
\end{enumerate}
\end{multicols}
\end{defproblem}

\begin{defproblem}{limita-32}

\end{defproblem}

\begin{defproblem}{limita-33}

\end{defproblem}

\begin{defproblem}{limita-34}

\end{defproblem}

\begin{defproblem}{limita-35}

\end{defproblem}